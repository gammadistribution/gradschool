\begin{problem}{14.3.1}
  Show that if $p(z)$ is a polynomial having degree at least 2, then $p(\infty) = \infty$.
  Use the definition of $p'(\infty)$ to show that $|p'(\infty)| < 1$ so that $\infty$
  is an attracting point of $p$. What happens if $p(z) = a_1z + a_0$ for some $a_1,a_0 \in \mathbb{C}$.
\end{problem}

\begin{proof}
  If $p(z)$ is a polynomial of degree at least 2, then $p$ is conjugate to the polynomial $q$
  under the map $h(z) = 1/z$ where
  \begin{align}\label{conj}
    q(z) = h^{-1} \circ p \circ h (z) = \frac{z^n}{a_0 z^n + a_1z^{n-1} + \dots + a_n}.
  \end{align}
  Since $q(0) = 0$, we see that $0$ is a fixed point of $q$. Note that $p$ and $q$
  are conjugate so that the map $h$ preserves fixed points. Under $h$, we see that $0$ maps to $\infty$
  so that $\infty$ is a fixed point of $p$, i.e.\ $p(\infty) = \infty$.

  From \eqref{conj}, we see that
  \begin{align*}
    |q'(z)| = \left|\frac{n z^{n-1} \sum_{k=0}^n a_k z^{n-k} - z^n \sum_{k=0}^{n-1} a_k(n-k) z^{n-k-1}  }{\left( \sum_{k=0}^n a_k z^{n-k} \right)^2}\right|.
  \end{align*}
  Thus, by definition, we have that $|p'(\infty)| = |q'(0)| = 0 < 1$ and $\infty$
  is an attracting fixed point.

  If $p(z) = a_1 z + a_0$ for some $a_1,a_0 \in \mathbb{C}$, then we see that $p$
  is conjugate to $q$ under the map $h(z) = 1/z$ where
  \begin{align*}
    q(z) = \frac{z}{a_0z + a_1}.
  \end{align*}
  Since $q(0)=0$, we know that $p(\infty) = \infty$ so that $\infty$ is a fixed point of $p$.
  However, we see that
  \begin{align*}
    q'(z) = \frac{(a_0z + a_1) - a_0z}{(a_0z + a_1)^2} = \frac{a_1}{(a_0z + a_1)^2}
  \end{align*}
  which implies that $|p'(\infty)| = |q'(0)| = |a_1|$.
  Thus, $\infty$ will be attracting under $p$ if and only if $|a_1| < 1$.
\end{proof}
\newpage
