\begin{problem}{10.3.7}
  The first three steps in the construction of the fractal shown are indicated below.
  Determine the fractal dimension.
\end{problem}

\begin{proof}
  To begin, partition the unit square into nine equal squares and
  let the bottom left corner of the unit square be positioned at coordinate $(0, 0)$. For the first
  iteration, remove
  the ``outer middle squares'', i.e.\ the squares of side-length 1/3 with top left upper-coordinate in the set
  $\{(0, 2/3), (1/3, 1), (1/3, 1/3), (2/3, 2/3)\}$. For each subsequent iteration, remove the outer middle squares of the
  squares remaining from the previous iteration.

  Thus, in the first iteration, in order to cover the fractal, we would require 5 squares of side-length 1/3.
  In the second iteration, in order to cover the fractal, we would require 25 squares of side-length 1/9
  and for the third iteration we would require 125 squares of side-length 1/27.

  Let $K$ be the fractal shown. In general, for the $n$-th iteration, we would require $N_{\delta_n}(K) = 5^n$
  squares of length $\delta_n = 3^{-n}$ in order to completely cover $K$. Therefore, we see that the fractal dimension of $K$ is
  \begin{align*}
    \dim(K) = \lim_{\delta \to 0^+} \frac{\log N_\delta(K)}{\log 1/\delta} = \lim_{n\to\infty}\frac{\log N_{\delta_n}(K)}{\log 1/\delta_n} = \lim_{n\to\infty}\frac{\log 5^n}{\log 3^n} = \frac{\log 5}{\log 3}.
  \end{align*}
\end{proof}
\newpage
