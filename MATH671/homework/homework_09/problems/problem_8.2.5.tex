\begin{problem}{8.2.5}
  \begin{enumerate}
    \item Show that if $p$ is a polynomial of degree $n$ having $n$ distinct
      fixed points, and negative Schwarzian derivative, then not all of the fixed points can
      be attracting.
    \item On the other hand, show that the logistic maps $L_\mu: \mathbb{R} \to \mathbb{R}$
      for $\mu > 2 + \sqrt{5}$ have negative Schwarzian derivative but have no attracting periodic orbits.
  \end{enumerate}
\end{problem}

\begin{proof}
  \begin{enumerate}
    \item Suppose to the contrary that $p$ is a polynomial of degree $n$ with $n$
      distinct fixed points and negative Schwarzian derivative but all of its fixed points
      are attracting. Let $x_1, \dots, x_n$ denote these attracting fixed points.

      Since $p$ is a polynomial, it is continuous, which implies that for each attracting
      fixed point $x_k$, its immediate basin of attraction $W_k$ is an open interval.
      Note that these fixed points are distinct and attracting so that the immediate basins of attraction
      of two fixed points $x_j$ and $x_k$ with $j\neq k$ are mutually exclusive, i.e.\ $W_j \cap W_k = \emptyset$
      for any $j\neq k$.

      Since $p\in C^3$ with negative Schwarzian derivative, we have by Singer's theorem
      that for every fixed point $x_k$, either $W_k$ is an unbounded interval,
      or the orbit of some critical point of $p$ is attracted to the orbit of $x_k$ under $f$.



    \item
  \end{enumerate}
\end{proof}
\newpage
