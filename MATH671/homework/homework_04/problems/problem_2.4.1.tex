\begin{problem}{2.4.1}
  Let $f_c(x) = x^2 + c$, $c\in\mathbb{R}$.
  \begin{enumerate}
    \item For what values of $c$ does $f_c$ have a super-attracting fixed point
      and what is the fixed point?
    \item For what values of $c$ does $f_c$ have a super-attracting 2-cycle
      and what is the 2-cycle?
    \item Show that if $f_c$ has a super-attracting 3-cycle, then $c$ satisfies
      the equation
      \begin{align*}
        c^3 + 2c^2 +c +1 = 0
      \end{align*}
      and the 3-cycle is given by $\{0, c, c^2 + c\}$.
  \end{enumerate}
\end{problem}

\begin{proof}
  \begin{enumerate}
    \item As was shown in problem 1.2.1, we know that $f_c:\mathbb{R}\to\mathbb{R}$ with $f_c(x) = x^2 + c$
      has two fixed points given by
      \begin{align}\label{fixed}
        x_1 = \frac{1 - \sqrt{1 - 4c}}{2}, \qquad         x_2 = \frac{1 + \sqrt{1 - 4c}}{2}
      \end{align}
      provided that $ c \leq 1/4$.

      The fixed point $x$ will be a super-attracting fixed point if $f_c'(x) = 0$.
      We note that $f_c'(x) = 2x$ so that $f_c'(x) = 0$ only if $x=0$.
      There is no real value of $c$ that will allow $x_2=0$ so $x_2$ is never a
      super-attracting fixed point. On the other hand, if $c=0$,
      then $x_1 = 0$ is a super-attracting fixed point.
    \item Note that $f_c$ will have a super-attracting 2-cycle if $f_c^2$ has a super-attracting
      period 2 point. A point $x$ will be a super-attracting period 2 point
      if $f_c^2(x) = x$ with $f_c(x) \neq x$ and if $(f_c^2(x))' = 0$.

      Since \eqref{fixed} are fixed points, we know that $(x- x_1)$ and $(x- x_2)$ must factor
      $f_c^2(x) - x$ so that
      \begin{align*}
        f_c^2(x) - x &= \left(x^2 + c\right)^2 + c - x \\
        &= \left(x-x_1\right)\left(x-x_2\right)
        \left(x - \frac{-1 + \sqrt{-3 - 4 c}}{2}\right)\left(x - \frac{-1 - \sqrt{-3 - 4 c}}{2}\right).
      \end{align*}
      Thus,
      $$\{x_3, x_4\} = \left\{\frac{-1 + \sqrt{-3 - 4 c}}{2}, \frac{-1 - \sqrt{-3 - 4 c}}{2}\right\}$$
      forms a 2-cycle of $f_c$. To analyze when this 2-cycle is super attracting, we analyze when
      \begin{align*}
        (f_c^2(x_3))' &= f_c'(x_3)f_c'(x_4) \\
        &= \left(-1 - \sqrt{-3 - 4 c}\right)\left(-1 + \sqrt{-3 - 4 c}\right) \\
        &= 4(1 + c) = 0.
      \end{align*}
      We readily see that $(f_c^2(x_3))' = 0$ only if $c= -1$ so that only $f_c(x) = x^2 - 1$
      has a super-attracting 2-cycle given by $\{0, -1\}$.
    \item If $f_c$ has a 3-cycle
      then $f_c$ has a period 3 point $x_0$ with $f_c^3(x_0) = x_0$ such that $f_c(x_0) = x_1 \neq x_0$
      and $f_c^2(x_0) = x_2 \neq x_0$. Note that $f_c'(x) = 2x$. Thus, this 3-cycle is super attracting
      if
      \begin{align*}
        (f_c^3(x_0))' = f_c'(x_0)f_c'(x_1)f_c'(x_2) = 2^3 x_0x_1x_2 = 0
      \end{align*}
      which implies that $x_0 = 0$, $x_1=0$, or $x_2=0$. Without loss of generality,
      we may assume that $x_0 = 0$. Using the fact that $f_c(x_0) = x_1 \neq x_0$ and $f_c^2(x_0) = x_2 \neq x_0$,
      we see that
      \begin{align*}
        x_1 &= f_c(x_0) = x_0^2 + c = c \\
        x_2 &= f_c^2(x_0) = (x_0^2 + c)^2 + c = c^2 + c
      \end{align*}
      In order for this to be a 3-cycle, we require that $f^3(x_0) = x_0 = 0$, i.e.\
      we require that
      \begin{align*}
        f_c^3(x_0) = f_c(f_c^2(x_0)) &= (c^2 + c)^2 + c \\
        &= c^4 + 2c^3 + c^2 + c \\
        &= c(c^3 + 2c^2 + c + 1) = 0.
      \end{align*}
      However we must have that $c \neq 0$ or $x_0 = 0$ would not generate a 3-cycle. Thus, we require that
      $(c^3 + 2c^2 + c + 1) = 0$. If this condition is met and $f_c$ has a super-attracting 3-cycle,
      then that 3-cycle is given by $\{0, c, c^2 + c\}$.
  \end{enumerate}
\end{proof}
\newpage
