\begin{problem}{7.3.5}
  Show that every quadratic polynomial $p(x) = a_2 x^2 + a_1 x + a_0$ is linearly conjugate
  to a unique polynomial of the form $f_c(x) = x^2 + c$.
\end{problem}

\begin{proof}
  In order for $p$ and $f_c$ to be linearly conjugate, we wish to find a function
  $h: \mathbb{R} \to \mathbb{R}$ of the form $h(x) = b_1 x + b_0$ such that $h \circ p = f \circ h$ with $b_1 \neq 0$.
  Note that any such $h$ is a continuous bijection so we need only check $h \circ p = f \circ h$.

  Checking, we have that
  \begin{align*}
    h \circ p (x) &= b_1 p(x) + b-0 \\
    &= b_1(a_2 x^2 + a_1 x + a_0) + b_0 \\
    &= a_2b_1 x^2 + a_1b_1 x + a_0b_1 + b_0
  \end{align*}
  and
  \begin{align*}
    f \circ h(x) &= \left(b_1 x + b_0\right)^2 + c \\
    &= b_1^2 x^2 + 2b_0 b_1 x + b_0^2 + c.
  \end{align*}
  Thus, $h\circ p = f \circ h$ if and only if the coefficients of the resulting
  polynomials are the same if and only if
  \begin{align*}
    b_1^2 - a_2b_1 &= 0 \\
    2b_0 b_1 - a_1b_1 &= 0 \\
    c + b_0^2 - a_0b_1 - b_0 &= 0.
  \end{align*}
  Since $b_1 \neq 0$, we can solve this system so that
  \begin{align*}
    b_1 &= a_2 \\
    b_0 &= \frac{a_1}{2} \\
    c &= a_0b_1 + b_0 - b_0^2 \\
    &= a_0a_2 + \frac{a_1}{2} - \frac{a_1^2}{4}.
  \end{align*}
  Therefore, $p(x) = a_2 x^2 + a_1 x + a_0$ is linearly conjugate to
  $f_c(x) = x^2 + c$
  via $h(x) = a_2 x + a_1 /2$ if $c = a_0a_2 + a_1/2 - a_1^2/4$.

  To show that $f_c$ is unique, suppose that $p(x) = a_2 x^2 + a_1 x + a_0$
  is linearly conjugate to some other quadratic polynomial $g(x) = d_2 x^2 + d_1 x + d_0$
  via $h(x) = b_1 x + b_0 =a_2 x + a_1 /2$. Then we have that $h \circ p = g \circ h$
  and equating coefficients we see that
  \begin{align*}
    d_0 &= \frac{a_2 b_0^2 + b_0 b_1 - a1 b_0 b_1 + a_0 b_1^2}{b_1}\\
    d_1 &= \frac{-2 a_2 b_0 + a_1 b_1}{b_1} \\
    d_2 &= \frac{a_2}{b_1}.
  \end{align*}
  Using the fact that $b_1 = a_2$ and $b_0 = a_1/2$, we have that $d_0 = a_0a_2 + a_1/2 - a_1^2/4$,
  $d_1 = 0$, and $d_2 = 1$. Thus, $g(x) = x^2 + a_0a_2 + a_1/2 - a_1^2/4= f_c(x)$ and
  $f_c(x)$ is unique.

  % To show that $f_c$ is unique, suppose that $p(x) = a_2 x^2 + a_1 x + a_0$
  % is linearly conjugate to both $f_{c_1}(x) = x^2 + c_1$ and $f_{c_2}(x) = x^2 + c_2$.
  % Then there exist linear functions $h_1(x) = d_1 x + d_0$ and $h_2(x) = e_1 x + e_0$
  % with $d_1, e_1 \neq 0$
  % such that
  % \begin{align*}
  %   h_1 \circ p &= f_{c_1} \circ h_1\\
  %   h_2 \circ p &= f_{c_2} \circ h_2.
  % \end{align*}
  % Equating the resulting polynomials from the above two equations shows that
  % $c_1 = a_0d_1 + d_0 - d_0^2$ and $c_2 = a_0e_1 + e_0 - e_0^2$. However, we also have
  % that $d_1 = a_2 = e_1$ and $d_0 = a_1/2 = e_0$. Therefore, $c_1 = c_2$ and the polynomial $f_c$ that is
  % linearly conjugate to $p(x)$ is unique.
\end{proof}
\newpage
