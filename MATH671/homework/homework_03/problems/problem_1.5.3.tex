\begin{problem}{1.5.3}
  \begin{enumerate}
    \item Show that $f(x) = -2x^3 + 2x^2 +x$ has two non-hyperbolic fixed points and determine their stability.
    \item If $x=0$ and $x=1$ are non-hyperbolic fixed points for $f:\mathbb{R}\to\mathbb{R}$ for $f(x) = ax^3 + bx^2 + cx + d$,
      find all possible values of $a, b, c,$ and $d$.
    \item Write down the function $f(x)$ in each case of (ii)\ above and determine the stability of the fixed points.
  \end{enumerate}
\end{problem}

\begin{proof}
  \begin{enumerate}
    \item The fixed points of $f(x)$ are the roots of the function
      \begin{align*}
        g(x) &= f(x) - x \\
        &= -2x^3 + 2x^2 + x - x \\
        &= -2x^2(x-1).
      \end{align*}
      From the factorization of $g(x)$, we clearly see that its roots are given by $x=0$ and $x=1$.
      Note that $f'(x) = -6x^2 + 4x + 1$. From this we see that $f'(0) = 1$ and $f'(1)=-1$, implying that both
      fixed points are non-hyperbolic fixed points.

      Using $f''(x) = -12x + 4$, we see that $f''(0) = 4 > 0$ so that by theorem 1.5.3 (i) the fixed point $x=0$ is one-sided stable to the left of $x =0$.

      To determine the stability of $x=1$, we note that the Schwarzian derivative of $f(x)$
      when $f'(x) = -1$ is given by $Sf(x) = -f'''(x) - 3f''(x)^2/2$. Thus, $f'(1) = -1$ and
      $Sf(1) = 12 - 96 < 0$ so that by
      theorem 1.5.7, the fixed point $x=1$ is asymptotically stable.
    \item Suppose that $f(x) = ax^3 + bx^2 + cx + d$. We know that if $x=0$ and $x=1$ are fixed points
      of $f(x)$ then
      \begin{align}\label{co_fixed}
        \begin{array}{lll}
          f(0) &= d &= 0 \\
          f(1) &= a + b + c &= 1.
        \end{array}
      \end{align}
      The fixed points $x=0$ and $x=1$ are non-hyperbolic fixed points if $|f'(0)| = |f'(1)| = 1$.
      Using $f(x)$, we see that $f'(x) = 3ax^2 + 2bx + c$. Thus, the fixed points are non-hyperbolic
      if
      \begin{align}\label{co_hyper}
        \begin{array}{lll}
          \left|f'(0)\right| &= |c| &= 1 \\
          \left|f'(1)\right| &= |3a + 2b + c| &= 1.
        \end{array}
      \end{align}
      From \eqref{co_fixed}, we see that $a = 1 -b -c$ and substituting into \eqref{co_hyper} we have that $|f'(1)|$ reduces to
      \begin{align}\label{red_hyper}
        \left|f'(1)\right| = |3(1 - b - c) + 2b + c| = |3 -b - 2c| = 1.
      \end{align}
      Note that \eqref{co_hyper} tells us there are two cases to consider, the case that
      $c = 1$ and the case that $c=-1$.

      If $c=1$, then \eqref{red_hyper} tells us that
      \begin{align*}
        |3-b-2c| = |1 -b| = 1
      \end{align*}
      from which we gather that $b=0$ or $b=2$. If $b=0$, then using \eqref{co_fixed}, we see that $a=0$. On the other hand, if
      $b=2$, we see that we must have that $a=-2$.

      If $c=-1$, then \eqref{red_hyper} tells us that
      \begin{align*}
        |3-b-2c| = |5 -b| = 1
      \end{align*}
      from which we gather that $b=4$ or $b=6$. If $b=4$, then using \eqref{co_fixed}, we see that $a=-2$. On the other hand, if
      $b=6$, we see that we must have that $a=-4$.
    \item The previous remarks allow us to explicitly write
      out the function $f:\mathbb{R}\to\mathbb{R}$ for $f(x) = ax^3 + bx^2 + cx + d$ when $f(x)$ has two non-hyperbolic fixed points $x=0$ and $x=1$.
      We write the four possibilities as follows:
      \begin{align*}
        f_1(x) &= x \\
        f_2(x) &= -2x^3 + 2x^2 + x \\
        f_3(x) &= -2x^3 + 4x^2 - x \\
        f_4(x) &= -4x^3 + 6x^2 - x.
      \end{align*}
      We now evaluate the stability of the fixed points $x = 0$ and $x = 1$ for $f_1(x)$.
      From the definition of stability, we have that a point $c$ is stable for $f$ if for all $\epsilon > 0$,
      there exists $\delta >0$ such that if $x \in I = \mathbb{R}$ and $|x - c| < \delta$, then
      $|f^n(x) - c| < \epsilon$ for every positive integer $n$. Note that $f_1^n(x) = x$ for every positive integer $n$. It is clear from the
      definition of stability that by taking $\delta = \epsilon$, the fixed points $x=0$ and $x=1$ are stable. They are not however asymptotically stable since
      these are not attracting fixed points. This is clear because there is no neighborhood of either fixed points such that
      the iterates of $f_1$ for points in that neighborhood converge to $x=0$ or $x=1$. Intuitively, this is because
      every point of $f_1(x)$ is a fixed point.

      We will now evaluate the stability of $x=0$ and $x=1$ for the remaining three functions.
      Recall that the Schwarzian derivative of a function $g(x)$ when $g'(x)=-1$ is given by
      \begin{align*}
        Sg(x) = -g'''(x) - \frac{3g''(x)^2}{2}.
      \end{align*}
      Now, if $f(x) = ax^3 + bx^2 + cx$, then
      \begin{align*}
        f'(x) &= 3ax^2 + 2bx + c\\
        f''(x) &= 6ax + 2b \\
        f'''(x) &= 6a.
      \end{align*}
      Thus, for $x =0$, we can see that
      \begin{align*}
        \begin{array}{rlrlrl}
          f_2'(0) &= \phantom{-}1  & f_3'(0) &= -1            & f_4'(0) &= -1 \\
          f_2''(0) &= \phantom{-}4 & f_3''(0) &= \phantom{-}8 & f_4''(0) &= \phantom{-}12 \\
          f_2'''(0) &= -12         & f_3'''(0) &= -12         & f_4'''(0) &= -24
        \end{array}
      \end{align*}
      We see that for $f_2$, we have that $f_2'(0) = 1$ and $f_2'(0) > 0$ so
      by theorem 1.5.3 we have that the fixed point $x=0$ is one-sided stable to the left of $x=0$.
      For $f_3$ and $f_4$, we have that $f_3'(0) = f_4'(0) = -1$ as well as $Sf_3(0) = -84 < 0$
      and $Sf_4(0) = -192 < 0$. Thus, by theorem 1.5.7, $x=0$
      is an asymptotically stable fixed point for $f_3$ and $f_4$.

      Similarly, for $x=1$, we can see that
      \begin{align*}
        \begin{array}{rlrlrl}
          f_2'(1) &= -1    & f_3'(1) &= \phantom{-}1  & f_4'(1) &= -1 \\
          f_2''(1) &= -8   & f_3''(1) &= -4           & f_4''(1) &= -12 \\
          f_2'''(1) &= -12 & f_3'''(1) &= -12         & f_4'''(1) &= -24
        \end{array}
      \end{align*}
      We see that for $f_3$, we have that $f_3'(1) = 1$ and $f_3''(1) < 0$ so
      by theorem 1.5.3 we have that the fixed point $x=1$ is one-sided stable to the right of $x=1$.
      For $f_2$ and $f_4$, we have that $f_2'(1) = f_4'(1) = -1$ as well as $Sf_2(1) = -84 < 0$
      and $Sf_4(1) = -192 < 0$. Thus, by theorem 1.5.7, $x=1$
      is an asymptotically stable fixed point for $f_2$ and $f_4$.
  \end{enumerate}
\end{proof}
\newpage
