\begin{problem}{2.6.9}
  \begin{enumerate}
    \item Use the results of section 2.6 to show that the logistic map $L_4(x) = 4x(1-x)$
      cannot have a super-attracting cycle.
    \item Find a point $x_0 \in (0, 1)$ which is not a periodic point for $L_4$.
  \end{enumerate}
\end{problem}

\begin{proof}
  \begin{enumerate}
  \item Suppose that $k > 1$ and $x_k$ is a period $k$ point so that $\{x_1, x_2, \dots, x_k\}$
    is a $k$-cycle with $L_4^i(x_1) = x_i$ for $0 < i < k$ and $L_4^k(x_1) = x_1$. This cycle will be super-attracting if
    \begin{align*}
      \prod_{i=1}^k L_4'(x_i) = 0.
    \end{align*}
    Note that $L_4'(x) = 4 - 8x = 0$ only if $x=1/2$. Thus, the cycle will be super attracting
    if and only if $x_i = 1/2$ for some $i=1,\dots k$. Note that the point $x = 1/2$ does not generate a cycle since
    $L_4(1/2) = 1$ and $L_4^n(1/2) = 0$ for $n > 1$ so $x_1 \neq 1/2$.

    We will now demonstrate that there is no point $x\in[0, 1]$ such that $L_4^n(x) = 1/2$ for $n>0$.
    It has been shown previously that
    \begin{align*}
      L_4^n(x) = \sin^2\left(2^n\sin^{-1}\left(\sqrt{x}\right)\right)= \sin^2(\theta)
    \end{align*}
    for some $\theta \in (0, \pi/2]$. Note that for $\theta_1, \theta_2 \in(0, \pi/2]$, we have that
    $\sin^2(\theta_1) = 1/2$ if and only if $\theta_1 = \pi/4$ and
    $\sin^2(\theta_2) = \pi/4$ if and only if $\theta_2 = \sin^{-1}\left(\sqrt{\pi}/2\right) > 1$. However, since $\theta_2 > 1$,
    there is no $\theta\in (0, \pi/2]$ such that $\sin^2(\theta) = \theta_2$.

    So there is no $x\in[0,1]$ such that $L_4^n(x) = \theta_2$ for any $n>0$ and hence no $n>0$ such that $L_4^n(x) = 1/2$.
    Thus, $x_i =L_4^i(x_1) \neq 1/2$ for any $i > 0$ so that $L_4'(x_i)\neq 0$.
    Therefore, $L_4$ has no super-attracting cycle.
    \item As was shown previously, $x=1/2$ is such that $L_4(x) = 1$ and $L_4^n(x) = 0\neq 1/2$ for $n > 1$.
      Therefore $x=1/2$ is not a periodic point of $L_4$.
  \end{enumerate}
\end{proof}
\newpage
