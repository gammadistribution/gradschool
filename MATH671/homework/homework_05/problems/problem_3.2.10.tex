\begin{problem}{3.2.10}
  Let $f : \mathbb{R} \to  \mathbb{R}$. Write down all the possibilities for a 4-cycle
  $\{a, b, c, d\}$ with $a < b < c < d$ for $f$
  (e.g.\ $f (a) = c$, $f (c) = d$, $f (d) = b$, and $f (b) = a$).
  Indicate which are mirror images, and which give rise to a 3-cycle.
\end{problem}

\begin{proof}
  Note that if $x = a,b,c,d$, then $f(x) \neq x$ otherwise $x$ would be a fixed
  point and would not generate a 4-cycle. So, first consider that $a$ generates the 4-cycle
  and $f(a) = b$. Then $f(b) \neq a$ otherwise $\{a, b\}$ would be a 2-cycle of $f$.
  Thus, either $f(b) = c$ of $f(b) = d$. If $f(b) = c$, then $f(c) \neq a$ otherwise $\{a, b, c\}$
  would be a 3-cycle and $f(c) \neq b$ otherwise $\{b, c\}$ would be a 2-cycle. So, $f(c) = d$
  and thus $f(d) = a$ if the set of points $\{a,b,c,d\}$ generates a 4-cycle. If, on
  the other hand $f(b) = d$, then $f(d) \neq a$ otherwise $\{a,b,d\}$ be a 3-cycle
  and $f(d) \neq b$ otherwise $\{b, d\}$ would be a 2-cycle. So $f(d) = c$ and $f(c) = a$
  if the set of points $\{a,b,c,d\}$ generates a 4-cycle.

  Therefore, if $f(a) = b$, we have two possible 4-cycles
  \begin{align*}
    \{a,b,c,d\} \text{  and  } \{a,b,d,c\}.
  \end{align*}
  If $f(a) = c$, we can use similar reasoning to see that $\{a,c,b,d\}$ and $\{a,c,d,b\}$
  are 4-cycles and if $f(a) = d$, then $\{a,d,b,c\}$ and $\{a,d,c,b\}$ are 4-cycles.
  Note, these are the only possible 4-cycles of $f$.

  The mirror image of a 4-cycle $\{x_1, x_2, x_3, x_4\}$ is the 4-cycle such
  that $f(x_4) = x_3$, $f(x_3) = x_2$, $f(x_2) = x_1$, and $f(x_1) = x_4$, i.e.\
  the 4-cycle $\{x_4, x_3, x_2, x_1\}$. Therefore,
  $\{a,b,c,d\}$ and $\{a,d,c,b\}$ are mirror images, $\{a,b,d,c\}$ and $\{a,c,d,b\}$ are mirror images,
  and lastly $\{a,c,b,d\}$ and $\{a,d,b,c\}$ are mirror images.

  Proposition 3.1.7 tells us that for $I$, an interval, and $f : I \to I$, a continuous map,
  if $I_1$ and $I_2$ are closed sub intervals of $I$ with at most one point in common
  and $I_2 \subset f(I_1)$ and $I_1 \cup I_2 \subset f(I_2)$, then $f$ has a 3-cycle.
  Throughout, we assume that our function $f$ is continuous. Let $I_1 = [a,b]$, $I_2=[b,c]$,
  and $I_3 = [c,d]$.

  If $\{a,b,c,d\}$ is a 4-cycle of $f$, then we see that
  \begin{align*}
    [c, d] = I_3 \subset f(I_2) = [c,d] \text{\quad and\quad}  [b, d] = I_2 \cup I_3 \subset f(I_3) = [a,d]
  \end{align*}
  so that a 3-cycle is generated by the proposition.
  Similarly, for the 4-cycles $\{a,b,d,c\}, \{a,c,d,b\}$, and $\{a,d,c,b\}$ we see under these 4-cycles that
  \begin{align*}
    [b, c] = I_2 \subset f(I_3) = [a,c] \text{\quad and\quad}  [b, d] = I_2 \cup I_3 \subset f(I_2) = [a,d],\\
    [a, b] = I_1 \subset f(I_2) = [a,d] \text{\quad and\quad}  [a, c] = I_1 \cup I_2 \subset f(I_1) = [a,c], \\
    [a, b] = I_1 \subset f(I_2) = [a,b] \text{\quad and\quad}  [a, c] = I_1 \cup I_2 \subset f(I_1) = [a,d],
  \end{align*}
  respectively, so that these 4-cycles give rise to 3-cycles by our proposition.
  Since the other 4-cycles do not meet the criteria of the proposition, they do not generate 3-cycles.

\end{proof}
\newpage
