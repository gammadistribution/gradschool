\begin{problem}{1.4.4}
  Let $S_\mu (x) = \mu \sin(x)$, $0 \leq x \leq 2\pi$, $0 < \mu \leq \pi$ and $C_\mu (x) = \mu \cos(x)$,
  $-\pi \leq x \leq \pi$ and $-\pi \leq \mu \leq \pi$, $\mu \neq 0$.
  \begin{enumerate}
    \item Show that $S_\mu$ has a super-attracting fixed point at $x = \pi/2$, when $\mu = \pi/2$.
    \item Find the corresponding values for $C_\mu$ having a super-attracting fixed point.
  \end{enumerate}
\end{problem}

\begin{proof}
  Recall that if $c$ is a fixed point of a differentiable function $f$, then $c$ is a super-attracting
  fixed point if $f'(c) = 0$.

  \begin{enumerate}
    \item Suppose that $\mu = \pi/2$. Since $S_\mu(\pi/2) = (\pi/2) \sin(\pi/2) = \pi/2$,
      we readily see that if $\mu = \pi/2,$ then $x=\pi /2$ is a fixed point of $S_\mu(x)$.
      Note that $S_\mu '(x) = \mu \cos(x)$. From this we gather that if $\mu = \pi/2$,
      then for the fixed point $x = \pi/2$, we have that $S_\mu ' (x) = (\pi/2) \cos(\pi/2) = 0$.
      Therefore, the fixed point $x=\pi/2$ is a super-attracting fixed point.
    \item We now investigate the super-attracting fixed points of $C_\mu(x)$. The
      definition of $C_\mu(x)$ shows that $C_\mu'(x) = - \mu\sin(x)$ from which
      we can gather that $C_\mu'(x) = 0$ for $x\in [-\pi, \pi]$ if $x = k \pi$ for
      $k\in \{-1, 0, 1\}$. Note that these are the possible super-attracting
      fixed points of $C_\mu(x)$, we must still determine which
      of these possible super-attracting fixed points are indeed fixed points, i.e.\ we
      must determine which points satisfy $C_\mu(x) = x$. If $x = k \pi$ for
      $k\in \{-1, 0, 1\}$, then
      \begin{align*}
        C_\mu(k\pi ) = \mu\cos(k\pi) =(-1)^k\mu.
      \end{align*}
      Thus, $C_\mu(k\pi) = (-1)^k \mu = k \pi$, if $\mu = (-1)^k k\pi$. Therefore, if $x, \mu \in [-\pi, \pi]$ with $\mu \neq 0$,
      then the points $x_1 = -\pi$ and $x_2 = \pi$, with corresponding
      $\mu$-values $\mu_1=\pi$ and $\mu_2 = -\pi$, are super-attracting fixed points.
      Note that $x = 0$ is not a super-attracting fixed point
      since it is not a fixed point, that is $C_\mu(0) = 0$ only if $\mu = 0$, which violates our initial conditions.
  \end{enumerate}
\end{proof}
\newpage
