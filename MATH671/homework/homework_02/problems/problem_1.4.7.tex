\begin{problem}{1.4.7}
  Let $N_f$ be the Newton function of the map $f(x) = x^2 + 1$. Clearly there are no
  fixed points of the Newton function as there are no zeros of $f$. Show that there are
  points $c$ where $N_f^2 (c) = c$ (called \emph{period 2-points} of $N_f$).
\end{problem}

\begin{proof}
  The Newton function for a function $f$ is defined by
  \begin{align*}
    N_f(x) = x - \frac{f(x)}{f'(x)}.
  \end{align*}
  Thus, for the function $f(x) = x^2 + 1$,
  we have that
  \begin{align*}
    N_f(x) &= x - \frac{f(x)}{f'(x)}\\
    &= x - \frac{x^2 + 1}{2x} \\
    &= \frac{x^2 - 1}{2x}.
  \end{align*}
  Using this definition of $N_f(x)$, we can readily see that
  \begin{align*}
    N_f^2(x) = N_f(N_f(x))
    &= \frac{\left(\frac{x^2 - 1}{2x}\right)^2 - 1}{2\left(\frac{x^2 - 1}{2x}\right)} \\
    &= \frac{x^4-2x^2 - 4x^2 +1}{4x^2} \cdot \frac{x}{x^2-1} \\
    &= \frac{x^4 - 6x^2 +1}{4x(x^2 - 1)}.
  \end{align*}

  The points $x$ such that $N_f^2(x) = x$ are the solutions to the equation
  \begin{align*}
    N_f^2(x) - x = \frac{x^4 - 6x^2 +1}{4x(x^2 - 1)} - x = \frac{-3x^4-2x^2+1}{4x(x^2-1)} = 0.
  \end{align*}
  We readily see that $x = \pm 3^{-1/2}$ are the real solutions to the above equation.
  Therefore, $x = \pm 3^{-1/2}$ satisfy $N_f^2(x) = x$ and are period 2-points.
\end{proof}
\newpage
