\begin{problem}{1.4.2}
  Let $f: \mathbb{R} \to \mathbb{R}$. If $f'(x)$ exists with $f'(x) \neq 1$ for
  all $x \in \mathbb{R}$, prove that $f$ has at most one fixed point.
  (Hint: Use the Mean Value Theorem).
\end{problem}

\begin{proof}
  Suppose to the contrary that for all $x\in\mathbb{R}$ we have that $f'(x)$ exists
  with $f'(x) \neq 1$, but $f$ has at least two distinct fixed points, $c_1$ and $c_2$, say.
  The Mean Value Theorem states that if a function $g$ is continuous on an interval
  $[a, b]$ and differentiable on the interval $(a, b)$, then there exists a point $c \in (a, b)$
  such that
  \begin{align*}
    g'(c) = \frac{g(b)- g(a)}{b - a}.
  \end{align*}
  By our supposition, we have that the function $f$ is continuous and differentiable
  on any interval and, in particular, it is continuous on $[c_1, c_2]$ and differentiable
  on $(c_1, c_2)$. By the Mean Value Theorem, there exists a point $c_3 \in (c_1, c_2)$
  such that
  \begin{align}\label{MVT}
    f'(c_3) = \frac{f(c_2)- f(c_1)}{c_2 - c_1}.
  \end{align}
  However, since $c_1$ and $c_2$ are fixed points of $f$, we know that $f(c_2)- f(c_1) = c_2 - c_1$
  and we gather from \eqref{MVT} that
  \begin{align*}
    f'(c_3) = \frac{f(c_2)- f(c_1)}{c_2 - c_1} = \frac{c_2 - c_1}{c_2 - c_1} = 1.
  \end{align*}
  However, this is in contradiction to our supposition that $f'(x) \neq 1$ for
  any $x \in \mathbb{R}$. Therefore, we must conclude that for a function
  $f: \mathbb{R} \to \mathbb{R}$, if for all $x\in\mathbb{R}$ we have that $f'(x)$ exists
  with $f'(x) \neq 1$, then $f$ has at most one fixed point.
\end{proof}
\newpage
