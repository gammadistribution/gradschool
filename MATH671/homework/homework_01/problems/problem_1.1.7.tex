\begin{problem}{1.1.7}
  Let $f(x) = x^2 + bx + c$. Give conditions on $b$ and $c$ for $f: [0, 1] \to [0, 1]$
  to be a dynamical system. Hint: Recall that the maximum and minimum values of
  a continuous function defined on a closed interval $[a, b]$ occur either at
  the end points or at the critical points of the function.
\end{problem}

\begin{proof}
  The function $f(x) = x^2 + bx + c$ for $f: [0, 1] \to [0, 1]$ is a dynamical
  system if the image of the function is contained in its domain, i.e.\ if
  $f([0,1]) \subseteq [0,1]$. The minimum and maximum values of a continuous function
  occur either at the end points of the domain or at the critical points of the function.
  Thus, for the continuous function $f$, if we ensure that the evaluation of $f$ at
  $x=0$, $x=1$, and the critical points of $f$ are contained in $[0, 1]$ then the
  image of $f$ will necessarily be contained in $[0, 1]$ and $f$ will be a
  dynamical system.

  At the end points of the domain we have that $f(0) = c$ and $f(1) = b + c + 1$. Thus, in order for $f$ to be
  a dynamical system, we must have that $c \in [0, 1]$ and $b + c \in [-1, 0]$.

  The only critical point of the function $f$ is found when $f'(x) = 0$ or when $x = -b/2$.
  Thus, we require that $f(-b/2) = -b^2/4 + c \in [0, 1]$. This reduces to
  requiring that $4c - 4 \leq b^2 \leq 4c$.
  Thus, when $b\in\mathbb{R}$, we must have that $b \in [-2\sqrt{c}, 2\sqrt{c}]$.

  Combining all of these inequalities shows that in order for the image of $f$
  to be contained in the domain of $f$, we must have that $c \in [0, 1]$ and $b \in [-2\sqrt{c}, -c]$.
  Therefore, for the function $f(x) = x^2 + bx + c$ for $f: [0, 1] \to [0, 1]$ to be a dynamical system we must have that $0 \leq c \leq 1$
  and $-2\sqrt{c} \leq b \leq -c$.

\end{proof}
\newpage
