\begin{problem}{1.2.6}
  Consider the eventual fixed points of the logistic map $L_{\mu}: [0,1] \to [0,1]$,
  $L_\mu(x)  = \mu x (1-x)$ for $0 < \mu < 4$.
  \begin{enumerate}
    \item Show that there are no eventual fixed points associated with the fixed
      point $x=0$, other than $x=1$.
    \item Show that for $1 < \mu \leq 2,$ the only eventual fixed point associated
      with the fixed point $x = 1 - 1/\mu$ is $x = 1/\mu$.
    \item Show that there are additional eventual fixed points associated with $x = 1 - 1/\mu$
      when $2 < \mu < 3$.
    \item Investigate the eventual fixed points of the logistic map when $\mu = 5/2$.
  \end{enumerate}
\end{problem}

\begin{proof}
  \begin{enumerate}
    \item It is clear that $x=1$ is an eventual fixed
      point since $x=0$ is a fixed point and $L_\mu(1) = 0$. To see that this
      is the only eventual fixed point associated to $x=0$, it suffices to see that no point
      in the interval $(0, 1)$ maps to either 0 or 1 under $L_\mu$, i.e.\
      for $y\in(0,1)$, the equations $L_\mu(y) = 0$ and $L_\mu(y) = 1$ have no solutions.
      Therefore, since no $y \in D_L_\mu = [0, 1]$ besides $y=0$ and $y=1$ maps to 0 or 1,
      there are no other eventual fixed points associated to $x=0$.
    \item
  \end{enumerate}
\end{proof}
\newpage
