\begin{problem}{1.2.6}
  Consider the eventual fixed points of the logistic map $L_{\mu}: [0,1] \to [0,1]$,
  $L_\mu(x)  = \mu x (1-x)$ for $0 < \mu < 4$.
  \begin{enumerate}
    \item Show that there are no eventual fixed points associated with the fixed
      point $x=0$, other than $x=1$.
    \item Show that for $1 < \mu \leq 2,$ the only eventual fixed point associated
      with the fixed point $x = 1 - 1/\mu$ is $x = 1/\mu$.
    \item Show that there are additional eventual fixed points associated with $x = 1 - 1/\mu$
      when $2 < \mu < 3$.
    \item Investigate the eventual fixed points of the logistic map when $\mu = 5/2$.
  \end{enumerate}
\end{problem}

\begin{proof}
  \begin{enumerate}
    \item It is clear that $x=1$ is an eventual fixed
      point since $x=0$ is a fixed point and $L_\mu(1) = 0$.
      This is the only eventual fixed point associated to $x=0$ since no point
      in the interval $(0, 1)$ maps to either 0 or 1 under $L_\mu$, i.e.\
      for $y\in(0,1)$, the equations $L_\mu(y) = 0$ and $L_\mu(y) = 1$
      have no solutions. Therefore, since no $y \in D_{L_\mu} = [0, 1]$ besides
      $y=0$ and $y=1$ maps to 0 or 1, there are no other eventual fixed points
      associated to $x=0$.
    \item Let $1 < \mu \leq 2$. It is clear that $x = 1/\mu$ is an eventual
      fixed point since $x = 1 - 1/\mu$ is a fixed point and
      $L_\mu(1/\mu) = 1 - 1/\mu$.
      We will now demonstrate that this is the only eventual fixed point
      associated to $x = 1 - 1/\mu$. Note that
      for $x \in [0, 1]$, the only solution to $L_\mu(x) = 1 - 1/\mu$ is $x = 1/\mu$.
      Therefore, in order for a point $x \in [0, 1]$ to be an eventual fixed point
      associated to $x = 1 - 1/\mu$, we must have that $x$ either maps to $1/\mu$
      or eventually maps to $1/\mu$, i.e.\ for $x \in [0, 1]$ we must have that
      $L_\mu(x) = 1/\mu$ has a solution. However, if $1 < \mu \leq 2$,
      then $L_\mu(x) = 1/\mu$ has no real solutions for $x \in [0, 1]$ and  so
      there are no other eventual fixed points associated to $x = 1 - 1/\mu$.
    \item Now suppose that $2 < \mu < 3$. Recall that $x = 1/\mu$ is an eventual fixed point.
      Note that
      $$y = \frac{1}{2} - \frac{1}{2}\sqrt{1 - \frac{4}{m^2}} \in [0, 1]$$
      satisfies $L_\mu(y) = 1/\mu$. Thus, $L_\mu^2(y) = 1 - 1/\mu$ where $1 - 1/\mu$ is a fixed point.
      Therefore, if $2 < \mu < 3$, then there are additional eventual fixed points associated
      to $x = 1 - 1/\mu$ besides $x = 1/\mu$.
    \item We proceed to describe all eventual
      fixed points of $L_\mu$ by first identifying all fixed points of the function.
      Suppose $\{x_{0_n}\}$ is the set of fixed points of $L_\mu$ where $n$ is the number
      of fixed points. To find all eventual fixed
      points associated to the fixed point $x_{0_k}$,
      first find the pre-image of $x_{0_k}$ minus the point $x_{0_k}$, i.e.\ find
      $$L_\mu^{-1}\left(x_{0_k}\right) = \{x \neq x_{0_k}\ |\ L_\mu(x) = x_{0_k} \}.$$
      Note that each point in $L_\mu^{-1}(x_{0_k})$ will be an eventual fixed point of $x_{0_k}$.
      Denote the set of eventual fixed points associated to $x_{0_k}$ by $x_{1_k}$.
      If the set is empty, then there are no eventual fixed points associated to $x_{0_k}$.
      If the set is non-empty, continue the process by finding for each point in
      the set of eventual fixed points $x_{1_k}$, the set
      of eventual fixed points associated to the eventual fixed point $x_{1_k}$,
      i.e.\ $x_{2_k} = L_\mu^{-1}\left(x_{1_k}\right) = L_\mu^{-1}\left(L_\mu^{-1}\left(x_{0_k}\right)\right)$.
      Again, every point in $x_{2_k}$ is an eventual fixed point of $x_{0_k}$.
      Continue this process indefinitely until the pre-image $x_{m_k}$ is empty.

      Now suppose that $\mu = 5/2$. The fixed points of $L_\mu$ are found by finding
      the roots of $g(x) = L_\mu(x) - x = (3/2)x - (5/2)x^2$ for $g:[0, 1] \to [0, 1]$.
      It is clear that the two roots of $g$ are given by $x=0$
      and $x =3/5$. Thus, denote the fixed points of $L_\mu$ by $\{x_{0_0}, x_{0_1}\} = \{0, 3/5\}$.

      We will now find all eventual fixed points associated to the fixed point
      $x_{0_0} = 0$. As was shown previously, the only eventual fixed points associated to
      $x_{0_0}$ is $x = 1$. Thus, $x_{1_0} = \{1\}$ and we are done.

      Now we will find all eventual fixed points associated to the fixed point
      $x_{0_1} = 3/5$.  Solving the equation $L_\mu(x) = x_{0_1}$ shows us that the pre-image
      of $x_{0_1}$ minus the point $x_{0_1}$ is given by $L_\mu^{-1}\left(x_{0_1}\right) = \{2/5 \}$, i.e.\
      $x_{1_1} = \{2/5\}$. Continuing, we solve the equation $L_\mu(x) = 2/5$ and find that
      $L_\mu^{-1}(x_{1_1}) = \{1/5, 4/5\}$. Thus $x_{2_1} = \{1/5, 4/5\}$.
      Since, $L_\mu(x) =  4/5$ has no real solutions, there are no eventual fixed points associated to the eventual fixed point $4/5$.
      However, solving $L_\mu(x) =  1/5$ we see that there are real solutions associated to this eventual fixed point
      so $x_{3_1} = \{1/10 (5 - \sqrt{17}), 1/10 (5 + \sqrt{17})\}$.

      Under further investigation the above sequence continues in a similar pattern. The equation
      $L_\mu(x) =  1/10 (5 + \sqrt{17})$ will have no real solutions but
      $L_\mu(x) =  1/10 (5 - \sqrt{17})$ does. Thus, we see using these solutions that
      $x_{4_1} = \{1/10 (5 - \sqrt{5 + 4 \sqrt{17}}), 1/10 (5 +  \sqrt{5 + 4 \sqrt{17}})\}$.
      This process repeats indefinitely
      so that two more eventual fixed points are found with the set of eventual fixed points $x_{k_1}$.
      The union of all such points as described above  make up the eventual
      fixed points associated to $x = 3/5$.

  \end{enumerate}
\end{proof}
\newpage
