\begin{problem}{2}
  Let $\displaystyle g_\mu(x) = \mu x \frac{(1-x)}{(1+x)}$, for $\mu > 0$.

  \begin{enumerate}
    \item Show that $g_\mu$ has a maximum at $x = \sqrt{2} - 1$ and the maximum value is
      $\mu(3 - 2\sqrt{2})$.
    \item Deduce that $g_\mu$ is a dynamical system on $[0, 1]$
      for $0 \leq \mu \leq 3 + 2\sqrt{2}$, i.e. $g_\mu([0,1])\subseteq[0,1]$.
    \item Find the fixed points of $g_\mu$ for $\mu \geq 1$.
    \item Find $g_\mu'$ and determine whether the fixed points are attracting
      or repelling.
    \item Use a graphing utility to graph $g_\mu^2$ and $g_\mu^3$
      and estimate when a period 2 point is created.
  \end{enumerate}
\end{problem}

\begin{proof}
  \begin{enumerate}
    \item If $\displaystyle g_\mu(x) = \mu x \frac{(1-x)}{(1+x)}$, then we see that
      \begin{align}\label{gmu}
        g_\mu'(x) &= \mu \left[\frac{(1-x)}{(1+x)} - \frac{2x}{(1+x)^2} \right] \notag \\
        &= \mu\left[\frac{-x^2-2x+1}{(1+x)^2}\right].
      \end{align}
      Thus, $g_\mu'(x) = 0$ if $x = \pm \sqrt{2} - 1$. Since $g_\mu'(0) = \mu > 0$ with $0 < \sqrt{2} - 1$
      and $g_\mu'(1) = -\mu/2 < 0$ for $\sqrt{2} -1 < 1$, we see that $x = \sqrt{2} - 1$ is a local maximum
      of $g_\mu(x)$. The maximum value is thus given by
      \begin{align*}
        g_\mu(\sqrt{2} - 1) = \mu (\sqrt{2} - 1) \frac{(1-(\sqrt{2} - 1))}{(1+(\sqrt{2} - 1))} = \mu(3 - 2\sqrt{2}).
      \end{align*}

    \item The function $g_\mu: [0, 1] \to [0, 1]$ will be a dynamical
      system for $0 \leq \mu \leq 3 + 2\sqrt{2}$ if $g_\mu([0,1]) \subseteq [0,1]$.
      Note that on $[0, 1]$, we have that the global minimum of $g_\mu$ is 0
      and can easily see using the previous result that the global maximum
      of $g_\mu$ is $\mu(3 - 2\sqrt{2})$. Thus, since $g_\mu$ is continuous, we must have that
      $g_\mu([0,1]) = [0, \mu(3 - 2\sqrt{2})]$. If $0 \leq \mu \leq 3 + 2\sqrt{2}$, we see that
      $$0 \leq \mu(3 - 2\sqrt{2}) \leq (3 + 2\sqrt{2})(3 - 2\sqrt{2}) = 1.$$
      Therefore,
      $g_\mu([0,1]) = [0, \mu(3 - 2\sqrt{2})] \subseteq [0, 1]$ and $g_\mu$ is a
      dynamical system on $[0, 1]$.

    \item Suppose that $\mu \geq 1$. The fixed points of $g_\mu$ are the roots of
      the function
      \begin{align*}
        f(x) = g_\mu(x) - x = - \frac{x[x(\mu+1) - (\mu - 1)]}{(x+1)}.
      \end{align*}
      Thus, the fixed points of $g_\mu$ are given by
      \begin{align}\label{gmu_fixed}
        x_0 = 0 \text{\quad and\quad} x_1 = \frac{\mu-1}{\mu+1}.
      \end{align}

    \item Recall that a fixed point $c$ of a function $f$ that is hyperbolic is attracting if $|f'(c)| < 1$ and
      repelling if $|f'(c)| > 1$. The derivative of $g_\mu$ is provided by \eqref{gmu}.
      Thus, we readily see that for the fixed points provided by \eqref{gmu_fixed}
      that
      \begin{align*}
        |g_\mu'(x_0)| = |g_\mu'(0)| = |\mu|
      \end{align*}
      and
      \begin{align*}
        |g_\mu'(x_1)| &= \left|g_\mu'\left(\frac{\mu-1}{\mu+1}\right)\right| \\
        &= \frac{1}{2}\left|\left(-\mu + \frac{1}{\mu} + 2\right)\right|.
      \end{align*}

      Consider $\mu \geq 1$. We see that if $\mu > 1$ then the fixed point $x_0$ will
      be a hyperbolic fixed point and will be repelling. If, however, $\mu = 1$, we see that $g_\mu'(x_0) = 1$
      and $x_0$ is a non-hyperbolic fixed point. We rely on a previous theorem that states that
      we can use the second and third derivative of $g_\mu$ in order to classify the non-hyperbolic fixed point.
      Note that
      \begin{align}\label{gmu_derivs}
        g_\mu''(x) = - \frac{4\mu}{(1 + x)^3} \text{\quad and\quad} g_\mu'''(x) = \frac{12\mu}{(1+x)^4}.
      \end{align}
      Since $g_\mu''(x_0) = -4\mu =-4 < 0$ for $\mu = 1$, the fixed point $x_0 = 0$ is
      one-sided asymptotically stable to the right of 0.

      For the fixed point $x_1$, we see that if
      $1 < \mu < 2 + \sqrt{5}$, then $|g_\mu'(x_1)| < 1$ so that
      $x_1$ is a hyperbolic, attracting fixed point. On the other hand, if $2 + \sqrt{5} < \mu$,
      then $|g_\mu'(x_1)| > 1$ so that $x_1$ is a hyperbolic, repelling fixed point.
      In the case that $\mu = 1$ or $\mu = 2 + \sqrt{5}$, the fixed point $x_1$
      is non-hyperbolic.

      If $\mu = 1$, we see that $x_1 = 0 = x_0$ and so it must have the same classification as $x_0$ when $\mu = 1$, i.e.\
      it is a non-hyperbolic fixed point that is one-sided asymptotically stable to the right of 0.
      If $\mu = 2 + \sqrt{5}$, then we see that $g_\mu'(x_1) = -1$.
      Note that we can use the Schwarzian derivative of $g_\mu$ to classify this non-hyperbolic fixed point.
      The Schwarzian derivative of $g_\mu$ evaluated at $x_1$ is given by
      \begin{align*}
        Sg_\mu(x_1) &= -g_\mu'''(x_1) - \frac{3g_\mu''(x_1)^2}{2} \\
        &= 6-6\sqrt{5} - \frac{3(-4)^2}{2}\\
        &= -18 - 6\sqrt{5}.
      \end{align*}
      Since $Sg_\mu(x_1) <0$, the fixed point $x_1$ is asymptotically stable
      when $\mu = 2 + \sqrt{5}$.
    \item Using the Mathematica $\texttt{Manipulate}$ command, we can plot the parametric
      families $g_\mu^2$ and $g_\mu^3$ for $0 \leq \mu \leq 3 + 2\sqrt{2}$. After plotting
      these families we see that a bifurcation point for the system occurs approximately
      when $\mu \approx 4.23607$. For values of $\mu > 4.23607$ a 2-cycle is born for
      the dynamical system.
  \end{enumerate}
\end{proof}
\newpage
