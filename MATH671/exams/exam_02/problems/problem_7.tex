\begin{problem}{7}
  Let $0 <\mu < \lambda < 1$ and let $h: [0, 1] \to [0, 1]$ be a homeomorphism
  with $h \circ L_\mu(x) = L_\lambda \circ h (x)$ for all $x\in [0, 1]$.
  \begin{enumerate}
    \item Show that $h$ is orientation-preserving.
    \item Show that $h(x) + h(1-x) = 1$ for all $x\in [0, 1]$. Deduce that $h(1/2) = 1/2$.
    \item Show that $h(\mu/4) = \lambda /4$ and $h(x) > x$ for $0 < x < 1/2$ and $h(x) < x$ for
      $1/2 < x < 1$.
  \end{enumerate}
\end{problem}

\begin{proof}
  \begin{enumerate}
    \item Since $h$ is a homeomorphism it is a bijection from $[0,1]$ to $[0,1]$.
      This implies that $h$ is either montonically increasing or montonically decreasing.
      If $L_\alpha$ is the logistic map from $[0,1]$ to $[0,1]$ with $0 < \alpha < 1$, then $x=0$
      is its only fixed point. Since $h$ is a conjugacy from $L_\mu$ to $L_\lambda$ it maps the fixed points
      of $L_\mu$ to the fixed points of $L_\lambda$. Since both $\mu, \lambda < 1$, the only fixed point of both $L_\mu$
      and $L_\lambda$ is $x=0$. This implies that $h(0) = 0$ or that $h$ is monotonically increasing.
      Therefore $h$ is orientation-preserving.

    \item

    \item
  \end{enumerate}
\end{proof}
\newpage
