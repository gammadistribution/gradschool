\begin{problem}{2}
  Let $K_c$ be the filled-in Julia set of $f_c(z) = z^2+c$.
  \begin{enumerate}
    \item Find the fixed points and the period 2 points of $f_{-6}$.
    \item Show that $2\sqrt{2} \in K_{-6}$ and find another point in $K_{-6}$,
      distinct from those found so far.
    \item Do any of the points you have found lie in the Julia set of $f_{-6}$?
    \item Is $-6\in\mathcal{M}$ where $\mathcal{M}$ is the Mandelbrot set?
  \end{enumerate}
\end{problem}

\begin{proof}
  \begin{enumerate}
    \item The fixed points of $f_{-6}$ are the solutions to
      \begin{align*}
        f_{-6}(z) - z = z^2 - z - 6  = 0.
      \end{align*}
      Thus, the fixed points of $f_{-6}$ are $z_0 = 3$ and $z_1 = -2$.
      The period 2 points are the solutions to
      \begin{align*}
        f_{-6}^2(z) - z = (z^2 - 6)^2 - z - 6  = 0
      \end{align*}
      that are also not fixed points of $f_{-6}$. Factoring $f_{-6}^2(z) - z$, we see that
      \begin{align*}
        f_{-6}^2(z) - z = (z- 3)(z + 2)(z^2 + z - 5).
      \end{align*}
      Thus, the period 2 points of $f_{-6}$ are the solutions to $z^2 + z - 5 = 0$, i.e.\
      the period 2 points of $f_{-6}$ are
      \begin{align*}
        z_2 = \frac{-1 - \sqrt{21}}{2}, \quad z_3 = \frac{-1 + \sqrt{21}}{2}.
      \end{align*}

    \item Recall that for a polynomial $p(z)$ with $\text{deg}(p) > 1$, the filled-in Julia
      set of $p(z)$ is the set of all points that do not converge to $\infty$ under iteration of $p$.

      Note that $2\sqrt{2}$ is an eventual fixed point of $f_{-6}$. We see that
      $f^{2}_{-6}(2\sqrt{2}) = -2$ so that $f^{k}_{-6}(2\sqrt{2}) = -2$ for $k > 2$.
      This implies that $2\sqrt{2}$ does not converge to $\infty$ under iteration of
      $f_{-6}$ so that $2\sqrt{2}$ is in the filled-in Julia set of $f_{-6}$, i.e.\
      $2\sqrt{2} \in K_{-6}$.

      For reasons similar to those listed above, we see that $-3$ is an eventual fixed point of
      $f_{-6}$, i.e.\ $f_{-6}(-3) = 3$, so that $-3 \in K_{-6}$.

    \item For a polynomial $p(z)$ with $\text{deg}(p) > 1$, the Julia
      set of $p(z)$ is the boundary of the basin of attraction of $\infty$.

      Since all of the points listed do not converge to $\infty$ under iteration of $f_{-6}$,
      we see that none of the listed points belong to the Julia set of $f_{-6}$.

    \item The definition of the Mandelbrot set
      is the set of all $c \in \mathbb{C}$ such that the orbit of 0 is bounded under iteration by $f_{c}$.
      It was shown previously that $c \in \mathcal{M}$ if and only if $|f_c^n(0)| \leq 2$ for all $n > 0$.
      For $f_{-6}$, we see that $f_{-6}(0) = -6$ where $|f_{-6}(0)| > 2$. Therefore,
      we must have that $-6 \notin \mathcal{M}$.
  \end{enumerate}
\end{proof}
\newpage
