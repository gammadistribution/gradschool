\begin{problem}{6}
  Let $f(x) = ax^3+bx+c$ where $a$ and $b$ satisfy $a/b > 0$. Denote by $N_f$
  the corresponding Newton function.
  \begin{enumerate}
    \item Show that $N_f$ has a unique fixed point.
    \item Show that $N_f$ cannot have any period 2 points.
    \item Why does it follow that $N_f$ has no points of period $n$ for $n>2$?
  \end{enumerate}
\end{problem}

\begin{proof}
  \begin{enumerate}
    \item Recall that the fixed points of $N_f$ are the roots of $f$.
      The discriminant of the polynomial $f$ is given by $D = -4 ab^3 -27 a^2 c^2$. Note that
      if $a/b > 0$ then $D < 0$. Therefore, $f$ only has one real root
      and as a consequence, $N_f$ has a unique fixed point, say $p$.
    \item If $f(x) = ax^3+bx+c$, then $f'(x) \neq 0$ for any $x\in\mathbb{R}$ if $a/b > 0$.
      Thus, all iterates of $N_f$ are well-defined. Since $f''(x)$ is bounded and the derivative of $f$ is non-zero on any finite interval,
      we have that the iterates of $N_f$ will converged to a root of $f$. Since $p$ is the only root,
      it must be a globally attracting fixed point of $N_f$. Thus,
      we have that $N_f^n(x)$ will converge to $p$ for all finite $x$.
      This implies that $\lim_n N_f^n(x) = p$ for all $x\neq p$.
      Therefore, since the limit of the iterates exist, we cannot have that $N_f$ has a period 2 point.
    \item If to the contrary, $N_f$ has a point of period $n > 2$,
      then since $n \triangleright 2$ in Sharkovsky's ordering, we must have
      by Sharkovsky's Theorem that $N_f$ has a point of period 2 . However, this is
      contradictory to the fact that $N_f$ has no points of period 2. Therefore,
      $N_f$ has no points of period $n > 2$.
  \end{enumerate}
\end{proof}
\newpage
