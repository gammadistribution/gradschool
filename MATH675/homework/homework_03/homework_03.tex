\documentclass[12pt]{article}
\usepackage[margin=1in]{geometry}
\usepackage{amsmath, amsthm, amssymb, amsfonts, enumitem, graphicx}
\usepackage{fancyhdr}
\usepackage{breqn}

\theoremstyle{definition}
\newtheorem{problem}{Problem}
\renewcommand*{\proofname}{Solution}
\renewcommand{\theenumi}{\alph{enumi}}

\newcommand{\vect}[1]{\boldsymbol{#1}}

\newenvironment{case}{%
  \let\olditem\item%
  \renewcommand\item[1][]{\olditem \textbf{##1} \\}%
  \begin{enumerate}[label=\textbf{Case \arabic*:},itemindent=*,leftmargin=0em]}{\end{enumerate}%
}

\pagestyle{fancy}
\fancyhf{}
\rhead{Homework Assignment 3}
\lhead{Matthew Tiger}
\cfoot{\thepage}


\title{Homework Assignment 3}
\author{Matthew Tiger}


\begin{document}


\maketitle


% Problem 1
\begin{problem}
  \begin{enumerate}
    \item Give an example of an asymptotic relation $f(x) \sim g(x)\ (x \to \infty)$
      that cannot be exponentiated; that is $e^{f(x)} \sim e^{g(x)}\ (x \to \infty)$ is false.
    \item Show that if $f(x) - g(x) \ll 1\ (x \to \infty)$, then $e^{f(x)} \sim e^{g(x)}\ (x \to \infty)$.
  \end{enumerate}
\end{problem}

\begin{proof}
  \begin{enumerate}
    \item Note that for $x\to\infty$ we have that $e^{f(x)} \not\sim e^{g(x)}$ if and only if
      \begin{align*}
        \lim_{x \to \infty}\frac{e^{f(x)}}{e^{g(x)}} = \lim_{x \to \infty} e^{f(x) - g(x)} \neq 1.
      \end{align*}
      Thus, if $\lim_{x\to\infty} f(x) - g(x) \neq 0$, then $\lim_{x \to \infty} e^{f(x) - g(x)} \neq 1$ and $e^{f(x)} \not\sim e^{g(x)}$.
      Therefore, take for instance the functions $f(x) = x + 1$ and $g(x) = x$. These functions are clearly asymptotic
      as
      \begin{align*}
        \lim_{x\to\infty} \frac{f(x)}{g(x)} = \lim_{x\to\infty} \frac{x + 1}{x} = \lim_{x\to\infty} 1 + \frac{1}{x} = 1.
      \end{align*}
      However,
      \begin{align*}
        \lim_{x \to \infty}\frac{e^{f(x)}}{e^{g(x)}} =         \lim_{x \to \infty} e^{f(x) - g(x)} = \lim_{x \to \infty}e^{(x+1) - x} = e \neq 1
      \end{align*}
      so that the asymptotic relation between $f(x)$ and $g(x)$ cannot be exponentiated.
    \item
      Suppose that $f(x) - g(x) \ll 1$ as $x\to\infty$. Then
      \begin{align*}
        \lim_{x\to\infty}\frac{f(x) - g(x)}{1} = \lim_{x\to\infty} f(x) - g(x) = 0.
      \end{align*}
      If this is true then we must have that
      \begin{align*}
        \lim_{x \to \infty}\frac{e^{f(x)}}{e^{g(x)}} = \lim_{x \to \infty} e^{f(x) - g(x)} = e^{\lim_{x \to \infty} f(x) - g(x)} = e^{0} = 1
      \end{align*}
      or that $e^{f(x)} \sim e^{g(x)}$ and we are done.

  \end{enumerate}
\end{proof}
\newpage


% Problem 2
\begin{problem}
  Find and classify all the singular points (including the point at $\infty$) of the equations:
  \begin{align*}
    x(1-x)y'' + [2-(a+b)x]y' - aby = 0, \qquad (x^2 + 1)y'' - xy = 0.
  \end{align*}
  Here, $a, b \in \mathbb{R}$.
\end{problem}

\begin{proof}
\end{proof}
\newpage


% Problem 3
\begin{problem}
  Find the Taylor series solution of the IVP
  \begin{align*}
    (1-x^3)y''' + 2xy' = 0,\quad y(0) = 3, y'(0) = 3, y''(0) = 0.
  \end{align*}
\end{problem}

\begin{proof}
\end{proof}
\newpage


% Problem 4
\begin{problem}
  Find two linearly independent solutions to $x(1-x)y'' - 3xy'-y = 0$.
\end{problem}

\begin{proof}
\end{proof}
\newpage


% Problem 5
\begin{problem}
  Find two linearly independent solutions to $x^2y'' + 3xy' + (1-2x)y = 0$.
\end{problem}

\begin{proof}
\end{proof}
\newpage


% Problem 6
\begin{problem}
  Find the leading behavior of both solutions of $x^5y'' - y = 0$ near $x=0$.
\end{problem}

\begin{proof}
\end{proof}
\newpage


% Problem 7
\begin{problem}
  Find the first four terms in the asymptotic series for the solutions of $y'' = e^{-2/x}y$ as $x\to + \infty$.

  \textbf{Hint:} When you are performing the asymptotic analysis to extract the leading
  behavior of the solution as $x\to  + \infty$, you may (and probably want) to replace
  $e^{-2/x}$ with a reasonable simpler approximation.
\end{problem}

\begin{proof}
\end{proof}
\newpage

\end{document}