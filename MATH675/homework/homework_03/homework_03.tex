\documentclass[12pt]{article}
\usepackage[margin=1in]{geometry}
\usepackage{amsmath, amsthm, amssymb, amsfonts, enumitem, graphicx}
\usepackage{fancyhdr}
\usepackage{breqn}

\theoremstyle{definition}
\newtheorem{problem}{Problem}
\renewcommand*{\proofname}{Solution}
\renewcommand{\theenumi}{\alph{enumi}}

\newcommand{\vect}[1]{\boldsymbol{#1}}

\newenvironment{case}{%
  \let\olditem\item%
  \renewcommand\item[1][]{\olditem \textbf{##1} \\}%
  \begin{enumerate}[label=\textbf{Case \arabic*:},itemindent=*,leftmargin=0em]}{\end{enumerate}%
}

\pagestyle{fancy}
\fancyhf{}
\rhead{Homework Assignment 3}
\lhead{Matthew Tiger}
\cfoot{\thepage}


\title{Homework Assignment 3}
\author{Matthew Tiger}


\begin{document}


\maketitle


% Problem 1
\begin{problem}
  \begin{enumerate}
    \item Give an example of an asymptotic relation $f(x) \sim g(x)\ (x \to \infty)$
      that cannot be exponentiated; that is $e^{f(x)} \sim e^{g(x)}\ (x \to \infty)$ is false.
    \item Show that if $f(x) - g(x) \ll 1\ (x \to \infty)$, then $e^{f(x)} \sim e^{g(x)}\ (x \to \infty)$.
  \end{enumerate}
\end{problem}

\begin{proof}
  \begin{enumerate}
    \item Note that for $x\to\infty$ we have that $e^{f(x)} \not\sim e^{g(x)}$ if and only if
      \begin{align*}
        \lim_{x \to \infty}\frac{e^{f(x)}}{e^{g(x)}} = \lim_{x \to \infty} e^{f(x) - g(x)} \neq 1.
      \end{align*}
      Thus, if $\lim_{x\to\infty} f(x) - g(x) \neq 0$, then $\lim_{x \to \infty} e^{f(x) - g(x)} \neq 1$ and $e^{f(x)} \not\sim e^{g(x)}$.
      Therefore, take for instance the functions $f(x) = x + 1$ and $g(x) = x$. These functions are clearly asymptotic
      as
      \begin{align*}
        \lim_{x\to\infty} \frac{f(x)}{g(x)} = \lim_{x\to\infty} \frac{x + 1}{x} = \lim_{x\to\infty} 1 + \frac{1}{x} = 1.
      \end{align*}
      However,
      \begin{align*}
        \lim_{x \to \infty}\frac{e^{f(x)}}{e^{g(x)}} =         \lim_{x \to \infty} e^{f(x) - g(x)} = \lim_{x \to \infty}e^{(x+1) - x} = e \neq 1
      \end{align*}
      so that the asymptotic relation between $f(x)$ and $g(x)$ cannot be exponentiated.
    \item
      Suppose that $f(x) - g(x) \ll 1$ as $x\to\infty$. Then
      \begin{align*}
        \lim_{x\to\infty}\frac{f(x) - g(x)}{1} = \lim_{x\to\infty} f(x) - g(x) = 0.
      \end{align*}
      If this is true then we must have that
      \begin{align*}
        \lim_{x \to \infty}\frac{e^{f(x)}}{e^{g(x)}} = \lim_{x \to \infty} e^{f(x) - g(x)} = e^{\lim_{x \to \infty} f(x) - g(x)} = e^{0} = 1
      \end{align*}
      or that $e^{f(x)} \sim e^{g(x)}$ and we are done.

  \end{enumerate}
\end{proof}
\newpage


% Problem 2
\begin{problem}
  Find and classify all the singular points (including the point at $\infty$) of the equations:
  \begin{align*}
    x(1-x)y'' + [2-(a+b)x]y' - aby = 0, \qquad (x^2 + 1)y'' - xy = 0.
  \end{align*}
  Here, $a, b \in \mathbb{R}$.
\end{problem}

\begin{proof}
  If a homogeneous differential equation is of the form
  \begin{align}\label{std_form}
    y^{(n)}(x) + p_{n-1}(x)y^{(n-1)}(x) + \dots + p_1(x)y^{(1)}(x) + p_0(x)y = 0
  \end{align}
  then the point $x = x_0$ is:
  \begin{itemize}
    \item \emph{ordinary} if the coefficient functions $p_i(x)$ for $i=0,\dots,n-1$
      are all analytic in a neighborhood of $x=x_0$.
    \item \emph{singular} if some coefficient function $p_i(x)$ for $i=0,\dots,n-1$
      is not analytic in some neighborhood of $x=x_0$.
  \end{itemize}
  If the point $x=x_0$ is singular, then the point is:
  \begin{itemize}
    \item \emph{regular singular} if the functions $(x-x_0)p_{n-1}(x),\dots,(x-x_0)^{n-1}p_1(x),(x-x_0)^n p_0(x)$ are all analytic in a neighborhood of $x=x_0$.
    \item \emph{irregular singular} if one of the functions $(x-x_0)p_{n-1}(x),\dots,(x-x_0)^{n-1}p_1(x),(x-x_0)^n p_0(x)$ are not analytic in some neighborhood of $x=x_0$.
  \end{itemize}

  \begin{enumerate}
    \item We begin by transforming the differential equation $x(1-x)y'' + [2-(a+b)x]y' - aby = 0$
      into the form presented in \eqref{std_form}:
      \begin{align*}
        y'' + \left[\frac{2}{x(1-x)} - \frac{a+b}{1-x}\right]y' + \left[-\frac{ab}{x(1-x)}\right]y = 0.
      \end{align*}
      Thus, we identify the coefficient functions
      \begin{align*}
        p_1(x) = \frac{2}{x(1-x)} - \frac{a+b}{1-x},\quad p_0(x) = -\frac{ab}{x(1-x)}.
      \end{align*}
      In order to classify all finite points, we see that these coefficient functions are analytic everywhere on the real line except for the points
      $x_0=0$ and $x_1=1$. To classify the singular points $x_i$ for $i=0,1$, we must determine if the functions
      $(x-x_i)p_1(x)$ and $(x-x_i)^2 p_0(x)$ are analytic.

      For the point $x_0 = 0$, we know that
      \begin{align*}
        x p_1(x) &= \frac{2}{(1-x)} - \frac{(a+b)x}{1-x} \\
        x^2 p_0(x) &= -\frac{abx}{(1-x)}
      \end{align*}
      both of which are analytic at $x = 0$ showing that $x_0 = 0$ is a regular singular point.

      For the point $x_0 = 1$, we know that
      \begin{align*}
        (x-1)p_1(x) &= -\frac{2}{x} + a + b \\
        (x-1)^2 p_0(x) &= \frac{ab(x-1)}{x}
      \end{align*}
      both of which are analytic at $x = 1$ showing that $x_0 = 1$ is a regular singular point.

      In order to classify the point at $\infty$, we make the transformation $x=\frac{1}{t}$ and classifying the point $t=0$.
      Thus, the equation becomes
      \begin{align*}
        t^4 y'' + \left[2t^3 - \frac{2t^4}{t-1} + \frac{(a+b)t^3}{t-1}\right]y' + \left[-\frac{abt^2}{t-1}\right]y = 0.
      \end{align*}
      The coefficient functions are then
      \begin{align*}
        p_1(t) = \frac{2}{t} - \frac{2}{t-1} + \frac{a+b}{t(t-1)},\quad p_0(t) = -\frac{ab}{t^2(t-1)}.
      \end{align*}
      It is clear that $x_2 = \infty$ is a singular point since the coefficient functions are not analytic at $t=0$.
      Since both $tp_1(t)$ and $t^2p_0(t)$ are analytic at $t=0$, the point $x_2 = \infty$ is a regular singular point.
    \item We rewrite the differential equation $(x^2 + 1)y'' - xy = 0$ in the form presented in \eqref{std_form}:
      \begin{align*}
        y'' + \left[- \frac{x}{x^2+1}\right]y = 0.
      \end{align*}
      Thus, the coefficient function is
      \begin{align*}
        p_0(x) = -\frac{x}{x^2+1}=-\frac{x}{(x-i)(x+i)}.
      \end{align*}
      In order to classify all finite points, we see that this coefficient function is analytic everywhere
      on the real line so that all finite real points are ordinary.

      In order to classify the point at $\infty$, we make the transformation $x=\frac{1}{t}$ and classifying the point $t=0$.
      Thus, the equation becomes
      \begin{align*}
        t^4 y'' + 2t^3y' - \frac{t}{t^2+1}y = 0.
      \end{align*}
      The coefficient functions are then
      \begin{align*}
        p_1(t) = \frac{2}{t},\quad p_0(t) = -\frac{1}{t^3(t^2+1)}.
      \end{align*}
      The point $t=0$ is clearly a singular point since both coefficient functions are not analytic at $t=0$.
      Since $t^2p_0(t)$ is also not analytic at $t=0$, the point $x_2 = \infty$ is an irregular singular point.
  \end{enumerate}
\end{proof}
\newpage


% Problem 3
\begin{problem}
  Find the Taylor series solution of the IVP
  \begin{align*}
    Ly = (1-x^3)y''' + 2xy' = 0,\quad y(0) = 3, y'(0) = 3, y''(0) = 0.
  \end{align*}
\end{problem}

\begin{proof}
  Note that the point $x=0$ is an ordinary point of this differential equation. Therefore, we may assume the solution to the differential
  equation is a Taylor series of the form
  \begin{align*}
    y(x) = \sum_{n=0}^\infty a_n x^n.
  \end{align*}
  Using this form, we see that the differential equation becomes
  \begin{align}\label{series}
    Ly &= (1-x^3)\sum_{n=0}^\infty n(n-1)(n-2) a_n x^{n-3} + 2x \sum_{n=0}^\infty n a_n x^{n-1} \notag \\
    &= \sum_{n=0}^\infty n(n-1)(n-2) a_n x^{n-3} - \sum_{n=0}^\infty n(n-1)(n-2) a_n x^{n} + \sum_{n=0}^\infty 2n a_n x^{n} \notag \\
    &= \sum_{n=0}^\infty n(n-1)(n-2) a_n x^{n-3} - \sum_{n=3}^\infty (n-3)(n-4)(n-5) a_{n-3} x^{n-3} + \sum_{n=3}^\infty 2(n-3) a_{n-3} x^{n-3} \notag \\
    &= \sum_{n=3}^\infty \left[n(n-1)(n-2) a_n - ((n-3)(n-4)(n-5) - 2(n-3)) a_{n-3}\right] x^{n-3} = 0.
  \end{align}
  Thus, we see that the series in \eqref{series} satisfies the differential equation
  if and only if the coefficients of the series are identically zero. From this we derive the recurrence relation
  \begin{align*}
      n(n-1)(n-2)a_{n} &= 0 &n &=0,1,2 \\
      a_{n} &= \frac{\left[(n-3)(n-4)(n-5) - 2(n-3)\right]a_{n-3}}{n(n-1)(n-2)} &n &= 3,4,5,\dots
  \end{align*}
  where the coefficients $a_0$, $a_1$, and $a_2$ are arbitrary constants. The solutions
  of the second equation are
  \begin{align*}
    a_{3n} &= 0, \\
    a_{3n+1} &= a_1\left[\left(\frac{(1\cdot 0\cdot -1) - (2\cdot1)}{4\cdot 3\cdot 2}\right)\left(\frac{(4\cdot 3\cdot 2) - (2\cdot4)}{7\cdot 6\cdot 5}\right)\dots\right. \\
    &\phantom{=}\ \ \dots\left(\frac{(3n-5)\cdot (3n-6)\cdot (3n-7) - 2\cdot(3n-5)}{(3n-2)(3n-3)(3n-4)}\right)\dots \\
    &\phantom{=}\ \ \left.\dots\left(\frac{(3n-2)\cdot (3n-3)\cdot (3n-4) - 2\cdot(3n-2)}{(3n+1)(3n)(3n-1)}\right)\right] \\
    &= a_1\left[\frac{\Pi_{k=1}^{n} (3k-2)(3k-3)(3k-4) - 2(3k-2)}{(3n+1)!}\right],\\
  \end{align*}
  \begin{align*}
    a_{3n+2} &= a_2\left[\left(\frac{(2\cdot 1\cdot 0) - (2\cdot2)}{5\cdot 4\cdot 3}\right)\left(\frac{(5\cdot 4\cdot 3) - (2\cdot5)}{8\cdot 7\cdot 6}\right)\dots\right. \\
    &\phantom{=}\ \ \dots\left(\frac{(3n-4)\cdot (3n-5)\cdot (3n-6) - 2\cdot(3n-4)}{(3n-1)(3n-2)(3n-3)}\right)\dots \\
    &\phantom{=}\ \ \left.\dots\left(\frac{(3n-1)\cdot (3n-2)\cdot (3n-3) - 2\cdot(3n-1)}{(3n+2)(3n+1)(3n)}\right)\right] \\
    &= a_2\left[\frac{2\Pi_{k=1}^{n} (3k-1)(3k-2)(3k-3) - 2(3k-1)}{(3n+2)!}\right].
  \end{align*}

  Therefore, the general solution to the differential equation $Ly = 0$ is
  \begin{align}\label{gen_sol}
    y(x) &= a_0 + a_1x + a_2x^2 + \sum_{n=3}^{\infty}a_nx^n \notag \\
    &= a_0 + a_1x + a_2x^2 + \sum_{n=1}^{\infty}a_{3n+1}x^{3n+1} + \sum_{n=1}^\infty a_{3n+2}x^{3n+2} \notag\\
    &= a_0 + a_1\left[x + \sum_{n=1}^{\infty}\frac{a_{3n+1}}{a_1}x^{3n+1}\right] + a_2\left[x^2 + \sum_{n=1}^\infty \frac{a_{3n+2}}{a_2}x^{3n+2} \right].
  \end{align}
  Using the initial conditions $y(0) = y'(0) = 3, y''(0) = 0$ and the general solution \eqref{gen_sol},
  we see that
  \begin{align*}
    y'(x) &= a_1 + 2a_2x + \sum_{n=1}^\infty (3n+1)a_{3n+1} x^{3n} +\sum_{n=1}^\infty(3n+2)a_{3n+2}x^{3n+1} \\
    y''(x) &= 2a_2 + \sum_{n=1}^\infty (3n+1)(3n)a_{3n+1} x^{3n-1} +\sum_{n=1}^\infty(3n+2)(3n+1)a_{3n+2}x^{3n}
  \end{align*}
  which implies that
  \begin{align*}
    y(0) &= a_0 = 3 \\
    y'(0) &= a_1 = 3 \\
    y'(0) &= 2a_2 = 0.
  \end{align*}
  Therefore, the solution to the IVP is
  \begin{align*}
    y(x) = 3 + 3x + 3\sum_{n=1}^{\infty}\left[\frac{\Pi_{k=1}^{n} (3k-2)(3k-3)(3k-4) - 2(3k-2)}{(3n+1)!}\right]x^{3n+1}.
  \end{align*}
\end{proof}
\newpage


% Problem 4
\begin{problem}
  Find two linearly independent solutions to $x(1-x)y'' - 3xy'-y = 0$ about $x=0$.
\end{problem}

\begin{proof}
\end{proof}
\newpage


% Problem 5
\begin{problem}
  Find two linearly independent solutions to $x^2y'' + 3xy' + (1-2x)y = 0$ about $x=0$.
\end{problem}

\begin{proof}
\end{proof}
\newpage


% Problem 6
\begin{problem}
  Find the leading behavior of both solutions of $x^5y'' - y = 0$ near $x=0$.
\end{problem}

\begin{proof}
\end{proof}
\newpage


% Problem 7
\begin{problem}
  Find the first four terms in the asymptotic series for the solutions of $y'' = e^{-2/x}y$ as $x\to + \infty$.

  \textbf{Hint:} When you are performing the asymptotic analysis to extract the leading
  behavior of the solution as $x\to  + \infty$, you may (and probably want) to replace
  $e^{-2/x}$ with a reasonable simpler approximation.
\end{problem}

\begin{proof}
\end{proof}
\newpage

\end{document}