\documentclass[12pt]{article}
\usepackage[margin=1in]{geometry}
\usepackage{amsmath, amsthm, amssymb, amsfonts, enumitem, graphicx}
\usepackage{fancyhdr}
\usepackage{breqn}

\theoremstyle{definition}
\newtheorem{problem}{Problem}
\renewcommand*{\proofname}{Solution}
\renewcommand{\theenumi}{\alph{enumi}}

\newcommand{\vect}[1]{\boldsymbol{#1}}

\newenvironment{case}{%
  \let\olditem\item%
  \renewcommand\item[1][]{\olditem \textbf{##1} \\}%
  \begin{enumerate}[label=\textbf{Case \arabic*:},itemindent=*,leftmargin=0em]}{\end{enumerate}%
}

\pagestyle{fancy}
\fancyhf{}
\rhead{Homework Assignment 3}
\lhead{Matthew Tiger}
\cfoot{\thepage}


\title{Homework Assignment 3}
\author{Matthew Tiger}


\begin{document}


\maketitle


% Problem 1
\begin{problem}
  \begin{enumerate}
    \item Give an example of an asymptotic relation $f(x) \sim g(x)\ (x \to \infty)$
      that cannot be exponentiated; that is $e^{f(x)} \sim e^{g(x)}\ (x \to \infty)$ is false.
    \item Show that if $f(x) - g(x) \ll 1\ (x \to \infty)$, then $e^{f(x)} \sim e^{g(x)}\ (x \to \infty)$.
  \end{enumerate}
\end{problem}

\begin{proof}
  \begin{enumerate}
    \item Note that for $x\to\infty$ we have that $e^{f(x)} \not\sim e^{g(x)}$ if and only if
      \begin{align*}
        \lim_{x \to \infty}\frac{e^{f(x)}}{e^{g(x)}} = \lim_{x \to \infty} e^{f(x) - g(x)} \neq 1.
      \end{align*}
      Thus, if $\lim_{x\to\infty} f(x) - g(x) \neq 0$, then $\lim_{x \to \infty} e^{f(x) - g(x)} \neq 1$ and $e^{f(x)} \not\sim e^{g(x)}$.
      Therefore, take for instance the functions $f(x) = x + 1$ and $g(x) = x$. These functions are clearly asymptotic
      as
      \begin{align*}
        \lim_{x\to\infty} \frac{f(x)}{g(x)} = \lim_{x\to\infty} \frac{x + 1}{x} = \lim_{x\to\infty} 1 + \frac{1}{x} = 1.
      \end{align*}
      However,
      \begin{align*}
        \lim_{x \to \infty}\frac{e^{f(x)}}{e^{g(x)}} =         \lim_{x \to \infty} e^{f(x) - g(x)} = \lim_{x \to \infty}e^{(x+1) - x} = e \neq 1
      \end{align*}
      so that the asymptotic relation between $f(x)$ and $g(x)$ cannot be exponentiated.
    \item
      Suppose that $f(x) - g(x) \ll 1$ as $x\to\infty$. Then
      \begin{align*}
        \lim_{x\to\infty}\frac{f(x) - g(x)}{1} = \lim_{x\to\infty} f(x) - g(x) = 0.
      \end{align*}
      If this is true then we must have that
      \begin{align*}
        \lim_{x \to \infty}\frac{e^{f(x)}}{e^{g(x)}} = \lim_{x \to \infty} e^{f(x) - g(x)} = e^{\lim_{x \to \infty} f(x) - g(x)} = e^{0} = 1
      \end{align*}
      or that $e^{f(x)} \sim e^{g(x)}$ and we are done.

  \end{enumerate}
\end{proof}
\newpage


% Problem 2
\begin{problem}
  Find and classify all the singular points (including the point at $\infty$) of the equations:
  \begin{align*}
    x(1-x)y'' + [2-(a+b)x]y' - aby = 0, \qquad (x^2 + 1)y'' - xy = 0.
  \end{align*}
  Here, $a, b \in \mathbb{R}$.
\end{problem}

\begin{proof}
  If a homogeneous differential equation is of the form
  \begin{align}\label{std_form}
    y^{(n)}(x) + p_{n-1}(x)y^{(n-1)}(x) + \dots + p_1(x)y^{(1)}(x) + p_0(x)y = 0
  \end{align}
  then the point $x = x_0$ is:
  \begin{itemize}
    \item \emph{ordinary} if the coefficient functions $p_i(x)$ for $i=0,\dots,n-1$
      are all analytic in a neighborhood of $x=x_0$.
    \item \emph{singular} if some coefficient function $p_i(x)$ for $i=0,\dots,n-1$
      is not analytic in some neighborhood of $x=x_0$.
  \end{itemize}
  If the point $x=x_0$ is singular, then the point is:
  \begin{itemize}
    \item \emph{regular singular} if the functions $(x-x_0)p_{n-1}(x),\dots,(x-x_0)^{n-1}p_1(x),(x-x_0)^n p_0(x)$ are all analytic in a neighborhood of $x=x_0$.
    \item \emph{irregular singular} if one of the functions $(x-x_0)p_{n-1}(x),\dots,(x-x_0)^{n-1}p_1(x),(x-x_0)^n p_0(x)$ are not analytic in some neighborhood of $x=x_0$.
  \end{itemize}

  \begin{enumerate}
    \item We begin by transforming the differential equation $x(1-x)y'' + [2-(a+b)x]y' - aby = 0$
      into the form presented in \eqref{std_form}:
      \begin{align*}
        y'' + \left[\frac{2}{x(1-x)} - \frac{a+b}{1-x}\right]y' + \left[-\frac{ab}{x(1-x)}\right]y = 0.
      \end{align*}
      Thus, we identify the coefficient functions
      \begin{align*}
        p_1(x) = \frac{2}{x(1-x)} - \frac{a+b}{1-x},\quad p_0(x) = -\frac{ab}{x(1-x)}.
      \end{align*}
      In order to classify all finite points, we see that these coefficient functions are analytic everywhere on the real line except for the points
      $x_0=0$ and $x_1=1$. To classify the singular points $x_i$ for $i=0,1$, we must determine if the functions
      $(x-x_i)p_1(x)$ and $(x-x_i)^2 p_0(x)$ are analytic.

      For the point $x_0 = 0$, we know that
      \begin{align*}
        x p_1(x) &= \frac{2}{(1-x)} - \frac{(a+b)x}{1-x} \\
        x^2 p_0(x) &= -\frac{abx}{(1-x)}
      \end{align*}
      both of which are analytic at $x = 0$ showing that $x_0 = 0$ is a regular singular point.

      For the point $x_0 = 1$, we know that
      \begin{align*}
        (x-1)p_1(x) &= -\frac{2}{x} + a + b \\
        (x-1)^2 p_0(x) &= \frac{ab(x-1)}{x}
      \end{align*}
      both of which are analytic at $x = 1$ showing that $x_0 = 1$ is a regular singular point.

      In order to classify the point at $\infty$, we make the transformation $x=\frac{1}{t}$ and classifying the point $t=0$.
      Thus, the equation becomes
      \begin{align*}
        t^4 y'' + \left[2t^3 - \frac{2t^4}{t-1} + \frac{(a+b)t^3}{t-1}\right]y' + \left[-\frac{abt^2}{t-1}\right]y = 0.
      \end{align*}
      The coefficient functions are then
      \begin{align*}
        p_1(t) = \frac{2}{t} - \frac{2}{t-1} + \frac{a+b}{t(t-1)},\quad p_0(t) = -\frac{ab}{t^2(t-1)}.
      \end{align*}
      It is clear that $x_2 = \infty$ is a singular point since the coefficient functions are not analytic at $t=0$.
      Since both $tp_1(t)$ and $t^2p_0(t)$ are analytic at $t=0$, the point $x_2 = \infty$ is a regular singular point.
    \item We rewrite the differential equation $(x^2 + 1)y'' - xy = 0$ in the form presented in \eqref{std_form}:
      \begin{align*}
        y'' + \left[- \frac{x}{x^2+1}\right]y = 0.
      \end{align*}
      Thus, the coefficient function is
      \begin{align*}
        p_0(x) = -\frac{x}{x^2+1}=-\frac{x}{(x-i)(x+i)}.
      \end{align*}
      In order to classify all finite points, we see that this coefficient function is analytic everywhere
      on the real line so that all finite real points are ordinary.

      In order to classify the point at $\infty$, we make the transformation $x=\frac{1}{t}$ and classifying the point $t=0$.
      Thus, the equation becomes
      \begin{align*}
        t^4 y'' + 2t^3y' - \frac{t}{t^2+1}y = 0.
      \end{align*}
      The coefficient functions are then
      \begin{align*}
        p_1(t) = \frac{2}{t},\quad p_0(t) = -\frac{1}{t^3(t^2+1)}.
      \end{align*}
      The point $t=0$ is clearly a singular point since both coefficient functions are not analytic at $t=0$.
      Since $t^2p_0(t)$ is also not analytic at $t=0$, the point $x_2 = \infty$ is an irregular singular point.
  \end{enumerate}
\end{proof}
\newpage


% Problem 3
\begin{problem}
  Find the Taylor series solution of the IVP
  \begin{align*}
    Ly = (1-x^3)y''' + 2xy' = 0,\quad y(0) = 3, y'(0) = 3, y''(0) = 0.
  \end{align*}
\end{problem}

\begin{proof}
  Note that the point $x=0$ is an ordinary point of this differential equation. Therefore, we may assume the solution to the differential
  equation is a Taylor series of the form
  \begin{align*}
    y(x) = \sum_{n=0}^\infty a_n x^n.
  \end{align*}
  Using this form, we see that the differential equation becomes
  \begin{align}\label{series}
    Ly &= (1-x^3)\sum_{n=0}^\infty n(n-1)(n-2) a_n x^{n-3} + 2x \sum_{n=0}^\infty n a_n x^{n-1} \notag \\
    &= \sum_{n=0}^\infty n(n-1)(n-2) a_n x^{n-3} - \sum_{n=0}^\infty n(n-1)(n-2) a_n x^{n} + \sum_{n=0}^\infty 2n a_n x^{n} \notag \\
    &= \sum_{n=0}^\infty n(n-1)(n-2) a_n x^{n-3} - \sum_{n=3}^\infty (n-3)(n-4)(n-5) a_{n-3} x^{n-3} + \sum_{n=3}^\infty 2(n-3) a_{n-3} x^{n-3} \notag \\
    &= \sum_{n=3}^\infty \left[n(n-1)(n-2) a_n - ((n-3)(n-4)(n-5) - 2(n-3)) a_{n-3}\right] x^{n-3} = 0.
  \end{align}
  Thus, we see that the series in \eqref{series} satisfies the differential equation
  if and only if the coefficients of the series are identically zero. From this we derive the recurrence relation
  \begin{align*}
      n(n-1)(n-2)a_{n} &= 0 &n &=0,1,2 \\
      a_{n} &= \frac{\left[(n-3)(n-4)(n-5) - 2(n-3)\right]a_{n-3}}{n(n-1)(n-2)} &n &= 3,4,5,\dots
  \end{align*}
  where the coefficients $a_0$, $a_1$, and $a_2$ are arbitrary constants. The solutions
  of the second equation are
  \begin{align*}
    a_{3n} &= 0, \\
    a_{3n+1} &= a_1\left[\left(\frac{(1\cdot 0\cdot -1) - (2\cdot1)}{4\cdot 3\cdot 2}\right)\left(\frac{(4\cdot 3\cdot 2) - (2\cdot4)}{7\cdot 6\cdot 5}\right)\dots\right. \\
    &\phantom{=}\ \ \dots\left(\frac{(3n-5)\cdot (3n-6)\cdot (3n-7) - 2\cdot(3n-5)}{(3n-2)(3n-3)(3n-4)}\right)\dots \\
    &\phantom{=}\ \ \left.\dots\left(\frac{(3n-2)\cdot (3n-3)\cdot (3n-4) - 2\cdot(3n-2)}{(3n+1)(3n)(3n-1)}\right)\right] \\
    &= a_1\left[\frac{\Pi_{k=1}^{n} (3k-2)(3k-3)(3k-4) - 2(3k-2)}{(3n+1)!}\right],\\
  \end{align*}
  \begin{align*}
    a_{3n+2} &= a_2\left[\left(\frac{(2\cdot 1\cdot 0) - (2\cdot2)}{5\cdot 4\cdot 3}\right)\left(\frac{(5\cdot 4\cdot 3) - (2\cdot5)}{8\cdot 7\cdot 6}\right)\dots\right. \\
    &\phantom{=}\ \ \dots\left(\frac{(3n-4)\cdot (3n-5)\cdot (3n-6) - 2\cdot(3n-4)}{(3n-1)(3n-2)(3n-3)}\right)\dots \\
    &\phantom{=}\ \ \left.\dots\left(\frac{(3n-1)\cdot (3n-2)\cdot (3n-3) - 2\cdot(3n-1)}{(3n+2)(3n+1)(3n)}\right)\right] \\
    &= a_2\left[\frac{2\Pi_{k=1}^{n} (3k-1)(3k-2)(3k-3) - 2(3k-1)}{(3n+2)!}\right].
  \end{align*}

  Therefore, the general solution to the differential equation $Ly = 0$ is
  \begin{align}\label{gen_sol}
    y(x) &= a_0 + a_1x + a_2x^2 + \sum_{n=3}^{\infty}a_nx^n \notag \\
    &= a_0 + a_1x + a_2x^2 + \sum_{n=1}^{\infty}a_{3n+1}x^{3n+1} + \sum_{n=1}^\infty a_{3n+2}x^{3n+2} \notag\\
    &= a_0 + a_1\left[x + \sum_{n=1}^{\infty}\frac{a_{3n+1}}{a_1}x^{3n+1}\right] + a_2\left[x^2 + \sum_{n=1}^\infty \frac{a_{3n+2}}{a_2}x^{3n+2} \right].
  \end{align}
  Using the initial conditions $y(0) = y'(0) = 3, y''(0) = 0$ and the general solution \eqref{gen_sol},
  we see that
  \begin{align*}
    y'(x) &= a_1 + 2a_2x + \sum_{n=1}^\infty (3n+1)a_{3n+1} x^{3n} +\sum_{n=1}^\infty(3n+2)a_{3n+2}x^{3n+1} \\
    y''(x) &= 2a_2 + \sum_{n=1}^\infty (3n+1)(3n)a_{3n+1} x^{3n-1} +\sum_{n=1}^\infty(3n+2)(3n+1)a_{3n+2}x^{3n}
  \end{align*}
  which implies that
  \begin{align*}
    y(0) &= a_0 = 3 \\
    y'(0) &= a_1 = 3 \\
    y'(0) &= 2a_2 = 0.
  \end{align*}
  Therefore, the solution to the IVP is
  \begin{align*}
    y(x) = 3 + 3x + 3\sum_{n=1}^{\infty}\left[\frac{\Pi_{k=1}^{n} (3k-2)(3k-3)(3k-4) - 2(3k-2)}{(3n+1)!}\right]x^{3n+1}.
  \end{align*}
\end{proof}
\newpage


% Problem 4
\begin{problem}
  Find two linearly independent solutions to $x(1-x)y'' - 3xy'-y = 0$ about $x=0$.
\end{problem}

\begin{proof}
  Suppose that we have a second-order differential equation in the following
  form
  \begin{align}\label{form}
    Ly = y'' + \left[\frac{p(x)}{x-x_0}\right]y' + \left[\frac{q(x)}{(x-x_0)^2}\right]y = 0
  \end{align}
  where $p(x)$ and $q(x)$ are analytic at the point $x=x_0$, i.e.\ $p(x)$ and $q(x)$ have power series representations centered at $x=x_0$, say
  \begin{align*}
    p(x) = \sum_{n=0}^\infty p_n x^n, \quad     q(x) = \sum_{n=0}^\infty q_n x^n.
  \end{align*}
  Clearly $x=x_0$ is a regular singular point of the differential equation $Ly = 0$. As such, the results of Fuchs show that there is at least one solution to the
  differential equation $Ly = 0$ of the form
  \begin{align}\label{frob}
    y(x) = (x-x_0)^\alpha \sum_{n=0}^\infty a_n (x-x_0)^n =   \sum_{n=0}^\infty a_n (x-x_0)^{n+\alpha}
  \end{align}
  where $\alpha\in\mathbb{R}$. Substituting this expression for $y(x)$ into the differential equation $Ly=0$ yields
  \begin{align*}
    Ly &= (x-x_0)^{\alpha-2}\left[\sum_{n=0}^\infty\left[(n+\alpha)(n+\alpha-1)a_n + \sum_{k=0}^n(k+\alpha)p_{n-k}a_k + \sum_{k=0}^nq_{n-k}a_k\right](x-x_0)^n \right]
  \end{align*}
  which implies that if $Ly=0$ then the coefficients of the above series must be zero yielding the following recurrence relation:
  \begin{align}\label{rel}
    \left[\alpha(\alpha-1)+ \alpha p_{0} + q_{0}\right] a_0&= 0 &n &=0 \notag \\
    \left[(n+\alpha)(n+\alpha-1) + (n+\alpha)p_0 +q_0\right] a_n &= - \sum_{k=0}^{n-1}\left[(k+\alpha)p_{n-k} + q_{n-k}\right]a_k  &n &\geq 1.
  \end{align}
  Thus, in order to generate a non-trival solution that satisfies this relation,
  we require that $\alpha$ must be a root of the \emph{indicial polynomial}
  \begin{align*}
    P(z) = z(z-1)+ z p_{0} + q_{0}
  \end{align*}
  since $a_0 \neq 0$ by convention.
  This choice of $\alpha$ then ensures that the solution \eqref{frob} will then be a non-trivial solution that satisfies the differential equation $Ly=0$.
  Depending on the roots of the indicial polynomial, a second linearly independent solution of the same form of the differential equation can be found.

  Note that for the differential equation
  \begin{align}\label{diff_eq_1}
    Ly = x(1-x)y'' - 3xy'-y = 0,
  \end{align}
  the point $x=0$ is a regular singular point. Thus, we begin by rewriting the equation $Ly=0$
  in the form presented in \eqref{form}:
  \begin{align*}
    Ly &= y'' + \left[\frac{-\frac{3x}{1-x}}{x}\right]y' + \left[\frac{-\frac{x}{1-x}}{x^2}\right]y \\
    &= y'' + \left[\frac{p(x)}{x}\right]y' + \left[\frac{q(x)}{x^2}\right]y = 0.
  \end{align*}
  From this we identify the functions $p(x) = -\frac{3x}{1-x}$ and $q(x) = -\frac{x}{1-x}$ which are
  both analytic at the singular point $x=0$. Note that
  $p(x)$ and $q(x)$ have the following power series representations centered at $x=0$
  \begin{align*}
    p(x) &= -3x\sum_{n=0}^\infty x^n = \sum_{n=1}^\infty -3 x^{n} \\
    q(x) &= -x\sum_{n=0}^\infty x^n = \sum_{n=1}^\infty - x^{n}.
  \end{align*}
  Using these power series representations, we see that $p_0 = q_0 = 0$, and that the indicial polynomial of
  the differential equation $Ly=0$ is
  \begin{align*}
    P(z) = z(z-1)
  \end{align*}
  the roots of which are $\alpha_1 = 1$ and $\alpha_2 = 0$. Using $\alpha_1 = 1$, we know that one solution of
  the differential equation is of the form \eqref{frob}, i.e.\ there is a solution $y_1(x)$ of the form
  \begin{align*}
    y_1(x) = \sum_{n=0}^\infty a_n x^{n+1}
  \end{align*}
  where the coefficients $a_n$ satisfy the recurrence relation \eqref{rel} with $a_0 \neq 0$. Using the recurrence relation with $\alpha = 1$ we see that for $n>0$,
  \begin{align*}
    a_n &= -\frac{\sum_{k=0}^{n-1} [(k+1)p_{n-k} + q_{n-k}]a_k}{n(n+1)} \\
    &= \frac{\sum_{k=0}^{n-1} (3k + 4)a_k}{n(n+1)}
  \end{align*}
  since $p_{n-k} = -3$ and $q_{n-k} = -1$ for $0\leq k \leq n-1$. It can easily be proven through induction that the solution to this newly obtained recurrence
  relation is $a_n = (n+1)a_0$ for $n\geq0$. Therefore, the first solution to the differential equation \eqref{diff_eq_1} is
  \begin{align*}
    y_1(x) = a_0 \sum_{n=0}^\infty (n+1)x^{n+1}
  \end{align*}

  In order to find the second solution to the differential equation, we note that
  the roots of the indicial polynomial differ by a positive integer. Since the roots of the indicial polynomial differ by a positive integer,
  there are two linearly independent solutions to the differential equation;
  however, without further analysis, we are unable to tell if the other solution is in the form \eqref{frob}.
  Since $P(\alpha_1) = P(N + \alpha_2) = 0$, we have from the recurrence relations \eqref{rel} that
  \begin{align*}
    P(N+\alpha_2)a_N = 0a_N = -\sum_{k=0}^{N-1}\left[(k+\alpha_2) p_{N-k} + q_{N-k}\right]a_k
  \end{align*}
  which is consistent only if the right hand side is equivalently zero. In our case, $N=1$, $\alpha_2 = 0$
  and we see that the right hand side evaluates to $-q_1a_0 \neq 0$ since $q_1 = -1 \neq 0$ and $a_0 \neq 0$ by assumption. Thus, the second solution to the differential equation is not of the form
  \eqref{frob}. To find the second solution we generalize the Frobenius solution
  and substitute the solution into the original differential equation to obtain
  \begin{align*}
    Ly(x,\alpha) = a_0 x^{\alpha - 2}P(\alpha).
  \end{align*}
  Note that if $\alpha=\alpha_1 = 1$, then the solution is the solution already obtained to the homogeneous equation $Ly(x,\alpha) = 0$.
  Differentiating both sides of the equation with respect to $\alpha$ yields that
  \begin{align}\label{gen_eqn}
    L\left[\left.\frac{\partial}{\partial\alpha}y(x,\alpha)\right|_{\alpha=\alpha_1}\right] = a_0P'(\alpha_1) x^{\alpha_2+N-2}
  \end{align}
  so that $\left.\frac{\partial}{\partial\alpha}y(x,\alpha)\right|_{\alpha=\alpha_1}$ is a particular solution of the above differential equation.
  We can construct a second particular solution and subtract it from this particular solution to obtain the solution to the homogeneous equation. Note that the second particular
  solution has a Frobenius expansion $\sum_{n=0}^{\infty}c_n x^{n+\alpha_2}$.
  Using this form of the solution and substituting into \eqref{gen_eqn} and equating coefficients with $x^{\alpha_2+N-2}$ yields the following
  relations
  \begin{align}\label{sec_rel}
    c_n &=- \frac{\sum_{k=0}^{n-1} [(k + \alpha_2)p_{n-k} + q_{n-k}]c_k}{P(\alpha_2 + n)} &n&\neq0,N \notag\\
    a_0 &= \frac{1}{P'(\alpha_1)}\sum_{k=0}^{N-1}[(k + \alpha_2)p_{N-k} + q_{N-k}]c_k &n&=N
  \end{align}
  where $c_0 \neq 0$ and $c_N\neq 0$ are arbitrary. The second solution to the homogeneous differential equation  $Ly=0$ is then
  \begin{align*}
    y_2(x) &= \sum_{n=0}^\infty c_n x^{n+\alpha_2} - \left.\frac{\partial}{\partial\alpha}y(x,\alpha)\right|_{\alpha=\alpha_1} \\
    &= \sum_{n=0}^\infty c_n x^{n+\alpha_2} - \log x\sum_{n=0}^{\infty}a_n x^{n+\alpha_1} - \sum_{n=0}^\infty\left.\frac{\partial a_n(\alpha)}{\partial \alpha}\right|_{\alpha=\alpha_1} x^{n+\alpha_1}
  \end{align*}
  where $a_n$ is the sequence of coefficients defining the first solution to the homogeneous differential equation.

  For this problem $N=1$, $\alpha_1 = 1$, and $\alpha_2 = 0$. Using the recurrence relations \eqref{sec_rel} we readily see that
  $c_0 = -a_0$, $c_1$ is arbitrary, and that for $n>1$:
  \begin{align*}
    c_n &= -\frac{\sum_{k=0}^{n-1}[kp_{n-k}+q_{n-k}]c_k}{n(n-1)}\\
    &= \frac{\sum_{k=0}^{n-1}(3k+1)c_k}{n(n-1)}
  \end{align*}
  It can be proven through induction that $c_n = n(c_0 + 4c_1)/4 = n(4c_1 - a_0)/4$ completely determining the coefficients of the power series in the second solution.

  We now investigate the partial derivative of $a_n(\alpha)$ with respect to $\alpha$.
  Recall that for this problem the indicial polynomial is given by $P(z) = z(z-1)$.
  Looking, for a closed form representation of the solution to $a_n(\alpha)$ we see that for $a_0(\alpha) = a_0 \neq 0$,
  \begin{align*}
    a_1(\alpha) &= \frac{(3\alpha + 1)a_0}{P(1+\alpha)} &\text{for $n = 1$} \\
    a_n(\alpha) &= \frac{(3\alpha + 1)a_0 \Pi_{k=1}^{n-1} [P(k+\alpha) + 3(k + \alpha) + 1]}{\Pi_{k=1}^n P(k+\alpha)} & \text{for $ n > 1$}
  \end{align*}

  Using this recurrence relation in conjunction with Mathematica we see that the sequence
  \begin{align*}
    \left.\frac{\partial a_n(\alpha)}{\partial \alpha}\right|_{\alpha=\alpha_1} = b_n =
    \begin{cases}
      a_0 &\text{for $n=0$} \\
      -\frac{3a_0}{2} &\text{for $n=1$} \\
      \frac{3a_0(2n^4 +11n^3+18n^2-71n+40)}{2n(n+1)^2(n+2)} &\text{for $n>1$}
    \end{cases}.
  \end{align*}

  Therefore, the solution to the differential equation \eqref{diff_eq_1} is

  \begin{align*}
    y_2(x) &= \sum_{n=0}^\infty\frac{n(4c_1-a_0)}{4}x^n - \log x \sum_{n=0}^\infty a_0(n+1)x^{n+1} - \sum_{n=0}^\infty\left.\frac{\partial a_n(\alpha)}{\partial \alpha}\right|_{\alpha=\alpha_1} x^{n+\alpha_1}\\
    &= c_1\sum_{n=0}^\infty n x^n - a_0 \left[\sum_{n=0}^\infty \frac{n}{4} x^n+ \log x \sum_{n=0}^\infty (n+1)x^{n+1} + \sum_{n=0}^\infty\frac{b_n}{a_0} x ^{n+1}\right].
  \end{align*}
\end{proof}
\newpage


% Problem 5
\begin{problem}
  Find two linearly independent solutions to $x^2y'' + 3xy' + (1-2x)y = 0$ about $x=0$.
\end{problem}

\begin{proof}
  Since $x=0$ is a regular point of the differential equation, we can use the methods outlined in problem 4 and see that the differential equation
  \begin{align*}
    Ly = x^2y'' + 3xy' + (1-2x)y = 0
  \end{align*}
  can be expressed in the form \eqref{form} as
  \begin{align*}
    Ly = y'' + \left[\frac{3}{x}\right]y' + \left[\frac{1-2x}{x^2}\right]y = 0.
  \end{align*}
  We therefore identify the functions $p(x) = 3$ and $q(x) = 1-2x$ both analytic at $x=0$. Thus
  $p_0 = 3$, $p_n = 0$ for $n>0$ and $q_0 = 1$, $q_1 = -2$, $q_n = 0$ for $n>1$. The indicial polynomial
  associated to the differential equation is then
  \begin{align*}
    P(z) = z(z-1)+p_0z+q_0 = (z+1)^2.
  \end{align*}
  The roots of this polynomial are then $\alpha_1=-1$ with multiplicity 2. From this we gather that the recurrence relation \eqref{rel} is
  \begin{align*}
    \left[(n-1)(n-2) + 3(n-1) +1\right]a_n &= -\sum_{k=0}^{n-1}\left[(k-1)p_{n-k} + q_{n-k}\right]a_k = -q_1a_{n-1}
  \end{align*}
  since $p_{n-k} = 0$ for ${n-k}\geq1$ and $q_{n-k}=0$ for ${n-k}\geq 2$. From this recurrence relation
  we see that $a_n = -q_1a_{n-1}$ and
  \begin{align*}
    a_n = \frac{2^n a_0}{(n!)^2}.
  \end{align*}
  Therefore, the first solution to the differential equation $Ly=0$ is
  \begin{align*}
    y_1(x) = \sum_{n=0}^{\infty} a_nx^{n-1} = a_0 \sum_{n=0}^{\infty} \frac{2^n}{(n!)^2}x^{n-1}
  \end{align*}
  which is a Bessel function.

  Since the roots of the indicial polynomial are repeated, the second solution to the differential equation
  is of the form
  \begin{align*}
    y_2(x) &= y_1(x,\alpha_1)\log x + \sum_{n=0}^{\infty}\left.\frac{\partial a_n(\alpha)}{\partial \alpha}\right|_{\alpha=\alpha_1}x^{n-1} \\
    &= a_0\log x\sum_{n=0}^{\infty} \frac{2^n}{(n!)^2}x^{n-1} + \sum_{n=0}^{\infty}\left.\frac{\partial a_n(\alpha)}{\partial \alpha}\right|_{\alpha=\alpha_1}x^{n-1}.
  \end{align*}

  Thus, we need to determine the coefficients of the second series. We do so by parametrizing the recurrence relation \eqref{rel} in terms of $\alpha$.
  This results in the following recurrence relation
  \begin{align*}
    \left[(n+\alpha)(n+\alpha-1) + 3(n+\alpha) +1  \right] a_n(\alpha) &= - \sum_{k=0}^{n-1}\left[(k+\alpha)p_{n-k} + q_{n-k}\right]a_k(\alpha)\implies & &\\
    (n+\alpha +1)^2 a_n(\alpha)&=2a_{n-1}(\alpha) &n &\geq1.
  \end{align*}
  Letting $a_0(\alpha) = a_0$,  we see that
  \begin{align*}
    a_n(\alpha) &= \frac{2^n a_0}{(n+\alpha+1)^2(n+\alpha)^2\dots(\alpha+2)^2} \\
    &=\frac{2^n a_0 \Gamma(\alpha+1)^2}{\Gamma(n+\alpha+1)^2}.
  \end{align*}
  Thus, using $\Gamma(0)=1$ and Mathematica, we have that
  \begin{align*}
    \left.\frac{\partial a_n(\alpha)}{\partial \alpha}\right|_{\alpha=\alpha_1} = \frac{2^{n+1} a_0 \psi(n)}{\Gamma(n)^2}
  \end{align*}
  where $\psi(z) = \Gamma'(z)/\Gamma(z)$. Therefore the second solution to the differential equation $Ly=0$ is
  \begin{align*}
    y_2(x)&=a_0\log x\sum_{n=0}^{\infty} \frac{2^n}{(n!)^2}x^{n-1} + \sum_{n=0}^{\infty}\left.\frac{\partial a_n(\alpha)}{\partial \alpha}\right|_{\alpha=\alpha_1}x^{n-1}\\
    &=a_0\log x\sum_{n=0}^{\infty} \frac{2^n}{(n!)^2}x^{n-1} + a_0\sum_{n=0}^{\infty}\frac{2^{n+1} \psi(n)}{\Gamma(n)^2}x^{n-1}.
  \end{align*}
\end{proof}
\newpage


% Problem 6
\begin{problem}
  Find the leading behavior of both solutions of $x^5y'' - y = 0$ near $x=0$.
\end{problem}

\begin{proof}
  Note that the homogeneous linear differential equation
  \begin{align*}
    Ly = x^5y'' - y = y'' + [0]y' + \left[-\frac{1}{x^5}\right] y = 0
  \end{align*}
  has an irregular singular point at $x=0$. Thus, the previous methods used to obtain a solution may not be used.

  In order to determine the leading behavior of the solutions to $Ly=0$ at the point $x=0$, we must determine the controlling
  factor of these solutions. The controlling factor is usually in the form of an exponential which suggests we try the substitution $y(x) = e^{S(x)}$. Using this substitution in the original
  differential equation yields the differential equation
  \begin{align*}
    S'' + (S')^2 - \frac{1}{x^5} = 0.
  \end{align*}
  Making the simplifying assumption that $S'' \ll (S')^2 $ the above differential equation becomes
  \begin{align*}
    (S')^2 \sim x^{-5}.
  \end{align*}
  Consider the asymptotic relation
  \begin{align*}
    (S')^2 \sim x^{-5} \quad (x\to 0^+) \implies
    S' \sim \pm x^{-5/2} \quad (x\to 0^+)
  \end{align*}
  The solution to this equation is
  \begin{align}\label{control}
    S(x) \sim \mp\frac{2x^{-3/2}}{3} \quad (x\to 0^+).
  \end{align}
  Thus, from this relation we have that
  \begin{align}\label{eq}
    S(x) = \mp\frac{2x^{-3/2}}{3} + C(x) \quad (x\to 0^+)
  \end{align}
  for some function $C(x)$ that satisfies $C(x) \ll \mp x^{-3/2}$ as $x\to 0^+$.
  This solution is only valid if the relation $S'' \ll (S')^2$. Thus, we will check the validity of the controlling factors \eqref{control}
  by using the equation \eqref{eq} and verifying the relation $S'' \ll (S')^2$.

  Suppose first that $S(x) = -\frac{2x^{-3/2}}{3} + C(x),\ C(x) \ll -x^{-3/2}, \ (x\to 0+)$. From this we see that
  \begin{align*}
    S'(x) &= x^{-5/2} + C'(x) \implies (S'(x))^2 = x^{-5} + 2x^{-5/2}C'(x) + (C'(x))^2 \\
    S''(x) &= -\frac{5x^{-7/2}}{2} + C''(x).
  \end{align*}
  If $C(x) \ll -x^{-3/2}$ as $x\to 0^+$, then
  \begin{align*}
    C'(x) \ll x^{-5/2} \quad \text{and} \quad C''(x) \ll x^{-7/2}.
  \end{align*}
  From this it is clear that
  \begin{align*}
    S''(x) = -\frac{5x^{-7/2}}{2}+ C''(x) \ll x^{-5} \quad (x\to 0^+)
  \end{align*}
  since
  \begin{align*}
    \lim_{x\to 0^+} \frac{-\frac{5x^{-7/2}}{2}+ C''(x)}{x^{-5}} &=  \lim_{x\to 0+}-\frac{5x^{3/2}}{2} + x^{3/2}\left(\frac{C''(x)}{x^{-7/2}}\right) = 0.
  \end{align*}
  It is also clear that
  \begin{align*}
    (S'(x))^2 \sim x^{-5}\quad (x\to 0^+)
  \end{align*}
  since
  \begin{align*}
    \lim_{x\to 0^+} \frac{(S'(x))^2}{x^{-5}} = \lim_{x\to 0^+} = 1 + 2\left(\frac{C'(x)}{x^{-5/2}}\right) + \left(\frac{C'(x)}{x^{-5/2}}\right)^2 = 1
  \end{align*}
  Since $(S')^2 \sim x^{-5}$, we have that $S'' \ll x^{-5} \sim (S')^2$ and the solution is valid. Therefore, the leading behavior of the first solution is
  \begin{align*}
    y_1(x) \sim e^{-2(x^{-3/2})/3 + C(x)},\quad C(x) \ll -x^{-3/2} \quad(x\to 0^+).
  \end{align*}

  The leading behavior of the second solution is obtained similarly.
  Suppose that $S(x) = \frac{2x^{-3/2}}{3} + C(x),\ C(x) \ll x^{-3/2}, \ (x\to 0+)$. From this we see that
  \begin{align*}
    S'(x) &= -x^{-5/2} + C'(x) \implies (S'(x))^2 = x^{-5} -2x^{-5/2}C'(x) + (C'(x))^2 \\
    S''(x) &= \frac{5x^{-7/2}}{2} + C''(x).
  \end{align*}
  If $C(x) \ll x^{-3/2}$ as $x\to 0^+$, then
  \begin{align*}
    C'(x) \ll x^{-5/2} \quad \text{and} \quad C''(x) \ll x^{-7/2}.
  \end{align*}
  From this it is clear that
  \begin{align*}
    S''(x) = \frac{5x^{-7/2}}{2}+ C''(x) \ll x^{-5} \quad (x\to 0^+)
  \end{align*}
  since
  \begin{align*}
    \lim_{x\to 0^+} \frac{\frac{5x^{-7/2}}{2}+ C''(x)}{x^{-5}} &=  \lim_{x\to 0+}\frac{5x^{3/2}}{2} + x^{3/2}\left(\frac{C''(x)}{x^{-7/2}}\right) = 0.
  \end{align*}
  It is also clear that
  \begin{align*}
    (S'(x))^2 \sim x^{-5}\quad (x\to 0^+)
  \end{align*}
  since
  \begin{align*}
    \lim_{x\to 0^+} \frac{(S'(x))^2}{x^{-5}} = \lim_{x\to 0^+} = 1 - 2\left(\frac{C'(x)}{x^{-5/2}}\right) + \left(\frac{C'(x)}{x^{-5/2}}\right)^2 = 1
  \end{align*}
  Since $(S')^2 \sim x^{-5}$, we have that $S'' \ll x^{-5} \sim (S')^2$ and the solution is valid. Therefore, the leading behavior of the second solution is
  \begin{align*}
    y_2(x) \sim e^{2(x^{-3/2})/3 + C(x)},\quad C(x) \ll x^{-3/2} \quad(x\to 0^+).
  \end{align*}

\end{proof}
\newpage


% Problem 7
\begin{problem}
  Find the first four terms in the asymptotic series for the solutions of $y'' = e^{-2/x}y$ as $x\to + \infty$.

  \textbf{Hint:} When you are performing the asymptotic analysis to extract the leading
  behavior of the solution as $x\to  + \infty$, you may (and probably want) to replace
  $e^{-2/x}$ with a reasonable simpler approximation.
\end{problem}

\begin{proof}
  Note that the point $+\infty$ is an irregular singular point of the differential equation
  \begin{align*}
    y'' = e^{-2/x}y.
  \end{align*}
  Since for $x\to+\infty$
  \begin{align*}
    e^{-2/x} = 1 - \frac{2}{x} + O\left(\frac{1}{x^2}\right),
  \end{align*}
  we know that $e^{-2/x} \sim 1 - \frac{2}{x}\ (x\to +\infty)$. We thus want to find the leading behavior of the
  asymptotic relation
  \begin{align*}
    y'' \sim \left(1-\frac{2}{x}\right)y \quad (x\to+\infty).
  \end{align*}
  Assuming a solution of the form $y(x) = e^{S(x)}$ where $S'' \ll (S')^2$ as $x\to+\infty$, we receive the asymptotic
  differential equation
  \begin{align*}
    S' \sim \pm \sqrt{\frac{2}{x} -1} \quad (x\to+\infty).
  \end{align*}
  Using Mathematica, the solution to this equation is
  \begin{align*}
    S(x) = \pm x\sqrt{\frac{2}{x} - 1}\mp \tan^{-1}\left(\frac{(x-1)\sqrt{\frac{2}{x} -1 }}{x-2}\right) + C(x)
  \end{align*}
  where $C(x) \ll \pm x\sqrt{\frac{2}{x} - 1}\mp \tan^{-1}\left(\frac{(x-1)\sqrt{\frac{2}{x} -1 }}{x-2}\right)$.
  Thus the leading behavior of the solutions to the equation are
  \begin{align*}
    y(x) &\sim c_1 \exp{\left[ x\sqrt{\frac{2}{x} - 1}\- \tan^{-1}\left(\frac{(x-1)\sqrt{\frac{2}{x} -1 }}{x-2}\right)\right]} = c_1e^{S_1(x)} \\
    y(x) &\sim c_2 \exp{\left[- x\sqrt{\frac{2}{x} - 1}+ \tan^{-1}\left(\frac{(x-1)\sqrt{\frac{2}{x} -1 }}{x-2}\right)\right]} = c_2e^{S_2(x)}
  \end{align*}

  We determine the asymptotic expansion for the solution with leading behavior, say $y(x) \sim c_1e^{S_1(x)}$,
  by substituting
  \begin{align*}
    y(x) = c_1 e^{S_1(x)}w(x)
  \end{align*}
  into the original differential equation.
  This yields the following differential equation
  \begin{align*}
    w'' + 2S_1' w' + [S_1'' + (S_1')^2-1]w = 0.
  \end{align*}
  We desire a solution to this differential equation of the form $w(x) = 1 + \varepsilon(x)$ with $\varepsilon(x) \ll 1$ as $(x\to+\infty)$. Note that $\varepsilon(x)$
  satisfies
  \begin{align*}
    \varepsilon'' + 2S_1' \varepsilon' + [S_1'' + (S_1')^2-1]\varepsilon &= -[S_1'' + (S_1')^2-1] \implies\\
    \varepsilon'' + 2\left[\sqrt{\frac{2}{x}-1}\right] \varepsilon' + \left[\frac{2}{x}-2-\frac{1}{x^2\sqrt{\frac{2}{x}-1}}\right]\varepsilon &= -\left[\frac{2}{x}-2-\frac{1}{x^2\sqrt{\frac{2}{x}-1}}\right]
  \end{align*}
  Making asymptotic assumptions on $\varepsilon$ will reduce the complexity of the
  above differential equation. Assume $\varepsilon'' \ll \left[\sqrt{\frac{2}{x}-1}\right] \varepsilon'$.
  Then, since $\varepsilon \ll 1$, the above equation reduces to
  \begin{align*}
    2\left[\sqrt{\frac{2}{x}-1}\right] \varepsilon' \sim -\left[\frac{2}{x}-2-\frac{1}{x^2\sqrt{\frac{2}{x}-1}}\right] \quad (x\to+\infty)
  \end{align*}
  so that, according to Mathematica, the leading behavior of $\varepsilon$ is
  \begin{align*}
    \varepsilon \sim - x\sqrt{-1 + \frac{2}{x}} - \frac{\log(2 - x)}{4} + \frac{\log(x)}{4} \quad (x\to+\infty).
  \end{align*}
  This process can be repeated to determine more terms in the asymptotic series of the solution.
\end{proof}


\end{document}