\documentclass[12pt]{article}
\usepackage[margin=1in]{geometry}
\usepackage{amsmath, amsthm, amssymb, amsfonts, enumitem, graphicx}
\usepackage{fancyhdr}
\usepackage{breqn}

\theoremstyle{definition}
\newtheorem{problem}{Problem}
\renewcommand*{\proofname}{Solution}
\renewcommand{\theenumi}{\alph{enumi}}

\newcommand{\vect}[1]{\boldsymbol{#1}}

\newenvironment{case}{%
  \let\olditem\item%
  \renewcommand\item[1][]{\olditem \textbf{##1} \\}%
  \begin{enumerate}[label=\textbf{Case \arabic*:},itemindent=*,leftmargin=0em]}{\end{enumerate}%
}

\pagestyle{fancy}
\fancyhf{}
\rhead{Homework Assignment 2}
\lhead{Matthew Tiger}
\cfoot{\thepage}


\title{Homework Assignment 2}
\author{Matthew Tiger}


\begin{document}


\maketitle


% Problem 1
\begin{problem}
  Use the method of variation of parameters to find the general solution of
  \begin{align*}
    y'' + 2y' + 2y = \sin x.
  \end{align*}
\end{problem}

\begin{proof}
  Suppose that $Ly = y'' + 2y' + 2y$. The general solution to $Ly = \sin x$ is
  given by $y = y_0 + y_h$ where $y_0$ is a particular solution of $Ly = \sin x$
  and $y_h$ is the solution to the homogeneous equation $Ly = 0$.

  The characteristic equation of the equation $Ly = 0$ is $m(x) = x^2 + 2x + 2$,
  the roots of which are $m_1 = -1 - i$ and $m_2 = -1 + i$. As the roots of the characteristic equation are complex,
  the solution to $Ly = 0$ is given by
  \begin{align}\label{hom}
    y_h =  c_1e^{-x}\sin x + c_2 e^{-x}\cos x.
  \end{align}

  The method of variation of parameters can be used to find a particular
  solution $y_0$. We wish to find functions $u_1(x), u_2(x)$
  such that
  \begin{align}\label{part}
    y_0 = u_1(x) y_1(x) + u_2(x) y_2(x)
  \end{align}
  satisfies $Ly_0 = \sin x$ where $y_1(x)$ and $y_2(x)$ are solutions to the
  homogeneous equation $Ly = 0$. If the functions
  $u_1(x)$ and $u_2(x)$ are solutions to the system
  \begin{align}\label{system}
    \left\{
      \begin{array}{lllll}
        u_1'y_1 &+& u_2'y_2 &=& 0 \\
        u_1' y_1' &+& u_2' y_2' &=& \sin x
      \end{array}
    \right.
  \end{align}
  then \eqref{part} will satisfy the original differential equation $Ly = \sin x$
  equation. The solution to the system \eqref{system} is
  \begin{align}\label{solution}
    u_1(x) = -\int \frac{y_2(x)\sin x }{W[\{y_1, y_2\}]} dx \qquad u_2(x) = \int \frac{y_1(x)\sin x}{W[\{y_1, y_2\}]} dx
  \end{align}
  where $W[\{y_1, y_2\}]$ is the Wronskian of the functions $y_1$ and $y_2$.

  Using \eqref{hom}, we know that $y_1(x) = e^{-x}\sin x$ and $y_2(x) = e^{-x}\cos x$
  so the particular solution has the form $y_0 = u_1(x)e^{-x}\sin x + u_2(x)e^{-x}\cos x$.
  Further, the Wronskian of $y_1$ and $y_2$ is
  \begin{align*}
    W[\{y_1, y_2\}] =
    \left|
      \begin{array}{ll}
        e^{-x}\sin x & e^{-x}\cos x \\
        e^{-x}\cos x - e^{-x}\sin x & -e^{-x}\cos x - e^{-x}\sin x
      \end{array}
    \right|
    = -e^{-2x}.
  \end{align*}
  Thus, using \eqref{solution}, we know that
  \begin{align*}
    u_1(x) &= -\int \frac{y_2(x)\sin x }{W[\{y_1, y_2\}]} dx \\
    &= \int \frac{e^{-x}\cos x\sin x }{e^{-2x}} dx \\
    &= \frac{e^{x}}{10}\left(-2\cos 2x + \sin 2x\right) + C
  \end{align*}
  and
  \begin{align*}
    u_2(x) &= \int \frac{y_1(x)\sin x }{W[\{y_1, y_2\}]} dx \\
    &= -\int \frac{e^{-x}\sin^2 x }{e^{-2x}} dx \\
    &= \frac{e^{x}}{10}\left(-5 +\cos 2x + 2\sin 2x\right)  + C.
  \end{align*}
  Therefore, a particular solution to $Ly=\sin x$ is
  \begin{align*}
    y_0(x) =
    \frac{1}{10}\left(-2\cos 2x + \sin 2x\right)\sin x +
    \frac{1}{10}\left(-5 +\cos 2x + 2\sin 2x\right)\cos x
  \end{align*}
  and the general solution to $Ly = \sin x$ is
  \begin{align}
    y(x) &= y_0(x) + y_h(x) \notag \\
    &= \frac{1}{10}\left(-2\cos 2x + \sin 2x\right)\sin x +
    \frac{1}{10}\left(-5 +\cos 2x + 2\sin 2x\right)\cos x \notag \\
    &\quad +c_1e^{-x}\sin x + c_2 e^{-x}\cos x
  \end{align}
\end{proof}
\newpage


% Problem 2
\begin{problem}
  Find the Green function of the IVP
  \begin{align*}
    y'' + 2y' + 2y = f(x), \quad y(0) = y'(0) = 0.
  \end{align*}
\end{problem}

\begin{proof}
  Let $Ly = f(x)$ denote the differential equation $y'' + 2y' + 2y = f(x)$ together with the
  initial conditions $y(0) = y'(0) = 0$.
  The Green function $G(x, a)$ of the IVP $Ly = f(x)$ is defined by the equations
  \begin{align*}
    \frac{\partial^2 G(x, a)}{\partial x^2} + \frac{2\partial G(x, a)}{\partial x} + 2G(x, a) = \delta(x- a), \qquad G(0, a) = 0, \qquad \frac{\partial G}{\partial x}(0, a) = 0
  \end{align*}
  where $\delta(x-a)$ is the Dirac Delta function such that
  $\int_{-\infty}^{\infty} \delta(x-a)f(x) dx = f(x)$. Note that $G(x, a)$ is continuous
  at $x=a$ and $\partial G /\partial x$ has a jump discontinuity of magnitude 1 at $x=a$.

  If $y_1$ and $y_2$ are linearly independent solutions of the homogeneous equation $Ly = 0$,
  then
  \begin{align*}
    G(x, a) =
    \begin{cases}
      A_1 y_1 + A_2 y_2 & \text{if $x < a$} \\
      B_1 y_1 + B_2 y_2 & \text{if $x > a$}
    \end{cases}
  \end{align*}
  where $A_1, A_2, B_1$, and $B_2$ are undetermined functions.
  The continuity of $G(x, a)$ at $x = a$ gives the equation
  \begin{align*}
    A_1 y_1(a) + A_2 y_2(a) = B_1 y_1(a) + B_2 y_2(a).
  \end{align*}
  Further, the fact that  $\partial G /\partial x$ has a jump discontinuity of magnitude 1 at $x=a$
  yields the second equation
  \begin{align*}
    (B_1 y_1'(a) + B_2 y_2'(a)) - (A_1 y_1'(a) + A_2 y_2'(a)) = 1.
  \end{align*}
  Combining these equations, we see that $A_1$, $A_2$, $B_1$, and $B_2$ are
  given by
  \begin{align*}
    B_1 &= A_1 - \frac{y_2(a)}{W[y_1(a),y_2(a)]} \\
    B_2 &= A_2 + \frac{y_1(a)}{W[y_1(a),y_2(a)]}
  \end{align*}

  From \eqref{hom}, we know that the linearly independent solutions to the
  homogeneous equation $Ly = 0$ are $y_1(x) = e^{-x}\sin x$ and $y_2(x) = e^{-x}\cos x$.
  Also, the Wronskian of these solutions is $W[y_1(a), y_2(a)] = -e^{-2a}$. Thus,
  \begin{align*}
    B_1 &= A_1 - \frac{y_2(a)}{W[y_1(a),y_2(a)]} = A_1 + e^{a}\cos a \\
    B_2 &= A_2 + \frac{y_1(a)}{W[y_1(a),y_2(a)]} = A_2 - e^{a}\sin a
  \end{align*}
  Using the two initial conditions, we can uniquely determine $A_1$ and $A_2$
  since $G(x, a) = A_1y_1(a) + A_2y_2(a)$ satisfies $LG = f(x)$. Since $y(0) = 0$
  we see that $A_2 = 0$ and since $y'(0) = 0$ we see that $A_1 - A_2 = 0$ implying that
  $A_1 = A_2 = 0$. Therefore, the Green function for the IVP $Ly = f(x)$ is
  \begin{align}\label{green}
    G(x, a) =
    \begin{cases}
      0 & \text{if $x < a$} \\
      e^{a-x}\left(\sin x\cos a  - \cos x \sin a \right) = e^{a-x}\sin(x-a) & \text{if $x > a$}
    \end{cases}
  \end{align}
\end{proof}
\newpage


% Problem 3
\begin{problem}
  Use your answer to Problem 2 to solve the IVP
  \begin{align*}
    y'' + 2y' + 2y = \sin x, \quad y(0) = y'(0) = 0.
  \end{align*}
\end{problem}

\begin{proof}
  If we can find the Green function $G(x,a)$ associated to the IVP we know that
  the particular solution to the IVP can be represented as
  \begin{align*}
    y_p(x) &= \int_{-\infty}^{\infty}G(x,a)\sin(a) da.
  \end{align*}
  Note that in problem 2, the Green function \eqref{green} is precisely the function
  we are after. For that Green function, we know that $G(x,a) = 0$ if $x>a$ so that
  \begin{align*}
    y_p(x) = \int_{-\infty}^{\infty}G(x,a)\sin(a)da = \int_{-\infty}^{x}G(x,a)\sin(a)da.
  \end{align*}
  Using the expression for the Green function found in \eqref{green}, we see that
  \begin{align*}
    y_p(x) &= \int_{-\infty}^{x}G(x,a)\sin(a)da \\
    &= \int_{-\infty}^{x} e^{a-x}\sin(x-a)\sin(a)da\\
    &= \frac{-2\cos x + \sin x}{5}.
  \end{align*}
  Since the homogeneous solution is equivalently zero under these initial values,
  the solution to the differential equation is
  \begin{align*}
    y(x) = y_h(x) + y_p(x) = \frac{-2\cos x + \sin x}{5}.
  \end{align*}

\end{proof}
\newpage


% Problem 4
\begin{problem}
  Show that if $y_1$, $y_2$, and $y_3$ are three linearly independent solutions
  of the linear ODE
  \begin{align*}
    y''' + p_2(x)y'' + p_1(x)y' + p_0(x) y = 0
  \end{align*}
  and $u_1$, $u_2$, $u_3$ are solutions of the system
  \begin{align}\label{u_system}
    \left\{
      \begin{array}{lllllll}
        u_1'y_1 &+& u_2'y_2 &+& u_3'y_3 &=& 0, \\
        u_1'y_1' &+& u_2'y_2' &+& u_3'y_3' &=& 0, \\
        u_1'y_1'' &+& u_2'y_2'' &+& u_3'y_3'' &=& f(x), \\
      \end{array}
    \right.
  \end{align}
  then the function $u = u_1 y_1 + u_2 y_2 + u_3 y_3$ is a solution of
  \begin{align*}
    Ly = y''' + p_2(x)y'' + p_1(x)y' + p_0(x) y = f(x)
  \end{align*}
\end{problem}

\begin{proof}
  We wish to show that $y = \sum_{i=1}^3 u_i y_i$ is a solution of the equation
  $Ly = f(x)$ given that $y_i$ are linearly independent solutions of the homogeneous
  equation $Ly = 0$ and $u_i$ are solutions of the system \eqref{u_system}.
  Using the form $y = \sum_{i=1}^3 u_i y_i$, we see that
  \begin{align*}
    y' &= \sum_{i=1}^3 u_i y_i' + u_i' y_i \\
    y'' &= \sum_{i=1}^3 u_i y_i'' + 2u_i'y_i' + u_i'' y_i \\
    y''' &= \sum_{i=1}^3 u_i y_i''' + 3u_i'y_i'' + 3u_i''y_i' + u_i''' y_i.
  \end{align*}
  Thus, we find that for $y = \sum_{i=1}^3 u_i y_i$,
  \begin{align*}
    Ly &= \sum_{i=1}^3 u_i y_i''' + 3u_i'y_i'' + 3u_i''y_i' + u_i''' y_i
    + p_2(x)\sum_{i=1}^3 u_i y_i'' + 2u_i'y_i' + u_i'' y_i \\
    &\quad + p_1(x)\sum_{i=1}^3 u_i y_i' + u_i' y_i
    + p_0(x)\sum_{i=1}^3 u_i y_i \\
    &= \sum_{i=1}^3 u_i[y_i''' + p_2(x)u_i'' + p_1(x)y_i' + p_0(x)y_i] \\
    &\quad + \sum_{i=1}^3 3u_i'y_i'' + 3u_i''y_i' + u_i'''y_i + 2p_2(x)u_i'y_i' + p_2(x)u_i''y_i + p_1(x)u_i'y_i.
  \end{align*}
  Since $y_i$ are solutions of the homogeneous equation $Ly = 0$, we see that the first sum is 0 and
  \begin{align}\label{pro}
    Ly = \sum_{i=1}^3 3u_i'y_i'' + 3u_i''y_i' + u_i'''y_i + 2p_2(x)u_i'y_i' + p_2(x)u_i''y_i + p_1(x)u_i'y_i.
  \end{align}
  We also know that since $u_1$, $u_2$, and $u_3$ are solutions of the system \eqref{u_system} the following implications are true
  \begin{align*}
    \sum_{i=1}^3 u_i'y_i = 0 &\implies \left[\sum_{i=1}^3 u_i'y_i\right]' = \sum_{i=1}^3 u_i''y_i + u_i'y_i' = 0 \\
    \sum_{i=1}^3 u_i''y_i + u_i'y_i' = 0 &\implies \left[\sum_{i=1}^3 u_i''y_i + u_i'y_i'\right]' = \sum_{i=1}^3 u_i'''y_i + 2u_i''y_i' + u_i'y_i'' = 0 \\
    \sum_{i=1}^3 u_i'y_i' = 0 &\implies \left[\sum_{i=1}^3 u_i'y_i'\right]' = \sum_{i=1}^3 u_i''y_i' + u_i'y_i'' = 0
  \end{align*}
  Rearranging the terms of \eqref{pro} and using the above relations we see that
  \begin{align*}
    Ly &= \sum_{i=1}^3 u_i' y_i'' + \left[\sum_{i=1}^{3}u_i'''y_i + 2u_i''y_i' + u_i'y_i''\right] + \left[\sum_{i=1}^{3}u_i'y_i'' + u_i''y_i'\right] \\
    &\quad + p_2(x)\left[\sum_{i=1}^{3}u_i'y_i' + u_i''y_i\right] + p_2(x)\left[\sum_{i=1}^{3}u_i'y_i'\right] + p_1(x)\left[\sum_{i=1}^{3}u_i'y_i\right] \\
    &= \sum_{i=1}^3 u_i' y_i''
  \end{align*}
  where every term in brackets is 0 as a consequence of the above derived relations or
  the fact that $u_1$, $u_2$, and $u_3$ are solutions of the system \eqref{u_system}.
  From the third equation of the system \eqref{u_system} we know that $\sum_{i=1}^3 u_i' y_i'' = f(x)$.
  Therefore, for $y = \sum_{i=1}^{3} u_i y_i$ satisfying the assumptions of the problem,
  \begin{align*}
    Ly = \sum_{i=1}^3 u_i' y_i'' = f(x)
  \end{align*}
  showing that $y$ is a solution of the equation $Ly = f(x)$.
\end{proof}
\newpage


% Problem 5
\begin{problem}
  Find the eigenvalues and the respective eigenfunctions for the BVP
  \begin{align*}
    x^2y'' + xy' + \lambda y = 0, \quad y'(1) = 0, \quad y'(b) = 0
  \end{align*}
  where $b > 1$.
\end{problem}

\begin{proof}
  The differential equation stated in this problem is an Euler differential
  equation. The equation can be transformed into a constant coefficient second
  order linear differential equation by making the substitution $x(t) = e^t$
  and rewriting the differential equation in terms of the independent variable
  $t$.

  To see this, we note that
  \begin{align*}
    \frac{d}{dt}\left[y(x(t))\right]
    &= \frac{dy(x(t))}{dx}\frac{dx(t)}{dt} \\
    &= \frac{dy(x(t))}{dx}\frac{dx(t)}{dt} \\
    &= y'(x(t)) x(t)
  \end{align*}
  since $x'(t) = [e^t]' = e^t = x(t)$. Similarly, using the above relation,
  \begin{align*}
    \frac{d^2}{dt^2}[y(x(t))]
    &= \frac{d}{dt}\left[\frac{dy(x(t))}{dt}\right] \\
    &= \frac{d}{dt}\left[\frac{dy(x(t))}{dx}\right] x(t) + \frac{dy(x(t))}{dx}\frac{d}{dt}\left[x(t)\right] \\
    &= \left[\frac{dy(x(t))}{dx}\frac{dx(t)}{dt}\right] x(t) + \left[\frac{dy(x(t))}{dx}\right] x(t) \\
    &= x(t)^2 \frac{d^2y(x(t))}{dx^2} + x(t)\frac{dy(x(t))}{dx} \\
    &= x(t)^2 y''(x(t)) + x(t)y'(x(t)).
  \end{align*}
  Thus, the original differential equation in the independent variable $x$ can
  be written as the following differential equation in the independent variable $t$
  after making the change of variables $x(t) = e^t$:
  \begin{align}\label{second}
    \left[x^2y''(x) + xy'(x)\right] + \lambda y(x) = \left[y''(x(t))\right] + \lambda y(x(t)) = 0.
  \end{align}

  The characteristic equation of the homogeneous second order linear differential
  equation in the variable $t$ is given by
  \begin{align}\label{char}
    m(z) = z^2 + \lambda.
  \end{align}
  The roots of $m(z)$ are $z_1=\sqrt{-\lambda}$ and $z_2=-\sqrt{-\lambda}$.
  The solution to \eqref{second} is thus dependent on the value of $\lambda$
  and as such there are three cases to consider, when $\lambda < 0$, $\lambda = 0$,
  and $\lambda > 0$.
  \rule{\textwidth}{1pt}

  \begin{case}
    \item[$\lambda < 0$] If $\lambda < 0$, then $\sqrt{-\lambda}$ is a positive real number
      and the roots of the characteristic equation \eqref{char} are real and distinct.
      Thus, the solution to \eqref{second} is
      \begin{align*}
        y(t) = c_1 e^{\sqrt{-\lambda}t} + c_2 e^{-\sqrt{-\lambda}t}.
      \end{align*}
      Using the substitution $t = \log x$, the solution to the differential
      equation with respect to $x$ becomes
      \begin{align*}
        y(t(x)) = y(x)
        &= c_1 e^{\sqrt{-\lambda}\log x} + c_2 e^{-\sqrt{-\lambda}\log x} \\
        &= c_1 x^{\sqrt{-\lambda}} + c_2 x^{-\sqrt{-\lambda}}
      \end{align*}
      For this solution, we see that
      \begin{align*}
        y'(x) = c_1\sqrt{-\lambda} x^{\sqrt{-\lambda} - 1} - c_2\sqrt{-\lambda} x^{-\sqrt{-\lambda} - 1}
      \end{align*}

      In this case, the initial condition $y'(1) = 0$ shows that
      \begin{align*}
        y'(1) = c_1 \sqrt{-\lambda} - c_2 \sqrt{-\lambda} = \sqrt{-\lambda}(c_1 - c_2) = 0.
      \end{align*}
      Since $\lambda < 0$, we know that $\sqrt{-\lambda} \neq 0$ and so $c_1 - c_2 = 0$
      or that $c_1 = c_2$.

      The initial condition $y'(b) = 0$ for $b > 1$ together with the fact that
      $c_1 = c_2$ shows that
      \begin{align*}
        y'(b) &= c_1\sqrt{-\lambda} b^{\sqrt{-\lambda} - 1} - c_2\sqrt{-\lambda} b^{-\sqrt{-\lambda} - 1} \\
        &= c_1\sqrt{-\lambda} \left(b^{\sqrt{-\lambda} - 1} - b^{-\sqrt{-\lambda} - 1}\right) = 0
      \end{align*}
      showing that since  $\lambda < 0$ we must have that $c_1 = 0$ since $\sqrt{-\lambda}\neq 0$
      and $b^{\sqrt{-\lambda} - 1} \neq b^{-\sqrt{-\lambda} - 1}$. Therefore, for $\lambda < 0$,
      the only solution to the differential equation is the trivial solution and
      in this case there are no eigenvalues of this equation.
    \item[$\lambda = 0$]
      If $\lambda = 0$, then the root of the characteristic equation \eqref{char}
      is $z = 0$ with multiplicity 2. As this is a repeated root, the solution to
      \eqref{second} is
      \begin{align*}
        y(t) = c_1 + c_2 t.
      \end{align*}
      Making the substitution $t = \log x$, we see that
      \begin{align*}
        y(t(x)) = y(x) = c_1 + c_2 \log x.
      \end{align*}
      In this case, we see that $y'(x) = c_2x^{-1}$. Using the initial condition
      that $y'(1) = 0$, we see that $c_2 = 0$. Similarly, the condition $y'(b) = 0$ for $b > 1$
      yields the same result. Thus, $c_1$ is free
      and we see that $y(x) = c_1$ is a non-trivial solution to this problem. Therefore,
      $\lambda_0 = 0$ is an eigenvalue of this differential equation with associated
      eigenfunction $y_{\lambda_0}(x) = 1$.
    \item[$\lambda > 0$]
      If $\lambda > 0$, then the roots to the characteristic equation \eqref{char}
      are $z_1 = i \sqrt{\lambda}$ and $z_2 = -i \sqrt{\lambda}$ which are complex roots.
      Thus, the solution to \eqref{second} is
      \begin{align*}
        y(t) = c_1\cos\left(t\sqrt{\lambda}\right) + c_2\sin\left(t\sqrt{\lambda}\right).
      \end{align*}
      Making the substitution $t = \log x$, we see that
      \begin{align*}
        y(t(x)) = y(x) = c_1\cos\left(\sqrt{\lambda}\log x\right) + c_2\sin\left(\sqrt{\lambda}\log x\right).
      \end{align*}
      In this case, we see that
      \begin{align*}
        y'(x) = -c_1x^{-1}\sqrt{\lambda}\sin\left(\sqrt{\lambda}\log x\right) + c_2x^{-1}\sqrt{\lambda}\cos\left(\sqrt{\lambda}\log x\right).
      \end{align*}
      The initial condition $y'(1) = 0$ shows that
      \begin{align*}
        y'(x) = -c_1\sqrt{\lambda}\sin\left(0\right) + c_2\sqrt{\lambda}\cos\left(0\right) = c_2\sqrt{\lambda} = 0.
      \end{align*}
      Since $\sqrt{\lambda} > 0$, we must have that $c_2 = 0$. This fact, combined
      with $y'(b) = 0$ for $b>1$, shows that
      \begin{align*}
        y'(b) = -c_1\sqrt{\lambda}\sin\left(\sqrt{\lambda}\log b\right) = 0 \implies c_1\sin\left(\sqrt{\lambda}\log b\right) = 0.
      \end{align*}
      So either $c_1 = 0$, which leads to the trivial solution, or $\sqrt{\lambda}\log b = n\pi$ for $n=1,2,\dots$. Since $\lambda > 0$ no
      other values of $n$ will yield $\sin\left(\sqrt{\lambda}\log b\right) = 0$. Thus,
      \begin{align*}
        \lambda_n = \left(\frac{n\pi}{\log b}\right)^2 \qquad \text{for $n=1,2,\dots$}
      \end{align*}
      are eigenvalues associated to this problem with associated eigenfunctions
      \begin{align*}
        y_{\lambda_n}(x) = \cos\left(\frac{n\pi\log x}{\log b}\right) \qquad \text{for $b>1$ and $n=1,2,\dots$}.
      \end{align*}
  \end{case}
  \rule{\textwidth}{1pt}

  We have therefore exhausted all cases and found all eigenvalues associated to
  the differential equation $x^2y''(x) + xy'(x) + \lambda y(x) = 0$ along with
  their eigenfunctions.
\end{proof}


\end{document}