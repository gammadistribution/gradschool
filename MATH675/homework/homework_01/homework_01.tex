\documentclass[12pt]{article}
\usepackage[margin=1in]{geometry}
\usepackage{amsmath, amsthm, amssymb, amsfonts, enumitem, graphicx}

\theoremstyle{definition}
\newtheorem{problem}{Problem}
\renewcommand*{\proofname}{Solution}
\renewcommand{\theenumi}{\alph{enumi}}

\newcommand{\vect}[1]{\boldsymbol{#1}}

\title{Homework Assignment 1}
\author{Matthew Tiger}


\begin{document}


\maketitle


% Problem 1
\begin{problem}
  Solve the IVP:
  \begin{align*}
    y' = y^2\cos(x),\quad y(0)=2.
  \end{align*}
\end{problem}

\begin{proof}
  Note that this is a separable differential equation and after separating we
  see that
  \begin{align*}
    \frac{dy}{y^2} &= \cos(x)dx \\
    \int \frac{dy}{y^2} &= \int \cos(x)dx \\
    -\frac{1}{y} &= \sin(x) + c_1
  \end{align*}
  so that $$y = -\frac{1}{\sin(x) + c_1}$$ is the general solution to the differential
  equation. Using the initial value
  $y(0) = 2$ and solving for $c_1$ we see that $c_1 = -1/2$ and the solution to the IVP
  is given by
  \begin{align*}
    y = -\frac{1}{\sin(x) - 1/2}.
  \end{align*}
\end{proof}
\newpage


% Problem 2
\begin{problem}
  Review solutions of first-order linear ODEs (p. 14) and solve the IVP:
  \begin{align*}
    y' - xy = x^3,\quad y(1) = \frac{1}{2}.
  \end{align*}
\end{problem}

\begin{proof}
  The solution to the first-order linear ODE
  \begin{align*}
    y'(x) + p_0(x)y(x) = f(x)
  \end{align*}
  is given by
  \begin{align*}
    y(x) = \frac{c_1}{I(x)} + \frac{1}{I(x)} \int_0^x f(t)I(t)dt, \quad I(x) = \exp{\left(\int_0^x p_0(t) dt\right)}.
  \end{align*}
  For this problem, we set $p_0(x) = -x$ and $f(x) = x^3$ and see that
  \begin{align*}
    I(x) = \exp{\left(\int_0^x p_0(t) dt\right)} = \exp{\left(\int_0^x -t dt \right)} = \exp{\left(-\frac{x^2}{2}\right)}.
  \end{align*}
  Thus the general solution to the ODE $y' - xy = x^3$ is given by
  \begin{align*}
    y &= \frac{c_1}{\exp{\left(-\frac{x^2}{2}\right)}} + \frac{1}{\exp{\left(-\frac{x^2}{2}\right)}}\int_0^x t^3 \exp{\left(-\frac{t^2}{2}\right)}dt \\
    &= \frac{c_1}{\exp{\left(-\frac{x^2}{2}\right)}} - \frac{\exp{\left(-\frac{x^2}{2}\right)}}{\exp{\left(-\frac{x^2}{2}\right)}}(2 + x^2) \\
    &= \frac{c_1}{\exp{\left(-\frac{x^2}{2}\right)}} - (2+x^2)
  \end{align*}
  Using the initial value $y(1) = \frac{1}{2}$, we see that $c_1 = \frac{7}{2}\exp{\left(-\frac{1}{2}\right)}$
  and the solution to the IVP is
  \begin{align*}
    y = \frac{7\exp{\left(-\frac{1}{2}\right)}}{2\exp{\left(-\frac{x^2}{2}\right)}} - (2 + x^2).
  \end{align*}
\end{proof}
\newpage


% Problem 3
\begin{problem}
  Let $Ly = y^{(4)} - 4 y''' + 3y'' +4y' -4y$.
  \begin{enumerate}
    \item Find the general solutions of the homogeneous ODE $Ly = 0$.
    \item Solve the IVP:
      \begin{align*}
        Ly = 0,\quad y(0) = 0, \quad y'(0)= -7, \quad y''(0) = 5, \quad y'''(0) = 9.
      \end{align*}
    \item Solve the BVP:
      \begin{align*}
        Ly = 0,\quad y(0) = 1, \quad \lim_{x\to\infty}y(x) = 0.
      \end{align*}
      Is this BVP well-posed?
    \item Solve the BVP:
      \begin{align*}
        Ly = 0,\quad y(0) = 1, \quad \lim_{x\to-\infty}y(x) = 0.
      \end{align*}
      Is this BVP well-posed?
  \end{enumerate}
\end{problem}

\begin{proof}
  \begin{enumerate}
    \item The characteristic equation associated to the homogeneous ODE $Ly=0$
      is $m(x) = x^4 - 4 x^3 + 3x^2 + 4x - 4$. The roots of the characteristic
      polynomial are $r_1 = -1$, $r_2=1$, $r_3=2$, and $r_4=2$.

      Therefore, the general solution of the homogeneous ODE is
      \begin{align}\label{homo_ode}
        y(x) = c_1 e^{-x} + c_2 e^{x} + c_3e^{2x} +c_4xe^{2x}.
      \end{align}
    \item Through an abuse of notation, we note that the matrix associated to the Wronskian
      of this equation as function of x is given by
      \begin{align*}
        W(x) =
        \begin{bmatrix}
          e^{-x} & e^{x} & e^{2x} & xe^{2x} \\
          -e^{-x} & e^{x} & 2e^{2x} & e^{2x} + 2xe^{2x} \\
          e^{-x} & e^{x} & 4e^{2x} & 4e^{2x} + 4xe^{2x} \\
          -e^{-x} & e^{x} & 8e^{2x} & 12e^{2x} + 8xe^{2x} \\
        \end{bmatrix}.
      \end{align*}
      The solution to the IVP is determined by particular values of the coefficients in the
      general solution \eqref{homo_ode}. These coefficients are found as the
      solution to the system of equations
      $W(0)\vect{c} = \vect{b}$ where
      \begin{align*}
        \vect{c} = \begin{bmatrix}c_1 \\ c_2 \\ c_3 \\ c_4 \end{bmatrix}\quad\text{and}\quad
        \vect{b} = \begin{bmatrix} 0 \\ -7 \\ 5 \\ 9 \end{bmatrix}.
      \end{align*}
      The solution to this system is given by $\vect{c} = \langle 4,-3,-1,2 \rangle$.
      Therefore, the solution to the IVP is
      \begin{align*}
        y(x) = 4 e^{-x} -3e^{x} -e^{2x} +2xe^{2x}.
      \end{align*}
    \item The general solution to the ODE, $y(x)$, is given by \eqref{homo_ode}.
      The second condition that $\lim_{x\to\infty} y(x) = 0$ can not be satisfied
      by the general solution since $\lim_{x\to\infty} e^{ax} = \infty$ for $a>0$. Therefore,
      the BVP as stated is not well-posed.
    \item The general solution to the ODE, $y(x)$, is given by \eqref{homo_ode}.
      The second condition that $\lim_{x\to-\infty} y(x) = 0$ can not be satisfied
      by the general solution since $\lim_{x\to-\infty} e^{ax} = \infty$ for $a<0$. Therefore,
      the BVP as stated is not well-posed.
  \end{enumerate}
\end{proof}
\newpage


% Problem 4
\begin{problem}
  Read \S1.6 and then solve the ODEs:
  $$xy' + 2y = x^2\sqrt{y}, \quad y' = \frac{4x^3-6xy^2-2xy}{x^2+6x^2y-3y^2}, \quad y' + y^2 +(2x+1)y + 1 + x + x^2 = 0.$$
\end{problem}

\begin{proof}
  We begin with the differential equation
  \begin{align*}
    xy' + 2y = x^2\sqrt{y}.
  \end{align*}
  Note that this equation can be rewritten as
  \begin{align}\label{bernoulli}
    y' = \left(-\frac{2}{x}\right) y + x y^{1/2},
  \end{align}
  which is a Bernoulli equation with $P=1/2$. Dividing \eqref{bernoulli} by $y^{1/2}$
  and making the substitution $u(x) = y(x)^{1-1/2}$ yields the new linear differential
  equation
  \begin{align*}
    u'(x) = -\left(\frac{1}{x}\right)u(x) + \frac{x}{2}.
  \end{align*}
  The solution to this linear equation is $u(x) = x^2/6 + c_1/x$ suggesting that
  \begin{align*}
    y(x)=u(x)^2 = \left(\frac{x^2}{6} + \frac{c_1}{x}\right)^2
  \end{align*}
  is the solution to \eqref{bernoulli}.

  Let us next investigate
  \begin{align*}
    y' = \frac{4x^3-6xy^2-2xy}{x^2+6x^2y-3y^2}.
  \end{align*}
  Note that this equation can be written as
  \begin{align*}
   -(4x^3-6xy^2-2xy) +(x^2+6x^2y-3y^2)y'(x) = 0.
  \end{align*}
  Identifying $M(x,y) = -(4x^3-6xy^2-2xy)$ and $N(x,y)=(x^2+6x^2y-3y^2)$, we notice
  that
  \begin{align*}
    \frac{\partial M(x,y)}{\partial y} = 12xy + 2x = \frac{\partial N(x,y)}{\partial x}
  \end{align*}
  making this equation exact.
  The solution to the exact differential equation is then $f(x,y) = c_1$ where
  $f_x = M(x,y)$ and $f_y = N(x,y)$. Thus,
  \begin{align}\label{exact}
    f(x,y) = \int f_x(x,y) dx = -\int (4x^3-6xy^2-2xy) dx = -x^4 +3x^2y^2 +x^2y + h(y).
  \end{align}
  In order to find out what $h(y)$ is, we take the partial derivative of
  \eqref{exact} and compare it with $N(x,y)$. Doing so, we see that
  \begin{align*}
    f_y(x,y) = x^2+6x^2y+h'(y) = x^2+6x^2y-3y^2 = N(x,y)
  \end{align*}
  implying that $h'(y) = -3y^2$ and that $h(y) = -y^3$. Therefore,
  the solution to the differential equation is
  \begin{align*}
    f(x,y) = -x^4 +3x^2y^2 +x^2y -y^3 = c_1.
  \end{align*}

  Finally let us investigate the differential equation
  \begin{align*}
    y' + y^2 +(2x+1)y + 1 + x + x^2 = 0.
  \end{align*}
  This equation can be rewritten as
  \begin{align}\label{ricatti}
    y' = -y^2 -(2x+1)y - (1 + x + x^2)
  \end{align}
  which is a Riccati equation. The procedure to find the solution of such
  equations is to produce a particular solution $y_p(x)$ to the equation and then find the
  general solution which will be in the form $y(x) = y_p(x) + u(x)$ by using this formula in
  the original equation. Note that $y_p(x) = -x$ is a particular solution
  of \eqref{ricatti}. Thus the general solution is of the form $y(x) = -x + u(x)$.

  Making this substitution reveals the following Bernoulli equation in $u(x)$:
  \begin{align*}
    u'(x) = -u(x) -u(x)^2
  \end{align*}
  The solution to this differential equation is $u(x) = -(e^{c_1}/(-e^x + e^{c_1}))$.
  Therefore, the general solution to \eqref{ricatti} is
  \begin{align*}
    y(x)=-x-\frac{e^{c_1}}{-e^x + e^{c_1}}.
  \end{align*}
\end{proof}
\newpage


% Problem 5
\begin{problem}
  \begin{enumerate}
    \item Use mathematical induction to prove Leibnitz's differentiation rule:
      \begin{align*}
        D^{k}(fg) = \sum_{j=0}^k \binom{k}{j}(D^j f) (D^{k-j}g).
      \end{align*}
      Here $f = f(x)$ and $g=g(x)$ are $k$-times differentiable functions and $D^k = \frac{d^k}{dx^k}$.
    \item Consider the constant-coefficient ODE
      \begin{align}\label{leibnitz_ode}
        Ly = D^n y + p_{n-1}D^{n-1}y + \dots + p_1 Dy + p_0 y = 0,
      \end{align}
      where $p_0, p_1, \dots, p_{n-1}$ are real numbers. Let $r$ be a double root of the characteristic polynomial
      $P(z) = z^n + p_{n-1} z^{n-1} + \dots + p_1 z + p_0.$ Use Leibnitz’'s rule to
      show that the function $xe^{rx}$ is a solution of \eqref{leibnitz_ode}.
    \item Let $r$ be a triple root of the characteristic polynomial $P(z)$ from part (b).
      Use Leibnitz'’s rule to show that the function $x^2 e^{rx}$
      is then also a solution of \eqref{leibnitz_ode}.
    \item Let $r$ be a real number. Show that the functions $e^{rx}$, $xe^{rx}$, and $x^2 e^{rx}$
      are linearly independent on $\mathbb{R}$.
  \end{enumerate}
\end{problem}

\begin{proof}
  \begin{enumerate}
    \item Suppose that $k=1$. Then our formula yields
      \begin{align*}
        D(fg) &= \sum_{j=0}^1 \binom{1}{j}(D^j f) (D^{1-j}g) = fD(g) + D(f)g,
      \end{align*}
      which is the product rule for derivatives and the base case is established.

      Now suppose the formula holds for $k=n$. Then, using the linear properties of the derivative, we see that
      \begin{align}\label{induction}
        D^{n+1}(fg) = D(D^{n}(fg)) &= D\left(\sum_{j=0}^n \binom{n}{j}(D^j f) (D^{n-j}g)\right) \notag\\
        &=\sum_{j=0}^n \binom{n}{j}D((D^j f) (D^{n-j}g))
      \end{align}
      Using the product rule, we note that
      \begin{align*}
        D((D^j f) (D^{n-j}g)) = (D^j f) (D^{(n + 1)-j}g) + (D ^{j+1}f)(D^{n-j}g)
      \end{align*}
      and replacing in \eqref{induction} we have that
      \begin{align}\label{pascal}
        \sum_{j=0}^n \binom{n}{j}D((D^j f) (D^{n-j}g)) \notag
        &= \sum_{j=0}^n \binom{n}{j} \left[(D^j f) (D^{(n + 1)-j}g) + (D ^{j+1}f)(D^{n-j}g)\right] \notag\\
        &= \sum_{j=0}^n \binom{n}{j} (D^j f) (D^{(n + 1)-j}g) + \sum_{j=0}^n \binom{n}{j} (D ^{j+1}f)(D^{n-j}g) \notag\\
        &= \sum_{j=0}^n \binom{n}{j} (D^j f) (D^{(n + 1)-j}g) + \sum_{j=1}^{n+1} \binom{n}{j-1} (D ^{j}f)(D^{(n+1)-j}g).
      \end{align}
      Combining terms along with Pascal's rule allows us to combine the binomial coefficients in \eqref{pascal} and thus
      \begin{align*}
        D^{n+1}(fg) &= (D^0f) (D^{n+1}g) + \sum_{j=1}^{n} \binom{n+1}{j}(D^j f) (D^{(n + 1)-j}g) + (D^{n+1}f) (D^0g) \\
        &= \sum_{j=0}^{n+1} \binom{n+1}{j}(D^j f) (D^{k-j}g).
      \end{align*}
      Therefore, the formula holds for $k=n+1$ and the rule holds.
    \item We wish to see if $y(x) = x e^{rx}$ is a solution of \eqref{leibnitz_ode} given
      that $r$ is a double root of the characteristic polynomial. Using Leibnitz's formula, note that
      for $k > 0$
      \begin{align}\label{roots}
        D^k y(x) = D^k(xe^{rx}) &= \sum_{j=0}^k \binom{k}{j} (D^{j}x)(D^{k-j}e^{rx}) \notag \\
        &= x D^{k} e^{rx} + k D^{k-1}e^{rx} \notag \\
        &= x r^k e^{rx} + k r^{k-1}e^{rx} = e^{rx} (xr^k + kr^{k-1})
      \end{align}
      since $D^j x = 0$ if $j > 1$. Using the formula in \eqref{roots}
      and replacing into the ODE, we see that for $y(x) = xe^{rx}$
      \begin{align*}
        Ly(x) &= e^{rx}(xr^n + nr^{n-1}) + p_{n-1}e^{rx}(xr^{n-1} + (n-1)r^{n-2}) + \dots + p_1 e^{rx}(xr+1) + p_0 xe^{rx} \\
        &= e^{rx}\left[(xr^n + nr^{n-1}) + p_{n-1}(xr^{n-1} + (n-1)r^{n-2}) + \dots + p_1 (xr+1) + p_0 x\right] \\
        &= e^{rx}\left[x(r^n + p_{n-1}r^{n-1} +\dots +p_1r + p_0) + (nr^{n-1} + p_{n-1}(n-1)r^{n-2} + \dots + p_1)\right] \\
        &= e^{rx}\left[P(r) + P'(r)\right].
      \end{align*}
      Since $r$ is a root with multiplicity 2 of the polynomial $P(x)$, we
      know that $P(x) = (x-r)^2q(x)$   and $P'(x) = 2(x-r)q(x) + (x-r)^2q'(x)$
      where the degree of $q(x)$ is $n-2$.
      This shows that $P(r) = P'(r) = 0$ and that $Ly(x) = 0$ for $y(x) = xe^{rx}$, i.e.\
      $y(x)$ is a solution of the differential equation.
    \item We wish to similarly see if $y(x) = x^2 e^{rx}$ is a solution of \eqref{leibnitz_ode} given
      that $r$ is a triple root of the characteristic polynomial. Using Leibnitz's formula, note that
      for $k > 0$
      \begin{align}\label{roots_2}
        D^k y(x) = D^k(x^2e^{rx}) &= \sum_{j=0}^k \binom{k}{j} (D^{j}x^2)(D^{k-j}e^{rx}) \notag \\
        &= x^2 D^{k} e^{rx} + 2kx D^{k-1}e^{rx} + k(k-1)D^{k-2}e^{rx}  \notag \\
        &= x^2 r^k e^{rx} + 2kx r^{k-1}e^{rx} + k(k-1) r^{k-2} e^{rx} \notag \\
        &= e^{rx} \left(x^2r^k + 2kxr^{k-1} + k(k-1) r^{k-2}\right)
      \end{align}
      since $D^j x = 0$ if $j > 2$. Using the formula in \eqref{roots_2}
      and replacing into the ODE, we see that for $y(x) = x^2e^{rx}$
      \begin{align*}
        Ly(x) &= e^{rx}\left(x^2r^n + 2nxr^{n-1} + n(n-1)r^{n-2}\right) +\\
        &\quad+ p_{n-1}e^{rx}\left(x^2r^{n-1} + 2(n-1)xr^{n-2} + (n-1)(n-2)r^{n-3}\right) + \dots +\\
        &\quad+ p_1 e^{rx}(x^2r+2x) + p_0 x^2e^{rx} \\
        &= e^{rx}x^2(r^n + p_{n-1}r^{n-1} +\dots + p_0) + \\
        &\quad + e^{rx}2x(nr^{n-1} + p_{n-1}(n-1)r^{n-2} + \dots + p_1) + \\
        &\quad + e^{rx}(n(n-1)r^{n-2} + (n-1)(n-2)r^{n-3} + \dots + 2p_2) \\
        &= e^{rx}\left[P(r) + 2P'(r) + P''(r)\right].
      \end{align*}
      Using the same argument as in (b), we know that
      since $r$ is a root with multiplicity 3 of the polynomial $P(x)$, we
      see $P(r) = P'(r) = P''(r) = 0$ and that $Ly(x) = 0$ for $y(x) = x^2e^{rx}$, i.e.\
      $y(x)$ is a solution of the differential equation.
    \item Note that $e^{rx}$, $xe^{rx}$, and $x^2 e^{rx}$
      are linearly independent on $\mathbb{R}$ if the Wronskian of these functions is nonzero. It is clear that
      \begin{align*}
        W(x) =
        e^{rx}
        \begin{vmatrix}
          1 & x & x^2  \\
          r & (xr + 1) &  (rx^2 + 2x) \\
          r^2 & (r^2x + 2r) & (r^2x^2 + 4rx + 2)  \\
        \end{vmatrix} = 2e^{3rx} \neq 0
      \end{align*}
      if $x \in \mathbb{R}$. Therefore, the functions are linearly independent.
  \end{enumerate}
\end{proof}
\newpage


% Problem 6
\begin{problem}
  Use the formula for the derivative of a determinant from the lectures, other
  properties of determinants, and the linear ODE (1.3.1) to verify identity
  (1.3.4) in the textbook.
\end{problem}

\begin{proof}
\end{proof}



\end{document}