\documentclass[12pt]{article}
\usepackage[margin=1in]{geometry}
\usepackage{amsmath, amsthm, amssymb, amsfonts, enumitem, graphicx}

\theoremstyle{definition}
\newtheorem{problem}{Problem}
\renewcommand*{\proofname}{Solution}
\renewcommand{\theenumi}{\alph{enumi}}

\newcommand{\vect}[1]{\boldsymbol{#1}}

\title{Homework Assignment 1}
\author{Matthew Tiger}


\begin{document}


\maketitle


% Problem 1
\begin{problem}
  Solve the IVP:
  \begin{align*}
    y' = y^2\cos(x),\quad y(0)=2.
  \end{align*}
\end{problem}

\begin{proof}
\end{proof}
\newpage


% Problem 2
\begin{problem}
  Review solutions of first-order linear ODEs (p. 14) and solve the IVP:'
  \begin{align*}
    y' - xy = x^3,\quad y(1) = \frac{1}{2}.
  \end{align*}
\end{problem}

\begin{proof}
\end{proof}
\newpage


% Problem 3
\begin{problem}
  Let $Ly = y^{(4)} - 4 y''' + 3y'' +4y' -4y$.
  \begin{enumerate}
    \item Find the general solutions of the homogeneous ODE $Ly = 0$.
    \item Solve the IVP:
      \begin{align*}
        Ly = 0,\quad y(0) = 0, \quad y'(0)= -7, \quad y''(0) = 5, \quad y'''(0) = 9.
      \end{align*}
    \item Solve the BVP:
      \begin{align*}
        Ly = 0,\quad y(0) = 1, \quad \lim_{x\to\infty}y(x) = 0.
      \end{align*}
      Is this BVP well-posed?
    \item Solve the BVP:
      \begin{align*}
        Ly = 0,\quad y(0) = 1, \quad \lim_{x\to-\infty}y(x) = 0.
      \end{align*}
      Is this BVP well-posed?
  \end{enumerate}
\end{problem}

\begin{proof}
\end{proof}
\newpage


% Problem 4
\begin{problem}
  Read \S1.6 and then solve the ODEs:
  $$xy' + 2y = x^2\sqrt{y}, \quad y' = \frac{4x^3-6xy^2-2xy}{x^2+6x^2y-3y^2}, \quad y' + y^2 +(2x+1)y + 1 + x + x^2 = 0.$$
\end{problem}

\begin{proof}
\end{proof}
\newpage


% Problem 5
\begin{problem}
  \begin{enumerate}
    \item Use mathematical induction to prove Leibnitz's differentiation rule:
      \begin{align*}
        D^{k}(fg) = \sum_{j=0}^k \binom{k}{j}(D^j f) (D^{k-j}g).
      \end{align*}
      Here $f = f(x)$ and $g=g(x)$ are $k$-times differentiable functions and $D^k = \frac{d^k}{dx^k}$.
    \item Consider the constant-coefficient ODE
      \begin{align}\label{leibnitz_ode}
        D^n y + p_{n-1}D^{n-1}y + \dots + p_1 Dy + p_0 y = 0,
      \end{align}
      where $p_0, p_1, \dots, p_{n-1}$ are real numbers. Let $r$ be a double root of the characteristic polynomial
      $P(z) = z^n + p_{n-1} z^{n-1} + \dots + p_1 z + p_0.$ Use Leibnitz’'s rule to
      show that the function $xe^{rx}$ is a solution of \eqref{leibnitz_ode}.
    \item Let $r$ be a triple root of the characteristic polynomial $P(z)$ from part (b).
      Use Leibnitz'’s rule to show that the function $x^2 e^{rx}$
      is then also a solution of \eqref{leibnitz_ode}.
    \item Let $r$ be a real number. Show that the functions $e^{rx}$, $xe^{rx}$, and $x^2 e^{rx}$
      are linearly independent on $\mathbb{R}$.
  \end{enumerate}
\end{problem}

\begin{proof}
\end{proof}
\newpage


% Problem 6
\begin{problem}
  Use the formula for the derivative of a determinant from the lectures, other
  properties of determinants, and the linear ODE (1.3.1) to verify identity
  (1.3.4) in the textbook.
\end{problem}

\begin{proof}
\end{proof}



\end{document}