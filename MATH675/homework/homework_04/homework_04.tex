\documentclass[12pt]{article}
\usepackage[margin=1in]{geometry}
\usepackage{amsmath, amsthm, amssymb, amsfonts, enumitem, graphicx}
\usepackage{fancyhdr}
\usepackage{breqn}

\theoremstyle{definition}
\newtheorem{problem}{Problem}
\renewcommand*{\proofname}{Solution}
\renewcommand{\theenumi}{\alph{enumi}}

\newcommand{\vect}[1]{\boldsymbol{#1}}
\newcommand{\bairy}{\text{Bi}}
\newenvironment{case}{%
  \let\olditem\item%
  \renewcommand\item[1][]{\olditem \textbf{##1} \\}%
  \begin{enumerate}[label=\textbf{Case \arabic*:},itemindent=*,leftmargin=0em]}{\end{enumerate}%
}

\pagestyle{fancy}
\fancyhf{}
\rhead{Homework Assignment 4}
\lhead{Matthew Tiger}
\cfoot{\thepage}


\title{Homework Assignment 4}
\author{Matthew Tiger}


\begin{document}


\maketitle


% Problem 1
\begin{problem}
  Find the first three terms in the asymptotic expansions of $x \to 0^+$ of the following
  integrals:
  \begin{align*}
    \int_x^1 \cos(xt)dt, \qquad \int_0^{1/x} e^{-t^2}dt.
  \end{align*}
\end{problem}

\begin{proof}
  If the function $f(t,x)$ possesses the asymptotic expansion
  \begin{align*}
    f(t,x) \sim \sum_{n=0}^\infty f_n(t) (x- x_0)^{\alpha n} \qquad \text{as $x \to x_0$}
  \end{align*}
  for some $\alpha > 0$, uniformly for $a \leq t \leq b$,
  then the asymptotic expansion of the integral
  \begin{align*}
    I(x) = \int_a^b f(t,x) dt
  \end{align*}
  as $x \to x_0$ is given by
  \begin{align*}
    I(x) \sim \sum_{n=0}^\infty(x- x_0)^{\alpha n} \int_a^b f_n(t) dt  \qquad \text{as $x \to x_0$}.
  \end{align*}

  We begin with finding the first three terms of the asymptotic expansion of the integral
  \begin{align*}
    I_1(x) = \int_x^1 \cos(xt)dt \qquad \text{as $x\to 0^+$}.
  \end{align*}
  Note that $f(t,x) = \cos(xt)$ has the following asymptotic expansion
  as $x \to 0^+$:
  \begin{align*}
    f(t, x) = \cos(xt) \sim 1 - \frac{t^2x^2}{2} + \frac{t^4 x^4}{24}.
  \end{align*}
  This expansion converges uniformly for all $x \leq t \leq 1$ as $x \to 0^+$.
  Therefore, we have that the first three terms of the asymptotic expansion
  of $I_1(x)$ as $x\to 0^+$ are given by
  \begin{align*}
    I_1(x) \sim \int_x^1 dt - \frac{x^2}{2} \int_x^1 t^2 dt + \frac{x^4}{24}  \int_x^1 t^4 dt = (1-x) - \frac{x^2}{2}\left[\frac{1-x^3}{3}\right] + \frac{x^4}{24}\left[\frac{1-x^5}{5}\right].
  \end{align*}

  Similar to what was shown above, we have that if
  \begin{align*}
    f(t,x) \sim f_0(t) \qquad \text{as $x \to x_0$}
  \end{align*}
  uniformly for $a \leq t \leq b$, then the asymptotic expansion of the integral
  is given by
  \begin{align*}
    I(x) = \int_a^b f(t,x) dt \sim \int_a^b f_0(t) dt \qquad \text{as $x \to x_0$}.
  \end{align*}

  Let us continue by finding the first three terms of the asymptotic expansion of the integral
  \begin{align*}
    I_2(x) = \int_0^{1/x} e^{-t^2}dt \qquad \text{as $x\to 0^+$}.
  \end{align*}
  Note that $f(t,x) = e^{-t^2}$ has the following asymptotic expansion
  as $x \to 0^+$:
  \begin{align*}
    f(t, x) = e^{-t^2} \sim 1 - t^2 + \frac{t^4}{2}.
  \end{align*}
  This expansion converges uniformly for all finite points, so it converges uniformly for $0 \leq t \leq 1/x$ as $x \to 0^+$.
  Therefore, we may integrate the expansion term by term and we have that the first three terms of the asymptotic expansion
  of $I_2(x)$ as $x\to 0^+$ are given by
  \begin{align*}
    I_2(x) \sim \int_0^{1/x}dt - \int_0^{1/x}t^2 dt + \frac{1}{2}\int_0^{1/x} t^4 dt = \frac{1}{x} - \frac{1}{3x^3} + \frac{1}{10x^5}.
  \end{align*}

\end{proof}
\newpage


% Problem 2
\begin{problem}
  Find the full asymptotic behavior as $x \to 0^+$ of the following integral:
  \begin{align*}
    \int_{0}^1 \frac{e^{-t}}{1 + x^2t^3}dt
  \end{align*}
\end{problem}

\begin{proof}
  Note that the function $f(t,x) = e^{-t}/(1 + x^2t^3)$ has the asymptotic expansion
  \begin{align*}
    f(t,x) = \frac{e^{-t}}{1 + x^2t^3} \sim e^{-t}\sum_{n=0}^\infty \left[(-1)^n t^{3n}\right]x^{2n} \qquad \text{as $x\to0^+$.}
  \end{align*}
  Note that this asymptotic expansion converges uniformly for $0\leq x \leq t < 1-\epsilon$ for all $\epsilon > 0$. To see this,
  we note that for $0 < m < n$, we have that
  \begin{align*}
    \left|\sum_{k=m+1}^n(-1)^k (x^2 t^3)^k\right| < \sum_{k=m+1}^n (1-\epsilon)^{5k}.
  \end{align*}
  Since $(1-\epsilon)^5 < 1$, we have that its geometric series converges and we can make it as small as we wish. Thus, by
  the Cauchy criterion we have uniform convergence for $0\leq x \leq t < 1-\epsilon$ for all $\epsilon > 0$.

  Per the discussion in Problem 1, using this uniformly convergent asymptotic expansion, we have that
  as $x \to 0^+$
  \begin{align*}
    \int_{0}^1 \frac{e^{-t}}{1 + x^2t^3}dt\sim \sum_{n=0}^\infty (-1)^n x^{2n} \int_0^1 e^{-t}t^{3n} dt = \sum_{n=0}^\infty (-1)^n x^{2n} \left[\Gamma(3n+1) - \Gamma(3n+1, 1)\right]
  \end{align*}
  where $\Gamma(a,k) = \int_{k}^\infty t^{a-1}e^{-t} dt$.
\end{proof}
\newpage

% Problem 3
\begin{problem}
  Find the full asymptotic expansion of $\int_0^x \bairy(t) dt$ as $x \to +\infty$.
\end{problem}

\begin{proof}
  Note that for $x\to + \infty$, the integral above can be written as
  \begin{align}\label{bairy_eqn}
    \int_0^x \bairy(t) dt = \int_0^1 \bairy(t) dt + \int_1^x \bairy(t) dt
  \end{align}
  Thus, the asymptotic expansion of the integral depends only on the second integral on the right.
  The Airy function $\bairy(t)$ satisfies the differential equation $y'' = ty$. Using this differential equation and integrating
  the integral on the right by parts we see that
  \begin{align*}
    \int_1^x \bairy(t) dt &= \int_1^x \frac{1}{t}\bairy''(t) dt \\
    &= \frac{1}{x}\bairy'(x) - \bairy'(1) + \int_1^x \frac{1}{t^2} \bairy'(t) dt.
  \end{align*}
  Note that it is clear that as $x\to + \infty$ the following relations hold
  \begin{align*}
    \bairy'(1) &\ll \frac{1}{x}\bairy'(x)\\
    \int_0^1 \bairy(t)dt &\ll \int_0^x \bairy(t)dt.
  \end{align*}
  Thus, from equation \eqref{bairy_eqn} and the above relations, we have that as $x\to + \infty$
  \begin{align}\label{double_asym}
    \int_0^x \bairy(t) dt \sim \frac{1}{x}\bairy'(x) + \int_1^x \frac{1}{t^2} \bairy'(t) dt.
  \end{align}
  However, upon further investigation we see that as $x \to +\infty$
  \begin{align}\label{lhosp}
    \int_1^x \frac{1}{t^2} \bairy'(t) dt \ll \frac{1}{x}\bairy'(x).
  \end{align}
  To see that this is true, we integrate the integral on the left by parts which yields
  \begin{align*}
    f(x) = \int_1^x \frac{1}{t^2} \bairy'(t) dt = x^{-2}\bairy(x) - \bairy(1) + 2\int_{1}^x t^{-3}\bairy(t)dt.
  \end{align*}
  In comparing the function $f(x)$ with the function $g(x) = x^{-1}\bairy'(x)$ as $x \to +\infty$, we see that
  \begin{align*}
    \lim_{x\to+\infty} \frac{f(x)}{g(x)} = \frac{+\infty}{+\infty}
  \end{align*}
  an indeterminate form. Thus, applying L\rq H\^opital's rule, we see that derivatives of $f(x)$ and $g(x)$ are
  \begin{align*}
    f'(x) &= -2x^{-3}\bairy(x) + x^{-2}\bairy'(x) + 2\left[x^{-3}\bairy(x) - \bairy(1) \right] \\
    &= x^{-2}\bairy'(x) - 2 \bairy(1) \\
    g'(x) &= -x^{-2}\bairy'(x) + x^{-1}\bairy''(x)
  \end{align*}
  and that
  \begin{align*}
    \lim_{x\to + \infty}\frac{f'(x)}{g'(x)} =\frac{x^{-2}\bairy'(x) -2\bairy(1)}{-x^{-2}\bairy'(x) + x^{-1}\bairy''(x)} = \frac{1}{1 + \frac{x^{-1}\bairy''(x)}{x^{-2}\bairy'(x)}} = 0
  \end{align*}
  Therefore, we must have that relation \eqref{lhosp} is true and that relation \eqref{double_asym} reduces to
  \begin{align*}
    \int_0^x \bairy(t) dt \sim \frac{1}{x}\bairy'(x) \qquad (x \to +\infty).
  \end{align*}

  Note that the asymptotic expansion of $\bairy(x)$ as $x \to +\infty$ is given by
  \begin{align*}
    \bairy(x) \sim \pi^{-1/2}x^{-1/4} \exp{\left(\frac{2x^{3/2}}{3}\right)} \sum_{n=0}^\infty c_n x^{-3n/2}
  \end{align*}
  where
  \begin{align*}
    c_n = \frac{1}{2\pi}\left(\frac{3}{4}\right)^n \frac{\Gamma(n + 5/6)\Gamma(n + 1/6)}{n!}.
  \end{align*}
  Thus, we see that as $x \to +\infty$,
  \begin{align*}
    \bairy'(x) &\sim \pi^{-1/2}\exp{\left(\frac{2x^{3/2}}{3}\right)}\left[\left(x^{3/2} - \frac{1}{4}\right)x^{-5/4} \sum_{n=0}^{\infty}c_n x^{-3n/2}  + x^{-1/4}\sum_{n=0}^{\infty}\frac{-3n c_n}{2} x^{-3n/2 - 1}\right] \\
    &= \pi^{-1/2}\exp{\left(\frac{2x^{3/2}}{3}\right)}\left(x^{3/2} + \frac{3}{4}\right)\sum_{n=0}^{\infty}\left(1 - \frac{3n}{2}\right)c_n x^{-3n/2-5/4}.
  \end{align*}
  Therefore, we can readily see that the full asymptotic behavior as $x \to +\infty$ of the integral of the problem is given by
  \begin{align*}
    \int_0^x \bairy(t) dt \sim     \frac{\bairy'(x)}{x} \sim \pi^{-1/2}\exp{\left(\frac{2x^{3/2}}{3}\right)}\left(x^{3/2} + \frac{3}{4}\right)\sum_{n=0}^{\infty}\left(1 - \frac{3n}{2}\right)c_n x^{-3n/2-9/4}.
  \end{align*}
\end{proof}
\newpage


% Problem 4
\begin{problem}
  Find the first five terms in the asymptotic expansion as $x \to +\infty$ of the integral
  \begin{align*}
    \int_0^{\pi/4} e^{-xt^2}\sqrt{\tan t}   dt
  \end{align*}
  \begin{enumerate}
    \item by using a suitable change of variables and then applying Watson's lemma.
    \item by applying Laplace's method directly to the given integral.
  \end{enumerate}
\end{problem}

\begin{proof}
  \begin{enumerate}
    \item Watson's lemma provides a formula for an asymptotic expansion as $x\to +\infty$ for integrals of the form
      \begin{align}\label{watson_form}
        I(x) = \int_0^b f(s) e^{-xs} ds \qquad b > 0
      \end{align}
      where the function $f(s)$ is continuous on the interval $0\leq s \leq b$ and has the asymptotic expansion
      \begin{align*}
        f(s) \sim s^{\alpha} \sum_{n=0}^\infty a_ns^{\beta n}\qquad (s \to 0^+)
      \end{align*}
      with $\alpha > -1$ and $\beta > 0$. Given these assumptions, Watson's lemma states that
      \begin{align}\label{watsons}
        I(x) \sim \sum_{n=0}^\infty \frac{a_n \Gamma(\alpha + \beta n +1)}{x^{\alpha + \beta n + 1}} \qquad (x \to + \infty).
      \end{align}

      For the integral
      \begin{align*}
        I(x) = \int_0^{\pi/4} e^{-xt^2} \sqrt{\tan t} dt,
      \end{align*}
      we proceed by making the change of variables $s = t^2$. The integral then becomes
      \begin{align*}
        I(x) = \int_0^{\sqrt{\pi}/2} 2^{-1}s^{-1/2}\sqrt{\tan s^{1/2}} e^{-xs}ds.
      \end{align*}
      Identifying the function $f(s) = 2^{-1}s^{-1/2}\sqrt{\tan s^{1/2}}$, we see that the above integral is of the form \eqref{watson_form}
      with $f(s)$ being continuous on $0 \leq s \leq \sqrt{\pi}/2$. Further, the function $f(s)$ has the following asymptotic
      expansion
      \begin{align*}
        f(s) \sim \frac{1}{2}s^{-1/4} + \frac{1}{12}s^{3/4} + \frac{19}{720}s^{7/4} + \frac{55}{6048} s^{11/4} + \frac{11813}{3628800}s^{15/4}  \qquad (s \to 0^+).
      \end{align*}
      Therefore, identifying $\alpha = -1/4$ and $\beta = 1$, we see that by Watson's lemma the first five terms in the asymptotic
      expansion of $I(x)$ as $x\to +\infty$ is given by
      \begin{align*}
        I(x) &\sim \frac{\Gamma\left(\frac{3}{4}\right)}{2}x^{-3/4} + \frac{\Gamma\left(\frac{7}{4}\right)}{12}x^{-7/4} + \frac{19\Gamma\left(\frac{11}{4}\right)}{720}x^{-11/4} + \frac{55\Gamma\left(\frac{15}{4}\right)}{6048} x^{-15/4}+ \frac{11813\Gamma\left(\frac{19}{4}\right)}{3628800}x^{-19/4}.
      \end{align*}

    \item Laplace's method states that, as $x\to +\infty$, for an integral of the form
      \begin{align*}
        I(x) = \int_a^b f(t)e^{x\phi(t)} dt
      \end{align*}
      where $f(t)$ and $\phi(t)$ are real continuous functions, the integral $I(x)$ is asymptotic to the integral of $f(t)e^{x\phi(t)}$ over some small
      neighborhood of the point where $\phi(t)$ obtains its maximum over the interval $[a, b]$.

      Identifying the function $f(t) = \sqrt{\tan t}$ and $\phi(t) = -t^2$,
      both real and continuous on the interval $[0, \pi/4]$, we see that $\phi(t)$
      obtains its maximum at the point $t = 0$ on the same interval. However the function $f(t)$ vanishes at $t=0$.
      Nevertheless Laplace's method may still be used since any contribution to the
      integral outside of the interval $[0, \epsilon]$ is subdominant for any $\epsilon > 0$.
      Thus, all of the assumptions of Laplace's method are satisfied and we have that for small $\epsilon > 0$,
      \begin{align*}
        I(x) = \int_0^{\pi/4} e^{-xt^2}\sqrt{\tan t}   dt \sim \int_0^{\epsilon} e^{-xt^2}\sqrt{\tan t}   dt \qquad (x\to +\infty).
      \end{align*}
      Since $\epsilon > 0$ is small, we may replace the function $f(t)$ with the asymptotic expansion about $t=0$
      \begin{align*}
        \sqrt{\tan t} \sim t^{1/2} + \frac{1}{6}t^{5/2} + \frac{19}{360}t^{9/2} + \frac{55}{3024}t^{13/2} + \frac{11813}{1814400}t^{17/2} \qquad (t \to 0^+)
      \end{align*}
      so that, as $x \to +\infty$, the first five terms in the asymptotic expansion of the integral are
      \begin{align*}
        I(x) & \sim \int_0^\epsilon \left[t^{1/2} + \frac{1}{6}t^{5/2} + \frac{19}{360}t^{9/2} + \frac{55}{3024}t^{13/2} + \frac{11813}{1814400}t^{17/2}\right]e^{-xt^2}dt\\
        &\sim \int_0^\infty \left[t^{1/2} + \frac{1}{6}t^{5/2} + \frac{19}{360}t^{9/2} + \frac{55}{3024}t^{13/2} + \frac{11813}{1814400}t^{17/2}\right]e^{-xt^2}dt\\
        &= \frac{\Gamma\left(\frac{3}{4}\right)}{2}x^{-3/4} + \frac{\Gamma\left(\frac{7}{4}\right)}{12}x^{-7/4} + \frac{19\Gamma\left(\frac{11}{4}\right)}{720}x^{-11/4} + \frac{55\Gamma\left(\frac{15}{4}\right)}{6048} x^{-15/4} + \frac{11813\Gamma\left(\frac{19}{4}\right)}{3628800}x^{-19/4}.
      \end{align*}
  \end{enumerate}
\end{proof}
\newpage


% Problem 5
\begin{problem}
  Use Laplace's method of moving maxima to obtain the first two terms in the asymptotic
  expansion as $x \to +\infty$ of the integral
  \begin{align}\label{exp_int}
    \int_{0}^\infty \exp\left[-t - \frac{x}{\sqrt{t}}\right] dt.
  \end{align}
\end{problem}

\begin{proof}
\end{proof}
\newpage


% Problem 6
\begin{problem}
  Let $f(x, t)$ be differentiable in $x$ and continuous in $(x, t)$ on $I \times J$,
  where $I$ and $J$ are intervals, and suppose that there exist functions $g(t)$ and $g_1(t)$
  that are integrable on $J$ such that for all $(x,t) \in I \times J$ we have that
  \begin{align*}
    |f(x,t)|\leq g(t) \quad \text{and} \quad |\partial_x f(x,t)| \leq g_1(t).
  \end{align*}
  Then
  \begin{align*}
    \frac{d}{dx} \int_J f(x,t) dt = \int_J \partial_x f(x,t) dt.
  \end{align*}
  \begin{enumerate}
    \item Let $0 < a < b < \infty$. Use the above theorem to show that if $x \in (a,b)$, then
      \begin{align*}
        \frac{d^3}{dx^3} \int_0^\infty \exp\left[-t - \frac{x}{\sqrt{t}}\right] dt = -\int_0^\infty t^{-3/2}\exp\left[-t - \frac{x}{\sqrt{t}}\right] dt.
      \end{align*}
    \item Use integration by parts to show that
      \begin{align*}
        \int_0^\infty \exp\left[-t - \frac{x}{\sqrt{t}}\right] dt = \frac{x}{2}\int_0^\infty t^{-3/2}\exp\left[-t - \frac{x}{\sqrt{t}}\right] dt.
      \end{align*}
    \item Combine parts (a) and (b) to prove that integral \eqref{exp_int} is a solution of the
      differential equation $xy''' + 2y = 0$ that also satisfies the initial condition $y(0) = 1$.
      Then use integration by parts to give an easy direct proof that the integral also satisfies
      the condition $y(+\infty) = 0$.
  \end{enumerate}
\end{problem}

\begin{proof}
\end{proof}
\newpage


% Problem 7
\begin{problem}
  \begin{enumerate}
    \item Find the leading behavior as $x\to + \infty$ of Laplace integrals of the form
      \begin{align*}
        \int_a^b (t-a) ^\alpha g(t) e^{x\phi(t)} dt
      \end{align*}
      where $\phi(t)$ has a maximum at $t = a$, $g(a) = 1$. Suppose further that $\alpha > -1$ and $\phi'(a) < 0$.
    \item Repeat the analysis of part (a) when $\alpha > -1$
      and $\phi'(a) = \phi''(a) = \dots = \phi^{(p-1)}(a) = 0$ and $\phi^{(p)}(a) < 0$.
  \end{enumerate}
\end{problem}

\begin{proof}
\end{proof}
\newpage


\end{document}