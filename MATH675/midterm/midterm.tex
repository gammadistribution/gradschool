\documentclass[12pt]{article}
\usepackage[margin=1in]{geometry}
\usepackage{amsmath, amsthm, amssymb, amsfonts, enumitem, graphicx}
\usepackage{fancyhdr}
\usepackage{breqn}

\theoremstyle{definition}
\newtheorem{problem}{Problem}
\renewcommand*{\proofname}{Solution}
\renewcommand{\theenumi}{\alph{enumi}}

\newcommand{\vect}[1]{\boldsymbol{#1}}

\newenvironment{case}{%
  \let\olditem\item%
  \renewcommand\item[1][]{\olditem \textbf{##1} \\}%
  \begin{enumerate}[label=\textbf{Case \arabic*:},itemindent=*,leftmargin=0em]}{\end{enumerate}%
}

\pagestyle{fancy}
\fancyhf{}
\rhead{Test 1}
\lhead{Matthew Tiger}
\cfoot{\thepage}


\title{Test 1}
\author{Matthew Tiger}


\begin{document}


\maketitle


% Problem 1
\begin{problem}
  \begin{enumerate}
    \item Use the Frobenius method to find a series expansion of $x=-2$
      of the general solution of the equation
      \begin{align*}
        x(x+2)y'' + (x+1)y' - 4y = 0.
      \end{align*}
    \item Use your answer to part a.\ to find a series solution of the BVP
      \begin{align*}
        x(x+2)y'' + (x+1)y' - 4y = 0,\quad y(-2) = y(-1.5) = 1.
      \end{align*}
  \end{enumerate}
\end{problem}

\begin{proof}
\end{proof}
\newpage


% Problem 2
\begin{problem}
  \begin{enumerate}
    \item Transform the equation $x(x+2)y'' + (x+1)y' - 4y = 0$ to the form
      \begin{align}\label{transform}
        \ddot{y} + t^{-1}p(t)\dot{y} + t^{-2}q(t)y = 0
      \end{align}
      and use the result to determine whether the point at $\infty$ is an ordinary, regular
      singular, or irregular singular point for the original equation.
    \item Apply an appropriate method to equation \eqref{transform}  to obtain
      two series that represent linearly independent solutions of the original
      equation as $x \to +\infty$.
  \end{enumerate}
\end{problem}

\begin{proof}
\end{proof}
\newpage


% Problem 2
\begin{problem}
  Find the first three terms in the asymptotic expansion as $x \to +\infty$
  of a solution of the equation
  \begin{align*}
    y''' + \frac{y'}{x^3} = x.
  \end{align*}
\end{problem}

\begin{proof}
\end{proof}


\end{document}