\documentclass[12pt]{article}
\usepackage[margin=1in]{geometry}
\usepackage{amsmath, amsthm, amssymb, amsfonts, enumitem, graphicx}
\usepackage{fancyhdr}
\usepackage{breqn}

\theoremstyle{definition}
\newtheorem{problem}{Problem}
\renewcommand*{\proofname}{Solution}
\renewcommand{\theenumi}{\alph{enumi}}

\newcommand{\vect}[1]{\boldsymbol{#1}}

\newenvironment{case}{%
  \let\olditem\item%
  \renewcommand\item[1][]{\olditem \textbf{##1} \\}%
  \begin{enumerate}[label=\textbf{Case \arabic*:},itemindent=*,leftmargin=0em]}{\end{enumerate}%
}

\pagestyle{fancy}
\fancyhf{}
\rhead{Test 1}
\lhead{Matthew Tiger}
\cfoot{\thepage}


\title{Test 1}
\author{Matthew Tiger}


\begin{document}


\maketitle


% Problem 1
\begin{problem}
  \begin{enumerate}
    \item Use the Frobenius method to find a series expansion of $x=-2$
      of the general solution of the equation
      \begin{align}\label{diff_eq}
        x(x+2)y'' + (x+1)y' - 4y = 0.
      \end{align}
    \item Use your answer to part a.\ to find a series solution of the BVP
      \begin{align*}
        x(x+2)y'' + (x+1)y' - 4y = 0,\quad y(-2) = y(-1.5) = 1.
      \end{align*}
  \end{enumerate}
\end{problem}

\begin{proof}
  \begin{enumerate}
    \item Note that the differential equation \eqref{diff_eq} may be written as
      \begin{align*}
        y'' + \left[\frac{x+1}{x(x+2)}\right] y' + \left[\frac{-4}{x(x+2)}\right]y = 0.
      \end{align*}
      Since the functions
      \begin{align*}
        p_1(x) = \frac{x+1}{x(x+2)}, \quad p_0(x) = \frac{-4}{x(x+2)}
      \end{align*}
      are not analytic at $x=-2$, but both $(x+2)p_1(x)$ and $(x+2)^2p_0(x)$ are
      analytic at that point, we classify the point $x=-2$ as a regular singular point.

      As such, we rewrite equation \eqref{diff_eq} as
      \begin{align*}
        y'' + \left[\frac{p(x)}{x+2}\right] y' + \left[\frac{q(x)}{(x+2)^2}\right]y = 0.
      \end{align*}
      where $p(x)$ and $q(x)$ are defined as the following analytic functions at $x=-2$:
      \begin{align*}
        p(x) = \frac{x+1}{x}, \quad q(x) = -\frac{4(x+2)}{x}.
      \end{align*}
      Using the following power series expansion of the function $f(x) = 1/x$ about $x=-2$,
      \begin{align*}
        \frac{1}{x} = \sum_{n=0}^{\infty}\left[\frac{-1}{2^{n+1}}\right](x+2)^n,
      \end{align*}
      we may write the power series expansions of the analytic functions $p(x)$ and $q(x)$ about $x=-2$ as follows:
      \begin{align}\label{power}
        p(x) &= \sum_{n=0}^{\infty} p_n (x+2)^n = \frac{1}{2} + \sum_{n=1}^{\infty} \left[\frac{-1}{2^{n+1}}\right] (x+2)^n \notag \\
        q(x) &= \sum_{n=0}^{\infty} q_n (x+2)^n = \sum_{n=1}^{\infty} \left[\frac{1}{2^{n-2}}\right] (x+2)^n.
      \end{align}
      Identifying $p_0 = 1/2$ and $q_0 = 0$, the indicial polynomial associated to the differential equation \eqref{diff_eq} is
      \begin{align*}
        P(\alpha) = \alpha^2 + (p_0 - 1)\alpha + q_0 = \alpha\left(\alpha - \frac{1}{2}\right).
      \end{align*}
      The two roots to the indicial polynomial are $\alpha_1 = 1/2$ and $\alpha_2 = 0$. Since the roots of the indicial polynomial
      do not differ by an integer, there exist two linearly independent solutions in Frobenius form.
      Therefore, the two linearly independent solutions are
      \begin{align}\label{lin_sol}
        y_1(x) = \sum_{n=0}^\infty a_n (x+2)^{n + \alpha_1}, \quad y_2(x) = \sum_{n=0}^\infty b_n (x+2)^{n + \alpha_2}
      \end{align}
      where the sequence $a_n$ satisfies the recurrence relations
      \begin{align}\label{a_rel}
        P(\alpha_1)a_0 &= 0 & & \notag \\
        P(\alpha_1 + n) a_n &= -\sum_{k=0}^{n-1}\left[(\alpha_1 + k)p_{n-k} + q_{n-k}\right]a_k, &n&=1,2,\dots
      \end{align}
      with $a_0 \neq 0$ and the sequence $b_n$ satisfies the recurrence relations
      \begin{align}\label{b_rel}
        P(\alpha_2)b_0 &= 0 & & \notag \\
        P(\alpha_2 + n) b_n &= -\sum_{k=0}^{n-1}\left[(\alpha_2 + k)p_{n-k} + q_{n-k}\right]b_k, &n&=1,2,\dots
      \end{align}
      with $b_0 \neq 0$.
      Thus, we need only solve the recurrence relations \eqref{a_rel} and \eqref{b_rel} to completely determine the
      linearly independent solutions \eqref{lin_sol}.

      % Find solution y_1(x)
      The sequence defining the solution $y_1(x)$ associated to the root $\alpha_1 = 1/2$ satisfies recurrence relation \eqref{a_rel}.
      Since $P(\alpha_1) = 0$, the first equation of the recurrence relation \eqref{a_rel} is satisfied and
      using the sequences defining the analytic functions $p(x)$ and $q(x)$, we have that the other equation becomes
      \begin{align}\label{a_rel_red}
        P(n + 1/2) a_n &= -\sum_{k=0}^{n-1}\left[-\frac{(k + 1/2)}{2^{n-k+1}} + \frac{1}{2^{n-k-2}}\right]a_k & & \notag\\
        &=\sum_{k=0}^{n-1}\left[\frac{2k-15}{2^{n-k+2}}\right]a_k, &n&=1,2,\dots.
      \end{align}
      We can prove through induction that the above relation satisfies the formula
      \begin{align*}
        a_n &= \frac{4n^2-4n-15}{8n^2+4n}a_{n-1}, &n&=1,2,\dots.
      \end{align*}
      To see this we can note that
      \begin{align*}
        a_1 &= \frac{4 - 4 - 15}{8 + 4}a_0 = -\frac{5}{4}a_0
      \end{align*}
      and have established that the formula holds for $n=1$. Now suppose the formula
      holds for general $n > 1$. Using our supposition, we see from relation \eqref{a_rel_red} that
      \begin{align*}
        P(n + 1 + 1/2) a_{n+1} &= \sum_{k=0}^{n}\left[\frac{2k-15}{2^{n-k+3}}\right]a_k \\
        &= \frac{1}{2}\sum_{k=0}^{n-1}\left[\frac{2k-15}{2^{n-k+2}}\right]a_k + \frac{2n-15}{8}a_n \\
        &= \left[\frac{P(n + 1/2)}{2} + \frac{2n-15}{8}\right]a_n.
      \end{align*}
      Performing some algebra on this expression we see that
      \begin{align*}
        a_{n+1} = \frac{4(n+1)^2-4(n+1)-15}{8(n+1)^2+4(n+1)}a_{n}
      \end{align*}
      and the formula holds for $n+1$ completing the proof.
      Mathematica reports that the solution to this recurrence relation is
      \begin{align*}
        a_n = \left[-\frac{\Gamma(2)}{\Gamma(-1/2)}\frac{(2n+3)\Gamma(n-3/2)}{2^{n+1}\Gamma(n+1)}\right]a_0 = \frac{1}{\sqrt{2\pi}}\left[\frac{(2n+3)\Gamma(n-3/2)}{2^{n+1}\Gamma(n+1)}\right]a_0
      \end{align*}
      Therefore, using \eqref{lin_sol}, the solution to the differential equation \eqref{diff_eq} associated to the root $\alpha_1 = 1/2$
      is
      \begin{align}\label{y1_sol}
        y_1(x) &= a_0 \sum_{n=0}^\infty \left[\frac{(2n+3)\Gamma(n-3/2)}{\sqrt{2\pi}2^{n+1}\Gamma(n+1)}\right](x+2)^{n+1/2} \notag \\
        &= a_0\left[\frac{-(x+1)\sqrt{-x(x+2)}}{\sqrt{2}}\right]
      \end{align}
      which has radius of convergence 2 centered at $x=-2$.

      % Find solution y_2(x)
      We now look to identify the solution $y_2(x)$.
      The sequence defining the solution $y_2(x)$ associated to the root $\alpha_2 = 0$ satisfies recurrence relation \eqref{b_rel}.
      Since $P(\alpha_2) = 0$, the first equation of the recurrence relation \eqref{b_rel} is satisfied and
      using the sequences defining the analytic functions $p(x)$ and $q(x)$, we have that the other equation becomes
      \begin{align}\label{b_rel_red}
        P(n) b_n &= -\sum_{k=0}^{n-1}\left[-\frac{k}{2^{n-k+1}} + \frac{1}{2^{n-k-2}}\right]b_k & & \notag\\
        &=\sum_{k=0}^{n-1}\left[\frac{k-8}{2^{n-k+1}}\right]b_k, &n&=1,2,\dots.
      \end{align}
      We can prove through induction that the above relation satisfies the formula
      \begin{align*}
        b_n &= \frac{-n^2+2n+3}{-2n^2+n}b_{n-1}, &n&=1,2,\dots.
      \end{align*}
      To see this we can note that
      \begin{align*}
        b_1 &= \frac{-1 +2 +3}{-2 + 1}b_0 = -4b_0
      \end{align*}
      and have established that the formula holds for $n=1$. Now suppose the formula
      holds for general $n > 1$. Using our supposition, we see from relation \eqref{b_rel_red} that
      \begin{align*}
        P(n + 1) b_{n+1} &= \sum_{k=0}^{n}\left[\frac{k-8}{2^{n-k+2}}\right]b_k \\
        &= \frac{1}{2}\sum_{k=0}^{n-1}\left[\frac{k-8}{2^{n-k+1}}\right]b_k + \frac{n-8}{4}b_n \\
        &= \left[\frac{P(n)}{2} + \frac{n-8}{4}\right]b_n.
      \end{align*}
      Performing some algebra on this expression we see that
      \begin{align*}
        b_{n+1} = \frac{-(n+1)^2+2(n+1)+3}{-2(n+1)^2+(n+1)}b_{n}
      \end{align*}
      and the formula holds for $n+1$ completing the proof.

      Note that $b_3 = 0$ which implies that $b_n = 0$ for $n\geq 3$ and that
      \begin{align*}
        b_n =
        \begin{cases}
          b_1 = -4b_0 \\
          b_2 = 2b_0 \\
          b_n = 0 &\text{for $n\geq 3$}
        \end{cases}.
      \end{align*}
      Therefore, using \eqref{lin_sol}, the solution to the differential equation \eqref{diff_eq} associated to the root $\alpha_2 = 0$
      is
      \begin{align}\label{y2_sol}
        y_2(x) &= \sum_{n=0}^\infty b_n(x+2)^{n} \notag \\
        &= b_0\left[1 -4(x+2) +2(x+2)^2\right]\notag\\
        &= b_0\left[2x^2+4x+1\right].
      \end{align}

      The general solution to the differential equation is then
      \begin{align}\label{gen_sol}
        y(x) = a_0\left[\frac{-(x+1)\sqrt{-x(x+2)}}{\sqrt{2}}\right] + b_0\left[2x^2+4x+1\right].
      \end{align}

      \item Note that \eqref{gen_sol} is the general solution to the BVP.
        So, for $-2 \leq x \leq -3/2$,
        \begin{align*}
          y(x) = a_0\left[\frac{-(x+1)\sqrt{-x(x+2)}}{\sqrt{2}}\right] + b_0\left[2x^2+4x+1\right].
        \end{align*}
        satisfies the differential equation. From the boundary conditions $y(-2) = y(-3/2) = 1$
        we see that
        \begin{align*}
          y(-2) &= b_0 = 1 \\
          y(-3/2) &= \frac{a_0\sqrt{3}}{4\sqrt{2}}  - \frac{b_0}{2} = 1
        \end{align*}
        from which we readily see that $a_0 = 2\sqrt{6}$ and $b_0 = 1$. Therefore,
        the solution to the BVP is
        \begin{align*}
          y(x) = 2\sqrt{6}\left[\frac{-(x+1)\sqrt{-x(x+2)}}{\sqrt{2}}\right] + \left[2x^2+4x+1\right].
        \end{align*}
  \end{enumerate}
\end{proof}
\newpage


% Problem 2
\begin{problem}
  \begin{enumerate}
    \item Transform the equation $x(x+2)y'' + (x+1)y' - 4y = 0$ to the form
      \begin{align}\label{transform}
        \ddot{y} + t^{-1}p(t)\dot{y} + t^{-2}q(t)y = 0
      \end{align}
      and use the result to determine whether the point at $\infty$ is an ordinary, regular
      singular, or irregular singular point for the original equation.
    \item Apply an appropriate method to equation \eqref{transform}  to obtain
      two series that represent linearly independent solutions of the original
      equation as $x \to +\infty$.
  \end{enumerate}
\end{problem}

\begin{proof}
\end{proof}
\newpage


% Problem 3
\begin{problem}
  Find the first three terms in the asymptotic expansion as $x \to +\infty$
  of a solution of the equation
  \begin{align*}
    y''' + \frac{y'}{x^3} = x.
  \end{align*}
\end{problem}

\begin{proof}
\end{proof}


\end{document}