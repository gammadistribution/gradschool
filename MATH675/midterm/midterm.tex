\documentclass[12pt]{article}
\usepackage[margin=1in]{geometry}
\usepackage{amsmath, amsthm, amssymb, amsfonts, enumitem, graphicx}
\usepackage{fancyhdr}
\usepackage{breqn}

\theoremstyle{definition}
\newtheorem{problem}{Problem}
\renewcommand*{\proofname}{Solution}
\renewcommand{\theenumi}{\alph{enumi}}

\newcommand{\vect}[1]{\boldsymbol{#1}}

\newenvironment{case}{%
  \let\olditem\item%
  \renewcommand\item[1][]{\olditem \textbf{##1} \\}%
  \begin{enumerate}[label=\textbf{Case \arabic*:},itemindent=*,leftmargin=0em]}{\end{enumerate}%
}

\pagestyle{fancy}
\fancyhf{}
\rhead{Test 1}
\lhead{Matthew Tiger}
\cfoot{\thepage}


\title{Test 1}
\author{Matthew Tiger}


\begin{document}


\maketitle


% Problem 1
\begin{problem}
  \begin{enumerate}
    \item Use the Frobenius method to find a series expansion of $x=-2$
      of the general solution of the equation
      \begin{align}\label{diff_eq}
        x(x+2)y'' + (x+1)y' - 4y = 0.
      \end{align}
    \item Use your answer to part a.\ to find a series solution of the BVP
      \begin{align*}
        x(x+2)y'' + (x+1)y' - 4y = 0,\quad y(-2) = y(-1.5) = 1.
      \end{align*}
  \end{enumerate}
\end{problem}

\begin{proof}
  \begin{enumerate}
    \item Note that the differential equation \eqref{diff_eq} may be written as
      \begin{align*}
        y'' + \left[\frac{x+1}{x(x+2)}\right] y' + \left[\frac{-4}{x(x+2)}\right]y = 0.
      \end{align*}
      Since the functions
      \begin{align*}
        p_1(x) = \frac{x+1}{x(x+2)}, \quad p_0(x) = \frac{-4}{x(x+2)}
      \end{align*}
      are not analytic at $x=-2$, but both $(x+2)p_1(x)$ and $(x+2)^2p_0(x)$ are
      analytic at that point, we classify the point $x=-2$ as a regular singular point.

      As such, we rewrite equation \eqref{diff_eq} as
      \begin{align*}
        y'' + \left[\frac{p(x)}{x+2}\right] y' + \left[\frac{q(x)}{(x+2)^2}\right]y = 0.
      \end{align*}
      where $p(x)$ and $q(x)$ are defined as the following analytic functions at $x=-2$:
      \begin{align*}
        p(x) = \frac{x+1}{x}, \quad q(x) = -\frac{4(x+2)}{x}.
      \end{align*}
      Using the following power series expansion of the function $f(x) = 1/x$ about $x=-2$,
      \begin{align*}
        \frac{1}{x} = \sum_{n=0}^{\infty}\left[\frac{-1}{2^{n+1}}\right](x+2)^n,
      \end{align*}
      we may write the power series expansions of the analytic functions $p(x)$ and $q(x)$ about $x=-2$ as follows:
      \begin{align}\label{power}
        p(x) &= \sum_{n=0}^{\infty} p_n (x+2)^n = \frac{1}{2} + \sum_{n=1}^{\infty} \left[\frac{-1}{2^{n+1}}\right] (x+2)^n \notag \\
        q(x) &= \sum_{n=0}^{\infty} q_n (x+2)^n = \sum_{n=1}^{\infty} \left[\frac{1}{2^{n-2}}\right] (x+2)^n.
      \end{align}
      Identifying $p_0 = 1/2$ and $q_0 = 0$, the indicial polynomial associated to the differential equation \eqref{diff_eq} is
      \begin{align*}
        P(\alpha) = \alpha^2 + (p_0 - 1)\alpha + q_0 = \alpha\left(\alpha - \frac{1}{2}\right).
      \end{align*}
      The two roots to the indicial polynomial are $\alpha_1 = 1/2$ and $\alpha_2 = 0$. Since the roots of the indicial polynomial
      do not differ by an integer, there exist two linearly independent solutions in Frobenius form.
      Therefore, the two linearly independent solutions are
      \begin{align}\label{lin_sol}
        y_1(x) = \sum_{n=0}^\infty a_n (x+2)^{n + \alpha_1}, \quad y_2(x) = \sum_{n=0}^\infty b_n (x+2)^{n + \alpha_2}
      \end{align}
      where the sequence $a_n$ satisfies the recurrence relations
      \begin{align}\label{a_rel}
        P(\alpha_1)a_0 &= 0 & & \notag \\
        P(\alpha_1 + n) a_n &= -\sum_{k=0}^{n-1}\left[(\alpha_1 + k)p_{n-k} + q_{n-k}\right]a_k, &n&=1,2,\dots
      \end{align}
      with $a_0 \neq 0$ and the sequence $b_n$ satisfies the recurrence relations
      \begin{align}\label{b_rel}
        P(\alpha_2)b_0 &= 0 & & \notag \\
        P(\alpha_2 + n) b_n &= -\sum_{k=0}^{n-1}\left[(\alpha_2 + k)p_{n-k} + q_{n-k}\right]b_k, &n&=1,2,\dots
      \end{align}
      with $b_0 \neq 0$.
      Thus, we need only solve the recurrence relations \eqref{a_rel} and \eqref{b_rel} to completely determine the
      linearly independent solutions \eqref{lin_sol}.

      % Find solution y_1(x)
      The sequence defining the solution $y_1(x)$ associated to the root $\alpha_1 = 1/2$ satisfies recurrence relation \eqref{a_rel}.
      Since $P(\alpha_1) = 0$, the first equation of the recurrence relation \eqref{a_rel} is satisfied and
      using the sequences defining the analytic functions $p(x)$ and $q(x)$, we have that the other equation becomes
      \begin{align}\label{a_rel_red}
        P(n + 1/2) a_n &= -\sum_{k=0}^{n-1}\left[-\frac{(k + 1/2)}{2^{n-k+1}} + \frac{1}{2^{n-k-2}}\right]a_k & & \notag\\
        &=\sum_{k=0}^{n-1}\left[\frac{2k-15}{2^{n-k+2}}\right]a_k, &n&=1,2,\dots.
      \end{align}
      We can prove through induction that the above relation satisfies the formula
      \begin{align*}
        a_n &= \frac{4n^2-4n-15}{8n^2+4n}a_{n-1}, &n&=1,2,\dots.
      \end{align*}
      To see this we can note that
      \begin{align*}
        a_1 &= \frac{4 - 4 - 15}{8 + 4}a_0 = -\frac{5}{4}a_0
      \end{align*}
      and have established that the formula holds for $n=1$. Now suppose the formula
      holds for general $n > 1$. Using our supposition, we see from relation \eqref{a_rel_red} that
      \begin{align*}
        P(n + 1 + 1/2) a_{n+1} &= \sum_{k=0}^{n}\left[\frac{2k-15}{2^{n-k+3}}\right]a_k \\
        &= \frac{1}{2}\sum_{k=0}^{n-1}\left[\frac{2k-15}{2^{n-k+2}}\right]a_k + \frac{2n-15}{8}a_n \\
        &= \left[\frac{P(n + 1/2)}{2} + \frac{2n-15}{8}\right]a_n.
      \end{align*}
      Performing some algebra on this expression we see that
      \begin{align*}
        a_{n+1} = \frac{4(n+1)^2-4(n+1)-15}{8(n+1)^2+4(n+1)}a_{n}
      \end{align*}
      and the formula holds for $n+1$ completing the proof.
      Mathematica reports that the solution to this recurrence relation is
      \begin{align*}
        a_n = \left[-\frac{\Gamma(2)}{\Gamma(-1/2)}\frac{(2n+3)\Gamma(n-3/2)}{2^{n+1}\Gamma(n+1)}\right]a_0 = \frac{1}{\sqrt{2\pi}}\left[\frac{(2n+3)\Gamma(n-3/2)}{2^{n+1}\Gamma(n+1)}\right]a_0
      \end{align*}
      Therefore, using \eqref{lin_sol}, the solution to the differential equation \eqref{diff_eq} associated to the root $\alpha_1 = 1/2$
      is
      \begin{align}\label{y1_sol}
        y_1(x) &= a_0 \sum_{n=0}^\infty \left[\frac{(2n+3)\Gamma(n-3/2)}{\sqrt{2\pi}2^{n+1}\Gamma(n+1)}\right](x+2)^{n+1/2} \notag \\
        &= a_0\left[\frac{-(x+1)\sqrt{-x(x+2)}}{\sqrt{2}}\right]
      \end{align}
      which has radius of convergence 2 centered at $x=-2$.

      % Find solution y_2(x)
      We now look to identify the solution $y_2(x)$.
      The sequence defining the solution $y_2(x)$ associated to the root $\alpha_2 = 0$ satisfies recurrence relation \eqref{b_rel}.
      Since $P(\alpha_2) = 0$, the first equation of the recurrence relation \eqref{b_rel} is satisfied and
      using the sequences defining the analytic functions $p(x)$ and $q(x)$, we have that the other equation becomes
      \begin{align}\label{b_rel_red}
        P(n) b_n &= -\sum_{k=0}^{n-1}\left[-\frac{k}{2^{n-k+1}} + \frac{1}{2^{n-k-2}}\right]b_k & & \notag\\
        &=\sum_{k=0}^{n-1}\left[\frac{k-8}{2^{n-k+1}}\right]b_k, &n&=1,2,\dots.
      \end{align}
      We can prove through induction that the above relation satisfies the formula
      \begin{align*}
        b_n &= \frac{-n^2+2n+3}{-2n^2+n}b_{n-1}, &n&=1,2,\dots.
      \end{align*}
      To see this we can note that
      \begin{align*}
        b_1 &= \frac{-1 +2 +3}{-2 + 1}b_0 = -4b_0
      \end{align*}
      and have established that the formula holds for $n=1$. Now suppose the formula
      holds for general $n > 1$. Using our supposition, we see from relation \eqref{b_rel_red} that
      \begin{align*}
        P(n + 1) b_{n+1} &= \sum_{k=0}^{n}\left[\frac{k-8}{2^{n-k+2}}\right]b_k \\
        &= \frac{1}{2}\sum_{k=0}^{n-1}\left[\frac{k-8}{2^{n-k+1}}\right]b_k + \frac{n-8}{4}b_n \\
        &= \left[\frac{P(n)}{2} + \frac{n-8}{4}\right]b_n.
      \end{align*}
      Performing some algebra on this expression we see that
      \begin{align*}
        b_{n+1} = \frac{-(n+1)^2+2(n+1)+3}{-2(n+1)^2+(n+1)}b_{n}
      \end{align*}
      and the formula holds for $n+1$ completing the proof.

      Note that $b_3 = 0$ which implies that $b_n = 0$ for $n\geq 3$ and that
      \begin{align*}
        b_n =
        \begin{cases}
          b_1 = -4b_0 \\
          b_2 = 2b_0 \\
          b_n = 0 &\text{for $n\geq 3$}
        \end{cases}.
      \end{align*}
      Therefore, using \eqref{lin_sol}, the solution to the differential equation \eqref{diff_eq} associated to the root $\alpha_2 = 0$
      is
      \begin{align}\label{y2_sol}
        y_2(x) &= \sum_{n=0}^\infty b_n(x+2)^{n} \notag \\
        &= b_0\left[1 -4(x+2) +2(x+2)^2\right]\notag\\
        &= b_0\left[2x^2+4x+1\right].
      \end{align}

      The general solution to the differential equation is then
      \begin{align}\label{gen_sol}
        y(x) = a_0\left[\frac{-(x+1)\sqrt{-x(x+2)}}{\sqrt{2}}\right] + b_0\left[2x^2+4x+1\right].
      \end{align}

      \item Note that \eqref{gen_sol} is the general solution to the BVP.
        So, for $-2 \leq x \leq -3/2$,
        \begin{align*}
          y(x) = a_0\left[\frac{-(x+1)\sqrt{-x(x+2)}}{\sqrt{2}}\right] + b_0\left[2x^2+4x+1\right]
        \end{align*}
        satisfies the differential equation. From the boundary conditions $y(-2) = y(-3/2) = 1$
        we see that
        \begin{align*}
          y(-2) &= b_0 = 1 \\
          y(-3/2) &= \frac{a_0\sqrt{3}}{4\sqrt{2}}  - \frac{b_0}{2} = 1
        \end{align*}
        from which we readily see that $a_0 = 2\sqrt{6}$ and $b_0 = 1$. Therefore,
        the solution to the BVP is
        \begin{align*}
          y(x) = 2\sqrt{6}\left[\frac{-(x+1)\sqrt{-x(x+2)}}{\sqrt{2}}\right] + \left[2x^2+4x+1\right].
        \end{align*}
  \end{enumerate}
\end{proof}
\newpage


% Problem 2
\begin{problem}
  \begin{enumerate}
    \item Transform the equation $x(x+2)y'' + (x+1)y' - 4y = 0$ to the form
      \begin{align}\label{transform}
        \ddot{y} + t^{-1}p(t)\dot{y} + t^{-2}q(t)y = 0
      \end{align}
      and use the result to determine whether the point at $\infty$ is an ordinary, regular
      singular, or irregular singular point for the original equation.
    \item Apply an appropriate method to equation \eqref{transform}  to obtain
      two series that represent linearly independent solutions of the original
      equation as $x \to +\infty$.
  \end{enumerate}
\end{problem}

\begin{proof}
  \begin{enumerate}
    \item In order to investigate the point at $+\infty$, we map the point at
      $+\infty$ into $0$ and identify the point at $0$ in the resulting equation.
      We can complete the mapping by making the following transformations
      \begin{align*}
        x &= \frac{1}{t} \\
        y' &= -t^2 \dot{y} \\
        y'' &= t^4 \ddot{y} + 2t^3 \dot{y}.
      \end{align*}
      Thus, the differential equation \eqref{diff_eq} becomes
      \begin{align*}
        Ly &= x(x+2)y'' + (x+1)y' - 4y \notag \\
        &= \left(\frac{2t+1}{t^2}\right)\left(t^4 \ddot{y} + 2t^3 \dot{y}\right) -t(t + 1)\dot{y} - 4y \notag \\
        &= t^2(2t+1)\ddot{y} + t(3t+1)\dot{y} - 4y.
      \end{align*}
      We can write this differential equation in the form
      \begin{align}\label{diff_eq_mapped}
        t^2(2t+1)\ddot{y} + t(3t+1)\dot{y} - 4y = \ddot{y} + \left[\frac{3t+1}{t(2t+1)}\right]\dot{y} + \left[-\frac{4}{t^2(2t+1)}\right]y=0.
      \end{align}
      Identifying $p(t) = (3t+1)/(2t+1)$ and $q(t) = -4/(2t+1)$, we see that the equation is written as
      \begin{align*}
        \ddot{y} + \left[\frac{3t+1}{t(2t+1)}\right]\dot{y} + \left[-\frac{4}{t^2(2t+1)}\right]y = \ddot{y} + \left[\frac{p(t)}{t}\right]\dot{y} + \left[\frac{q(t)}{t^2}\right]y = 0.
      \end{align*}
      Note that $p(t)$ and $q(t)$ are both analytic at $t=0$.
      Since $t^{-1}p(t)$ and $t^{-2}q(t)$ are not analytic at $t=0$ but both
      $t(t^{-1}p(t))$ and $t^2(t^{-2}q(t))$ are analytic at that point, we
      see that the point $t=0$ is a regular singular point. As a result
      we conclude that $x=+\infty$ is also a regular singular point.
    \item After identifying that the point $t=0$ is a regular singular point of
      equation \eqref{diff_eq_mapped} we can expect that at least one solution to the equation
      is of Frobenius form. We begin by finding the power series expansions
      of the functions $p(t)$ and $q(t)$. Using the following power series expansion
      of $f(t) = 1/(2t+1)$ about $t=0$,
      \begin{align*}
        \frac{1}{2t+1} = \sum_{n=0}^\infty (-2)^n t^n,
      \end{align*}
      we see that the power series expansions of the analytic functions $p(t)$
      and $q(t)$ about $t=0$ are
      \begin{align}\label{pq_expansion}
        p(t) &= \sum_{n=0}^{\infty}p_n t^n = 1 + \sum_{n=1}^\infty \left[3(-2)^{n-1} + (-2)^{n}\right]t^n  = 1 + \sum_{n=1}^\infty \left[(-1)^{n+1}2^{n-1}\right]t^n \notag \\
        q(t) &= \sum_{n=0}^{\infty}q_n t^n = \sum_{n=0}^\infty2(-2)^{n+1}t^n = \sum_{n=0}^\infty(-1)^{n+1}2^{n+2}t^n.
      \end{align}
      Identifying $p_0 = 1$ and $q_0 = -4$, the indicial polynomial associated to the differential equation \eqref{diff_eq_mapped} is
      \begin{align*}
        P(\alpha) = \alpha^2 + (p_0 - 1)\alpha + q_0 = (\alpha-2)(\alpha+2).
      \end{align*}
      The two roots to the indicial polynomial are $\alpha_1 = 2$ and $\alpha_2 = -2$. Since these roots
      differ by a positive integer, the solution to the differential equation associated to the root $\alpha_1$ exists in Frobenius
      form and further analysis is needed to determine if the other solution to the differential equation associated to the root $\alpha_2$
      is also in Frobenius form. Using the sequences defining the power series expansions of $p(t)$ and $q(t)$ and setting $N = \alpha_1 - \alpha_2 = 4$, if
      \begin{align*}
        -\sum_{k=0}^{N-1} \left[(k + \alpha_2)p_{N-k} + q_{N-k}\right]a_k = 0
      \end{align*}
      with $a_0 \neq 0$, then the solution to the differential equation associated to the root $\alpha_2$ is
      in Frobenius form, otherwise only the solution associated to the root $\alpha_1$ is
      in Frobenius form.

      We proceed by finding $y_1(t)$, the solution to the differential equation associated to
      the root $\alpha_1 = 2$. This solution is in Frobenius form, i.e.\
      \begin{align}\label{frob_sol}
        y_1(t) = \sum_{n=0}^\infty a_n t^{n+2}
      \end{align}
      where $a_0\neq 0$ and $a_n$ satisfies the recurrence relation
      \begin{align}
        P(2)a_0 &= 0 & & \notag \\
        P(n+2) a_n &= -\sum_{k=0}^{n-1}\left[(k + 2)p_{n-k} + q_{n-k}\right]a_k, &n&=1,2,\dots
      \end{align}
      and $p_n, q_n$ are the sequences defining the power series expansions of $p(t)$ and $q(t)$.
      Thus, solving the recurrence relation completely identifies the first solution.
      The first equation of the recurrence relation is satisfied since $\alpha_1=2$ is a root
      of the indicial polynomial $P(z) = (z-2)(z+2)$. The other recurrence relation
      reduces to
      \begin{align}\label{a_y1_rel}
        P(n+2)a_n &= -\sum_{k=0}^{n-1}\left[(k + 2)p_{n-k} + q_{n-k}\right]a_k & & \notag \\
        &= \sum_{k=0}^{n-1} (-1)^{n-k+2}2^{n-k-1}(k+10)a_k&n&=1,2,\dots.
      \end{align}
      We can prove through induction that the above relation satisfies the formula
      \begin{align*}
        a_n &= \frac{-2n^2-5n-3}{n^2+4n}a_{n-1}, &n&=1,2,\dots.
      \end{align*}
      To see this we can note that
      \begin{align*}
        a_1 &= \frac{-2 - 5 - 3}{1 + 4}a_0 = -2a_0
      \end{align*}
      and have established that the formula holds for $n=1$. Now suppose the formula
      holds for general $n > 1$. Using our supposition, we see from relation \eqref{a_y1_rel} that
      \begin{align*}
        P(n + 3) a_{n+1} &= \sum_{k=0}^{n}(-1)^{n-k+2}2^{n-k-1}(k+10)a_k \\
        &= -2\sum_{k=0}^{n-1}(-1)^{n-k+2}2^{n-k-1}(k+10)a_k - (n+10)a_n \\
        &= -\left[2P(n+2) + (n+10) \right]a_n.
      \end{align*}
      Performing some algebra on this equation we see that
      \begin{align*}
        a_{n+1} = \frac{-2(n+1)^2-5(n+1)-3}{(n+1)^2+4(n+1)}a_{n-1}
      \end{align*}
      and the formula holds for $n+1$ completing the proof. Mathematica reports that
      the solution to this new recurrence relation is
      \begin{align*}
        a_n &= \left[\frac{\Gamma(5)}{\Gamma(5/2)}\frac{(-2)^{n}(n+1)\Gamma(n+5/2)}{\Gamma(n+5)}\right]a_0 \notag \\
        &= \left[\frac{32(-2)^{n}(n+1)\Gamma(n+5/2)}{\sqrt{\pi}\Gamma(n+5)}\right]a_0.
      \end{align*}
      Therefore, the solution to the differential equation associated to the root $\alpha_1=2$ is
      \begin{align*}
        y_1(t) &= a_0 \sum_{n=0}^\infty \left[\frac{32(-2)^{n}(n+1)\Gamma(n+5/2)}{\sqrt{\pi}\Gamma(n+5)}\right]t^{n+2} \\
        &= \begin{cases}
          0 & \text{for $t=0$}\\
          a_0\left[\frac{4\left(2+4t+t^2-2\sqrt{2t+1}-2t\sqrt{2t+1}\right)}{t^2}\right] & \text{for $0 < |t|\ < 1/2$}
          \end{cases}.
      \end{align*}
      which has radius of convergence $1/2$ centered at $t=0$.

      %Find y_2 solution
      To find the second solution we generalize the Frobenius solution
      and substitute the solution into the original differential equation to obtain
      \begin{align*}
        Ly(t,\alpha) = a_0 t^{\alpha - 2}P(\alpha).
      \end{align*}
      Note that if $\alpha=\alpha_1 = 2$, then the solution is the solution already obtained to the homogeneous equation $Ly(t,\alpha) = 0$.
      Differentiating both sides of the equation with respect to $\alpha$ yields that
      \begin{align}\label{gen_eqn}
        L\left[\left.\frac{\partial}{\partial\alpha}y(t,\alpha)\right|_{\alpha=\alpha_1}\right] = a_0P'(\alpha_1) t^{\alpha_2+N-2}
      \end{align}
      so that $\left.\frac{\partial}{\partial\alpha}y(t,\alpha)\right|_{\alpha=\alpha_1}$ is a particular solution of the above differential equation.
      We can construct a second particular solution and subtract it from this particular solution to obtain the solution to the homogeneous equation. Note that the second particular
      solution has a Frobenius expansion $\sum_{n=0}^{\infty}c_n t^{n+\alpha_2}$.
      Using this form of the solution and substituting into \eqref{gen_eqn} and equating coefficients with $t^{\alpha_2+N-2}$ yields the following
      relations
      \begin{align}\label{sec_rel}
        P(n + \alpha_2)c_n &=-\sum_{k=0}^{n-1} [(k + \alpha_2)p_{n-k} + q_{n-k}]c_k &n&\neq0,N \notag\\
        a_0 &= \frac{1}{P'(\alpha_1)}\sum_{k=0}^{N-1}[(k + \alpha_2)p_{N-k} + q_{N-k}]c_k &n&=N
      \end{align}
      where $c_0 \neq 0$ and $c_N\neq 0$ are arbitrary. The second solution to the homogeneous differential equation  $Ly=0$ is then
      \begin{align*}
        y_2(t) &= \sum_{n=0}^\infty c_n t^{n+\alpha_2} - \left.\frac{\partial}{\partial\alpha}y(t,\alpha)\right|_{\alpha=\alpha_1} \\
        &= \sum_{n=0}^\infty c_n t^{n+\alpha_2} - \log t\sum_{n=0}^{\infty}a_n t^{n+\alpha_1} - \sum_{n=0}^\infty\left.\frac{\partial a_n(\alpha)}{\partial \alpha}\right|_{\alpha=\alpha_1} t^{n+\alpha_1}
      \end{align*}
      where $a_n$ is the sequence of coefficients defining the first solution to the homogeneous differential equation.

      We begin by determining the sequence $c_n$. For this particular differential equation
      we have that $\alpha_1 = 2$, $\alpha_2 = -2$, and $N = \alpha_1 - \alpha_2 = 4$.
      We can see that the first equation of the recurrence relation \eqref{sec_rel} becomes
      \begin{align*}
        P(n - 2)c_n &=-\sum_{k=0}^{n-1} [(k - 2)p_{n-k} + q_{n-k}]c_k && \\
        &= \sum_{k=0}^{n-1} (-1)^{n-k+2}2^{n-k-1}(k+6)c_k &n&\neq0,4.
      \end{align*}
      We can see from this relation that $c_1 = 2c_0$, $c_2 = c_0 /2$, and $c_3=0$.
      This implies that the second equation of the recurrence relation \eqref{sec_rel} becomes
      \begin{align*}
        a_0 &= \frac{1}{P'(2)}\sum_{k=0}^{4-1}[(k - 2)p_{4-k} + q_{4-k}]c_k = \frac{1}{4} \sum_{k=0}^3(-1)^{4-k}2^{3-k}(k+6)c_k = 0.
      \end{align*}

      Continuing from the first equation for $n>4$, we see that the recurrence relation becomes
      \begin{align*}
        P(n - 2)c_n &= \sum_{k=0}^{n-1} (-1)^{n-k+2}2^{n-k-1}(k+6)c_k \\
        &= \sum_{k=0}^{4-1} (-1)^{n-k+2}2^{n-k-1}(k+6)c_k + \sum_{k=4}^{n-1} (-1)^{n-k+2}2^{n-k-1}(k+6)c_k\\
        &= \sum_{k=4}^{n-1} (-1)^{n-k+2}2^{n-k-1}(k+6)c_k
      \end{align*}
      since $\sum_{k=0}^{4-1} (-1)^{n-k+2}2^{n-k-1}(k+6)c_k = (-1)^{n-1}2^{n-3}(24c_0 - 28c_0 + 4 c_0) = 0$ from our definitions of
      $c_k$ for $0\leq k \leq 3$. Thus $c_n$ only depends on $c_4$ for $n>4$.
      In a very similar fashion as was shown for the sequence $a_n$, we see that
      \begin{align*}
        c_n = \frac{-2n^2 +11n-15}{n^2-4n}c_{n-1} \quad \text{for $n > 4$}.
      \end{align*}
      Mathematica reports that the solution to this recurrence relation
      is
      \begin{align*}
        c_n &= c_4\left[\frac{\Gamma(5)}{\Gamma(5/2)}\frac{(-1)^n 2^{n-4}(n-3)\Gamma(n-4+5/2)}{\Gamma(n-4+5)}\right] \\
        &= c_4\left[\frac{32(-1)^n 2^{n-4}(n-3)\Gamma(n-4+5/2)}{\sqrt{\pi}\Gamma(n-4+5)}\right] \\
      \end{align*}
      Thus, we have that $c_0, c_4 \neq 0$ and
      \begin{align*}
        c_n =
        \begin{cases}
          c_1 = 2c_0& \\
          c_2 = c_0/2&\\
          c_3 = 0&\\
          c_n = c_4\left[\frac{32(-1)^n 2^{n-4}(n-3)\Gamma(n-4+5/2)}{\sqrt{\pi}\Gamma(n-4+5)}\right] & \text{for $n\geq 4$}
        \end{cases}.
      \end{align*}
      Therefore, we have that the solution associated to the root $\alpha_2 = -2$ is
      \begin{align*}
        y_2(t) &= c_0t^{-2} + 2c_0 t^{-1} + \frac{c_0}{2} + c_4\sum_{n=4}^\infty\left[\frac{32(-1)^n 2^{n-4}(n-3)\Gamma(n-4+5/2)}{\sqrt{\pi}\Gamma(n-4+5)}\right]t^{n-2} \\
        &= c_0\left[t^{-2} + 2t^{-1} + \frac{1}{2}\right] + c_4 \left[\frac{4(2+4t+t^2-2\sqrt{2t+1}-2t\sqrt{2t+1})}{t^2}\right]
      \end{align*}
      and we have found both linearly independent solutions.
  \end{enumerate}
\end{proof}
\newpage


% Problem 3
\begin{problem}+
  Find the first three terms in the asymptotic expansion as $x \to +\infty$
  of a solution of the equation
  \begin{align*}
    Ly = y''' + \frac{y'}{x^3} = x.
  \end{align*}
\end{problem}

\begin{proof}
  We investigate the leading behavior of the solution $y(x)$ as $x\to + \infty$
  by employing the method of dominant balance. We begin by checking the balance
  $y''' \ll y'/x^3$ and $y'/x^3 \sim x$ as $x\to + \infty$. Thus, we have the asymptotic differential equation
  $y' \sim x^4$, of which the solution is $y \sim x^5/5$ as $x\to + \infty$. So, we see that
  \begin{align}\label{first_balance}
    y(x) = \frac{x^5}{5} + C_1(x), \qquad C_1(x) \ll x^5, \qquad (x\to+\infty).
  \end{align}
  We must check to see if this solution satisfies the assumption that $y''' \ll y'/x^3$ as $x\to + \infty$.
  If $y(x)$ is of the form presented in \eqref{first_balance}, then
  \begin{align*}
    y'(x) &= x^4 + C_1'(x), \qquad C_1'(x) \ll x^4, \qquad (x\to+\infty)  \\
    y'''(x) &= 12x^2 + C_1'''(x), \qquad C_1'''(x) \ll x^2, \qquad (x\to+\infty)
  \end{align*}
  Note, that these equations contradict the assumption that $y''' \ll y'/x^3$ as $x\to + \infty$, so this balance cannot happen.

  We proceed by checking the balance $y''' \sim x$ and $y' / x^3 \ll x$ as $x\to + \infty$. The solution to
  this asymptotic differential equation is given by $y \sim x^4 / 24$ as $x \to + \infty$. So we have that
  \begin{align}\label{second_balance}
    y(x) = \frac{x^4}{24} + C_1(x), \qquad C_1(x) \ll x^4, \qquad (x\to+\infty).
  \end{align}
  We now must check to see if this solution agrees with the assumption that $y' / x^3 \ll x$ as $x\to + \infty$.
  From equation \eqref{second_balance} we see that
  \begin{align*}
    y'(x) = \frac{x^3}{6} + C_1'(x), \qquad C_1'(x) \ll x^3, \qquad (x\to+\infty)
  \end{align*}
  from which we conclude that $y'(x)/x^3$ is indeed much much less than $x$ as $x\to + \infty$.
  Therefore, we have that the leading behavior of the solution is given by \eqref{second_balance}.

  We continue by using this form of $y(x)$ and replacing it into the original differential equation so that
  \begin{align*}
    Ly = C_1''' + \frac{C_1'}{x^3} = -1/6, \qquad C_1(x) \ll x^4, \qquad (x\to+\infty).
  \end{align*}
  Continuing with the method of dominant balance and taking cue from the previous equation,
  we check the balance $C_1''' \sim -1/6$ and $C_1'/x^3 \ll -1/6$ as $x\to + \infty$.
  The solution to this asymptotic differential equation is given by $C_1 \sim - x^3/36$ as $x\to + \infty$. Thus,
  \begin{align}\label{third}
    C_1(x) = -\frac{x^3}{36} + C_2(x), \qquad C_2(x) \ll x^3, \qquad (x\to+\infty).
  \end{align}
  We proceed by checking this solution against the assumption that $C_1'/x^3 \ll -1/6$. From equation \eqref{third} we see that
  \begin{align*}
    C_1'(x) = -\frac{x^2}{12} + C_2'(x), \qquad C_2'(x) \ll x^2, \qquad (x\to+\infty).
  \end{align*}
  This indeed shows that $C_1' / x^3 \ll -1/6$ and this solution is consistent with our assumptions.
  Thus, we have so far that the leading behavior of the solution to $Ly=x$ is
  \begin{align*}
    y(x) = \frac{x^4}{24} - \frac{x^3}{36} + C_2(x), \qquad C_2(x) \ll x^3, \qquad (x\to+\infty).
  \end{align*}

  We now peel off one final term by assuming $y(x)$ is of the form above so that the original
  differential equation becomes
  \begin{align*}
    Ly = C_2''' + \frac{C_2'}{x^3} = \frac{1}{12x}, \qquad C_2(x) \ll x^3, \qquad (x\to+\infty).
  \end{align*}
  Using the method of dominant balance we assume that $C_2''' \sim 1/(12x)$ and $C_2'/x^3 \ll 1/(12x)$ as $x\to + \infty$.
  The solution to this asymptotic differential equation is given by $C_2 \sim -x^2/16 +(x^2\log x)/24$ as $x\to + \infty$. Thus,
  \begin{align}\label{fourth}
    C_2(x) = -\frac{x^2}{16} +\frac{x^2\log x}{24} + C_3(x), \qquad C_3(x) \ll x^2\log x, \qquad (x\to+\infty).
  \end{align}
  From equation \eqref{fourth} we see that
  \begin{align*}
    C_2'(x) =-\frac{x}{12} + \frac{x\log x}{12} + C_3'(x), \qquad C_3'(x) \ll x + 2x\log x, \qquad (x\to+\infty)
  \end{align*}
  which indeed satisfies $C_2'/x^3 \ll 1/(12x)$ as $x\to + \infty$.

  Therefore, we have that the first three terms in the asymptotic expansion of $Ly = x$ as $x \to +\infty$ is
  \begin{align*}
    y(x) = \frac{x^4}{24} - \frac{x^3}{36} -\frac{x^2}{16} +\frac{x^2\log x}{24} + C_3(x), \qquad C_3(x) \ll x^2\log x.
  \end{align*}
\end{proof}


\end{document}