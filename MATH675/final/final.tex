\documentclass[12pt]{article}
\usepackage[margin=1in]{geometry}
\usepackage{amsmath, amsthm, amssymb, amsfonts, enumitem, graphicx}
\usepackage{fancyhdr}
\usepackage{breqn}

\theoremstyle{definition}
\newtheorem{problem}{Problem}
\renewcommand*{\proofname}{Solution}
\renewcommand{\theenumi}{\alph{enumi}}

\newcommand{\vect}[1]{\boldsymbol{#1}}

\newenvironment{case}{%
  \let\olditem\item%
  \renewcommand\item[1][]{\olditem \textbf{##1} \\}%
  \begin{enumerate}[label=\textbf{Case \arabic*:},itemindent=*,leftmargin=0em]}{\end{enumerate}%
}

\pagestyle{fancy}
\fancyhf{}
\rhead{Test 2}
\lhead{Matthew Tiger}
\cfoot{\thepage}


\title{Test 2}
\author{Matthew Tiger}


\begin{document}


\maketitle


% Problem 1
\begin{problem}
  Approximate the roots of $\varepsilon x^3 + x^2 +x - 2 = 0$, $\varepsilon \to 0^+$, with
  precision $O(\varepsilon^2)$.
\end{problem}

\begin{proof}
  When $\varepsilon = 0$, the unperturbed equation only has two roots while the original equation
  has three roots implying that this is a singular perturbation problem.

  To find the other root of this equation we must employ the method of dominant balance on the
  original equation. There are six possible two-term balances to consider as $\varepsilon \to 0^+$:
  \begin{enumerate}
    \item Suppose that $\varepsilon x^3 \sim x^2$ is the dominant balance and that $x \ll x^2$, $1 \ll x^2$.
      If $\varepsilon x^3 \sim x^2$ as $\varepsilon \to 0^+$, then $x = O(\varepsilon^{-1})$, which is consistent
      with the assumptions that $x \ll x^2$ and $1 \ll x^2$ and the balance itself is consistent.

    \item Suppose that $\varepsilon x^3 \sim x$ is the dominant balance and that $x^2 \ll x$, $1 \ll x$.
      If $\varepsilon x^3 \sim x$ as $\varepsilon \to 0^+$, then $x = O(\varepsilon^{-1/2})$.
      However, if $x = O(\varepsilon^{-1/2})$, then
      \begin{align*}
        \frac{x^2}{x} = \frac{\left(\varepsilon^{-1/2}\right)^2}{\varepsilon^{-1/2}} = \varepsilon^{-1/2}
      \end{align*}
      which implies that as $\varepsilon \to 0^+$ the assumption that $x^2 \ll x$ is false and that this balance is inconsistent.

    \item Suppose that $\varepsilon x^3 \sim 2$ is the dominant balance and that $x^2 \ll 1$, $x \ll 1$.
      If $\varepsilon x^3 \sim 2$ as $\varepsilon \to 0^+$, then $x = O(\varepsilon^{-1/3})$, which
      implies that as $\varepsilon \to 0^+$ the assumption that $x \ll 1$ is false and that this balance is inconsistent.

    \item Suppose that $x^2 \sim x$ is the dominant balance and that $\varepsilon x^3 \ll x$, $1 \ll x$.
      If $x^2 \sim x$ as $\varepsilon \to 0^+$, then $x \sim 1$ and $x = O(1)$, which is consistent with the assumptions
      that $\varepsilon x^3 \ll 1$ and $x^2 \ll 1$ and the balance is consistent. Using this balance
      will recover the root $x=1$ from the unperturbed equation and the root can be expanded with a perturbation series in $\varepsilon$ in the normal way.

    \item Suppose that $x^2 \sim 2$ is the dominant balance and that $\varepsilon x^3 \ll 1$, $x \ll 1$.
      If $ x^2 \sim 2$ as $\varepsilon \to 0^+$, then $x = O(1)$, which
      implies that as $\varepsilon \to 0^+$ the assumption that $x \ll 1$ is false and that this balance is inconsistent.

    \item Suppose that $x \sim -2$ is the dominant balance and that $\varepsilon x^3 \ll 1$, $x^2 \ll 1$.
      If $x \sim -2$ as $\varepsilon \to 0^+$, then $x = O(1)$, which is consistent with the assumptions
      that $\varepsilon x^3 \ll 1$ and $x^2 \ll 1$ and the balance is consistent. Using this balance
      will recover the root $x=-2$ from the unperturbed equation and the root can be expanded with a perturbation series in $\varepsilon$ in the normal way.
  \end{enumerate}
  Assuming the balance in case a., we see that the roots of the equation are of order $\varepsilon^{-1}$.
  Making the transformation $x = \varepsilon^{-1}y$ we see that the original equation becomes
  \begin{align}\label{new_pert}
    y^3 + y^2 + \varepsilon y -2 \varepsilon^2 = 0.
  \end{align}
  Suppose that the roots of the equation \eqref{new_pert} can be expressed in terms of $\varepsilon$, i.e.\
  \begin{align*}
    y = \sum_{n=0}^\infty a_n \varepsilon^n.
  \end{align*}
  Suppose $y = a_0 + a_1 \varepsilon + a_2 \varepsilon^2 + a_3 \varepsilon^3 + O(\varepsilon^4)$. Substituting $y$ into \eqref{new_pert} and equating coefficients of $\varepsilon$,
  we see that the following equations must be satisfied:
  \begin{align*}
    \begin{array}{ll}
      a_0^2 + a_0^3 &= 0 \\
      a_0 + 2 a_0 a_1 + 3 a_0^2 a_1 &= 0\\
      -2 + a_1 + a_1^2 + 3 a_0 a_1^2 + 2 a_0 a_2 + 3 a_0^2 a_2&= 0 \\
      a_1^3 + a_2 + 2 a_1 a_2 + 6 a_0 a_1 a_2 + 2 a_0 a_3 + 3 a_0^2 a_3 &= 0
    \end{array}
  \end{align*}

  When $a_0 = -1$, we see from the second equation that $a_1 = 1$, from the third equation that
  $a_2 = 3$, and from the fourth equation that $a_3 = 8$. Thus, the root $x_1$ to the original equation is given by
  \begin{align*}
    x_1 = \varepsilon^{-1}y = \varepsilon^{-1}\left(-1 + \varepsilon + 3 \varepsilon^2 + 8 \varepsilon^3 + O(\varepsilon^4)\right) = -\varepsilon^{-1} + 1 + 3 \varepsilon + 8 \varepsilon^2 + O(\varepsilon^3).
  \end{align*}
  accurate to precision $O(\varepsilon^2)$.

  When $a_0 = 0$, we have two possibilities from the above system of non-linear equations, either
  $a_1 = -2$, $a_2 = -8/3$, $a_3 = 0$ or $a_1=1$, $a_2 = -1/3$, $a_3 = 0$.
  Thus, the second and third roots $x_2, x_3$ to the original equation are given by
  \begin{align*}
    x_2 = \varepsilon^{-1}y = \varepsilon^{-1}\left(-2\varepsilon -\frac{8}{3} \varepsilon^2  + O(\varepsilon^4)\right) = -2 -\frac{8}{3} \varepsilon  + O(\varepsilon^3).
  \end{align*}
  and
  \begin{align*}
    x_3 = \varepsilon^{-1}y = \varepsilon^{-1}\left(\varepsilon -\frac{1}{3} \varepsilon^2  + O(\varepsilon^4)\right) = 1 -\frac{1}{3} \varepsilon  + O(\varepsilon^3).
  \end{align*}
  both accurate to precision $O(\varepsilon^2)$. Note that when $a_0 =0$ we have recovered the roots to the unperturbed equation and their actual order is $\varepsilon$.

  Therefore, the three roots to the equation $\varepsilon x^3 + x^2 +x - 2 = 0$, $\varepsilon \to 0^+$ are given by
  \begin{align*}
    x_1 &= -\varepsilon^{-1} + 1 + 3 \varepsilon + 8 \varepsilon^2 + O(\varepsilon^3) \\
    x_2 &= -2 -\frac{8}{3} \varepsilon  + O(\varepsilon^3) \\
    x_3 &= 1 -\frac{1}{3} \varepsilon  + O(\varepsilon^3)
  \end{align*}
  all accurate to precision $O(\varepsilon^2)$.
\end{proof}
\newpage


\begin{problem}
  Let
  \begin{align*}
    I(x) = \int_0^\infty \frac{t^{x-1}e^{-t}}{t^2+x^2} dt.
  \end{align*}

  \begin{enumerate}
    \item Use the method of movable maxima to transform $I(x)$ into an integral
      of the form
      \begin{align*}
        J(x) = g(x) \int_{0}^\infty f(s) e^{-x\phi(s)} ds
      \end{align*}
      which is amenable to analysis by Laplace's method.
    \item Use the result of part a.\ to obtain a three term approximation to $I(x)$ for
      $x\to +\infty$ by Laplace's method.
  \end{enumerate}
\end{problem}

\begin{proof}
\end{proof}
\newpage


\begin{problem}
  Use the method of stationary phase to obtain the leading behavior for $x\to+\infty$
  of the integral
  \begin{align*}
    I(x) = \int_0^1 e^{i x (t-\sin t)} dt.
  \end{align*}
\end{problem}

\begin{proof}
  We begin by noting that the integral $I(x)$ is a generalized Fourier integral
  which can be written as
  \begin{align*}
    I(x) = \int_0^1 f(t) e^{ix\psi(t)} dt
  \end{align*}
  where $f(t) = 1$ and $\psi(t) = t - \sin t$. The leading asymptotic behavior of
  such integrals as $x\to+\infty$ may be found, in general, using integration by parts.
  However, this method may fail at \emph{stationary points}, i.e.\ any point on
  the interval of definition such that $\psi'(t) = 0$. Note that $\psi'(t) = 0$
  when $t=2 \pi k$ for $k\in \mathbb{Z}$. Thus, for the integral
  $I(x)$, we note that $t = 0$ is the only stationary point.
  Thus, we proceed by writing $I(x)$ as follows:

  \begin{align*}
    I(x) = I_1(x) + I_2(x) = \int_0^\varepsilon f(t) e^{ix\psi(t)} dt + \int_\varepsilon^1 f(t) e^{ix\psi(t)} dt
  \end{align*}
  for some $\varepsilon > 0$.

  Since $I_2(x)$ does not have any stationary points and the
  function $f(t) = 1 \in L^1$ over the interval $[0, 1]$, we have by the Riemann-Lebesgue lemma that $I_2(x) \to 0$ as $x \to +\infty$. Thus,
  as $x\to +\infty$,
  \begin{align*}
    I(x) \sim I_1(x) = \int_0^\varepsilon f(t) e^{ix\psi(t)} dt = \int_0^\varepsilon e^{ix (t-\sin t)} dt.
  \end{align*}
  The leading behavior of $I(x)$ can be obtained by replacing $f(t)$ with $f(0) = 1$
  and $\psi(t) = t - \sin t$ with
  \begin{align*}
    \psi(0) + \frac{\psi'''(0)t^3}{6} = (0 - \sin 0) + \frac{(\cos 0) t^3}{6} = \frac{t^3}{6},
  \end{align*}
  since these are the parts that contribute the most to the integral. Thus,
  \begin{align*}
    I(x) \sim \int_0^\varepsilon e^{ix t^3/6} dt.
  \end{align*}
  Replacing $\varepsilon$ with $\infty$ introduces error terms that vanish as $x \to +\infty$ so that
  \begin{align*}
    I(x) \sim \int_0^\infty e^{ix t^3/6} dt.
  \end{align*}

  Making the substitution $u = x^{1/3}t/6^{1/3}$, we see that $du = (x^{1/3}/6^{1/3}) dt$, $u^3 = x t^3 / 6$ and that
  \begin{align*}
    I(x) \sim \int_0^\infty e^{ix t^3/6} dt = \left(\frac{6}{x}\right)^{1/3}\int_0^\infty e^{iu^3} du.
  \end{align*}
  Using the identity
  \begin{align*}
    \int_0^\infty e^{iu^3} du = e^{i \pi/6}\Gamma(4/3),
  \end{align*}
  we see that
  \begin{align*}
    I(x) \sim \left(\frac{6}{x}\right)^{1/3}\int_0^\infty e^{iu^3} du = \left(\frac{6}{x}\right)^{1/3}e^{i \pi/6}\Gamma(4/3).
  \end{align*}

  Therefore, as $x \to +\infty$,
  \begin{align*}
    I(x) \sim \left(\frac{6}{x}\right)^{1/3}e^{i \pi/6}\Gamma(4/3).
  \end{align*}
\end{proof}

\end{document}