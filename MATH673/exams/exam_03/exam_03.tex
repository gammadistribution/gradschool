\documentclass[12pt]{article}
\usepackage[margin=1in]{geometry}
\usepackage{amsmath, amsthm, amssymb, amsfonts, enumitem, graphicx, xifthen, mathrsfs, fancyhdr}

\theoremstyle{definition}
\newtheorem{innercustomthm}{Problem}
\newenvironment{problem}[1]
  {\renewcommand\theinnercustomthm{#1}\innercustomthm}
  {\endinnercustomthm}

\renewcommand*{\proofname}{Solution}
\renewcommand{\theenumi}{\roman{enumi}}
\newcommand{\fourier}[1]{\mathscr{F}\left\{ {#1} \right\}}
\newcommand{\fourierinv}[1]{\mathscr{F}^{-1}\left\{ {#1} \right\}}
\newcommand{\fouriercos}[1]{\mathscr{F}_c\left\{ {#1} \right\}}
\newcommand{\fouriercosinv}[1]{\mathscr{F}_c^{-1}\left\{ {#1} \right\}}
\newcommand{\fouriersin}[1]{\mathscr{F}_s\left\{ {#1} \right\}}
\newcommand{\fouriersininv}[1]{\mathscr{F}_s^{-1}\left\{ {#1} \right\}}
\newcommand{\der}[2]{
  \ifthenelse{\equal{#2}{}}{
    \frac{d}{d #1}
  }{
    \frac{d^{#2}}{d #1^{#2}}
  }
}
\newcommand{\norm}[1]{\left\Vert {#1} \right\Vert}
\renewcommand{\vec}[1]{\boldsymbol{#1}}
\newcommand{\laplace}[1]{\mathscr{L}\left\{ {#1} \right\}}
\newcommand{\laplaceinv}[1]{\mathscr{L}^{-1}\left\{ {#1} \right\}}
\newcommand{\hankel}[2]{\mathscr{H}_{ {#1} }\left\{ {#2} \right\}}
\newcommand{\hankelinv}[2]{\mathscr{H}^{-1}_{ {#1} }\left\{ {#2} \right\}}
\newcommand{\mellin}[1]{\mathscr{M}\left\{ {#1} \right\}}
\newcommand{\mellininv}[1]{\mathscr{M}^{-1}\left\{ {#1} \right\}}
\newcommand{\Z}[1]{\text{Z}\left\{ {#1} \right\}}
\newcommand{\Zinv}[1]{\text{Z}^{-1}\left\{ {#1} \right\}}


\pagestyle{fancy}
\fancyhf{}
\rhead{Exam 3}
\lhead{Matthew Tiger}
\cfoot{\thepage}


\title{Exam 3}
\author{Matthew Tiger}


\begin{document}


\maketitle


\begin{problem}{1}
  Find the Fourier Transforms of the following functions:
  \begin{enumerate}
    \item[a.] $f(x) = x^2 e^{-a|x|}$, $a > 0$,
    \item[b.] $\displaystyle f(x) = \left(1 - \frac{|x|}{2}\right)H\left(1 - \frac{|x|}{2}\right)$.
  \end{enumerate}
\end{problem}

\begin{proof}
  Recall that if $f(x) \in L^1(\mathbb{R})$, then the Fourier Transform of $f$ is
  defined to be
  \begin{align}\label{fourierdef}
    \fourier{f(x)} = F(k) = \frac{1}{\sqrt{2\pi}}\int_{-\infty}^\infty f(x) e^{-ikx} dx.
  \end{align}
  \begin{enumerate}
    \item[a.] If $f(x) = x^2 e^{-a|x|}$, $a > 0$, we see from \eqref{fourierdef}
      that, by definition, the Fourier Transform of $f$ is given by
      \begin{align*}
        \fourier{f(x)} = F(k) &= \frac{1}{\sqrt{2\pi}}\int_{-\infty}^\infty x^2 e^{-a|x|} e^{-ikx} dx \\
        &= \frac{1}{\sqrt{2\pi}}\left[\int_{-\infty}^0 x^2 e^{-(-a+ik)x} dx + \int_{0}^\infty x^2 e^{-(a+ik)x} dx\right].
      \end{align*}
      Now, we see by integration by parts that
      \begin{align*}
        \int_c^d x^2 e^{-(b+ik)x}dx
        &= \left.-\frac{x^2}{b+ik}e^{-(b+ik)x}\right\rvert_c^d + \frac{2}{b+ik}\int_c^d x e^{-(b+ik)x}dx \\
        &= \left.-\frac{x^2}{b+ik}e^{-(b+ik)x}\right\rvert_c^d + \frac{2}{b+ik}\left[\left.-\frac{x}{b+ik}e^{-(b+ik)x}\right\rvert_c^d + \frac{1}{b+ik}\int_c^d e^{-(b+ik)x}dx\right]\\
        &=\left.-\frac{x^2}{b+ik}e^{-(b+ik)x}\right\rvert_c^d + \frac{2}{b+ik}\left[\left.-\frac{x}{b+ik}e^{-(b+ik)x}\right\rvert_c^d - \left.\frac{1}{(b+ik)^2}e^{-(b+ik)x}\right\rvert_c^d\right].
      \end{align*}
      Thus,
      \begin{align*}
        \int_{-\infty}^0 x^2 e^{-(-a+ik)x} dx = -\frac{2}{(-a+ik)^3}
      \end{align*}
      and
      \begin{align*}
        \int_{0}^\infty x^2 e^{-(a+ik)x} dx = \frac{2}{(a+ik)^3}.
      \end{align*}
      Therefore,
      \begin{align*}
        \fourier{f(x)} = F(k)
        &= \frac{1}{\sqrt{2\pi}}\left[\int_{-\infty}^0 x^2 e^{-(-a+ik)x} dx + \int_{0}^\infty x^2 e^{-(a+ik)x} dx\right]\\
        &= \frac{1}{\sqrt{2\pi}}\left[-\frac{2}{(-a+ik)^3} + \frac{2}{(a+ik)^3}\right] \\
        &= \frac{1}{\sqrt{2\pi}}\left[\frac{4a(a^2-3k)}{(a^2+k^2)^3}\right] \\
        &= \sqrt{\frac{2}{\pi}}\left[\frac{2a(a^2-3k)}{(a^2+k^2)^3}\right].
      \end{align*}
    \item[b.] Recall that the Heaviside function $H$ is defined as
      \begin{align*}
        H(x) = \begin{cases}
          1 & \text{if $x > 0$}\\
          0 & \text{if $x < 0$}.
        \end{cases}
      \end{align*}
      Thus,
      \begin{align*}
        H\left(1-\frac{|x|}{2}\right) = \begin{cases}
          1 & \text{if $|x| < 2$}\\
          0 & \text{if $|x| > 2$}.
        \end{cases}
      \end{align*}

      If $\displaystyle f(x) = \left(1 - \frac{|x|}{2}\right)H\left(1 - \frac{|x|}{2}\right)$, we see from \eqref{fourierdef}
      that, by definition, the Fourier Transform of $f$ is given by
      \begin{align*}
        \fourier{f(x)} = F(k)
        &= \frac{1}{\sqrt{2\pi}}\int_{-\infty}^\infty\left(1 - \frac{|x|}{2}\right)H\left(1 - \frac{|x|}{2}\right) e^{-ikx} dx \\
        &= \frac{1}{\sqrt{2\pi}}\int_{-2}^2\left(1 - \frac{|x|}{2}\right)e^{-ikx} dx \\
        &= \frac{1}{\sqrt{2\pi}}\left[\int_{-2}^0 \left(1 + \frac{x}{2}\right) e^{-ikx}dx + \int_{0}^2 \left(1 - \frac{x}{2}\right)e^{-ikx} dx\right].
      \end{align*}
      Now, we see by integration by parts that
      \begin{align*}
        \int_c^d \left(1 \pm \frac{x}{2}\right) e^{-ikx}dx
        &= \left.\frac{i}{k}e^{-ikx}\right\rvert_c^d \pm \frac{1}{2}\int_c^d x e^{-ikx}dx \\
        &= \left.\frac{i}{k}e^{-ikx}\right\rvert_c^d \pm \frac{1}{2}\left[\left.\frac{ix}{k}e^{-ikx} \right\rvert_c^d - \frac{i}{k}\int_c^d e^{-ikx}dx\right] \\
        &= \left.\frac{i}{k}e^{-ikx}\right\rvert_c^d \pm \frac{1}{2}\left[\left.\frac{ix}{k}e^{-ikx} \right\rvert_c^d + \left.\frac{e^{-ikx}}{k^2} \right\rvert_c^d\right].
      \end{align*}
      Thus,
      \begin{align*}
        \int_{-2}^0 \left(1 + \frac{x}{2}\right) e^{-ikx}dx = \frac{1-e^{2ik}+2ik}{2k^2}
      \end{align*}
      and
      \begin{align*}
        \int_{0}^2 \left(1 - \frac{x}{2}\right) e^{-ikx}dx = \frac{1-e^{-2ik}-2ik}{2k^2}
      \end{align*}
      Therefore, using the definition of the complex exponential
      and various trigonometric identities, we have that
      \begin{align*}
        \fourier{f(x)} = F(k)
        &= \frac{1}{\sqrt{2\pi}}\left[\int_{-2}^0 \left(1 + \frac{x}{2}\right) e^{-ikx}dx + \int_{0}^2 \left(1 - \frac{x}{2}\right)e^{-ikx} dx\right] \\
        &= \frac{1}{\sqrt{2\pi}}\left[\frac{1-e^{2ik}+2ik}{2k^2} + \frac{1-e^{-2ik}-2ik}{2k^2} \right] \\
        &= \frac{1}{\sqrt{2\pi}}\left[\frac{1}{k^2} - \frac{e^{-2ik}+e^{2ik}}{2k^2} \right]\\
        &= \frac{1}{\sqrt{2\pi}}\left[\frac{1-\cos 2k}{k^2}\right] \\
        &= \frac{1}{\sqrt{2\pi}}\left[\frac{1-(\cos^2 k - \sin^2 k)}{k^2}\right] \\
        &= \frac{2\sin^2 k}{\sqrt{2\pi} k^2}.
       \end{align*}
  \end{enumerate}
\end{proof}
\newpage


\begin{problem}{2}
\end{problem}

\begin{proof}
\end{proof}
\newpage


\begin{problem}{3}
  Apply the Laplace transform to solve the following Initial Value Problems:
  \begin{enumerate}
    \item[a.] $y'' + 2 a y' + (a^2 + 4)y = f(t)$

      $y(0) = 1$, \quad $y'(0) = -a$.

    \item[b.] $u_{tt} = c^2 u_{xx} + \sin x$, \quad  $0 < x < \pi$, \quad $t > 0$

      $u(0, t) = u(\pi, t) = 1$, \quad $u(x, 0) = u_t(x, 0) = 0$.
  \end{enumerate}
\end{problem}

\begin{proof}
  Recall that if $\bar{y}(s) = \laplace{y(t)}$, then the Laplace transform of the
  $n$-th derivative of $y(t)$ is given by
  \begin{align}\label{laplace_nth_der}
    \laplace{y^{(n)}(t)} = s^n \bar{y}(s) - \sum_{k=0}^{n-1} s^{n-1-k} y^{(k)}(0).
  \end{align}

  \begin{enumerate}
    \item[a.] Suppose that $Ly \equiv y''(t) + 2a y'(t) + (a^2 + 4)y(t)$. Using \eqref{laplace_nth_der}, application of the Laplace transform
      to $Ly = f(t)$ yields that
      \begin{align*}
        \laplace{Ly} = (s^2 +2as +a^2+4)\bar{y}(s) -2 a y(0)  -s y(0) - y'(0) = \bar{f}(s) = \laplace{f(t)}.
      \end{align*}
      From the initial data, we see that this reduces to
      \begin{align*}
        (s^2 +2as +a^2+4)\bar{y}(s) -(s+a) = \bar{f}(s).
      \end{align*}
      Solving for $\bar{y}(s)$ yields
      \begin{align*}
        \bar{y}(s) = \frac{\bar{f}(s) + s + a}{s^2 +2as +a^2+4} = \frac{\bar{f}(s) + s + a}{(s+a+2i)(s+a-2i)}.
      \end{align*}

      Note that from our table of Laplace transforms that
      \begin{align*}
        \laplaceinv{\frac{a-b}{(s-a)(s-b)}} = e^{at}-e^{bt}
      \end{align*}
      and
      \begin{align*}
        \laplaceinv{\frac{s}{(s-a)(s-b)}} = \frac{ae^{at}-be^{bt}}{a-b}.
      \end{align*}

      Therefore, the solution to the original differential equation is given by
      \begin{align*}
        y(t) &= \laplaceinv{\bar{f}(s)} \\
        &= \laplaceinv{\frac{\bar{f}(s) + s + a}{(s+a+2i)(s+a-2i)}} \\
        &= \laplaceinv{\frac{\bar{f}(s)}{(s+a+2i)(s+a-2i)}} + \frac{e^{-(2i + a) t} }{4} \left[(2-i + ia)e^{4i t} + 2+i -ia\right].
      \end{align*}

    \item[b.]
      Let $u(x, t)$ be a function in $x$ and $t$. The Laplace transform of $u(x, t)$ with respect to $t$ is
      given by
      \begin{align*}
        \laplace{u(x, t)} = \bar{u}(x, s) = \int_0^\infty u(x, t) e^{-st} dt.
      \end{align*}
      From this definition, we see from previous theorems that
      \begin{align*}
        \laplace{\frac{\partial^n}{\partial t^n}\left[u(x, t)\right]} = s^n \bar{u}(x, s) - \sum_{k=0}^{n-1}s^{n-1-k} \frac{\partial^k}{\partial t^k}\left[u(x, 0)\right]
      \end{align*}
      Similarly, we see from the Leibniz integral rule that
      \begin{align*}
        \laplace{\frac{\partial^n}{\partial x^n}\left[u(x, t)\right]} = \der{x}{n}\left[\bar{u}(x, s)\right].
      \end{align*}

      Applying the Laplace transform with respect to $t$ to the differential equation yields that
      \begin{align*}
        \laplace{u_{tt} - c^2 u_{xx}} = s^2 \bar{u}(x,s) -s u(x, 0) - u_t(x, 0) -c^2 \frac{d^2\bar{u}(x,s)}{dx^2} = \frac{\sin x}{s} = \laplace{\sin x}.
      \end{align*}
      In light of the initial data, this equation reduces to
      \begin{align*}
        s^2 \bar{u}(x,s) -c^2 \frac{d^2\bar{u}(x,s)}{dx^2} = \frac{\sin x}{s},
      \end{align*}
      or, equivalently,
      \begin{align*}
        \frac{d^2\bar{u}(x,s)}{dx^2} - \left(\frac{s}{c}\right)^2 \bar{u}(x,s) = -\frac{\sin x}{sc^2}.
      \end{align*}
      The homogeneous solution to the above differential equation is easily seen to be
      \begin{align*}
        \bar{u}_h(x,s) = c_1 \exp{\left(-\frac{xs}{c}\right)} + c_2 \exp{\left(\frac{xs}{c}\right)}
      \end{align*}
      From the method of undetermined coefficients, assuming the particular solution of the equation is of the form $\bar{u}_p(x,s) = A \sin x$ for some unknown $A$,
      the particular solution of the transformed equation is given by
      \begin{align*}
        \bar{u}_p(x,s) = \frac{\sin x}{s(s^2+c^2)}.
      \end{align*}
      Therefore, the general solution to the transformed equation is given by
      \begin{align*}
        \bar{u}(x, s) = \bar{u}_h(x,s) + \bar{u}_p(x,s) = c_1 \exp{\left(-\frac{xs}{c}\right)} + c_2 \exp{\left(\frac{xs}{c}\right)} + \frac{\sin x}{s(s^2+c^2)}.
      \end{align*}

      Note that the transformed boundary data is given by $\displaystyle \bar{u}(0, s) = \bar{u}(\pi, s) = \frac{1}{s}$.
      Using the form of the solution to the transformed equation listed above, we see that in light of the transformed boundary data
      that
      \begin{align*}
        c_1 + c_2 &= \frac{1}{s} \\
        c_1 \exp{\left(-\frac{\pi s}{c}\right)} + c_2 \exp{\left(\frac{\pi s}{c}\right)} &= \frac{1}{s}
      \end{align*}

      After solving the above system, we therefore see that the solution to the transformed equation is given by
      \begin{align*}
        \bar{u}(x, s) &= c_1 \exp{\left(-\frac{xs}{c}\right)} + c_2 \exp{\left(\frac{xs}{c}\right)} + \frac{\sin x}{s(s^2+c^2)} \\
        &= \frac{\exp{\left(\frac{\pi s}{c}\right)}\exp{\left(-\frac{xs}{c}\right)}}{s\left(1+\exp{\left(\frac{\pi s}{c}\right)}\right)} + \frac{\exp{\left(\frac{xs}{c}\right)}}{s\left(1+\exp{\left(\frac{\pi s}{c}\right)}\right)} + \frac{\sin x}{s(s^2+c^2)}
      \end{align*}

      Therefore, the solution to the original differential equation is given by
      \begin{align*}
        u(x, t) = \laplaceinv{\bar{u}(x, s)}
        &= \laplaceinv{\frac{\exp{\left(\frac{\pi s}{c}\right)}\exp{\left(-\frac{xs}{c}\right)}}{s\left(1+\exp{\left(\frac{\pi s}{c}\right)}\right)} + \frac{\exp{\left(\frac{xs}{c}\right)}}{s\left(1+\exp{\left(\frac{\pi s}{c}\right)}\right)} + \frac{\sin x}{s(s^2+c^2)}}.
      \end{align*}


  \end{enumerate}
\end{proof}
\newpage


\begin{problem}{4}
  Show that the function $\displaystyle H(z) = \frac{z-i}{z+i}$ gives a conjugacy between
  the Newton map $N_{f_1}$ of $f_1(z) = z^2 + 1$ and the function $f_0(z) = z^2$. Deduce
  the Julia set of $N_{f_1}$ and show that it is chaotic on its Julia set.
\end{problem}

\begin{proof}
  Note that the Newton function $N_{f_1}$ of $f_1(z) = z^2 + 1$
  is given by
  \begin{align*}
    N_{f_1}(z) = z - \frac{f(z)}{f'(z)} = z - \frac{z^2+1}{2z} = \frac{z^2 - 1}{2z}.
  \end{align*}

  Let $D = \{ w \in \mathbb{C} \mid |w| > 1 \}$ and consider $f_0(z) = z^2$. Note
  that, $B_{f_0}(\infty)$, the basin of attraction of infinity for $f_0$, is $D$.
  Define $\displaystyle H(z) = \frac{z-i}{z+i}$. Then $H: H^{-1}(D) \to D$ is a homeomorphism,
  where $H^{-1}(D) = \{z\in\mathbb{C} \mid w = H(z),\ |w| > 1\}$.

  To see this, we will show that $H$ is a continuous
  bijection with continuous inverse. Suppose first that $H(z_1) = H(z_2)$. Then we have that
  \begin{align*}
    H(z_1) = \frac{z_1 - i}{z_1 + i} = \frac{z_2 - i}{z_2 + i} = H(z_2).
  \end{align*}
  This implies that
  \begin{align*}
    z_1 z_2 + i z_1 - i z_2 + 1 = z_1 z_2 -i z_1 + i z_2 + 1
  \end{align*}
  or that $2i(z_1 - z_2) = 0$. Since the complex numbers form an integral domain, we must
  have that $z_1 - z_2 = 0$ or that $z_1 = z_2$. Thus, $H$ is injective.

  Let $w\in D$ and let $\displaystyle z = - \frac{i(w+1)}{w-1} \in H^{-1}(D)$. Then we see that
  \begin{align*}
    H(z) = H\left(- \frac{i(w+1)}{w-1}\right) = \frac{- \frac{i(w+1)}{w-1} - i}{- \frac{i(w+1)}{w-1} + i} =  w
  \end{align*}
  so that $H$ is surjective.

  Thus $H$ is a bijection and we see that $H^{-1}: D \to H^{-1}(D)$ defined by
  $$H^{-1}(w) = - \frac{i(w+1)}{w-1}$$
  is the inverse of $H$. It is clear that $H$ is continuous at all points except at
  $z=-i$. However, $z=-i\notin H^{-1}(D)$ and so $H$ is continuous everywhere in its domain.
  Similarly, $H^{-1}$ is continuous everywhere except at $w=1$, but $w=1\notin D$. Therefore,
  $H^{-1}$ is continuous everywhere in its domain and $H$ is a homeomorphism.

  Now, the function $H$ will give a conjugacy between $N_{f_1}$ and $f_0$ if $f_0 \circ H = H \circ N_{f_1}$.
  We can easily verify that
  \begin{align*}
    f_0 \circ H (z) = f_0\left(\frac{z-i}{z+i}\right) = \frac{(z-i)^2}{(z+i)^2}
  \end{align*}
  and
  \begin{align*}
    H \circ N_{f_1}(z) = H\left(\frac{z^2-1}{2z}\right) &= \frac{\frac{z^2-1}{2z} - i}{\frac{z^2-1}{2z}+i} \\
    &= \frac{\frac{(z-i)^2}{2z}}{\frac{(z+i)^2}{2z}} \\
    &= \frac{(z-i)^2}{(z+i)^2}.
  \end{align*}
  Therefore, $f_0 \circ H = H \circ N_{f_1}$ and $H$ gives a conjugacy between $N_{f_1}$ and $f_0$.

  Since $D$ is the basin of attraction of infinity of $f_0$ and $H$ is a conjugacy between $N_{f_1}$ and $f_0$,
  we must have that $H^{-1}(D)$ is the basin of attraction of infinity for $N_{f_1}$. By definition, $K(N_{f_1})$,
  the filled-in Julia set of $N_{f_1}$, must be $K(N_{f_1}) = \mathbb{C} \ H^{-1}(D)$. The Julia set is then the
  boundary of this set.

\end{proof}
\newpage


\begin{problem}{5}
  Apply the Fourier Transform to solve the following initial value problem for the heat equation:
  \begin{align*}
    &\frac{\partial u}{\partial t} = a^2 \frac{\partial^2 u}{\partial x^2} + f(x, t), \quad -\infty < x < \infty,\\
    &u(x, 0) = \phi(x), \quad t >0.
  \end{align*}
\end{problem}

\begin{proof}
  Consider the function $u(x, t)$. The Fourier transform
  of $u$ with respect to $x$ is defined as
  \begin{align}\label{fouriermulti}
    \fourier{u(x, t)} = U(k, t) = \frac{1}{\sqrt{2\pi}}\int_{-\infty}^{\infty} e^{-ikx} u(x, t) dx.
  \end{align}

  From this definition and the Leibniz integral rule, we can see by induction that
  \begin{align}\label{partialy}
    \fourier{\frac{\partial^n }{\partial t^n }\left[u(x,t)\right]}
    &= \frac{1}{\sqrt{2\pi}}\int_{-\infty}^\infty \frac{\partial^n}{ \partial t^n}\left[ u(x, t)\right] e^{-ikx} dx \notag \\
    &= \der{t}{n}\left[\frac{1}{\sqrt{2\pi}} \int_{-\infty}^\infty u(x, t) e^{-ikx} dx \right] \notag \\
    &= \der{t}{n}\left[\fourier{u(x,t)}\right].
  \end{align}
  Similarly, we see from definition \eqref{fouriermulti} and previous
  theorems regarding the Fourier transform that
  \begin{align}\label{partialx}
    \fourier{\frac{\partial^n}{\partial x^n}\left[u(x, t)\right]}
    &= \frac{1}{\sqrt{2\pi}}\int_{-\infty}^\infty \frac{\partial^n}{\partial x^n}\left[u(x, t)\right] e^{-ikx} dx \notag \\
    &= (ik)^n \left[\frac{1}{\sqrt{2\pi}} \int_{-\infty}^\infty u(x,t)e^{-ikx} dx \right] \notag \\
    &= (ik)^n \fourier{u(x,y)}.
  \end{align}

  Now, applying the Fourier transform to the first equation yields that
  \begin{align*}
    \fourier{\frac{\partial u}{\partial t}} = \der{t}{}\left[U(k, t)\right] = -(ak)^2 U(k,t) + F(k, t) =  \fourier{a^2 \frac{\partial^2 u}{\partial x^2} + f(x, t)}.
  \end{align*}
  This results in a first-order non-homogeneous linear differential equation
  \begin{align*}
    \der{t}{}\left[U(k, t)\right] + (ak)^2 U(k,t) = F(k, t).
  \end{align*}
\end{proof}
\newpage


\begin{problem}{6}
  Let $f(x) = ax^3+bx+c$ where $a$ and $b$ satisfy $a/b > 0$. Denote by $N_f$
  the corresponding Newton function.
  \begin{enumerate}
    \item Show that $N_f$ has a unique fixed point.
    \item Show that $N_f$ cannot have any period 2 points.
    \item Why does it follow that $N_f$ has no points of period $n$ for $n>2$?
  \end{enumerate}
\end{problem}

\begin{proof}
  \begin{enumerate}
    \item Recall that the fixed points of $N_f$ are the roots of $f$.
      The discriminant of the polynomial $f$ is given by $D = -4 ab^3 -27 a^2 c^2$. Note that
      if $a/b > 0$ then $D < 0$. Therefore, $f$ only has one real root
      and as a consequence, $N_f$ has a unique fixed point, say $p$.
    \item If $f(x) = ax^3+bx+c$, then $f'(x) \neq 0$ for any $x\in\mathbb{R}$ if $a/b > 0$.
      Thus, all iterates of $N_f$ are well-defined. Since $f''(x)$ is bounded and the derivative of $f$ is non-zero on any finite interval,
      we have that the iterates of $N_f$ will converged to a root of $f$. Since $p$ is the only root,
      it must be a globally attracting fixed point of $N_f$. Thus,
      we have that $N_f^n(x)$ will converge to $p$ for all finite $x$.
      This implies that $\lim_n N_f^n(x) = p$ for all $x\neq p$.
      Therefore, since the limit of the iterates exist, we cannot have that $N_f$ has a period 2 point.
    \item If to the contrary, $N_f$ has a point of period $n > 2$,
      then since $n \triangleright 2$ in Sharkovsky's ordering, we must have
      by Sharkovsky's Theorem that $N_f$ has a point of period 2 . However, this is
      contradictory to the fact that $N_f$ has no points of period 2. Therefore,
      $N_f$ has no points of period $n > 2$.
  \end{enumerate}
\end{proof}
\newpage


\begin{problem}{7}
  \begin{enumerate}
    \item Show that the function $f(x) = -1/(x+1)$ has the property that $f^3(x) = x$
      for all $x\neq-1,0$.
    \item Let $f:\mathbb{R} \to \mathbb{R}$ be a function
      defined on a set $I$, with $f^3(x) = x$ for all $x\in I$. Set $g(x) = f^2(x)$.
      Show that $g^3(x) = x$ for all $x\in I$. Deduce a function different from that
      in a) that has this property.
    \item In general, show that such a function cannot have a 2-cycle.
    \item Deduce that a function $f:\mathbb{R}\to\mathbb{R}$ with the property $f^3(x) = x$
      cannot be continuous.
    \item Show that the inverse of $f$ must exist.
    \item If $f'(x)$ exists for all $x\in I$, show that the 3-cycles are non-hyperbolic where $f$ is not the identity map.
    \item Suppose that $\displaystyle f(x) = \frac{ax + b}{cx + d}$ satisfies $f^3(x) = x$. Show that if $f$ is not the identity map
      and $a\neq d$, then $a^2 +bc + ad +d^2 = 0$.
      \begin{enumerate}[label=\roman*)]
        \item Use this to find other functions with the property $f^3(x) = x$.
        \item Deduce that if $ad- bc > 0$, then such a function cannot have any fixed points.
      \end{enumerate}
  \end{enumerate}
\end{problem}

\begin{proof}
  \begin{enumerate}
    \item Let $x\neq -1, 0$. Then we see that for $f(x) = -1/(x+1)$, we have that
      \begin{align*}
        f^2(x) = f(f(x)) = -\frac{1}{-\frac{1}{x+1}+1} = \frac{-x-1}{x} = -1 - \frac{1}{x}
      \end{align*}
      and
      \begin{align*}
        f^3(x) = f(f^2(x)) = -\frac{1}{-1 - \frac{1}{x} +1} = x.
      \end{align*}
    \item Suppose that for all $x\in I$, with $I$ an interval, $f^3(x) = x$. Define $g(x) = f^2(x)$.
      Then we see that
      \begin{align*}
        g^3(x) = g(g(f^2(x)))=g(f^4(x))=f^6(x).
      \end{align*}
      Since $f^3(x) = x$, we have that
      \begin{align*}
        g^3(x) = f^6(x) = f^3(f^3(x)) = f^3(x) = x.
      \end{align*}
      Thus, a function with the property that $g^3(x) = x$ is given by $g:\mathbb{R} \to \mathbb{R}$
      defined by $g(x) = f^2(x)$ where $f(x) = -1/(x+1)$. Explicitly,
      \begin{align*}
        g(x) = f^2(x) = -1 - \frac{1}{x}
      \end{align*}
    \item Suppose to the contrary that for $f$ with $f^3(x) = x$
      that $f$ has a 2-cycle $\{x, f(x)\}$. Then we have that $f^2(x) = x$ with $f(x) \neq x$.
      Since $f^3(x) = x$, we see that
      \begin{align*}
        x = f^3(x) = f(f^2(x)) = f(x).
      \end{align*}
      However, this contradicts the assumption that $\{x, f(x)\}$ is a 2-cycle. Therefore, $f$ with
      such a property cannot have a 2-cycle.
    \item By Sharkovsky's Theorem, we see that if $f$ is a continuous function
      that has a period 3 point, then it must have a period 2 point since $3 \triangleright 2$
      in Sharkovsky's ordering. Since a function $f$ with the property that $f^3(x) = x$ for
      all $x\in I$, where $I$ is an interval, does not have a period 2 point,
      it must follow from Sharkovsky's Theorem that $f$ is not continuous.
    \item Let $f:\mathbb{R}\to\mathbb{R}$ satisfy $f^3(x) = x$. This function is injective and surjective.
      To see that $f$ is injective, suppose that $f(x_1) = f(x_2)$. Then we have that
      $f^2(f(x_1)) = f^2(f(x_2))$ since $f$ is a well-defined function. But this implies that
      $x_1 = f^3(x_1) = f^3(x_2) = x_2$, and so $f$ is injective.

      To see that $f$ is surjective, take $y \in \mathbb{R}$. Then $x = f^2(y)$
      satisfies $f(x) = f(f^2(y) = f^3(y) = y$ or that $y$ is in the range of $f$
      and so $f$ is surjective since $y$ was arbitrary.

      Therefore, since $f$ is injective and surjective, it must be a bijection
      and must have an inverse.
    \item Suppose that $f$ is not identically the identity map with $f^3(x) = x$.
      Let $\{x, f(x), f^2(x)\}$ be a 3-cycle of $f$. If $f^3(x) = x$, then $(f^3(x))' = 1$.
      This implies that
      \begin{align*}
        \left|f^3(x)'\right| = \left|f'(x)f'(f(x))f'(f^2(x))\right| = 1
      \end{align*}
      or that $\{x, f(x), f^2(x)\}$ is a non-hyperbolic 3-cycle.
  \end{enumerate}
\end{proof}
\newpage


\begin{problem}{8}
  Apply the Fourier Transform to solve the 3-dimensional wave problem
  \begin{align*}
    \frac{\partial^2 u}{\partial t^2} = a^2\left(\frac{\partial^2 u}{\partial x^2} + \frac{\partial^2 u}{\partial y^2} + \frac{\partial^2 u}{\partial z^2}\right), \quad -\infty < x, y, z < \infty,
  \end{align*}
  subject to the initial conditions
  \begin{align*}
    &u(x, y, z, t)\rvert_{t=0} = 0 \\
    &\left.\frac{\partial u(x, y, z, t)}{\partial t}\right\rvert_{t=0} = \delta(x, y, z).
  \end{align*}
\end{problem}

\begin{proof}
  Let $\vec{x} = (x_1, x_2, \dots, x_n)\in \mathbb{R}^n$ and suppose that $u(\vec{x}, t)$ is given. The
  Fourier transform of $u(\vec{x}, t)$ with respect to $\vec{x}$ is defined to be
  \begin{align}\label{multifourier}
    \fourier{u(\vec{x}, t)} = U(\vec{k}, t) = \frac{1}{(2\pi)^{n/2}}\int_{-\infty}^\infty u(\vec{x}, t) e^{-i\vec{x}\cdot\vec{k}} d\vec{x}
  \end{align}
  where $\vec{k}\in\mathbb{R}^n$.

  In order to investigate the Fourier transform of partials of $u(\vec{x}, t)$ with respect to a given component of $\vec{x}$, define the
  Fourier transform of $u(\vec{x}, t)$ with respect to $x_j$ as the following
  \begin{align*}
    \fourierwrt{u(\vec{x}, t)}{x_j} = \frac{1}{\sqrt{2\pi}}\int_{-\infty}^{\infty} u(\vec{x}, t) e^{-i x_jk_j }dx_j.
  \end{align*}
  Further, we will also use the function $\pi_{j}: \mathbb{R}^n \to \mathbb{R}^{n-1}$ defined as
  \begin{align*}
    \pi_{j}(\vec{x}) := (x_1, \dots, x_{j-1}, x_{j+1}, \dots, x_n)
  \end{align*}
  to aid in our description of the Fourier transform of partials of $u(\vec{x}, t)$.
  Now from definition \eqref{multifourier} and Leibniz's integral rule we see that
  \begin{align*}
    \fourier{\frac{\partial^n u(\vec{x}, t)}{\partial t^n}} &= \frac{1}{(2\pi)^{n/2}}\int_{-\infty}^\infty \frac{\partial^n }{\partial t^n}\left[u(\vec{x}, t)\right] e^{-i\vec{x}\cdot\vec{k}} d\vec{x} \\
    &= \der{t}{n}\left[\frac{1}{(2\pi)^{n/2}}\int_{-\infty}^\infty u(\vec{x}, t) e^{-i\vec{x}\cdot\vec{k}} d\vec{x} \right]\\
    &= \der{t}{n}\left[\fourier{u(\vec{x}, t)}\right].
  \end{align*}
  Similarly, from definition \eqref{multifourier} and previous results about the Fourier transform,
  we see that
  \begin{align*}
    \fourier{\frac{\partial^n u(\vec{x}, t)}{\partial x_j^n}}
    &= \frac{1}{(2\pi)^{n/2}}\int_{-\infty}^\infty \dots \int_{-\infty}^\infty\frac{\partial^n }{\partial x_j^n}\left[u(\vec{x}, t)\right] e^{-i x_1 k_1}\cdots e^{-i x_n k_n} dx_1 \dots dx_n \\
    &= \frac{1}{(2\pi)^{(n-1)/2}}\int_{-\infty}^\infty \fourierwrt{\frac{\partial^n }{\partial x_j^n}\left[u(\vec{x}, t)\right]}{x_j} e^{-i \pi_j(\vec{x}) \cdot \pi_j(\vec{k})} d\pi_j(\vec{x}) \\
    &= \frac{(ik_j)^n}{(2\pi)^{(n-1)/2}}\int_{-\infty}^\infty \fourierwrt{u(\vec{x}, t)}{x_j} e^{-i \pi_j(\vec{x}) \cdot \pi_j(\vec{k})} d\pi_j(\vec{x}) \\
    &= (ik_j)^n\fourier{u(\vec{x}, t)}.
  \end{align*}

  Now, define $\vec{x} = (x_1, x_2, x_3) = (x, y, z) \in\mathbb{R}^3$. The the system of differential equations
  of the function $u(\vec{x}, t) = u(x_1, x_2, x_3, t)$ becomes
  \begin{align*}
    a^2\left(\frac{\partial^2 u}{\partial x_1^2} + \frac{\partial^2 u}{\partial x_2^2} + \frac{\partial^2 u}{\partial x_3^2}\right) - \frac{\partial^2 u}{\partial t^2} = 0, \quad -\infty < x_1, x_2, x_3 < \infty,
  \end{align*}
  subject to the initial conditions
  \begin{align*}
    &u(\vec{x}, t)\rvert_{t=0} = 0, \qquad\left.\frac{\partial u(\vec{x}, t)}{\partial t}\right\rvert_{t=0} = \delta(\vec{x}).
  \end{align*}

  Applying the Fourier transform with respect to $\vec{x}$ to the first equation yields
  \begin{align*}
    \fourier{a^2\left(\frac{\partial^2 u}{\partial x_1^2} + \frac{\partial^2 u}{\partial x_2^2} + \frac{\partial^2 u}{\partial x_3^2}\right) - \frac{\partial^2 u}{\partial t^2}}
    =-a^2\norm{\vec{x}}^2U(\vec{k}, t)- \der{t}{2}\left[U(\vec{k}, t)\right] = 0 = \fourier{0}
  \end{align*}
  where $U(\vec{k}, t) = \fourier{u(\vec{x}, t)}$. Similarly, we deduce that
  the transformed initial conditions become
  \begin{align*}
    &\fourier{u(\vec{x}, t)\rvert_{t=0}} = U(\vec{k}, t)\rvert_{t=0} = 0 = \fourier{0}, \\
    &\fourier{\left.\frac{\partial u(\vec{x}, t)}{\partial t}\right\rvert_{t=0}} = \left.\der{t}{}\left[U(\vec{k}, t)\right]\right\rvert_{t=0} = \frac{1}{(2\pi)^{3/2}} = \fourier{\delta(\vec{x})}.
  \end{align*}

  We see that the first transformed equation is a second-order
  linear homogeneous ordinary differential equation, from which we readily see that the solution is
  \begin{align*}
    U(\vec{k}, t) = c_1 \cos(a\norm{\vec{k}} t) + c_2 \sin(a \norm{\vec{k}} t).
  \end{align*}
  Using this solution we see from the first transformed initial condition that
  \begin{align*}
    U(\vec{k}, t)\rvert_{t=0}
    &= \left.c_1 \cos(a\norm{\vec{k}} t) + c_2 \sin(a \norm{\vec{k}} t)\right\rvert_{t=0} = c_1 = 0.
  \end{align*}
  From the second transformed initial condition, we see using the above solution that
  \begin{align*}
    \left.\der{t}{}\left[U(\vec{k}, t)\right]\right\rvert_{t=0}
    &= -a\norm{\vec{k}}c_1\sin(a\norm{\vec{k}}t) + a\norm{\vec{k}}c_2\cos(a\norm{\vec{k}}t)\rvert_{t=0} \\
    &= a\norm{\vec{k}}c_2 \\
    &= \frac{1}{(2\pi)^{3/2}},
  \end{align*}
  or that $\displaystyle c_2 = \frac{1}{a\norm{\vec{k}}(2\pi)^{3/2}}.$ Therefore, the solution to the transformed
  system of differential equations is
  \begin{align*}
    U(\vec{k}, t) = \frac{\sin(a\norm{\vec{k}}t)}{a\norm{\vec{k}}(2\pi)^{3/2}}.
  \end{align*}

  Applying the inverse Fourier Transform to the above solution gives the solution to the original
  system
  \begin{align*}
    u(\vec{x}, t) &= \fourierinv{\frac{\sin(a\norm{\vec{k}}t)}{a\norm{\vec{k}}(2\pi)^{3/2}}}\\
    &= \frac{1}{(2\pi)^3}\int_{-\infty}^\infty \frac{\sin(a\norm{\vec{k}}t)}{a\norm{\vec{k}}} e^{i\vec{k}\cdot \vec{x}} d\vec{k} \\
    &= \frac{1}{(2\pi)^3}\int_{-\infty}^\infty\int_{-\infty}^\infty\int_{-\infty}^\infty \frac{\sin(a\norm{\vec{k}}t)}{a\norm{\vec{k}}} e^{ik_1 x_1} e^{ik_2 x_2} e^{ik_3 x_3}dk_1 dk_2 dk_3 .
  \end{align*}
\end{proof}
\newpage



\end{document}
