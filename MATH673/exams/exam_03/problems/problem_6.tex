\begin{problem}{6}
  Suppose that $I_n f(x)$ denotes the $n$-th repeated integral of $f(x)$ defined by
  \begin{align*}
    I_n f(x) = \int_x^\infty I_{n-1}f(t) dt
  \end{align*}
  and that $\mellin{f(x)} = \tilde{f}(p)$. Show that
  \begin{enumerate}
    \item[a.] $\displaystyle \mellin{\int_x^\infty f(t) dt} = \frac{1}{p}\tilde{f}(p+1)$,
    \item[b.] $\displaystyle \mellin{I_n f(x)} = \frac{\Gamma(p)}{\Gamma(p+n)} \tilde{f}(p + n)$
  \end{enumerate}
\end{problem}

\begin{proof}
  Recall that the Mellin transform of the function $f(x)$ is defined as
  \begin{align}
    \label{6:Mellin}
    \mellin{f(x)} = \int_0^\infty x^{p-1} f(x) dx.
  \end{align}

  \begin{enumerate}
    \item[a.] From the definition of the Mellin transform \eqref{6:Mellin}, we see that
      \begin{align*}
        \mellin{\int_x^\infty f(t) dt}  = \int_0^\infty x^{p-1} \left[\int_x^\infty f(t) dt\right] dx.
      \end{align*}
      Interchanging the order of integration from $t$ to $x$, we see that
      \begin{align*}
        \mellin{\int_x^\infty f(t) dt}  &= \int_0^\infty x^{p-1} \left[\int_x^\infty f(t) dt\right] dx \\
        &= \int_0^\infty f(t) \left[\int_0^t x^{p-1} dx\right] dt \\
        &= \frac{1}{p}\int_0^\infty t^{p}f(t)dt.
      \end{align*}
      If $\tilde{f}(p) = \mellin{f(x)}$, then from the definition of the Mellin transform \eqref{6:Mellin},
      we see that the above integral becomes
      \begin{align*}
        \mellin{\int_x^\infty f(t) dt}
        &= \frac{1}{p}\int_0^\infty t^{p}f(t)dt \\
        &= \frac{1}{p}\tilde{f}(p+1),
      \end{align*}
      and we are done.

    \item[b.] We will now prove the relation
      \begin{align*}
        \mellin{I_n f(x)} = \frac{\Gamma(p)}{\Gamma(p+n)} \tilde{f}(p + n)
      \end{align*}
      by induction. The results of the previous exercise show that
      \begin{align*}
        \mellin{I_1 f(x)}
        = \mellin{\int_x^\infty f(t) dt}
        = \frac{1}{p}\tilde{f}(p+1)
        = \frac{\Gamma(p)}{\Gamma(p+1)}\tilde{f}(p+1)
      \end{align*}
      and the base step is established.

      Now assume the relation holds for $n$, i.e.\ assume that
      \begin{align*}
        \mellin{I_n f(x)} = \frac{\Gamma(p)}{\Gamma(p+n)} \tilde{f}(p + n).
      \end{align*}

      Now, we see from the definition of the Mellin transform \eqref{6:Mellin} that
      \begin{align*}
        \mellin{I_{n+1} f(x)} &= \mellin{\int_x^\infty I_n f(t) dt} \\
        &= \int_0^\infty x^{p-1} \left[\int_x^\infty I_n f(t) dt \right] dx.
      \end{align*}
      Let $g(t) = I_n f(t)$. Then, proceeding as we did in establishing the result of
      the base step, interchanging the order of integration
      from $x$ to $t$ yields
      \begin{align}
        \label{6:induction}
        \mellin{I_{n+1} f(x)}
        &= \int_0^\infty x^{p-1} \left[\int_x^\infty g(t) dt \right] dx \notag \\
        &= \int_0^\infty g(t)  \left[\int_0^t x^{p-1}dx \right] dt \notag \\
        &= \frac{1}{p}\int_0^\infty t^p g(t)dt \notag \\
        &= \frac{1}{p}\tilde{g}(p + 1)
      \end{align}
      where $\tilde{g}(p) = \mellin{g(t)}$.

      From our assumption, we have that
      \begin{align*}
        \tilde{g}(p) = \mellin{g(t)} = \mellin{I_n f(x)} = \frac{\Gamma(p)}{\Gamma(p+n)} \tilde{f}(p + n).
      \end{align*}
      Thus, the integral in \eqref{6:induction} becomes
      \begin{align*}
        \mellin{I_{n+1} f(x)}
        &= \frac{1}{p}\tilde{g}(p + 1) \\
        &= \frac{\Gamma(p+1)}{p\Gamma(p+n+1)} \tilde{f}(p + n + 1) \\
        &= \frac{\Gamma(p)}{\Gamma(p+n+1)} \tilde{f}(p + n + 1).
      \end{align*}
      Therefore, the result holds for $n+1$ and the relation holds in general for all $n > 0$.

  \end{enumerate}
\end{proof}
\newpage
