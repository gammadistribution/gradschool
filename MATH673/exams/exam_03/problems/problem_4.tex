\begin{problem}{4}
  Solve the following Partial Differential Equation by the Mellin transform
  \begin{align*}
    &r^{2} \phi_{rr} + r \phi_r + \phi_{\theta \theta} = 0, \qquad 0 < r < \infty, \quad 0 < \theta < \pi \\
    &\phi(r, 0) =
    \begin{cases}
      (1-r)^2 & 0 < r < 1 \\
      0 & 1 < r
    \end{cases}\\
    &\phi(r, \pi) =
    \begin{cases}
      1 & 0 < r < 1 \\
      0 & 1 < r
    \end{cases},
  \end{align*}
\end{problem}

\begin{proof}
  Recall that if $\mellin{\phi(r, \theta)} = \tilde{\phi}(p, \theta)$, then the following property holds
  \begin{align*}
    \mellin{r^2 \frac{\partial^2}{\partial r^2}\left[\phi(r, \theta)\right] + r \frac{\partial}{\partial r}\left[\phi(r, \theta) \right]} = p^2 \tilde{\phi}(p, \theta).
  \end{align*}
  Thus, applying the Mellin transform to the Partial Differential Equation and using our table of Mellin transforms,
  we see that
  \begin{align*}
    &\der{\theta}{2}\left[\tilde{\phi}(p, \theta)\right] + p^2 \tilde{\phi}(p, \theta) = 0, \qquad 0 < p < \infty, \quad 0 < \theta < \pi \\
    &\tilde{\phi}(p, 0) = \frac{\Gamma(3)\Gamma(p)}{\Gamma(p + 3)},\quad \tilde{\phi}(p, \pi) = \frac{1}{p}.
  \end{align*}

  The solution to the resulting homogeneous linear Ordinary Differential Equation is
  \begin{align*}
    \tilde{\phi}(p, \theta) = c_1(p) \cos p\theta + c_2(p) \sin p\theta.
  \end{align*}
  Using the above solution and the transformed boundary conditions, we see that
  \begin{align*}
    c_1(p) &= \frac{\Gamma(3)\Gamma(p)}{\Gamma(p + 3)} \\
    c_1(p)\cos p\pi + c_2(p)\sin p \pi &= \frac{1}{p}.
  \end{align*}
  Solving, we see that
  \begin{align*}
    c_2(p) &= \left(\frac{1}{p} - \frac{\Gamma(3)\Gamma(p)}{\Gamma(p + 3)} \cos p\pi\right) \csc p\pi \\
    &= \frac{\csc p\pi}{p} - \frac{\Gamma(3)\Gamma(p)}{\Gamma(p + 3)} \cot p\pi.
  \end{align*}
  Thus, the solution to the transformed boundary value problem is
  \begin{align*}
    \tilde{\phi}(p, \theta) = \left[\frac{\Gamma(3)\Gamma(p)}{\Gamma(p + 3)}\right] \cos p\theta + \left[\frac{\csc p\pi}{p} - \frac{\Gamma(3)\Gamma(p)}{\Gamma(p + 3)} \cot p\pi\right] \sin p\theta.
  \end{align*}

  Therefore, the solution to the original boundary value problem is
  \begin{align*}
    \phi(r, \theta) &= \mellininv{\left[\frac{\Gamma(3)\Gamma(p)}{\Gamma(p + 3)}\right] \cos p\theta + \left[\frac{\csc p\pi}{p} - \frac{\Gamma(3)\Gamma(p)}{\Gamma(p + 3)} \cot p\pi\right] \sin p\theta} \\
    &= \frac{1}{2\pi i} \int_{c - i \infty}^{c + i \infty} r^{-p} \left\{\left[\frac{\Gamma(3)\Gamma(p)}{\Gamma(p + 3)}\right] \cos p\theta + \left[\frac{\csc p\pi}{p} - \frac{\Gamma(3)\Gamma(p)}{\Gamma(p + 3)} \cot p\pi\right] \sin p\theta\right\}dp.
  \end{align*}
\end{proof}
\newpage
