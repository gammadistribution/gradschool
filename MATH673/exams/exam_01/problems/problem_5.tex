\begin{problem}{5}
  Apply the Fourier Transform to solve the following initial value problem for the heat equation:
  \begin{align*}
    &\frac{\partial u}{\partial t} = a^2 \frac{\partial^2 u}{\partial x^2} + f(x, t), \quad -\infty < x < \infty,\\
    &u(x, 0) = \phi(x), \quad t >0.
  \end{align*}
\end{problem}

\begin{proof}
  Consider the function $u(x, t)$. The Fourier transform
  of $u$ with respect to $x$ is defined as
  \begin{align}\label{fouriermulti}
    \fourier{u(x, t)} = U(k, t) = \frac{1}{\sqrt{2\pi}}\int_{-\infty}^{\infty} e^{-ikx} u(x, t) dx.
  \end{align}

  From this definition and the Leibniz integral rule, we can see by induction that
  \begin{align}\label{partialy}
    \fourier{\frac{\partial^n }{\partial t^n }\left[u(x,t)\right]}
    &= \frac{1}{\sqrt{2\pi}}\int_{-\infty}^\infty \frac{\partial^n}{ \partial t^n}\left[ u(x, t)\right] e^{-ikx} dx \notag \\
    &= \der{t}{n}\left[\frac{1}{\sqrt{2\pi}} \int_{-\infty}^\infty u(x, t) e^{-ikx} dx \right] \notag \\
    &= \der{t}{n}\left[\fourier{u(x,t)}\right].
  \end{align}
  Similarly, we see from definition \eqref{fouriermulti} and previous
  theorems regarding the Fourier transform that
  \begin{align}\label{partialx}
    \fourier{\frac{\partial^n}{\partial x^n}\left[u(x, t)\right]}
    &= \frac{1}{\sqrt{2\pi}}\int_{-\infty}^\infty \frac{\partial^n}{\partial x^n}\left[u(x, t)\right] e^{-ikx} dx \notag \\
    &= (ik)^n \left[\frac{1}{\sqrt{2\pi}} \int_{-\infty}^\infty u(x,t)e^{-ikx} dx \right] \notag \\
    &= (ik)^n \fourier{u(x,y)}.
  \end{align}

  Now, applying the Fourier transform to the first equation yields that
  \begin{align*}
    \fourier{\frac{\partial u}{\partial t}} = \der{t}{}\left[U(k, t)\right] = -(ak)^2 U(k,t) + F(k, t) =  \fourier{a^2 \frac{\partial^2 u}{\partial x^2} + f(x, t)}.
  \end{align*}
  This results in a first-order non-homogeneous linear differential equation
  \begin{align*}
    \der{t}{}\left[U(k, t)\right] + (ak)^2 U(k,t) = F(k, t).
  \end{align*}

  Using well-established techniques, we see that the solution to the linear differential equation is
  \begin{align*}
    U(k, t) = c_1 e^{-(ak)^2 t} + e^{-(ak)^2 t}\int_0^t e^{(ak)^2 \xi} F(k, \xi) d\xi.
  \end{align*}
  Applying the Fourier Transform to the initial value equation shows that
  \begin{align*}
    \fourier{u(x, 0)} = U(k, 0) = \Phi(k) = \fourier{\phi(x)}
  \end{align*}
  Thus, from the transformed initial value equation, we see using the above solution that
  \begin{align*}
    U(k, 0) = c_1 = \Phi(k)
  \end{align*}
  Therefore, the solution to the transformed system of differential equations
  is
  \begin{align*}
    U(k, t) = \Phi(k)e^{-(ak)^2 t}+ e^{-(ak)^2 t}\int_0^t e^{(ak)^2 \xi} F(k, \xi) d\xi.
  \end{align*}

  Applying the inverse Fourier transform to the solution of the transformed system of differential equations
  yields that the solution to the original system is
  \begin{align*}
    u(x, t) = \fourierinv{U(k, t)} = \fourierinv{\Phi(k)e^{-(ak)^2 t}} + \fourierinv{e^{-(ak)^2 t}\int_0^t e^{(ak)^2 \xi} F(k, \xi) d\xi}
  \end{align*}
  Note from our table of Fourier transforms that for $b>0$
  \begin{align*}
    \fourier{e^{-b x^2}} =\frac{1}{\sqrt{2b}} \exp\left(-\frac{k^2}{4b}\right)
  \end{align*}
  Thus,
  \begin{align*}
    \fourierinv{e^{-a^2 t k^2}} = \fourierinv{\exp\left(- \frac{k^2}{4(1/4a^2 t)}\right)} = \sqrt{\frac{1}{2a^2 t}} \exp\left(-\frac{x^2}{4a^2 t}\right) = g(x, t)
  \end{align*}
  Now, from the Convolution Theorem, we have that
  \begin{align*}
    \fourierinv{\Phi(k)e^{-(ak)^2 t}} = (\phi \ast g)(x) = \int_{-\infty}^\infty \phi(x-\xi) g(\xi, t)d\xi.
  \end{align*}
  Therefore, using this identity and the definition of the inverse Fourier Transform, the solution to the original system
  of differential equations is
  \begin{align*}
    u(x, t) &= \fourierinv{\Phi(k)e^{-(ak)^2 t}} + \fourierinv{e^{-(ak)^2 t}\int_0^t e^{(ak)^2 \xi} F(k, \xi) d\xi} \\
    &= \int_{-\infty}^\infty \phi(x-\xi) g(\xi, t)d\xi + \int_{-\infty}^{\infty}\left[\int_0^t e^{(ak)^2 \xi} F(k, \xi) d\xi\right]e^{-(ak)^2 t} e^{ikx} dk.
  \end{align*}
\end{proof}
\newpage
