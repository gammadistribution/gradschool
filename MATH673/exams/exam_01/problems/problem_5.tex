\begin{problem}{5}
  Apply the Fourier Transform to solve the following initial value problem for the heat equation:
  \begin{align*}
    &\frac{\partial u}{\partial t} = a^2 \frac{\partial^2 u}{\partial x^2} + f(x, t), \quad -\infty < x < \infty,\\
    &u(x, 0) = \phi(x), \quad t >0.
  \end{align*}
\end{problem}

\begin{proof}
  Consider the function $u(x, t)$. The Fourier transform
  of $u$ with respect to $x$ is defined as
  \begin{align}\label{fouriermulti}
    \fourier{u(x, t)} = U(k, t) = \frac{1}{\sqrt{2\pi}}\int_{-\infty}^{\infty} e^{-ikx} u(x, t) dx.
  \end{align}

  From this definition and the Leibniz integral rule, we can see by induction that
  \begin{align}\label{partialy}
    \fourier{\frac{\partial^n }{\partial t^n }\left[u(x,t)\right]}
    &= \frac{1}{\sqrt{2\pi}}\int_{-\infty}^\infty \frac{\partial^n}{ \partial t^n}\left[ u(x, t)\right] e^{-ikx} dx \notag \\
    &= \der{t}{n}\left[\frac{1}{\sqrt{2\pi}} \int_{-\infty}^\infty u(x, t) e^{-ikx} dx \right] \notag \\
    &= \der{t}{n}\left[\fourier{u(x,t)}\right].
  \end{align}
  Similarly, we see from definition \eqref{fouriermulti} and previous
  theorems regarding the Fourier transform that
  \begin{align}\label{partialx}
    \fourier{\frac{\partial^n}{\partial x^n}\left[u(x, t)\right]}
    &= \frac{1}{\sqrt{2\pi}}\int_{-\infty}^\infty \frac{\partial^n}{\partial x^n}\left[u(x, t)\right] e^{-ikx} dx \notag \\
    &= (ik)^n \left[\frac{1}{\sqrt{2\pi}} \int_{-\infty}^\infty u(x,t)e^{-ikx} dx \right] \notag \\
    &= (ik)^n \fourier{u(x,y)}.
  \end{align}

  Now, applying the Fourier transform to the first equation yields that
  \begin{align*}
    \fourier{\frac{\partial u}{\partial t}} = \der{t}{}\left[U(k, t)\right] = -(ak)^2 U(k,t) + F(k, t) =  \fourier{a^2 \frac{\partial^2 u}{\partial x^2} + f(x, t)}.
  \end{align*}
  This results in a first-order non-homogeneous linear differential equation
  \begin{align*}
    \der{t}{}\left[U(k, t)\right] + (ak)^2 U(k,t) = F(k, t).
  \end{align*}
\end{proof}
\newpage
