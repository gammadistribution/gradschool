\begin{problem}{8}
  Apply the Fourier Transform to solve the 3-dimensional wave problem
  \begin{align*}
    \frac{\partial^2 u}{\partial t^2} = a^2\left(\frac{\partial^2 u}{\partial x^2} + \frac{\partial^2 u}{\partial y^2} + \frac{\partial^2 u}{\partial z^2}\right), \quad -\infty < x, y, z < \infty,
  \end{align*}
  subject to the initial conditions
  \begin{align*}
    &u(x, y, z, t)\rvert_{t=0} = 0 \\
    &\left.\frac{\partial u(x, y, z, t)}{\partial t}\right\rvert_{t=0} = \delta(x, y, z).
  \end{align*}
\end{problem}

\begin{proof}
  Let $\vec{x} = (x_1, x_2, \dots, x_n)\in \mathbb{R}^n$ and suppose that $u(\vec{x}, t)$ is given. The
  Fourier transform of $u(\vec{x}, t)$ with respect to $\vec{x}$ is defined to be
  \begin{align}\label{multifourier}
    \fourier{u(\vec{x}, t)} = U(\vec{k}, t) = \frac{1}{(2\pi)^{n/2}}\int_{-\infty}^\infty u(\vec{x}, t) e^{-i\vec{x}\cdot\vec{k}} d\vec{x}
  \end{align}
  where $\vec{k}\in\mathbb{R}^n$.

  In order to investigate the Fourier transform of partials of $u(\vec{x}, t)$ with respect to a given component of $\vec{x}$, define the
  Fourier transform of $u(\vec{x}, t)$ with respect to $x_j$ as the following
  \begin{align*}
    \fourierwrt{u(\vec{x}, t)}{x_j} = \frac{1}{\sqrt{2\pi}}\int_{-\infty}^{\infty} u(\vec{x}, t) e^{-i x_jk_j }dx_j.
  \end{align*}
  Further, we will also use the function $\pi_{j}: \mathbb{R}^n \to \mathbb{R}^{n-1}$ defined as
  \begin{align*}
    \pi_{j}(\vec{x}) := (x_1, \dots, x_{j-1}, x_{j+1}, \dots, x_n)
  \end{align*}
  to aid in our description of the Fourier transform of partials of $u(\vec{x}, t)$.
  Now from definition \eqref{multifourier} and Leibniz's integral rule we see that
  \begin{align*}
    \fourier{\frac{\partial^n u(\vec{x}, t)}{\partial t^n}} &= \frac{1}{(2\pi)^{n/2}}\int_{-\infty}^\infty \frac{\partial^n }{\partial t^n}\left[u(\vec{x}, t)\right] e^{-i\vec{x}\cdot\vec{k}} d\vec{x} \\
    &= \der{t}{n}\left[\frac{1}{(2\pi)^{n/2}}\int_{-\infty}^\infty u(\vec{x}, t) e^{-i\vec{x}\cdot\vec{k}} d\vec{x} \right]\\
    &= \der{t}{n}\left[\fourier{u(\vec{x}, t)}\right].
  \end{align*}
  Similarly, from definition \eqref{multifourier} and previous results about the Fourier transform,
  we see that
  \begin{align*}
    \fourier{\frac{\partial^n u(\vec{x}, t)}{\partial x_j^n}}
    &= \frac{1}{(2\pi)^{n/2}}\int_{-\infty}^\infty \dots \int_{-\infty}^\infty\frac{\partial^n }{\partial x_j^n}\left[u(\vec{x}, t)\right] e^{-i x_1 k_1}\cdots e^{-i x_n k_n} dx_1 \dots dx_n \\
    &= \frac{1}{(2\pi)^{(n-1)/2}}\int_{-\infty}^\infty \fourierwrt{\frac{\partial^n }{\partial x_j^n}\left[u(\vec{x}, t)\right]}{x_j} e^{-i \pi_j(\vec{x}) \cdot \pi_j(\vec{k})} d\pi_j(\vec{x}) \\
    &= \frac{(ik_j)^n}{(2\pi)^{(n-1)/2}}\int_{-\infty}^\infty \fourierwrt{u(\vec{x}, t)}{x_j} e^{-i \pi_j(\vec{x}) \cdot \pi_j(\vec{k})} d\pi_j(\vec{x}) \\
    &= (ik_j)^n\fourier{u(\vec{x}, t)}.
  \end{align*}

  Now, define $\vec{x} = (x_1, x_2, x_3) = (x, y, z) \in\mathbb{R}^3$. The the system of differential equations
  of the function $u(\vec{x}, t) = u(x_1, x_2, x_3, t)$ becomes
  \begin{align*}
    a^2\left(\frac{\partial^2 u}{\partial x_1^2} + \frac{\partial^2 u}{\partial x_2^2} + \frac{\partial^2 u}{\partial x_3^2}\right) - \frac{\partial^2 u}{\partial t^2} = 0, \quad -\infty < x_1, x_2, x_3 < \infty,
  \end{align*}
  subject to the initial conditions
  \begin{align*}
    &u(\vec{x}, t)\rvert_{t=0} = 0, \qquad\left.\frac{\partial u(\vec{x}, t)}{\partial t}\right\rvert_{t=0} = \delta(\vec{x}).
  \end{align*}

  Applying the Fourier transform with respect to $\vec{x}$ to the first equation yields
  \begin{align*}
    \fourier{a^2\left(\frac{\partial^2 u}{\partial x_1^2} + \frac{\partial^2 u}{\partial x_2^2} + \frac{\partial^2 u}{\partial x_3^2}\right) - \frac{\partial^2 u}{\partial t^2}}
    =-a^2\norm{\vec{x}}^2U(\vec{k}, t)- \der{t}{2}\left[U(\vec{k}, t)\right] = 0 = \fourier{0}
  \end{align*}
  where $U(\vec{k}, t) = \fourier{u(\vec{x}, t)}$. Similarly, we deduce that
  the transformed initial conditions become
  \begin{align*}
    &\fourier{u(\vec{x}, t)\rvert_{t=0}} = U(\vec{k}, t)\rvert_{t=0} = 0 = \fourier{0}, \\
    &\fourier{\left.\frac{\partial u(\vec{x}, t)}{\partial t}\right\rvert_{t=0}} = \left.\der{t}{}\left[U(\vec{k}, t)\right]\right\rvert_{t=0} = \frac{1}{(2\pi)^{3/2}} = \fourier{\delta(\vec{x})}.
  \end{align*}

  We see that the first transformed equation is a second-order
  linear homogeneous ordinary differential equation, from which we readily see that the solution is
  \begin{align*}
    U(\vec{k}, t) = c_1 \cos(a\norm{\vec{k}} t) + c_2 \sin(a \norm{\vec{k}} t).
  \end{align*}
  Using this solution we see from the first transformed initial condition that
  \begin{align*}
    U(\vec{k}, t)\rvert_{t=0}
    &= \left.c_1 \cos(a\norm{\vec{k}} t) + c_2 \sin(a \norm{\vec{k}} t)\right\rvert_{t=0} = c_1 = 0.
  \end{align*}
  From the second transformed initial condition, we see using the above solution that
  \begin{align*}
    \left.\der{t}{}\left[U(\vec{k}, t)\right]\right\rvert_{t=0}
    &= -a\norm{\vec{k}}c_1\sin(a\norm{\vec{k}}t) + a\norm{\vec{k}}c_2\cos(a\norm{\vec{k}}t)\rvert_{t=0} \\
    &= a\norm{\vec{k}}c_2 \\
    &= \frac{1}{(2\pi)^{3/2}},
  \end{align*}
  or that $\displaystyle c_2 = \frac{1}{a\norm{\vec{k}}(2\pi)^{3/2}}.$ Therefore, the solution to the transformed
  system of differential equations is
  \begin{align*}
    U(\vec{k}, t) = \frac{\sin(a\norm{\vec{k}}t)}{a\norm{\vec{k}}(2\pi)^{3/2}}.
  \end{align*}

  Applying the inverse Fourier Transform to the above solution gives the solution to the original
  system
  \begin{align*}
    u(\vec{x}, t) &= \fourierinv{\frac{\sin(a\norm{\vec{k}}t)}{a\norm{\vec{k}}(2\pi)^{3/2}}}\\
    &= \frac{1}{(2\pi)^3}\int_{-\infty}^\infty \frac{\sin(a\norm{\vec{k}}t)}{a\norm{\vec{k}}} e^{i\vec{k}\cdot \vec{x}} d\vec{k} \\
    &= \frac{1}{(2\pi)^3}\int_{-\infty}^\infty\int_{-\infty}^\infty\int_{-\infty}^\infty \frac{\sin(a\norm{\vec{k}}t)}{a\norm{\vec{k}}} e^{ik_1 x_1} e^{ik_2 x_2} e^{ik_3 x_3}dk_1 dk_2 dk_3 .
  \end{align*}
\end{proof}
\newpage
