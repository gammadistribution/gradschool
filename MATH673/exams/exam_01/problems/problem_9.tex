\begin{problem}{9}
  Show that if $E$ is a solution of an $m$-th order partial differential equation
  \begin{align*}
    P(\partial)u = \sum_{n=0}^m a_n \partial^n u = \sqrt{2\pi} \delta,
  \end{align*}
  where $\delta$ is the Dirac delta function, then $E \ast f$ is the solution of
  the partial differential equation $P(\partial)u = f$, where $\ast$ is the convolution.
\end{problem}

\begin{proof}
  Suppose that $E$ is a solution of the partial differential equation
  \begin{align*}
    P(\partial)u = \sum_{n=0}^m a_n \partial^n u = \sqrt{2\pi} \delta,
  \end{align*}
  i.e.\ $P(\partial)E = \sqrt{2\pi}\delta$. Then applying the Fourier Transform
  shows that
  \begin{align}\label{partial}
    \fourier{P(\partial)E} &= \sum_{n=0}^m a_n \fourier{\partial^n E}
    = \sum_{n=0}^m a_n (ik)^n \fourier{E} = 1 = \fourier{\sqrt{2\pi} \delta}.
  \end{align}
  Now, we apply the Fourier Transform to the differential
  equation replacing $u$ with $E \ast f$. Doing so yields
  \begin{align*}
    \fourier{P(\partial)(E\ast f)} &= \sum_{n=0}^m a_n \fourier{\partial^n (E\ast f)} \\
    &=  \sum_{n=0}^m a_n (ik)^n\fourier{E\ast f}.
  \end{align*}

  The Convolution Theorem states that $\fourier{E \ast f} = \fourier{E}\fourier{f}$ which implies that
  \begin{align*}
    \fourier{P(\partial)(E\ast f)}
    &=  \sum_{n=0}^m a_n (ik)^n\fourier{E\ast f} \\
    &=  \sum_{n=0}^m a_n (ik)^n\fourier{E}\fourier{f} \\
    &=  \fourier{f}\sum_{n=0}^m a_n (ik)^n\fourier{E} \\
    &= \fourier{f}\fourier{P(\partial)E}.
  \end{align*}
  Thus, from \eqref{partial} we see that
  \begin{align*}
    \fourier{P(\partial)(E\ast f)} =  \fourier{f}\fourier{P(\partial)E} = \fourier{f}.
  \end{align*}
  Therefore, applying the inverse Fourier Transform we have that
  \begin{align*}
    P(\partial)(E\ast f) = \fourierinv{\fourier{P(\partial)(E\ast f)}} = \fourierinv{\fourier{f}} = f
  \end{align*}
  or that $E\ast f$ is a solution of the partial differential equation $P(\partial)u =f$, and we are done.

\end{proof}
\newpage
