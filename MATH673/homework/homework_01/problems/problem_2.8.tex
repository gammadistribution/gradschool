\begin{problem}{2.8}
  Use the Fourier transform to solve the following ordinary differential equations
  for $-\infty < x < \infty$:
  \begin{enumerate}
    \item [a.] $y''(x) - y(x) + 2f(x) = 0,$ where $f(x) = 0$ when $x < -a$
      and when $x > a$ and its derivatives vanish at $x = \pm \infty$,
    \item [b.] $2 y''(x) + x y'(x) + y(x) = 0.$
  \end{enumerate}
\end{problem}

\begin{proof}
  Consider the following $n$-th order ordinary differential equation with constant
  coefficients:
  \begin{align*}
    L y(x) = f(x),
  \end{align*}
  where $L$ is the $n$-th order differential operation given by
  \begin{align*}
    L \equiv a_n \der{x}{n} + a_{n-1}\der{x}{n-1} + \dots + a_1 \der{x}{n} + a_0.
  \end{align*}

  Applying the Fourier transform to both sides of the ordinary differential equation gives
  \begin{align}\label{fourier_ode}
    \left[a_n (ik)^n + a_{n-1}(ik)^{n-1} + \dots +a_1(ik) + a_0\right]Y(k) = F(k)
  \end{align}
  where $Y(k) = \fourier{y(x)}$ and $F(k) = \fourier{f(x)}$. Set $P(z) = \sum_{k=0}^n a_k z^k$. Then
  \eqref{fourier_ode} may be rewritten as $P(ik)Y(k) = F(k),$ or equivalently
  \begin{align*}
    Y(k) = \frac{F(k)}{P(ik)} = F(k)Q(k)
  \end{align*}
  where $Q(k) = 1/P(ik)$. The Convolution theorem then tell us that $y(x)$, the solution to the
  ordinary differential equation, is given by
  \begin{align*}
    y(x) = \fourierinv{F(k)Q(k)}
    = \frac{1}{\sqrt{2\pi}}\int_{-\infty}^{\infty} f(\xi) q(x-\xi)d\xi,
  \end{align*}
  where $q(x) = \fourierinv{Q(k)}$ is known explicitly.

  \begin{enumerate}
    \item[a.] The ordinary differential equation $y''(x) - y(x) + 2f(x) = 0$ may be rewritten
      as
      \begin{align*}
        \left[a_2 \der{x}{2} + a_0\right] y(x) = g(x)
      \end{align*}
      where $a_2 = 1$, $a_0 = -1$, and $g(x) = -2f(x)$. Applying the Fourier transform
      to this ordinary differential equation results in the following equation
      \begin{align*}
        \left[a_2(ik)^2 + a_0\right]Y(k) = G(k),
      \end{align*}
      where $Y(k) = \fourier{y(x)}$ and $G(k) = \fourier{g(x)}$. Thus, we have that
      \begin{align*}
        Y(k) = G(k)Q(k)
      \end{align*}
      where
      \begin{align*}
        Q(k) = -\frac{1}{k^2+1}.
      \end{align*}

      By definition, the inverse Fourier transform of a function $Q(k)$ is given by
      \begin{align*}
        \fourierinv{Q(k)} = \frac{1}{\sqrt{2\pi}} \int_{-\infty}^{\infty} e^{i k x} Q(k) dk.
      \end{align*}

      Using Cauchy's Residue theorem allows us to evaluate this integral for $Q(k)$. The theorem
      states that the integral of $h(k) = e^{i k x} Q(k)$ around any closed contour
      is given in terms of the residues of $h$ at all singular points bounded by the contour.

      So, suppose first that $x > 0$ and consider the contour $C$ that winds counter clockwise along the straight line segment $[-a, a]$, say $C_1$,
      through the counter $C_2$ defined to be the half circle such that $z = i$ is bounded by this contour. Then,
      $h$ is analytic everywhere except at $k=i$ and by Cauchy's Residue theorem we have that
      \begin{align*}
        \int_C h(k) dk = \int_{C_1} h(k) dk + \int_{C_2} h(k) dk
        = 2\pi i\lim_{k\to i} (k-i)h(k)
        = 2\pi i\left[\frac{e^{-x}}{2i}\right]
        = \pi e^{-x}.
      \end{align*}

      Thus,
      \begin{align*}
        \int_{-a}^{a} h(k) dk = \pi e^{-x} - \int_{C_2} h(k) dk.
      \end{align*}
      It can be shown that
      \begin{align*}
        \left| \int_{C_2} h(k) dk \right| \leq \frac{a \pi}{a^2 -1}.
      \end{align*}
      As $a\to \infty$, we have that $a \pi/(a^2 -1) \to 0$ so
      \begin{align*}
        \int_{-\infty}^{\infty} h(k) dk = \pi e^{-x}.
      \end{align*}

      A similar argument follows when $x < 0$ by changing the contour to wind around $z=-i$ in a clockwise fashion to yield that
      \begin{align*}
        \int_{-\infty}^{\infty} h(k) dk = \pi e^{x}.
      \end{align*}

      Therefore, we have that
      \begin{align*}
        \fourierinv{Q(k)} &= -\frac{1}{\sqrt{2\pi}} \int_{-\infty}^{\infty} \frac{e^{i k x}}{k^2+1}  dk \\
        &= -\frac{\pi e^{-|x|}}{\sqrt{2\pi}}.
      \end{align*}

      Now, from the previous remarks, this implies that the solution to the ordinary
      differential equation is given by
      \begin{align*}
        y(x) &= \frac{1}{\sqrt{2\pi}}\int_{-\infty}^{\infty} g(\xi) q(x-\xi)d\xi \\
        &= \frac{1}{\sqrt{2\pi}}\frac{2\pi}{\sqrt{2\pi}}\int_{-a}^{a} f(\xi) e^{-|x-\xi|} d\xi \\
        &= \int_{-a}^{a} f(\xi) e^{-|x-\xi|} d\xi.
      \end{align*}
    \item[b.] We begin by applying the Fourier transform
      to the ordinary differential equation $2 y''(x) + x y'(x) + y(x) = 0$.
      Using the linearity and other properties of the Fourier transform, we see that the left hand side
      becomes
      \begin{align*}
        \fourier{2 y''(x) + x y'(x) + y(x)}
        &= 2\fourier{y''(x)} + \fourier{xy'(x)} + \fourier{y(x)} \\
        &= (1 - 2k^2)\fourier{y(x)} + \fourier{xy'(x)}.
      \end{align*}
      We have shown previously that $\displaystyle \fourier{x g(x)} = i \der{k}{} \fourier{g(x)}$
      and $\fourier{g'(x)} = ik \fourier{g(x)}$. Thus, we have that
      \begin{align*}
        \fourier{x y'(x)} &= i \der{k}{}\left[\fourier{y'(x)}\right] \\
        &= i^2\der{k}{}\left[kY(k)\right] \\
        &= - Y(k) - kY'(k).
      \end{align*}
      So, we see that
      \begin{align*}
        \fourier{2 y''(x) + x y'(x) + y(x)}
        &= (1 - 2k^2)\fourier{y(x)} + \fourier{xy'(x)} \\
        &= (1-2k^2)Y(k) - Y(k) - kY'(k) \\
        &= -2k^2Y(k) - kY'(k).
      \end{align*}

      Since $\fourier{0} = 0$, the ordinary differential equation under the Fourier
      transform becomes
      \begin{align*}
        -2k^2Y(k) - kY'(k) = 0
      \end{align*}
      or, equivalently,
      \begin{align*}
        Y'(k) = -2kY(k).
      \end{align*}

      This is a separable differential equation, the solution to which we readily see is
      \begin{align*}
        Y(k) = c_1 e^{-k^2}.
      \end{align*}

      Thus, we have that
      \begin{align*}
        y(x) = \fourierinv{c_1 e^{-k^2}} = c_1 \fourierinv{e^{-k^2}}
      \end{align*}
      Using a table of Fourier transforms, we have that
      \begin{align*}
        \fourierinv{ \frac{1}{\sqrt{2a}} \exp\left(-\frac{k^2}{4a}\right)} = \exp(-ax^2).
      \end{align*}
      Setting $a=1/4$, we have that
      \begin{align*}
        \sqrt{2}\fourierinv{ \exp\left(-k^2\right)} = \exp\left(-\frac{x^2}{4}\right).
      \end{align*}

      Therefore, the solution to the ordinary differential equation is given by
      \begin{align*}
        y(x) = c_1 \fourierinv{ \exp\left(-k^2\right)} = \frac{1}{c_1\sqrt{2}}e^{-\frac{x^2}{4}}.
      \end{align*}
  \end{enumerate}
\end{proof}
\newpage
