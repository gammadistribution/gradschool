\begin{problem}{2.3}
  Show that
  \begin{enumerate}
    \item [a.] $\displaystyle i \der{k}{} F(k) = \fourier{x f(x)}$
    \item [b.] $\displaystyle i^n \der{k}{n} F(k) = \fourier{x^n f(x)}$
  \end{enumerate}
\end{problem}

\begin{proof}
  \begin{enumerate}
    \item [a.] Recall from the definition of the Fourier transform in \eqref{fourier} that
      \begin{align*}
        F(k) = \fourier{f(x)} = \frac{1}{\sqrt{2\pi}}\int_{-\infty}^{\infty} e^{-ikx} f(x) dx.
      \end{align*}
      Using the Leibniz integral rule, we have that
      \begin{align*}
        \der{k}{}F(k) = \der{k}{}\fourier{f(x)} &= \frac{1}{\sqrt{2\pi}}\int_{-\infty}^{\infty} \frac{\partial }{\partial k}\left[e^{-ikx} f(x)\right] dx\\
        &= -\frac{1}{\sqrt{2\pi}}\int_{-\infty}^{\infty} ix e^{-ikx} f(x)dx \\
        &= - i \fourier{x f(x)}.
      \end{align*}
      Of course, this implies that
      \begin{align*}
        i \der{k}{}F(k) = \fourier{x f(x)}.
      \end{align*}
    \item [b.] Suppose that for $n  > 1$ we have that
      \begin{align}\label{ind}
        i^n \der{k}{n} F(k) = \fourier{x^n f(x)}.
      \end{align}
      Then by the Leibniz integral rule, we have that
      \begin{align*}
        i^n \der{k}{n+1} F(k) = \der{k}{}\left[i^n \der{k}{n} F(k)\right]
        &= \der{k}{}\left[\frac{1}{\sqrt{2\pi}} \int_{-\infty}^{\infty} x^n e^{-ikx} f(x) dx \right]\\
        &= \frac{1}{\sqrt{2\pi}}\int_{-\infty}^{\infty} \frac{\partial }{\partial k}\left[x^ne^{-ikx} f(x)\right] dx\\
        &= -i\frac{1}{\sqrt{2\pi}}\int_{-\infty}^{\infty} x^{n+1} e^{-ikx} f(x)dx \\
        &= -i \fourier{x^{n+1}f(x)}.
      \end{align*}
      This implies that
      \begin{align*}
        i^{n+1} \der{k}{n+1} F(k) = \fourier{x^{n+1} f(x)}
      \end{align*}
      and \eqref{ind} holds by induction.
  \end{enumerate}
\end{proof}
\newpage
