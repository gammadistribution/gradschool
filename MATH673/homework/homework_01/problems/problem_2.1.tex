\begin{problem}{2.1}
  Find the Fourier transforms of each of the following functions:
  \begin{enumerate}
    \item [c.] $f(x) = \delta ^{(n)}(x),$
    \item [f.] $\displaystyle f(x) = x \exp \left( -\frac{a x^2}{2} \right), a > 0,$
    \item [g.] $\displaystyle f(x) = x^2 \exp \left( -\frac{x^2}{2} \right).$
  \end{enumerate}
\end{problem}

\begin{proof}
  Recall that, by definition, we have that for a function $f(x) \in L^1(\mathbb{R})$,
  its Fourier transform is given by
  \begin{align}\label{fourier}
    \fourier{f(x)} = F(k) = \frac{1}{\sqrt{2 \pi}} \int_{-\infty}^{\infty} e^{-ikx}f(x) dx
  \end{align}
  where $k\in\mathbb{R}$.

  \begin{enumerate}
    \item [c.] The Dirac delta function $\delta(x)$ is defined such that for any good function $g(x)$ we have that
      \begin{align*}
        \int_{-\infty}^{\infty} \delta(x) g(x) dx = g(0).
      \end{align*}

      A good function is defined as a function in $C^{\infty}$ that decays sufficiently rapidly.
      Since it is clear that $\delta(x) \to 0$ as $|x| \to \infty$, we have by a previous theorem that
      \begin{align}\label{delta_der}
        \fourier{\delta'(x)} = ik \fourier{\delta(x)}.
      \end{align}

      By \eqref{fourier} and the definition of the Dirac delta function, we see that
      \begin{align*}
        \fourier{\delta(x)} &= \frac{1}{\sqrt{2\pi}}\int_{-\infty}^{\infty} e^{- i k x}\delta(x) dx = \frac{1}{\sqrt{2\pi}}.
      \end{align*}

      Thus, using \eqref{delta_der}, we can easily see by induction for $n > 1$ that
      \begin{align*}
        \fourier{\delta^{(n)}(x)} = ik \fourier{\delta^{(n-1)}(x)} = \dots = \frac{(ik)^n}{\sqrt{2 \pi}}.
      \end{align*}

    \item [f.] From \eqref{fourier}, we see that
      \begin{align*}
        \fourier{f(x)} &= \frac{1}{\sqrt{2 \pi}} \int_{-\infty}^{\infty} x \exp(-ikx) \exp\left(-\frac{ax^2}{2}\right) dx \\
        &= \frac{\exp\left( \frac{(ik)^2}{2a} \right)}{\sqrt{2 \pi}} \int_{-\infty}^{\infty} x \exp\left(-\frac{ax^2}{2} - ikx - \frac{(ik)^2}{2a}\right)  dx \\
        &= \frac{\exp\left( -\frac{k^2}{2a} \right)}{\sqrt{2 \pi}} \int_{-\infty}^{\infty} x \exp\left(-\frac{a}{2} \left(x + \frac{ik}{a}\right)^2 \right)  dx.
      \end{align*}

      Making the substitution $u = x + ik/a$, where $du = dx$, we have that
      \begin{align}\label{subst}
        \fourier{f(x)} &= \frac{\exp\left( -\frac{k^2}{2a} \right)}{\sqrt{2 \pi}} \int_{-\infty}^{\infty} \left(u - \frac{ik}{a}\right) \exp\left(-\frac{au^2}{2}  \right)  du \notag \\
        &= \frac{\exp\left( -\frac{k^2}{2a} \right)}{\sqrt{2 \pi}} \left[\int_{-\infty}^{\infty} u\exp\left(-\frac{au^2}{2}  \right) du - \frac{ik}{a}\int_{-\infty}^{\infty} \exp\left(-\frac{au^2}{2}  \right)du\right].
      \end{align}

      Since the function $g(x) = u\exp\left(-\frac{au^2}{2}\right)$ is odd, we know that
      \begin{align*}
        \int_{-\infty}^{\infty} u\exp\left(-\frac{au^2}{2}  \right) du = 0.
      \end{align*}

      Using the formula for the general Gaussian integral we have that
      \begin{align*}
        \int_{-\infty}^{\infty} \exp\left(-\frac{au^2}{2}  \right) = \frac{\sqrt{2\pi}}{\sqrt{a}}
      \end{align*}
      when $a > 0$.

      Combining, we see from \eqref{subst} that the Fourier transform of $\displaystyle f(x) = x \exp \left( -\frac{a x^2}{2} \right)$ for $a > 0$
      is
      \begin{align*}
        \fourier{f(x)} &= \frac{\exp\left( -\frac{k^2}{2a} \right)}{\sqrt{2 \pi}} \left[\int_{-\infty}^{\infty} u\exp\left(-\frac{au^2}{2}  \right) du - \frac{ik}{a}\int_{-\infty}^{\infty} \exp\left(-\frac{au^2}{2}  \right)du\right] \\
        &= \frac{\exp\left( -\frac{k^2}{2a} \right)}{\sqrt{2 \pi}} \left(- \frac{ik}{a}\right) \left(\frac{\sqrt{2\pi}}{\sqrt{a}}\right) \\
        &= - \frac{ik\exp\left( -\frac{k^2}{2a} \right)}{a\sqrt{a}} .
      \end{align*}

    \item [g.] From \eqref{fourier}, we see that
      \begin{align*}
        \fourier{f(x)} &= \frac{1}{\sqrt{2 \pi}} \int_{-\infty}^{\infty} x^2 \exp(-ikx) \exp \left( -\frac{x^2}{2} \right) dx \\
        &= \frac{\exp\left( \frac{(ik)^2}{2} \right)}{\sqrt{2 \pi}} \int_{-\infty}^{\infty} x^2 \exp\left(-\frac{x^2}{2} - ikx - \frac{(ik)^2}{2}\right)  dx \\
        &= \frac{\exp\left( -\frac{k^2}{2} \right)}{\sqrt{2 \pi}} \int_{-\infty}^{\infty} x^2 \exp\left(-\frac{1}{2} \left(x + ik\right)^2 \right)  dx.
      \end{align*}

      Making the substitution $u = x + ik$, where $du = dx$, we have that
      \begin{align}\label{subst_g}
        \fourier{f(x)} &= \frac{\exp\left( -\frac{k^2}{2} \right)}{\sqrt{2 \pi}} \int_{-\infty}^{\infty} \left(u - ik\right)^2 \exp\left(-\frac{u^2}{2}  \right)  du \notag \\
        &= \frac{\exp\left( -\frac{k^2}{2} \right)}{\sqrt{2 \pi}} \int_{-\infty}^{\infty} \left(u^2 -2 i k u - k^2\right) \exp\left(-\frac{u^2}{2}  \right)  du.
      \end{align}

      After distributing the exponential term to the polynomial in \eqref{subst_g} and splitting the integral using the operator's linearity,
      the first integral may be computed by parts by setting $w = u$ and $dv = -u \exp\left(-\frac{u^2}{2}\right) du$ as so:
      \begin{align*}
        \int_{-\infty}^{\infty} u^2 \exp\left(-\frac{u^2}{2}  \right) du = -\left[ \left. u \exp\left(-\frac{u^2}{2}\right) \right\rvert_{-\infty}^{\infty} - \int_{-\infty}^{\infty} \exp\left(-\frac{u^2}{2}\right) du \right] = \sqrt{2\pi}
      \end{align*}
      where the last equality follows from the formula for the general Gaussian integral.

      The other two resulting integrals in \eqref{subst_g} are calculated very similarly to their corresponding integrals in 2.1.f, i.e.\
      \begin{align*}
        2 i k \int_{-\infty}^{\infty} u \exp\left(-\frac{u^2}{2}  \right) du &= 0, \\
        k^2 \int_{-\infty}^{\infty} \exp\left(-\frac{u^2}{2}  \right) du &= k^2\sqrt{2\pi}.
      \end{align*}

      Therefore, we have that the Fourier transform of $\displaystyle f(x) = x^2 \exp \left( -\frac{x^2}{2} \right)$ is given by
      \begin{align*}
        \fourier{f(x)} &= \frac{\exp\left( -\frac{k^2}{2} \right)}{\sqrt{2 \pi}} \int_{-\infty}^{\infty} \left(u^2 -2 i k u - k^2\right) \exp\left(-\frac{u^2}{2}  \right)  du \\
        &= \frac{\exp\left( -\frac{k^2}{2} \right)}{\sqrt{2 \pi}}  \left[\sqrt{2\pi} -k^2\sqrt{2\pi}\right] \\
        &= (1-k^2)\exp\left( -\frac{k^2}{2} \right).
      \end{align*}

  \end{enumerate}
\end{proof}
\newpage
