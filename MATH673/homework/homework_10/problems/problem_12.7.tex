\begin{problem}{12.7}
  Solve the following difference equations:
  \begin{enumerate}
    \item[a.] $f(n+1)+3f(n)=n, \qquad f(0) = 1.$
    \item[e.] $f(n+2)-f(n+1)-6f(n)=0, \qquad f(0) = 0,\quad f(1)=3$
  \end{enumerate}
\end{problem}

\begin{proof}
  Recall that if $\Z{f(n)} = F(z)$ and $m \geq 0$, then the following property holds:
  \begin{align*}
    \Z{f(n+m)} = z^m \left[F(z) - \sum_{r=0}^{m-1}f(r) z^{-r}\right].
  \end{align*}

  \begin{enumerate}
    \item[a.] Applying the $Z$-transform to the difference equation, we have that
      \begin{align*}
        z F(z) -z f(0) + 3 F(z) = \frac{z}{(z-1)^2}.
      \end{align*}
      In light of the initial data, this reduces to
      \begin{align*}
        (z+3)F(z) - z = \frac{z}{(z-1)^2}.
      \end{align*}
      Solving the resulting algebraic equation yields
      \begin{align*}
        F(z) = \frac{z (z^2 -2z + 2)}{(z+3)(z-1)^2}
      \end{align*}
      Applying the method of partial fraction decomposition to this transformed function
      shows that
      \begin{align*}
        F(z)
        &= \frac{z (z^2 -2z + 2)}{(z+3)(z-1)^2} \\
        &= z\left[\frac{a_1}{z+3} + \frac{a_2}{z-1} + \frac{a_3}{(z-1)^2}\right] \\
        &= \frac{17}{16}\left(\frac{z}{z+3}\right) - \frac{1}{16}\left(\frac{z}{z-1}\right) + \frac{1}{4}\left[\frac{z}{(z-1)^2}\right].
      \end{align*}

      Therefore, using the fact that
      \begin{align*}
        \Z{a^n} = \frac{z}{z-a}
      \end{align*}
      and
      \begin{align*}
        \Z{n} = \frac{z}{(z-1)^2},
      \end{align*}
      we see that the solution to the original difference equation is
      \begin{align*}
        f(n) = \Zinv{F(z)}
        &= \frac{17}{16}\Zinv{\frac{z}{z+3}} - \frac{1}{16}\Zinv{\frac{z}{z-1}} + \frac{1}{4}\Zinv{\frac{z}{(z-1)^2}} \\
        &= \frac{17}{16}(-3)^n - \frac{1}{16} + \frac{1}{4}n \\
      \end{align*}

    \item[e.] Applying the $Z$-transform to the Initial Value Problem, we have that
      \begin{align*}
        z^2 F(z) - z^2f(0) - z f(1) - z F(z) + z f(0) - 6F(z) = 0.
      \end{align*}
      In light of the initial data, this reduces to
      \begin{align*}
        \left(z-3\right)\left(z+2\right) F(z) -3z = 0
      \end{align*}

      Thus, the solution to the transformed equation is
      \begin{align*}
        F(z) &= \frac{3z}{(z-3)(z+2)}.
      \end{align*}

      Applying the method of partial fraction decomposition to this transformed function shows that
      \begin{align*}
        F(z)
        &= \frac{3z}{(z-3)(z+2)} \\
        &= 3z\left[\frac{a_1}{z-3}+\frac{a_2}{z+2}\right] \\
        &= \frac{3}{5}\left[\frac{z}{z-3}-\frac{z}{z+2}\right].
      \end{align*}

      Therefore, using the fact that
      \begin{align*}
        \Z{a^n} = \frac{z}{z-a},
      \end{align*}
      we see that the solution to the original difference equation is
      \begin{align*}
        f(n) = \Zinv{F(z)}
        &= \frac{3}{5}\left[\Zinv{\frac{z}{z-3}}-\Zinv{\frac{z}{z+2}}\right] \\
        &= \frac{3}{5}\left[3^n - (-2)^n\right].
      \end{align*}
  \end{enumerate}
\end{proof}
\newpage
