\begin{problem}{7.9}
  Solve the problem of the electrified unit disk in the $(x,y)$ plane with center at the origin.
  The electric potential $u(r, z)$ is axisymmetric and satisfies the boundary value problem
  \begin{align*}
    &u_{rr} + \frac{1}{r} u_r + u_{zz} = 0, \qquad 0 < r < \infty, \quad 0 < z < \infty,\\
    &u(r, 0) = u_0, \qquad 0 \leq r < a \\
    &\frac{\partial u}{\partial z} = 0, \qquad \text{on $z=0$ for $a < r < \infty$},\\
    & u(r,z) \to 0 \qquad \text{as $z\to\infty$ for all $r$},
  \end{align*}
  where $u_0$ is constant. Show that the solution is
  \begin{align*}
    u(r,z) = \left(\frac{2au_0}{\pi}\right) \int_0^\infty k J_0(kr) \left(\frac{\sin ak}{k^2}\right)e^{-kz} dk.
  \end{align*}
\end{problem}

\begin{proof}
  In order to find the solution to the boundary value problem, we will apply
  the 0-th order Hankel transform to the system of differential equations.

  Let $\tilde{u}_0(k, z) = \hankel{0}{u(r, z)}$. Then
  from a previous theorem we have that
  \begin{align}\label{hankel0}
    \hankel{0}{u_{rr} + \frac{1}{r}u_r} = -k^2 \tilde{u}_0(k, z).
  \end{align}
  Thus, from the above result in combination with Leibniz's integral rule, we see that
  applying the 0-th order Hankel transform to the boundary value problem yields
  \begin{align*}
    &\der{z}{2}\left[\tilde{u}_0(k, z)\right] - k^2 \tilde{u}_0(k, z) = 0, \qquad 0 < r < \infty, \quad 0 < z < \infty.
  \end{align*}
  This is a homogeneous linear differential equation and we readily see that
  the solution to the equation is
  \begin{align}\label{homo}
    \tilde{u}_0(k, z) = c_1 e^{-kz} + c_2 e^{kz}.
  \end{align}

  Note that the boundary conditions
  \begin{align*}
    &u(r, 0) = u_0, \qquad 0 \leq r < a \\
    &\der{z}{}\left[\tilde{u}_0(k, z)\right] = 0, \qquad \text{on $z=0$ for $a < r < \infty$}
  \end{align*}
  are equivalent to
  \begin{align*}
    &u(r, 0) = u_0H(a - r), \qquad 0 \leq r < \infty \\
    &\der{z}{}\left[\tilde{u}_0(k, z)\right] = H(a-r), \qquad \text{on $z=0$ for $0 < r < \infty$}.
  \end{align*}
  Thus, we see that the transformed boundary conditions are
  \begin{align*}
    &\tilde{u}_0(k, 0) = \frac{a u_0}{k}J_1(ak),\qquad 0 \leq r < \infty \\
    &\der{z}{}\left[\tilde{u}_0(k, z)\right] = \frac{a }{k}J_1(ak), \qquad \text{on $z=0$ for $a < r < \infty$},\\
    &\tilde{u}_0(k, z) \to 0 \qquad \text{as $z\to\infty$ for all $k$}.
  \end{align*}
  % Note that if $\tilde{u}_0(k, z)$ is of the form \eqref{homo},
  % then $\tilde{u}_0(k, z) \to 0 \qquad \text{as $z\to\infty$ for all $k$}$
  % if and only if $c_2 = 0$. Thus, \eqref{homo} reduces to
  % \begin{align}\label{homo2}
  %   \tilde{u}_0(k, z) = c_1 e^{-kz}.
  % \end{align}

  Using \eqref{homo} and the first transformed boundary condition, we see that
  \begin{align*}
    c_1 + c_2 = \frac{a u_0}{k}J_1(ak).
  \end{align*}
  Similarly, from \eqref{homo} and the second transformed boundary condition, we see that
  \begin{align*}
    -kc_1 + kc_2 = 0.
  \end{align*}
\end{proof}
\newpage
