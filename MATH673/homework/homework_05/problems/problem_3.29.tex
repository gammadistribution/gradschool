\begin{problem}{3.29}
  Suppose that $\laplace{f(t)} = \bar{f}(s)$ and $\laplace{g(t)} = \bar{g}(s)$.
  Show that
  \begin{align*}
    \laplaceinv{s \bar{f}(s) \bar{g}(s)} &= f(0)g(t) + \int_0^t g(t-\tau) f'(\tau) d\tau \\
    \laplaceinv{s \bar{f}(s) \bar{g}(s)} &= g(0)f(t) + \int_0^t f(t-\tau) g'(\tau) d\tau.
  \end{align*}
\end{problem}

\begin{proof}
  We wish to show that
  \begin{align*}
    \laplaceinv{s \bar{f}(s) \bar{g}(s)} &= f(0)g(t) + \int_0^t g(t-\tau) f'(\tau) d\tau.
  \end{align*}
  This is equivalent to showing that
  \begin{align*}
    \laplace{f(0)g(t) + \int_0^t g(t-\tau)  f'(\tau) d\tau} = s\bar{f}(s)\bar{g}(s).
  \end{align*}
  Note that we have by the definition of the convolution that
  \begin{align*}
    \int_0^t g(t-\tau)  f'(\tau) d\tau = (g \ast f')(t).
  \end{align*}
  Thus,
  \begin{align*}
    \laplace{f(0)g(t) + \int_0^t g(t-\tau)  f'(\tau) d\tau} = \laplace{g(t)f(0) + (g \ast f')(t)}.
  \end{align*}

  Using the linearity of the Laplace transform in combination with the Convolution Theorem, we have that
  \begin{align*}
    \laplace{f(0)g(t) + \int_0^t g(t-\tau)  f'(\tau) d\tau}
    &= \laplace{g(t)f(0) + (g \ast f')(t)} \\
    &= f(0) \laplace{g(t)} + \laplace{g(t)}\laplace{f'(t)}.
  \end{align*}
  Recall that we have shown previously that
  \begin{align*}
    \laplace{f'(t)} = s\laplace{f(t)} - f(0).
  \end{align*}

  Therefore,
  \begin{align*}
    \laplace{f(0)g(t) + \int_0^t g(t-\tau)  f'(\tau) d\tau}
    &= f(0) \laplace{g(t)} + \laplace{g(t)}\laplace{f'(t)} \\
    &=\laplace{g(t)}\left(f(0) + s \laplace{f(t)} - f(0)\right) \\
    &= s\laplace{f(t)}\laplace{g(t)} \\
    &= s\bar{f}(s)\bar{g}(s).
  \end{align*}
  Note the same argument can be repeated by interchanging $f$ and $g$ to show that
  \begin{align*}
    \laplace{g(0)f(t) + \int_0^t f(t-\tau) g'(\tau) d\tau} = s \bar{f}(s) \bar{g}(s),
  \end{align*}
  and we are done.
\end{proof}
\newpage
