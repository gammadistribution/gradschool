\begin{problem}{8.10}
  Show that the integral equation
  \begin{align*}
    f(x) = h(x) + \int_0^\infty f(\xi) g\left(\frac{x}{\xi}\right) \frac{d\xi}{\xi}
  \end{align*}
  has the formal solution
  \begin{align*}
    f(x) = \frac{1}{2\pi i} \int_{c -i \infty}^{c + i \infty}\frac{x^{-p}\tilde{h}(p)}{1-\tilde{g}(p)} dp.
  \end{align*}
\end{problem}

\begin{proof}
  Define the convolution of the functions $f(x)$ and $g(x)$ as
  \begin{align*}
    (f \ast g)(x) = \int_0^\infty f(\xi) g\left(\frac{x}{\xi}\right) \frac{d\xi}{\xi}.
  \end{align*}
  Then the integral equation becomes
  \begin{align*}
    f(x) = h(x) + (f \ast g)(x)
  \end{align*}

  Now, let $\tilde{f}(p), \tilde{g}(p), $ and $\tilde{h}(p)$ denote the Mellin transforms
  of $f(x), g(x), $ and $h(x)$, respectively.
  Taking the Mellin transform of the integral equation shows that
  \begin{align*}
    \tilde{f}(p) &= \mellin{h(x) + (f \ast g)(x)} \\
    &= \tilde{h}(p) + \mellin{(f \ast g)(x)} \\
    &= \tilde{h}(p) + \tilde{f}(p) \tilde{g}(p)
  \end{align*}
  where we have used the Convolution Type Theorem which states that
  \begin{align*}
    \mellin{(f \ast g)(x)} = \tilde{f}(p) \tilde{g}(p).
  \end{align*}

  Thus, after taking the Mellin transform, the integral equation becomes an algebraic one
  in the variable $p$. Solving for $\tilde{f}(p)$ shows that
  \begin{align*}
    \tilde{f}(p) = \frac{\tilde{h}(p)}{1 - \tilde{g}(p)}.
  \end{align*}
  Therefore, the formal solution to the integral equation is
  \begin{align*}
    f(x) = \mellininv{\tilde{f}(p)} = \frac{1}{2\pi i} \int_{c -i \infty}^{c + i \infty}\frac{x^{-p}\tilde{h}(p)}{1-\tilde{g}(p)} dp.
  \end{align*}
\end{proof}
\newpage
