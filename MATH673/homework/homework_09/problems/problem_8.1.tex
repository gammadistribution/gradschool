\begin{problem}{8.1}
  Find the Mellin transform of each of the following functions:
  \begin{enumerate}
    \item[a.] $f(x) = H(a - x)$, $a > 0$,
    \item[b.] $f(x) = x^m e^{-nx}$, $m,n > 0$,
    \item[c.] $\displaystyle f(x) = \frac{1}{x^2 + 1}.$
  \end{enumerate}
\end{problem}

\begin{proof}
  The Mellin transform of the function $f(x)$ is defined to be
  \begin{align*}
    \mellin{f(x)} = \tilde{f}(p) = \int_0^\infty x^{p-1} f(x) dx.
  \end{align*}

  \begin{enumerate}
    \item[a.] Recall that the Heaviside function $H$ is defined as
      \begin{align*}
        H(a-x) =
        \begin{cases}
          1 & \text{if $x < a$} \\
          0 & \text{if $x > a$} \\
        \end{cases}.
      \end{align*}
       Therefore, from the definition of the Mellin transform, we have that for
       $f(x) = H(a-x)$ with $a > 0$,
       \begin{align*}
         \tilde{f}(p) = \mellin{f(x)} &= \int_0^\infty x^{p-1}H(a-x) dx \\
         &= \int_0^a x^{p-1} dx \\
         &= \frac{a^p}{p}.
       \end{align*}

    \item[b.] Let $f(x) = x^m g(x)$ where $g(x) = e^{-nx}$
      with $m,n > 0$ and let $\tilde{g}(p) = \mellin{g(x)}$.

      By the shifting property of the Mellin transform, we have that
      \begin{align*}
        \tilde{f}(p) = \mellin{f(x)} = \mellin{x^m g(x)} = \tilde{g}(p + m).
      \end{align*}

      From our table of Mellin transforms, we know that
      \begin{align*}
        \tilde{g}(p) = \mellin{g(x)} = \frac{\Gamma(p)}{n^p}
      \end{align*}
      where $\Re{\{p\}} > 0$.

      Therefore,
      \begin{align*}
        \tilde{f}(p) = \mellin{f(x)} = \tilde{g}(p + m) = \frac{\Gamma(p + m)}{n^{p+m}}
      \end{align*}
      where $\Re{\{p + m\}} > 0$.

    \item[c.] From our table of Mellin transforms,
      we see that
      \begin{align*}
        \mellin{\frac{1}{(x^a + 1)^s}} = \frac{\Gamma(p/a)\Gamma(s - p/a)}{a \Gamma(s)}.
      \end{align*}

      Therefore, for $f(x) = \frac{1}{x^2 + 1}$, identifying $a = 2$ and $s = 1$, we have that
      \begin{align*}
        \tilde{f}(p) = \mellin{f(x)} = \mellin{\frac{1}{x^2 + 1}}
        &= \frac{\Gamma(p/2)\Gamma(1 - p/2)}{2 \Gamma(1)} \\
        &= \frac{\Gamma(p/2)\Gamma(1 - p/2)}{2 }.
      \end{align*}

  \end{enumerate}
\end{proof}
\newpage
