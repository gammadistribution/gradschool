\documentclass{article}
\usepackage{amsmath}
\usepackage{amssymb}
\usepackage{./.sty/appendix}
\usepackage{graphicx}
\usepackage{./.sty/titlesec}
\usepackage[dotinlabels]{./.sty/titletoc}
\usepackage[]{./.sty/mcode}


\graphicspath{ {images/} }


\titlelabel{\thetitle.\quad}


\renewenvironment{abstract}
  {\small\quotation
  {\bfseries\noindent{\large\abstractname}\par\nobreak\smallskip}}
  {\endquotation}


\title{
	{A Numerical Solution to a Second Order Ordinary Differential Equation}\\
}


\author{
	Hunter, Marcus \\
	\and
	Tiger, Matthew \\
	\and
	Wigfield, Jacob \\
}


\begin{document}


\maketitle
\newpage


\tableofcontents
\newpage


\begin{section}{Introduction}
  The authors were tasked by the client with finding the solution to the following
  family of differential equations
  \[
  \begin{cases}
    -u''(x) + c u(x) = f(x) \\
    0 \leq x \leq 1 \\
    u(0) = \epsilon \\
    u(1) = \delta.
  \end{cases}
  \]
  Additionally, the client has also requested to be provided with a means of
  plotting the solution once obtained.

  Throughout this report, the above family of differential equations together with
  the interval of definition and initial conditions will be represented by
  $Lu = f$.

  Assumptions were placed on this family so that $c \in \mathbb{R}$ with $c > 0$
  and $f \in C^k([0,1])$ for sufficiently large $k$ so that $f$ is relatively
  well-behaved on the defined interval.

  In this report we will detail the analytical solution to this family of
  differential equations showing the the above problem is well-posed and
  explain why this solution is not amenable to practical use. We therefore
  provide a numerical scheme to approximate the solution to the family of
  differential equations and examine the convergence, consistency and stability
  of the numerical scheme. Using the solution provided by the numerical scheme,
  we then explore the different options for plotting the solution.
\end{section}


\begin{section}{Analytical Solution}
  The family of differential equations $Lu = f$ represents a second order linear
  differential equation and therefore well-known techniques can be used to find
  the solution $u(x)$.

  The solution $u(x)$ is given by $u(x) = u_h(x) + u_p(x)$ where $u_h(x)$ is
  the solution to the homogeneous equation $-u''(x) + c u(x) = 0$ and $u_p(x)$
  is a particular solution of $-u''(x) + c u(x) = f(x)$.

  To find the homogeneous solution, note that the characteristic equation of
  this family of differential equations is given by $-m^2 + c = 0$, the roots of
  which are $m_1 = \sqrt{c} = \omega $ and $m_2 = -\sqrt{c} = -\omega$. Note
  that since $c > 0$, these roots are real and distinct suggesting that the
  homogeneous solution is given by
  \begin{align}\label{homogeneous_solution}
    u_h(x) = c_1 e^{\omega x} + c_2 e^{-\omega x}.
  \end{align}

  To find the particular solution, we assume the particular solution is of the
  form $u_p(x) = \kappa(x) e^{\omega x}$ for some unknown function $\kappa(x)$.
  Thus,
  \[
    u_p''(x) = \kappa''(x) e^{\omega x} + 2 \omega \kappa'(x) e^{\omega x} + \omega^2 \kappa(x) e^{\omega x}
  \]
  and substituting the above into the original differential equation $Lu = f$
  with $u_p(x) = \kappa(x) e^{\omega x}$ we have
  \begin{align}\label{particular_step_one}
    \kappa''(x) + 2\omega\kappa'(x) = -f(x)e^{-\omega x}.
  \end{align}
  Making the substitution $\lambda(x) = \kappa'(x)$ into \eqref{particular_step_one}
  we can reduce the above second order linear differential equation into the
  first order linear differential equation
  \begin{align}\label{particular_step_two}
    \lambda'(x) + 2\omega\lambda(x) = -f(x)e^{-\omega x}.
  \end{align}
  The homogeneous solution to this first order differential equation is given by
  $\lambda_h(x) = c_3 e^{-2\omega x}$ suggesting the particular solution to the
  first order differential equation is of the form $\lambda_p(x) = \mu(x) e^{-2\omega x}$.

  Repeating the same process as above, we see that
  \[
    \lambda_p'(x) = \mu'(x) e^{-2 \omega x} - 2 \omega\mu(x) e^{-2 \omega x}
  \]
  and substituting into \eqref{particular_step_two} with $\lambda_p(x) = \mu(x) e^{-2\omega x}$
  we find that the first order linear differential equation becomes the separable first order differential
  equation
  \[
    \mu'(x) = -f(x)e^{\omega x}.
  \]
  We readily see the solution to the above differential equation is given by
  \[
    \mu(x) = - \int_{0}^x f(r) e^{\omega r} dr.
  \]
  As $\kappa'(x) = \lambda_p(x) = \mu(x)e^{-2\omega x}$, we deduce that
  \begin{align*}
    \kappa(x) = - \int_{0}^{x} e^{-2 \omega s} \left[ \int_{0}^s f(r) e^{\omega r} dr \right] ds
  \end{align*}
  and
  \begin{align}\label{particular_solution}
    u_p(x) = \kappa(x)e^{\omega x} = -e^{\omega x} \int_{0}^{x} e^{-2 \omega s} \left[ \int_{0}^s f(r) e^{\omega r} dr \right] ds.
  \end{align}
  Combining the homogeneous solution \eqref{homogeneous_solution} and the particular
  solution \eqref{particular_solution} we have that the general solution to
  $Lu = f$ is given by
  \begin{align}\label{general_analytical_solution}
    u(x) &= u_h(x) + u_p(x) \nonumber \\
    &= c_1 e^{\omega x} + c_2 e^{-\omega x} - e^{\omega x} \int_{0}^{x} e^{-2 \omega s} \left[ \int_{0}^s f(r) e^{\omega r} dr \right] ds.
  \end{align}
  Using the boundary values provided in $Lu=f$, the general solution is specified
  by the system of linear equations
  \begin{align*}
    u(0) &= c_1 + c_2 = \epsilon \\
    u(1) &= c_1 e^{\omega} + c_2 e^{-\omega} - e^{\omega} \int_{0}^{1} e^{-2 \omega s} \left[ \int_{0}^s f(r) e^{\omega r} dr \right] ds = \delta .
  \end{align*}
  The solution to this system in terms of the unknowns $c_1$ and $c_2$ is given
  by
  \begin{align*}
    c_1 = \frac{\epsilon e^{-\omega} - \delta - e^{\omega} \int_{0}^{1} e^{-2 \omega s} \left[ \int_{0}^s f(r) e^{\omega r} dr \right] ds}{e^{-\omega} - e^{\omega}}\\
    c_2 = \frac{-\epsilon e^{\omega} + \delta + e^{\omega} \int_{0}^{1} e^{-2 \omega s} \left[ \int_{0}^s f(r) e^{\omega r} dr \right] ds}{e^{-\omega} - e^{\omega}}.
  \end{align*}

  Using these constants in the general solution \eqref{general_analytical_solution}
  gives us the unique analytical solution to the family of differential equation $Lu = f$.
  Furthermore, we deduce that the problem is in fact well-posed.

  From this solution, we must make the following additional assumption on this
  problem: $f(x)$ must be integrable on the interval $[0, 1]$.

  As the analytical solution depends on the symbolic integration of $f(x)$, we
  will be unable to use this solution for functions $f(x)$ in which the
  closed-form of the integral is not known.
\end{section}


\begin{section}{Numerical Scheme}\label{sec:scheme}
  As mentioned in the previous section, the analytical solution is not practical
  to use for most functions $f(x)$. Thus, we present a numerical solution to
  approximate the analytical solution for the problem $Lu = f$.

  \begin{subsection}{Description}
    Our solution is derived from the method of finite differences. We define
    a finite set of points on the interval $[0, 1]$ called the grid $D_h$ where
    the parameter $h$ is the size of the grid where a smaller $h$ denotes a finer
    grid. For our purposes, we consider $h=1/N$ for positive $N$ and
    create the uniform grid
    \[
      D_h = \{x_n| x_n = hn \text{ for $0 \leq n \leq N$}\}.
    \]

    Define on this grid the discretized solution to the problem $Lu = f$ as
    $[u]_h = \{u(x_n)\}$ and define the discretized function $f^{(h)} = \{f(x_n)\}$.
    We wish to create a scheme $L_h$ that computes an approximate solution
    $u^{(h)} = \left\{u_0^{(h)}, u_1^{(h)}, \dots, u_N^{(h)}\right\}$ to the problem
    $Lu = f$, i.e.\ a scheme such that $L_h u^{(h)} = f^{(h)}$.

    Finding an approximation to $u''(x)$ should suggest how to construct
    the scheme $L_h$.
    To find an approximation for $u''(x)$, we investigate the Taylor expansion
    of $u(x)$ centered at $h$ and $-h$. These expansions are given by
    \begin{align*}
      u(x + h) &= u(x) + h u'(x) + \frac{h^2 u ''(x)}{2} + \frac{h^3 u^{(3)}(x)}{3!} + \frac{h^4 u^{(4)}(\xi_1)}{4!} \\
      u(x - h) &= u(x) - h u'(x) + \frac{h^2 u ''(x)}{2} - \frac{h^3 u^{(3)}(x)}{3!} + \frac{h^4 u^{(4)}(\xi_2)}{4!}
    \end{align*}
    where $x \leq \xi_1 \leq x + h$ and $x - h \leq \xi_2 \leq x$.
    Adding these two expressions and solving for $u''(x)$ shows that
    \begin{align}\label{second_deriv}
      u''(x) = \frac{u(x + h) - 2u(x) + u(x-h)}{h^2} - \frac{h^2(u^{(4)}(\xi_1) + u^{(4)}(\xi_2))}{4!}.
    \end{align}
    This suggests that we should define our numerical scheme by replacing $u''(x)$
    in $Lu = f$ with the approximation
    \[
      u''(x) \approx \frac{u(x + h) - 2u(x) + u(x-h)}{h^2}.
    \]
    Therefore, we define the numerical scheme as
    \begin{align*}
      L_h u^{(h)} = f^{(h)} :=
      \begin{cases}
        \frac{-u_{n+1} + 2 u_n - u_{n-1}}{h^2} + c u_n = f_n & \text{for $n= 1, \dots, N - 1$} \\
        u_0 = \epsilon \\
        u_N = \delta
      \end{cases}.
    \end{align*}
    For $n=1,\dots,N-1$, the scheme presents us with the recurrence relation
    \[
      -u_{n-1} + (2 + ch^2) u_n - u_{n+1} = h^2 f_n
    \]
    with initial conditions $u_0 = \epsilon$ and $u_N = \delta$. This recurrence
    relation is represented by the following system of equations
    \begin{align*}
      (2 + ch^2) u_1 - u_2 &= h^2f_1 + u_0 \\
      -u_1 + (2 + ch^2) u_2 - u_3 &= h^2f_2 \\
      -u_2 + (2 + ch^2) u_3 - u_4 &= h^2f_3 \\
      \vdots &  \\
      -u_{N-2} + (2 + ch^2) u_{N-1} &= h^2f_{N-1} + u_N.
    \end{align*}
    In matrix form, this system of equations becomes
    \begin{align*}
      \begin{bmatrix}
        2 + ch^2 & -1 & 0 & \hdots & 0 & 0\\
        -1 & 2 + ch^2 & -1 & \hdots & 0 & 0\\
        0 & -1 & 2 + ch^2 & \hdots & 0 & 0\\
        \vdots & \vdots & \vdots & \ddots & \vdots & \vdots \\
        0 & 0 & 0 & \hdots & -1 & 2+ch^2 \\
      \end{bmatrix}
      \begin{bmatrix}
        u_1 \\
        u_2 \\
        u_3 \\
        \vdots \\
        u_{N-1}
      \end{bmatrix}
      =
      \begin{bmatrix}
        h^2 f_1 + u_0 \\
        h^2 f_2 \\
        h^2 f_3 \\
        \vdots \\
        h^2 f_{N-1} + u_{N}
      \end{bmatrix}
    \end{align*}
    The solution to this system of equations paired with the initial conditions
    allows us to explicitly find $u^{(h)}$, our scheme's solution.

    In section \ref{sec:scheme_prop} we examine the convergence, consistency,
    and stability of this scheme in order to determine its usefulness in
    approximating the analytical solution to the problem $Lu = f$.
  \end{subsection}

  \begin{subsection}{Implementation}
    We will now discuss the implementation of the above numerical scheme and
    how to use the implementation to numerically approximate the solution to the
    problem $Lu = f$.
    \subsubsection{Discretized Solution}

    \subsubsection{Plotting}

  \end{subsection}

\end{section}


\begin{section}{Numerical Scheme Properties}\label{sec:scheme_prop}
  There are three main properties of the numerical scheme presented in section
  \ref{sec:scheme} that are important to the validity of the numerical solution,
  namely the convergence, consistency, and stability of the numerical scheme.
  We will now investigate these properties in detail to determine how well
  the numerical scheme approximates the analytical solution to our differential
  equation.

  \begin{subsection}{Convergence}
  \end{subsection}

  \begin{subsection}{Consistency}
    The consistency of a numerical scheme is a measure of how well the solution
    obtained by the scheme approximates the analytical solution as the grid
    the scheme is defined on becomes more refined. Formally,
    for a scheme $L_h u^{(h)} = f^{(h)}$ for the problem $Lu = f$, we say the
    scheme is \textit{consistent} if
    \begin{align}\label{consistency}
      ||L_h[u]_h - L_h u^{(h)}|| \to 0 \text{\ as $h \to 0$}
    \end{align}
    where $|| u^{(h)} || = \max_n |u_n|$.

    Using the expression for the second
    derivative in \eqref{second_deriv} and replacing it in our problem $Lu = f$, we
    see that after some rearranging
    \begin{align*}
      \frac{-u(x + h) + 2u(x) - u(x-h)}{h^2} + cu(x) = f(x) - \frac{h^2(u^{(4)}(\xi_1) + u^{(4)}(\xi_2))}{4!}.
    \end{align*}
    For the discretized analytical solution $[u]_h$ to our problem $Lu = f$ on the grid $D_h$,
    this equation then becomes
    \begin{align*}
      \frac{-u(x_{n+1}) + 2u(x_n) - u(x_{n-1})}{h^2} + cu(x_n) = f(x_n) - \frac{h^2(u^{(4)}(\xi_1) + u^{(4)}(\xi_2))}{4!}.
    \end{align*}
    From this equation we notice that the left side is precisely the evaluation
    of our numerical scheme for the discretized analytical solution, i.e.
    \begin{align}\label{discretized}
      L_h[u]_h = f_n - \frac{h^2(u^{(4)}(\xi_1) + u^{(4)}(\xi_2))}{4!}.
    \end{align}
    Combining the expression in \eqref{discretized} with the fact that
    $L_hu^{(h)} = f_n$, we see that
    \begin{align*}
      ||L_h[u]_h - L_hu^{(h)}||
      &= \left|\left| \left(f_n - \frac{h^2(u^{(4)}(\xi_1) + u^{(4)}(\xi_2))}{4!}\right) - f_n \right|\right| \\
      &= \left|\left| \frac{(u^{(4)}(\xi_1) + u^{(4)}(\xi_2))}{4!}\right|\right| h^2
    \end{align*}

    From the above equation it is clear that $||L_h[u]_h - L_hu^{(h)}|| \to 0$ as $h \to 0$.
    Therefore, according to the definition in \eqref{consistency}, we see that
    our scheme $L_h[u]_h = f^{(h)}$ is consistent.

    If we make the assumption that the analytical solution's fourth derivative
    is bounded, i.e. $|u^{(4)}(x)| \leq M$ for all $0 \leq x \leq 1$, then
    \begin{align}\label{inequality}
      ||L_h[u]_h - L_hu^{(h)}||
      &= \left|\left| \frac{(u^{(4)}(\xi_1) + u^{(4)}(\xi_2))}{4!}\right|\right| h^2 \leq \frac{2M}{4!}h^2.
    \end{align}
    Moreover, from the inequality in \eqref{inequality}, we see that
    \begin{align}\label{order}
      ||L_h[u]_h - L_hu^{(h)}|| \leq \frac{2M}{4!}h^2 = Ch^2
    \end{align}
    where the constant $C$ does not depend on $h$. In this case, we then say that
    the scheme $L_h u^{(h)} = f^{(h)}$ has \textit{order of consistency} 2.

    We thus conclude that as our scheme is consistent, it does in fact approximate
    the analytical solution to the problem $Lu = f$ and approaches the analytical
    solution as we refine the grid $D_h$.
  \end{subsection}

  \begin{subsection}{Stability}
  \end{subsection}

\end{section}


\begin{section}{Worked Example}
  Specify a polynomial for f and set c and show how to implement solution and
  then plot it.
\end{section}


\newpage
\begin{appendices}
  \section{Numerical Scheme Program}
  The following is the m-function \texttt{numerical\_scheme.m} for use in MATLAB
  to compute the numerical solution to the problem $Lu = f$.
  \lstinputlisting{../NumericalSolution/numerical_scheme.m}

  Note that the above m-function is dependent on the m-function
  \texttt{uniform\_nodes.m} and that it must be in MATLAB's path when using
  \texttt{numerical\_scheme.m}.
  \lstinputlisting{../NumericalSolution/uniform_nodes.m}
\end{appendices}


\end{document}
