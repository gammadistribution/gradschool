The family of differential equations $Lu = f$ represents a second order linear
differential equation and therefore well-known techniques can be used to find
the solution $u(x)$.

The solution $u(x)$ is given by $u(x) = u_h(x) + u_p(x)$ where $u_h(x)$ is
the solution to the homogeneous equation $-u''(x) + c u(x) = 0$ and $u_p(x)$
is a particular solution of $-u''(x) + c u(x) = f(x)$.

To find the homogeneous solution, note that the characteristic equation of
this family of differential equations is given by $-m^2 + c = 0$, the roots of
which are $m_1 = \sqrt{c} = \omega $ and $m_2 = -\sqrt{c} = -\omega$. Note
that since $c > 0$, these roots are real and distinct suggesting that the
homogeneous solution is given by
\begin{align}\label{homogeneous_solution}
  u_h(x) = c_1 e^{\omega x} + c_2 e^{-\omega x}.
\end{align}

To find the particular solution, we assume the particular solution is of the
form $u_p(x) = \kappa(x) e^{\omega x}$ for some unknown function $\kappa(x)$.
Thus,
\[
u_p''(x) = \kappa''(x) e^{\omega x} + 2 \omega \kappa'(x) e^{\omega x} + \omega^2 \kappa(x) e^{\omega x}
\]
and substituting the above into the original differential equation $Lu = f$
with $u_p(x) = \kappa(x) e^{\omega x}$ we have
\begin{align}\label{particular_step_one}
  \kappa''(x) + 2\omega\kappa'(x) = -f(x)e^{-\omega x}.
\end{align}
Making the substitution $\lambda(x) = \kappa'(x)$ into \eqref{particular_step_one}
we can reduce the above second order linear differential equation into the
first order linear differential equation
\begin{align}\label{particular_step_two}
  \lambda'(x) + 2\omega\lambda(x) = -f(x)e^{-\omega x}.
\end{align}
The homogeneous solution to this first order differential equation is given by
$\lambda_h(x) = c_3 e^{-2\omega x}$ suggesting the particular solution to the
first order differential equation is of the form $\lambda_p(x) = \mu(x) e^{-2\omega x}$.

Repeating the same process as above, we see that
\[
\lambda_p'(x) = \mu'(x) e^{-2 \omega x} - 2 \omega\mu(x) e^{-2 \omega x}
\]
and substituting into \eqref{particular_step_two} with $\lambda_p(x) = \mu(x) e^{-2\omega x}$
we find that the first order linear differential equation becomes the separable first order differential
equation
\[
\mu'(x) = -f(x)e^{\omega x}.
\]
We readily see the solution to the above differential equation is given by
\[
\mu(x) = - \int_{0}^x f(r) e^{\omega r} dr.
\]
As $\kappa'(x) = \lambda_p(x) = \mu(x)e^{-2\omega x}$, we deduce that
\begin{align*}
  \kappa(x) = - \int_{0}^{x} e^{-2 \omega s} \left[ \int_{0}^s f(r) e^{\omega r} dr \right] ds
\end{align*}
and
\begin{align}\label{particular_solution}
  u_p(x) = \kappa(x)e^{\omega x} = -e^{\omega x} \int_{0}^{x} e^{-2 \omega s} \left[ \int_{0}^s f(r) e^{\omega r} dr \right] ds.
\end{align}
Combining the homogeneous solution \eqref{homogeneous_solution} and the particular
solution \eqref{particular_solution} we have that the general solution to
$Lu = f$ is given by
\begin{align}\label{general_analytical_solution}
  u(x) &= u_h(x) + u_p(x) \nonumber \\
  &= c_1 e^{\omega x} + c_2 e^{-\omega x} - e^{\omega x} \int_{0}^{x} e^{-2 \omega s} \left[ \int_{0}^s f(r) e^{\omega r} dr \right] ds.
\end{align}
Using the boundary values provided in $Lu=f$, the general solution is specified
by the system of linear equations
\begin{align*}
  u(0) &= c_1 + c_2 = \epsilon \\
  u(1) &= c_1 e^{\omega} + c_2 e^{-\omega} - e^{\omega} \int_{0}^{1} e^{-2 \omega s} \left[ \int_{0}^s f(r) e^{\omega r} dr \right] ds = \delta .
\end{align*}
The solution to this system in terms of the unknowns $c_1$ and $c_2$ is given
by
\begin{align*}
  c_1 = \frac{\epsilon e^{-\omega} - \delta - e^{\omega} \int_{0}^{1} e^{-2 \omega s} \left[ \int_{0}^s f(r) e^{\omega r} dr \right] ds}{e^{-\omega} - e^{\omega}}\\
  c_2 = \frac{-\epsilon e^{\omega} + \delta + e^{\omega} \int_{0}^{1} e^{-2 \omega s} \left[ \int_{0}^s f(r) e^{\omega r} dr \right] ds}{e^{-\omega} - e^{\omega}}.
\end{align*}

Using these constants in the general solution \eqref{general_analytical_solution}
gives us the unique analytical solution to the family of differential equation $Lu = f$.
Furthermore, we deduce that the problem is in fact well-posed.

From this solution, we must make the following additional assumption on this
problem: $f(x)$ must be integrable on the interval $[0, 1]$.

As the analytical solution depends on the symbolic integration of $f(x)$, we
will be unable to use this solution for functions $f(x)$ in which the
closed-form of the integral is not known.
