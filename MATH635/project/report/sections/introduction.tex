The authors were tasked by the client with finding the solution to the following
family of differential equations
\[
  \begin{cases}
    -u''(x) + c u(x) = f(x) \\
    0 \leq x \leq 1 \\
    u(0) = \epsilon \\
    u(1) = \delta.
  \end{cases}
\]
Additionally, the client has also requested to be provided with a means of
plotting the solution once obtained.

Throughout this report, the above family of differential equations together with
the interval of definition and initial conditions will be represented by
$Lu = f$ where $L$ can be thought of as the differential operator for the above family.

Assumptions were placed on this family so that $c \in \mathbb{R}$ with $c > 0$
and $f \in C^k([0,1])$ for sufficiently large $k$ so that $f$ is relatively
well-behaved on the defined interval.

In this report we will detail the analytical solution to this family of
differential equations showing that the above problem is well-posed and
explain why this solution is not amenable to practical use. We therefore
provide a numerical scheme to approximate the solution to the family of
differential equations and examine the convergence, consistency and stability
of the numerical scheme. Using the solution provided by the numerical scheme,
we then explore the method for plotting the solution.
