In order to efficiently use the numerical scheme just described we will need to
implement the scheme using computational software.

\subsubsection{Discretized Solution}
We have implemented the numerical scheme described above in MATLAB which can be
used by calling the m-function \texttt{numerical\_scheme.m}. We will now
describe the parameters necessary to call the function, how the function
computes the solution, and the results outputted by the function. Please refer
to appendix \ref{append_numerical} for the function definition in MATLAB.

This m-function requires the following parameters to compute the numerical solution:
\begin{itemize}
  \item \texttt{f} - MATLAB function that represents the function $f$ in the differential equation $Lu = f$.
  \item \texttt{c} - A real number that represents the constant $c$ in the differential equation $Lu = f$.
  \item \texttt{initials} - An array with two elements representing the initial conditions in the problem
    $Lu=f$. The first element of the array is $\epsilon$ and the second element of the array is $\delta$.
  \item \texttt{interval} - An array with two elements represent the endpoints of the interval of definition in
    the problem $Lu = f$.
  \item \texttt{subintervals} - An integer that represents the number of subintervals with which to construct
    the uniform nodes on the interval of definition. This corresponds to an
    $h$-value of $1/\texttt{subintervals}$ on the grid $D_h$.
\end{itemize}

Calling the function as follows
\[
  \texttt{numerical\_scheme(f, c, initials, interval, subintervals)}
\]
returns the array \texttt{[x, u]} where \texttt{x} is an array whose elements are
the nodes on the grid $D_h$ and \texttt{u} is an array whose elements are the
numerical solution obtained by the scheme $L_h u^{(h)} = f^{(h)}$ evaluated on the
nodes of the grid.

From these parameters, after verifying that $c$ is a positive real number,
the function creates and assigns to \texttt{x} the uniform
nodes equally spaced on the interval with width 1/\texttt{subintervals}. We then
construct the  coefficient matrix \texttt{A} and the right-hand side vector \texttt{b} of the equation
in \eqref{matrix_system}. Finally, we then assign to \texttt{u} the solution vector
which is given by \texttt{initials(1), inv(A) * b, initials(2)}.

Using this function then allows us to compute the numerical solution to the problem
$Lu = f$.

\subsubsection{Plotting}
