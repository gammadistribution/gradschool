The last property we wish to investigate for our numerical scheme is the
stability of the scheme. The stability of the scheme will ensure that small
changes in the right hand side of the scheme will only cause small changes in
the solution derived from the scheme. Formally, the scheme $L_h u^{(h)} = f^{(h)}$
will be called \textit{stable} if for any chosen $h_0 > 0$ and $\delta > 0$ such
that for any $h < h_0$ and for any $\varepsilon^{(h)}$ with
$\left|\left| \varepsilon^{(h)} \right|\right| < \delta$, the solution for the
finite-difference problem
$$L_h z^{(h)} = f^{(h)} + \varepsilon^{(h)}$$
is unique and the deviation from the solution of the unperturbed problem satisfies
the estimate
\begin{align}\label{stability_ineq}
  \left|\left| z^{(h)} - u^{(h)} \right|\right| \leq k \left|\left| \varepsilon^{(h)} \right|\right|
\end{align}
where $k$ does note depend on $h$ or $\varepsilon^{(h)}$.