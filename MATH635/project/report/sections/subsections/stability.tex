The last property we wish to investigate for our numerical scheme is the
stability of the scheme. The stability of the scheme will ensure that small
changes in the right hand side of the scheme will only cause small changes in
the solution derived from the scheme. Formally, the scheme $L_h u^{(h)} = f^{(h)}$
will be called \textit{stable} if for any chosen $h_0 > 0$ and $\delta > 0$ such
that for any $h < h_0$ and for any $\varepsilon^{(h)}$ with
$\left|\left| \varepsilon^{(h)} \right|\right| < \delta$, the solution for the
finite-difference problem
$$L_h z^{(h)} = f^{(h)} + \varepsilon^{(h)}$$
is unique and the deviation from the solution of the unperturbed problem satisfies
the estimate
\begin{align}\label{stability_ineq}
  \left|\left| z^{(h)} - u^{(h)} \right|\right| \leq k \left|\left| \varepsilon^{(h)} \right|\right|
\end{align}
where $k$ does not depend on $h$ or $\varepsilon^{(h)}$.

We now present strong numerical evidence that our scheme satisfies the above
inequality.
We begin by investigating this condition for our scheme $L_h u^{(h)} = f^{(h)}$ with
$f(x) = e^{0.5x}$, $c=1$, and initial conditions $u(0) = 0$ and $u(1) = 0$. The
following table shows the values of $\left|\left|z^{(h)} - u^{(h)}\right|\right|$
for the solutions to $L_h u^{(h)} = f^{(h)}$ and
$L_h z^{(h)} = f^{(h)} + \varepsilon^{(h)}$ for increasing values of
$\varepsilon^{(h)}$ up to $\varepsilon^{(h)} = 0.7$. The program used to create
the following table can be found in appendix \ref{append_stability}.

\begin{table}[h!]\label{stability_table}
  \centering
  \bgroup
  \def\arraystretch{1.5}
  \begin{tabular}{| l | c | c | c | c |}
    \hline
    $h$ & $\varepsilon^{(h)} = 0.1$ & $\varepsilon^{(h)} = 0.3$ & $\varepsilon^{(h)} = 0.5$ & $\varepsilon^{(h)} = 0.7$ \\
    \hline
    \texttt{1.0e-1} & \texttt{0.0113095845} &  \texttt{0.0339287535} &  \texttt{0.0565479225} &  \texttt{0.0791670915} \\
    \texttt{1.0e-3} & \texttt{0.0113181107} &  \texttt{0.0339543322} &  \texttt{0.0565905537} &  \texttt{0.0792267752} \\
    \texttt{3.0e-3} & \texttt{0.0113181115} &  \texttt{0.0339543345} &  \texttt{0.0565905575} &  \texttt{0.0792267806} \\
    \texttt{5.0e-3} & \texttt{0.0113181116} &  \texttt{0.0339543347} &  \texttt{0.0565905578} &  \texttt{0.0792267809} \\
    \texttt{7.0e-3} & \texttt{0.0113181116} &  \texttt{0.0339543348} &  \texttt{0.0565905580} &  \texttt{0.0792267811} \\
    \texttt{9.0e-3} & \texttt{0.0113181116} &  \texttt{0.0339543347} &  \texttt{0.0565905579} &  \texttt{0.0792267811} \\
    \texttt{1.1e-4} & \texttt{0.0113181116} &  \texttt{0.0339543348} &  \texttt{0.0565905579} &  \texttt{0.0792267811} \\
    \texttt{1.3e-4} & \texttt{0.0113181116} &  \texttt{0.0339543349} &  \texttt{0.0565905581} &  \texttt{0.0792267813} \\
    \texttt{1.5e-4} & \texttt{0.0113181116} &  \texttt{0.0339543349} &  \texttt{0.0565905581} &  \texttt{0.0792267813} \\
    \hline
  \end{tabular}
  \egroup
  \caption{Values of $\left|\left|z^{(h)} - u^{(h)}\right|\right|$ for various
    perturbations $\varepsilon^{(h)}$ for the function $f(x) = e^{0.5x}$ with $c=1$
    and initial values $u(0) = 0$ and $u(1) = 0$ rounded to the first 10
    significant digits.}
\end{table}

This table shows that as $h$ approaches $0$, the value $\left|\left|z^{(h)} - u^{(h)}\right|\right|$ approaches
a limit dependent upon the perturbation error $\varepsilon^{(h)}$.
In particular, this table shows that $\left|\left|z^{(h)} - u^{(h)}\right|\right| \leq k ||\varepsilon ^{(h)}||$ with the choice of $k=0.115$ for
any $h < h_0$ and $\varepsilon^{(h)} < \delta$ where we can specify any $h_0$ and
$\delta$ satisfying $h_0 < 0.1$ and $\delta < 0.7$.
Therefore, we have verification that the scheme $L_h u^{(h)} = f^{(h)}$ is stable for
this particular $f(x)$ and $c$ and the above mentioned initial conditions.

Further investigation shows that replacing $f(x) = e^{0.5x}$ with any function $f(x)$
produces the same values for $\left|\left|z^{(h)} - u^{(h)}\right|\right|$ as in
table \ref{stability_table}. Thus, we conclude that the scheme
$L_hu^{(h)} = f^{(h)}$ is stable for $c=1$ with initial conditions $u(0) = 0$ and $u(1) = 0$
for any $f(x)$.
