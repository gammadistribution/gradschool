The single most important property of the scheme presented in
\eqref{numerical_scheme} is its convergence to the analytical solution. We will
now rigorously define this notion of convergence.

But first, we must define the measure of the deviation between two solutions.
Let $U_h$ be the normed linear space of all functions defined on the grid $D_h$
as presented in \eqref{uniform_grid}. For $u^{h} \in U_h$, the equipped norm is
given as
\begin{align}\label{norm}
  ||u^{(h)}|| = \sup_n |u_n| = \max_n |u_n|.
\end{align}
With this definition, we can then precisely define the measure of deviation
between two solutions $a^{(h)}$ and $b^{(h)}$ as $||a^{(h)} - n^{(h)}||$.

Thus, we say that the solution $u^{(h)}$ given to us by the
scheme $L_h u^{(h)} = f^{(h)}$ \texttt{converges} to the discretized analytical
solution $[u]_h$ to the problem $Lu = f$ if
\begin{align}\label{convergence_def}
  \left|\left|[u]_h - u^{(h)}\right|\right| \to 0 \quad \text{as $h \to 0$}.
\end{align}

We now present strong numerical evidence that our scheme $L_h u^{(h)} = f^{(h)}$
converges to the discretized analytical solution to the problem $Lu = f$. The
program used to create the following tables can be found in appendix \ref{append_convergence}.

The following table shows the values of the computed norm $\left|\left|[u]_h - u^{(h)}\right|\right|$ for
increasingly smaller values of $h$ for $c=1$ on the interval $[0, 1]$ with the
initials conditions $u(0) = 1$ and $u(1) = 0.5$ with various definitions of
the function $f(x)$.

\begin{table}[h!]
  \centering
  \bgroup
  \def\arraystretch{1.5}
  \begin{tabular}{| l | c | c | c | c | c | c | c |}
    \hline
    $h$ & $x$ & $x^2$ & $x^3$ & $x^4$ & $x^5$ & $e^{0.5x}$ & $\sin(0.1 x)$\\
    \hline
    $10^{-1}$ & \texttt{0.2456e-4} & & & & & & \\
    $10^{-2}$ & & & & & & & \\
    $10^{-3}$ & & & & & & & \\
    $10^{-4}$ & & & & & & & \\
    \hline
  \end{tabular}
  \egroup
  \caption{Values of $\left|\left|[u]_h - u^{(h)}\right|\right|$ for various functions $f(x)$ with $c = 1$, $u(0) = 1$, and $u(1) = 0.5$.}
\end{table}

The above table suggests that the difference between the approximate solution
and the exact solution does tend toward zero as we refine the grid value $h$ and
that this happens regardless of the choice of $f(x)$.

A natural question would then be if the convergence is dependent upon the choice
of $c$. The following table shows the values of $\left|\left|[u]_h - u^{(h)}\right|\right|$for various
increasing values of $c$ with $f(x) = e^{0.5x}$ and $u(0) = 1$, $u(1) = 0.5$.

\begin{table}[h!]
  \centering
  \bgroup
  \def\arraystretch{1.5}
  \begin{tabular}{| l | c | c | c | c | c | c | c |}
    \hline
    $h$ & $1$ & $17$ & $59$ & $119$ & $409$ & $1307$ & $14639$\\
    \hline
    $10^{-1}$ & & & & & & & \\
    $10^{-2}$ & & & & & & & \\
    $10^{-3}$ & & & & & & & \\
    $10^{-4}$ & & & & & & & \\
    \hline
  \end{tabular}
  \egroup
  \caption{Values of $\left|\left|[u]_h - u^{(h)}\right|\right|$ for various values of $c$ for the function $f(x) = e^{0.5x}$ with $u(0) = 1$ and $u(1) = 0.5$.}
\end{table}

The above table suggests that the convergence is not dependent on $c$ for $f(x) = e^{0.5x}$.

The last aspect of convergence to investigate is the impact of the initial conditions
on the scheme's convergence convergence. The following table shows the values of $||[u]_h - u^{h}||$ for various
initial conditions with $f(x) = e^{0.5x}$ and $c=1$.

\begin{table}[h!]
  \centering
  \bgroup
  \def\arraystretch{1.5}
  \begin{tabular}{| l | c | c | c | c | c |}
    \hline
    & $u(0) = 0$ & $u(0) = 0.05$ & $u(0) = 0.1$ & $u(0) = 0.1$ & $u(0) = -0.1$ \\
    $h$ &  $u(1) = 0$ & $u(1) = 0.1$ & $u(1) = 0.05$ & $u(1) = 0.1$ & $u(1) = 0.1$\\
    \hline
    $10^{-1}$ & & & & & \\
    $10^{-2}$ & & & & & \\
    $10^{-3}$ & & & & & \\
    $10^{-4}$ & & & & & \\
    \hline
  \end{tabular}
  \egroup
  \caption{Values of $\left|\left|[u]_h - u^{(h)}\right|\right|$ for various initial values for the function $f(x) = e^{0.5x}$ with $c=1$.}
\end{table}

Again, from the table we conclude that the approximate solution converges to the
exact solution regardless of the choice in initial conditions for the function
$f(x) = e^{0.5x}$.