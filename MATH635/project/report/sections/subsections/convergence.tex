The single most important property of the scheme presented in
\eqref{numerical_scheme} is its convergence to the analytical solution. We will
now rigorously define this notion of convergence.

But first, we must define the measure of the deviation between two solutions.
Let $U_h$ be the normed linear space of all functions defined on the grid $D_h$
as presented in \eqref{uniform_grid}. For $u^{h} \in U_h$, the equipped norm is
given as
\begin{align}\label{norm}
  ||u^{(h)}|| = \sup_n |u_n| = \max_n |u_n|.
\end{align}
With this definition, we can then precisely define the measure of deviation
between two solutions $a^{(h)}$ and $b^{(h)}$ as $||a^{(h)} - n^{(h)}||$.

Thus, we say that the solution $u^{(h)}$ given to us by the
scheme $L_h u^{(h)} = f^{(h)}$ \textit{converges} to the discretized analytical
solution $[u]_h$ to the problem $Lu = f$ if
\begin{align}\label{convergence_def}
  \left|\left|[u]_h - u^{(h)}\right|\right| \to 0 \quad \text{as $h \to 0$}.
\end{align}

We now present strong numerical evidence that our scheme $L_h u^{(h)} = f^{(h)}$
converges to the discretized analytical solution to the problem $Lu = f$. The
program used to create the following tables can be found in appendix \ref{append_convergence}.

The following table shows the values of the computed norm $\left|\left|[u]_h - u^{(h)}\right|\right|$ for
decreasing values of $h$ for $c=1$ on the interval $[0, 1]$ with the
initials conditions $u(0) = 0$ and $u(1) = 0$ with various definitions of
the function $f(x)$.

\begin{table}[h!]
  \centering
  \bgroup
  \def\arraystretch{1.5}
  \begin{tabular}{| l | c | c | c | c |}
    \hline
    $f(x)$ & $h = 10^{-1}$ & $h = 10^{-2}$ & $h = 10^{-3}$ & $h = 10^{-4}$ \\
    \hline
    $x$ & \texttt{044.1459e-06} & \texttt{442.1934e-09} & \texttt{004.4227e-09} & \texttt{013.8432e-12} \\
    $x^2$ & \texttt{213.1776e-06} & \texttt{002.1377e-06} & \texttt{021.3777e-09} & \texttt{194.1320e-12} \\
    $x^3$ & \texttt{309.0214e-06} & \texttt{003.0954e-06} & \texttt{030.9542e-09} & \texttt{296.6352e-12} \\
    $x^4$ & \texttt{368.6967e-06} & \texttt{003.7352e-06} & \texttt{037.3546e-09} & \texttt{364.6753e-12} \\
    $x^5$ & \texttt{416.4482e-06} & \texttt{004.2060e-06} & \texttt{042.0633e-09} & \texttt{414.2863e-12} \\
    $e^{0.5x}$ & \texttt{140.4375e-06} & \texttt{001.4093e-06} & \texttt{014.0963e-09} & \texttt{056.2481e-12} \\
    $\sin(0.1x)$ &  \texttt{004.3631e-06} & \texttt{043.7038e-09} & \texttt{437.1167e-12} & \texttt{001.3406e-12} \\
    \hline
  \end{tabular}
  \egroup
  \caption{Values of $\left|\left|[u]_h - u^{(h)}\right|\right|$ for various functions
    $f(x)$ with $c = 1$, $u(0) = 0$, and $u(1) = 0$.}
\end{table}

The above table suggests that the difference between the approximate solution
and the exact solution does tend toward zero as we refine the grid value $h$ and
that this happens regardless of the choice of $f(x)$. Thus, the scheme converges
to the exact solution for the problem $Lu = f$ if $c=1$ with initial conditions
$u(0) = 0$ and $u(1) = 0$.

A natural question would then be if the convergence is dependent upon the choice
of $c$. As the convergence of the scheme does not appear to depend on $f(x)$, we
fix $f(x) = e^{0.5x}$ and initial conditions $u(0) = 0$ and $u(1) = 0$ and investigate
the values of $\left|\left|[u]_h - u^{(h)}\right|\right|$ for varying values of $c$.
The following table contains the above described computations

\begin{table}[h!]
  \centering
  \bgroup
  \def\arraystretch{1.5}
  \begin{tabular}{| l | c | c | c | c |}
    \hline
    $c$ & $h = 10^{-1}$ & $h = 10^{-2}$ & $h = 10^{-3}$ & $h = 10^{-4}$ \\
    \hline
    $10^{-7}$ & \texttt{033.6085e-06} &  \texttt{336.7078e-09} &   \texttt{003.3658e-09} &   \texttt{153.9025e-12} \\
    1 & \texttt{140.4375e-06} &  \texttt{001.4093e-06} &   \texttt{014.0963e-09} &   \texttt{056.2481e-12} \\
    17 & \texttt{299.5348e-06} &  \texttt{003.0371e-06} &   \texttt{030.3761e-09} &   \texttt{276.4544e-12} \\
    59 & \texttt{234.9940e-06} &  \texttt{002.5653e-06} &   \texttt{025.6691e-09} &   \texttt{256.1311e-12} \\
    119 & \texttt{229.1304e-06} &  \texttt{002.5310e-06} &   \texttt{025.3418e-09} &   \texttt{252.7861e-12} \\
    409 &\texttt{147.4121e-06} &  \texttt{002.5192e-06} &   \texttt{025.2848e-09} &   \texttt{252.1907e-12} \\
    1,307 & \texttt{050.1416e-06} &  \texttt{002.4901e-06} &   \texttt{025.2720e-09} &   \texttt{252.7783e-12} \\
    \hline
  \end{tabular}
  \egroup
  \caption{Values of $\left|\left|[u]_h - u^{(h)}\right|\right|$ for various values of
    $c$ for the function $f(x) = e^{0.5x}$ with $u(0) = 0$ and $u(1) = 0$.}
\end{table}

The evidence presented in the above table suggests that for decreasing values of $h$,
the value $\left|\left|[u]_h - u^{(h)}\right|\right|$ tends to 0. Thus, the convergence of the
scheme is not dependent on the value of $c$ in the problem $Lu = f$.

The last aspect of convergence to investigate is the impact of the initial conditions
in the problem $Lu = f$ on the scheme's convergence.
The following table shows the values of $||[u]_h - u^{h}||$ for various
initial conditions with $f(x) = e^{0.5x}$ and $c=1$.


From the table of values in Table \ref{tab:con} we see that for decreasing values
of $h$, the value $\left|\left|[u]_h - u^{(h)}\right|\right|$ tends to 0 and we
conclude that the approximate solution
converges to the exact solution regardless of the choice in initial conditions for the function
$f(x) = e^{0.5x}$.

Therefore, from the numerical evidence presented, the convergence of the
scheme $L_h u^{(h)} = f^{(h)}$ is not dependent on the function $f(x)$, the constant
$c$, nor the initial conditions found in the problem $Lu =f$ and we conclude that
the scheme is convergent.

\begin{table}[H]
  \centering
  \bgroup
  \def\arraystretch{1.5}
  \begin{tabular}{| l | c | c | c | c |}
    \hline
    \shortstack[l]{Initial \\ Conditions} & $h = 10^{-1}$ & $h = 10^{-2}$ & $h = 10^{-3}$ & $h = 10^{-4}$ \\
    \hline
    \shortstack[l]{$u(0) = 0$ \\ $u(1) = 0$} & \texttt{140.4375e-006} &   \texttt{001.4093e-06} & \texttt{014.0963e-09}  &  \texttt{056.2481e-12} \\
    \hline
    \shortstack[l]{$u(0) = 0.05$ \\ $u(1) = 0.1$} & \texttt{134.0422e-06} &  \texttt{001.3450e-06} &    \texttt{013.4545e-09} &   \texttt{007.8354e-12} \\
    \hline
    \shortstack[l]{$u(0) = 0.1$ \\ $u(1) = 0.05$} & \texttt{134.0422e-06} &  \texttt{001.3459e-06} &    \texttt{013.4629e-09} &   \texttt{013.0701e-12} \\
    \hline
    \shortstack[l]{$u(0) = 0.1$ \\ $u(1) = 0.1$} & \texttt{131.9104e-06} &  \texttt{001.3242e-06} &    \texttt{013.2460e-09} &   \texttt{016.8537e-12} \\
    \hline
    \shortstack[l]{$u(0) = -0.1$ \\ $u(1) = 0.1$} & \texttt{140.4375e-06} &  \texttt{001.4080e-06} &    \texttt{014.0822e-09} &   \texttt{057.2911e-12} \\
    \hline
  \end{tabular}
  \egroup
  \caption{Values of $\left|\left|[u]_h - u^{(h)}\right|\right|$ for various initial values
     for the function $f(x) = e^{0.5x}$ with $c=1$.}\label{tab:con}
\end{table}
