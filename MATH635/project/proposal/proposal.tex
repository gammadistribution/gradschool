\documentclass[12pt]{article}
\usepackage{amsmath}

\title{Numerical Solution Proposal}

\begin{document}
\maketitle

Consider the differential equation
\[
  - u''(t) + c u(t) = f(t)
\]
with initial conditions $u(0) = \epsilon$  and $u(1) = \delta$ for $0<|\epsilon|<0.1, 0<|\delta|<0.1$
where $c$ is a constant for $0\leq t \leq 1$.
We wish to find a solution to this differential equation.

\begin{enumerate}
  \item Determine if an analytical solution exists. Is the problem well-conditioned
    or ill-conditioned? This differential equation is
    a second order linear differential equation so techniques exist to find
    a closed form solution.
  \item Develop a numerical scheme to compute the function values.
  \item This scheme must be convergent. Is the scheme well-conditioned or ill-conditioned?
    Convergence will depend on the values of $c$ and $f(t)$. Stability will depend on $\epsilon$ and $\delta$.
  \item Once we have the scheme we will be able to compute the function for
    any given nodes. We can then plot the function using the values at those nodes
    and interpolate the values.
  \item Determine the relation between $c$ and $\epsilon$ and $\delta$.
\end{enumerate}


\end{document}