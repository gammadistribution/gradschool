\documentclass[12pt]{article}
\usepackage[margin=1in]{geometry}
\usepackage{amsmath, amsthm, amssymb, amsfonts, breqn, graphicx}


\theoremstyle{definition}
\newtheorem{problem}{Problem}
\renewcommand*{\proofname}{Solution}
\newenvironment{custompbm}[1]
  {\renewcommand\theproblem{#1}\problem}
  {\endproblem}
\renewcommand{\theenumi}{\alph{enumi}}


\newcommand{\E}{\text{E}}
\newcommand{\V}{\text{Var}}
\newcommand{\Co}[2]{\text{Cov}\left({#1}, {#2}\right)}
\newcommand{\pdf}{\text{pdf}}
\newcommand{\pmf}{\text{pmf}}
\newcommand{\me}{\mathrm{e}}
\newcommand*\diff{\mathop{}\!\mathrm{d}}
\newcommand{\vect}[1]{\boldsymbol{#1}}
\newcommand{\mx}[1][t]{\mu_X({#1})}
\newcommand{\gx}[2]{\gamma_X({#1}, {#2})}
\newcommand\norm[1]{\left\lVert#1\right\rVert}


\title{Homework Assignment 9}
\author{Matthew Tiger}


\begin{document}


\maketitle
\begin{problem} Verify that the forward Euler scheme (9.29) has first order
  accuracy on a smooth solution $u=u(x)$ of problem (9.30).
\end{problem}

\begin{proof}
  Suppose we have the problem $Lu = f$, as defined in 9.30 i.e.
  \[
      Lu =
      \begin{cases}
        \frac{du}{dx} - G(x, u),\ 0 < x \leq 1 \\
        0
      \end{cases}
      \ \text{and} \
      f =
      \begin{cases}
        0,\ 0 < x \leq 1 \\
        a
      \end{cases}.
  \]
  The forward Euler scheme $L_hu^{(h)} = f^{(h)}$ is given by
  \[
      L_hu^{(h)} =
      \begin{cases}
        \frac{u_{n+1} - u_{n}}{h} - G(x_n, u_n),\  n = 0, 1, \dots N-1 \\
        u_0
      \end{cases}
      \ \text{and} \
      f^{(h)} =
      \begin{cases}
        0,\  n = 0, 1, \dots N-1 \\
        a
      \end{cases}.
  \]
  Let $[u]_h$ denote the discretized solution to $Lu = f$. This scheme has
  first order accuracy if $\norm{L_h[u]_h - L_hu^{(h)}} \leq C h$ where
  $C$ is a constant that does not depend on $h$.

  Note that the Taylor series expansion of $u(x + h)$ centered at $x$ is given by
  \[
    u(x + h) = u(x) + u^{\prime}(x) h + \frac{u^{\prime \prime}(\xi) h^2}{2}
  \]
  for $x \leq \xi \leq x+h$. This implies that
  \[
    u^{\prime}(x) = \frac{u(x+h) - u(x)}{h} - \frac{u^{\prime \prime}(\xi) h}{2}
  \]
  or that
  \[
    u^{\prime}(x) - G(x, u) = \frac{u(x+h) - u(x)}{h} - \frac{u^{\prime \prime}(\xi) h}{2} - G(x, u).
  \]
  As $u^{\prime}(x) - G(x, u) = 0$ is the exact solution to $Lu = f$, we know that the
  discretized exact solution is given by
  \[
    u^{\prime}(x) - G(x, u) = \frac{u(x_{n+1}) - u(x_n)}{h} - \frac{u^{\prime \prime}(\xi(x_n)) h}{2} - G(x_n, u_n) = 0
  \]
  where $\xi(x_n)$ depends on the node $x_n$. But under the forward Euler scheme,
  $L_h[u]_h = \frac{u_{n+1} - u_{n}}{h} - G(x_n, u_n)$ so that
  \[
    u^{\prime}(x) - G(x, u) = L_h[u]_h - \frac{u^{\prime \prime}(\xi(x_n)) h}{2} = 0
  \]
  i.e.
  \[
    u^{\prime}(x) - G(x, u) = L_h[u]_h - L_h u^{(h)} = \frac{u^{\prime \prime}(\xi(x_n)) h}{2}
  \]
  since $L_h u^{(h)} = 0$. If $|u^{\prime \prime}(x)| \leq M$ for $x \in [0, 1]$, then
  the above implies that
  \[
    \norm{L_h[u]_h - L_h u^{(h)}} = \norm{\frac{u^{\prime \prime}(\xi(x_n)) h}{2}} \leq \frac{M}{2} h.
  \]
  As $M/2$ does not depend on $h$, we have shown $\norm{L_h[u]_h - L_h u^{(h)}} \leq C h$
  where $C = M/2$ and that the forward Euler scheme has first order of accuracy.
\end{proof}


\begin{problem} Verify that the Crank-Nicolson scheme (9.33) has second order
  accuracy on a smooth solution $u=u(x)$ of problem (9.30).
\end{problem}

\begin{proof}
  Suppose we have the problem $Lu = f$, as defined in 9.30 i.e.
  \[
      Lu =
      \begin{cases}
        \frac{du}{dx} - G(x, u),\ 0 < x \leq 1 \\
        0
      \end{cases}
      \ \text{and} \
      f =
      \begin{cases}
        0,\ o < x \leq 1 \\
        a
      \end{cases}.
  \]
  The Crank-Nicolson scheme $L_hu^{(h)} = f^{(h)}$ is given by
  \begin{align*}
      L_hu^{(h)} &=
      \begin{cases}
        \frac{u_{n+1} - u_{n}}{h} - \frac{1}{2}[G(x_n, u_n) + G(x_{n+1}, u_{n+1})],\  n = 0, \dots N-1 \\
        u_0
      \end{cases}
      \\ \text{and} \\
      f^{(h)} &=
      \begin{cases}
        0,\  n = 0, \dots N-1 \\
        a
      \end{cases}.
  \end{align*}
  Let $[u]_h$ denote the discretized solution to $Lu = f$. This scheme has
  second order accuracy if $\norm{L_h[u]_h - L_hu^{(h)}} \leq C h^2$ where
  $C$ is a constant that does not depend on $h$.

  From the problem $Lu = f$, we see that $\frac{du}{dx} = G(x, u(x))$ implies that
  \begin{align*}
    \frac{d^2 u}{dx^2} &= \frac{d}{dx}\left[G(x, u(x))\right] = \frac{\partial G(x, u(x))}{\partial x} + \frac{\partial G(x, u(x))}{\partial u} \frac{du}{dx} \\
    &= \frac{\partial G(x, u(x))}{\partial x} + \frac{\partial G(x, u(x))}{\partial u} G(x, u(x))
  \end{align*}
  The Taylor expansion of $u(x+h)$ centered at $x$ is given by
  \[
    u(x + h) = u(x) + u^{\prime}(x) h + \frac{u^{\prime \prime}(x) h^2}{2} + \frac{u^{(3)}(\xi_1) h^3}{6}
  \]
  for $x \leq \xi_1 \leq x+h$. This implies that
  \[
    u^{\prime}(x) - G(x, u(x)) = -G(x, u(x)) + \frac{u(x+h) - u(x)}{h} - \frac{u^{\prime \prime}(x) h}{2} - \frac{u^{(3)}(\xi_1) h^2}{6} = 0
  \]
  Since $u^{\prime \prime}(x) = \frac{\partial G(x, u(x))}{\partial x} + \frac{\partial G(x, u(x))}{\partial u} G(x, u(x))$ from our earlier calculation, we have that
  \begin{align}\label{first_taylor}
    u^{\prime}(x) - G(x, u(x)) &= \frac{u(x+h) - u(x)}{h} - \left[G(x, u(x)) + \frac{h}{2}\left(\frac{\partial G}{\partial x} + \frac{\partial G}{\partial u} G(x, u(x))\right)\right] \\
    &= \frac{u^{(3)}(\xi_1) h^2}{6}.\notag
  \end{align}
  We now wish to show that we can replace the expression $\frac{\partial G(x, u(x))}{\partial x} + \frac{\partial G(x, u(x))}{\partial u} G(x, u(x))$ with
  $\frac{1}{2}[G(x, u(x)) + G(x + h, u(x) + h G(x, u(x)))]$ by expressing the Taylor expansion of $G(x + h, u(x) + h G(x, u(x))) = G(x_1, y_1)$
  centered at $(x, u(x))$. This Taylor expansion is given by
  \begin{align*}
    G(x_1, y_1) &= G(x, u(x)) + h\left[\frac{\partial G(x, u(x))}{\partial x} + \frac{\partial G(x, u(x))}{\partial u} G(x, u(x))\right]\\
    &\phantom{=}\ + \left. h^2\frac{1}{2}\left[\frac{\partial^2 G(x, u(x))}{\partial x^2} + 2 \frac{\partial^2 G(x, u(x))}{\partial x \partial u}G(x, u(x)) + \frac{\partial ^2G(x, u(x))}{\partial u^2}G(x, u(x))^2\right] \right|_{x=\xi_2}\\
    &= G(x, u(x)) + \frac{\partial G(x, u(x))}{\partial x} h + \frac{\partial G(x, u(x))}{\partial u} h G(x, u(x)) + R h^2
  \end{align*}
  where $x \leq \xi_2 \leq x + h$ and $u(x) \leq u(\xi_2) \leq u(x) + hG(x, u(x))$
  and $R$ is the additional constant of the second order term. From the above identity we can see that
  \[
    \frac{1}{2}\left[G(x, u) + G(x + h, u + hG(x, u))\right] = G(x, u) + \frac{h}{2}\left[\frac{\partial G(x, u)}{\partial x} + \frac{\partial G(x, u)}{\partial u}G(x, u)\right] + \frac{R}{2}h^2
  \]
  where we have replaced $u(x)$ with $u$ to shorten the expression. Note that from this identity it is clear that
  \begin{align}\label{second_taylor}
    G(x, u) + \frac{h}{2}\left(\frac{\partial G(x, u)}{\partial x} + \frac{\partial G(x, u)}{\partial u}G(x, u)\right) = \frac{1}{2}\left[G(x, u) + G(x + h, u + hG(x, u))\right] - \frac{R}{2}h^2
  \end{align}
  and replacing \eqref{second_taylor} in \eqref{first_taylor} yields the exact solution to the problem $Lu = f$ as
  \begin{align}\label{final}
    u^{\prime}(x) - G(x, u(x)) &= \frac{u(x+h) - u(x)}{h} - \frac{1}{2}\left[G(x, u) + G(x + h, u + hG(x, u))\right] \\
    &= \left[\frac{u^{(3)}(\xi_1)}{6} +  \frac{R}{2}\right]h^2.\notag
  \end{align}
  If $[u]_h$ is the discretized solution of the problem $Lu =f$, then discretizing
  the exact solution shows that for $x + h = x_{n+1}$ and $G(x+h,u(x)+ h G(x, u(x)) = G(x_{n+1}, u_n + hG(x_n, u_n)) =G(x_{n+1}, u_{n+1})$
  we have from \eqref{final} that
  \begin{align*}
    u^{\prime}(x) - G(x, u(x)) &= \frac{u_{n+1} - u_n}{h} - \frac{1}{2}\left[\frac{1}{2}[G(x_n, u_n) + G(x_{n+1}, u_{n+1})]\right] \\
    &= L_h[u]_h - L_h u^{(h)} =  \left[\frac{u^{(3)}(\xi_1)}{6} +  \frac{R}{2}\right]h^2.\notag
  \end{align*}
  since $L_h u^{(h)} = 0$. Assuming all second order partials and mixed partials of $G(x, u(x))$ are bounded and that the third derivative of our function $u(x)$ is bounded,
  it is clear that $\norm{L_h[u]_h - L_hu^{(h)}} \leq C h^2$ where $C$ does not depend on $h$
  showing that this scheme has second order of accuracy.
\end{proof}


\begin{problem}
  Create a difference scheme that is not consistent.
\end{problem}

\begin{proof}
  Suppose we have the Cauchy problem $Lu = f$, as defined in 9.30 i.e.
  \[
      Lu =
      \begin{cases}
        \frac{du}{dx} - G(x, u),\ 0 < x \leq 1 \\
        0
      \end{cases}
      \ \text{and} \
      f =
      \begin{cases}
        0,\ 0 < x \leq 1 \\
        a
      \end{cases}.
  \]
  Define a variant of the forward Euler scheme $L_hu^{(h)} = f^{(h)}$ as folows
  \[
      L_hu^{(h)} =
      \begin{cases}
        \frac{u_{n+1} - u_{n}}{h} - G(x_n, u_n) + 1,\  n = 0, 1, \dots N-1 \\
        u_0
      \end{cases}
      \ \text{and} \
      f^{(h)} =
      \begin{cases}
        0,\  n = 0, 1, \dots N-1 \\
        a
      \end{cases}.
  \]
  Then it is clear that this scheme is inconsistent as the the residual will always
  be a constant and never vanish regardless of the grid we choose.
\end{proof}


\begin{problem}
  Prove that the scheme
  \[
    4 \frac{u_{n+1} - u_{n-1}}{2h} - 3 \frac{u_{n+1} - u_n}{h} + u_n = 0\text{,\quad $n = 1, 2, \dots, N-1$}
  \]
  with initial conditions $u_0 = 1$ and $u_1 = e^{-h}$ is consistent for the problem
  \[
    \frac{du}{dx} + u = 0\text{,\quad $0 \leq x \leq 1$}
  \]
  with initial condition $u(0) = 1$.
\end{problem}

\begin{proof}
  If $[u]_h$ is the discretized solution to the problem $Lu = f$ as defined above,
  then the scheme $L_h u^{(h)} = f^{(h)}$ is consistent if $\norm{L_h[u]_h - L_hu^{(h)}} \to 0$
  as $h \to 0$.

  Note that the Taylor series expansions of $u(x + h)$ and $u(x - h)$ centered at $x$ are given by
  \begin{align*}
    u(x + h) = u(x) + u^{\prime}(x) h + \frac{u^{\prime \prime}(\xi_1) h^2}{2} \\
    u(x - h) = u(x) - u^{\prime}(x) h + \frac{u^{\prime \prime}(\xi_2) h^2}{2}
  \end{align*}
  for $x \leq \xi_1 \leq x+h$ and $x-h \leq \xi_2 \leq x$.
  From these expansions we can see that
  \begin{align*}
    u^{\prime}(x) = \frac{u(x+h) - u(x-h)}{2h}  - \frac{1}{4}h(u^{\prime \prime}(\xi_1) - u^{\prime \prime}(\xi_2))
  \end{align*}
  and
  \begin{align*}
    u^{\prime}(x) = \frac{u(x+h) - u(x)}{h} - \frac{1}{2}hu^{\prime \prime}(\xi_3).
  \end{align*}
  This shows that
  \begin{align*}
    u^{\prime}(x) + u(x) = 4\frac{u(x+h) - u(x-h)}{2h} - 3\frac{u(x+h) - u(x)}{h} + u(x) + h\left(\frac{3}{2}u^{\prime \prime}(\xi_3) - (u^{\prime \prime}(\xi_1) - u^{\prime \prime}(\xi_2))\right)
  \end{align*}
  so that if $[u]_h$ is the discretized solution to the problem defined above,
  \begin{align*}
    u^{\prime}(x) + u(x) &= 4\frac{u(x_{n+1}) - u(x_{n-1})}{2h} - 3\frac{u(x_{n+1}) - u(x_n)}{h} + u(x_n) + h\left(\frac{3}{2}u^{\prime \prime}(\xi_3) - (u^{\prime \prime}(\xi_1) - u^{\prime \prime}(\xi_2))\right)\\
    &= L_h[u]_h + h\left(\frac{3}{2}u^{\prime \prime}(\xi_3) - (u^{\prime \prime}(\xi_1) - u^{\prime \prime}(\xi_2))\right) = 0.
  \end{align*}
  Combining the above and the fact that $L_h u^{(h)} = 0$, we see that
  \[
    \norm{L_h [u]_h - L_h u^{(h)}} = h\norm{(u^{\prime \prime}(\xi_1) - u^{\prime \prime}(\xi_2)) - \frac{3}{2}u^{\prime \prime}(\xi_3)}.
  \]
  If $|u^{\prime \prime}(x)| \leq M$, then $0 \leq \norm{L_h [u]_h - L_h u^{(h)}} \leq h \left(\frac{7}{2}M\right)$
  and it is then clear that $\norm{L_h [u]_h - L_h u^{(h)}} \to 0$ as $h \to 0$ showing the
  consistency of the scheme.
\end{proof}


\begin{problem}
  Prove that the scheme
  \[
    4 \frac{u_{n+1} - u_{n-1}}{2h} - 3 \frac{u_{n+1} - u_n}{h} + u_n = 0\text{,\quad $n = 1, 2, \dots, N-1$}
  \]
  with initial conditions $u_0 = 1$ and $u_1 = e^{-h}$ is divergent for the problem
  \[
    \frac{du}{dx} + u = 0\text{,\quad $0 \leq x \leq 1$}
  \]
  with initial condition $u(0) = 1$.
\end{problem}

\begin{proof}
  If $[u]_h$ is the discretized solution to the problem $Lu = f$ as defined above,
  then the scheme $L_h u^{(h)} = f^{(h)}$ is divergent if $\norm{[u]_h - u^{(h)}}$
  does not approach 0 as $h \to 0$.

  The exact solution to the problem $Lu = f$ with the initial condition
  $u(0) = 1$ is $u(x) = e^{-x}$. Hence, $[u]_h = [e^{-x_0}, e^{-x_1}, \dots, e^{-x_n},\dots, e^{-x_N}] = [e^{0}, e^{-h}, \dots, e^{-nh},\dots, e^{-1}]$.
  The solution to the difference scheme $L_h u^{(h)} = f^{(h)}$ given by
  $u^{(h)}$ and can be found by finding the explicit solution to the difference
  equation defined in the scheme.

  Note that this is a second order difference equation that can be rewritten as
  \[
    - u_{n+1} + (3 + h) u_n - 2 u_{n-1} = 0.
  \]
  The characteristic equation of this difference equation is given by $-m^2 + (3+h)m -2 = 0$.
  As this characteristic equation has distinct real roots, the general
  solution to the difference equation is $u_n = c_1 m_1^n + c_2 m_2 ^n$ where
  $m_1 = \frac{1}{2}(-\sqrt{h^2 + 6h +1} + h + 3)$ and $m_2=\frac{1}{2}(\sqrt{h^2 + 6h +1} + h + 3)$
  are the roots of the characteristic equation. Choosing the constants so that
  the initial conditions are satisfied gives us the general solution as
  \begin{align*}
    u_n^{(h)} &= u_0 \left[\frac{m_2(h)}{m_2(h) - m_1(h)}m_1(h)^n - \frac{m_1(h)}{m_2(h) - m_1(h)}m_2(h)^n\right] \\
    &\phantom{= }+ u_1\left[-\frac{1}{m_2(h) - m_1(h)}m_1(h)^n + \frac{1}{m_2(h) - m_1(h)}m_2(h)^n\right].
  \end{align*}
  Combining this general solution to the scheme and the exact solution to the
  problem we can clearly see that $\norm{[u]_h - u^{(h)}}$ does not approach 0 as
  $h\to 0$ and that the scheme is divergent.
\end{proof}


\end{document}
