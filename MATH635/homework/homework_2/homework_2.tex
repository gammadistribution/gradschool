\documentclass[12pt]{article}
\usepackage[margin=1in]{geometry}
\usepackage{amsmath, amsthm, amssymb, amsfonts, breqn}


\theoremstyle{definition}
\newtheorem{problem}{Problem}
\renewcommand*{\proofname}{Solution}

\title{Homework Assignment 2}
\author{Matthew Tiger}


\newcommand{\E}{\text{E}}
\newcommand{\V}{\text{Var}}
\newcommand{\pdf}{\text{pdf}}
\newcommand{\pmf}{\text{pmf}}
\newcommand{\me}{\mathrm{e}}
\newcommand*\diff{\mathop{}\!\mathrm{d}}
\newcommand{\vect}[1]{\boldsymbol{#1}}
\newcommand{\MAT}{\texttt{MATLAB}\hspace{1.25mm}}

\begin{document}


\maketitle


% Problem 1
\begin{problem}
  Let $A = \begin{bmatrix} 4 & 1 & 0 & 0 \\ 1 & 4 & 1 & 0 \\ 0 & 1 & 4 & 1 \\ 0 & 0 & 1 & 4 \\\end{bmatrix}$. With \MAT, compute $\det A$.
\end{problem}

\begin{proof}
  Entering the following into the \MAT console gives:
  \begin{verbatim}>> A = [4 1 0 0; 1 4 1 0; 0 1 4 1; 0 0 1 4];
>> det(A)

ans =

   209\end{verbatim}
 Therefore, $\det A = 209.$
\end{proof}


% Problem 2
\begin{problem}
  Determine the rank of $A = \begin{bmatrix} 2 & 1 & 3 & 3 \\ 1 & 0 & -1 & 0 \\ a & 2 & 1 & 1 \\ 4 & 3 & 2 & 4 \end{bmatrix}$ as a function of $a$.
  Check with \MAT and the function \texttt{rank} for different values of $a$.
\end{problem}

\begin{proof}
  We know that $A$ has full rank, i.e. $\text{rank}(A) = 4$, if
  $\det A \neq 0$. Running the following in the \MAT console shows:
  \begin{verbatim}
>> syms a;
>> A = [2 1 3 3; 1 0 -1 0; a 2 1 1; 4 3 2 4];
>> det(A)

ans =

5*a
  \end{verbatim}
  so that $\det A = 5a$. It is clear that $A$ has full rank if $a \neq 0$. When
  $a = 0$, we can see using the \MAT console that
  \begin{verbatim}
>> A = [2 1 3 3; 1 0 -1 0; 0 2 1 1; 4 3 2 4];
>> rank(A)

ans =

     3.
  \end{verbatim}
  This is because the Reduced Row Echelon form of the matrix has 3 leading entries as
  \MAT can verify:
  \begin{verbatim}
>> A = [2 1 3 3; 1 0 -1 0; 0 2 1 1; 4 3 2 4];
>> rref(A)

ans =

    1.0000         0         0    0.5556
         0    1.0000         0    0.2222
         0         0    1.0000    0.5556
         0         0         0         0
  \end{verbatim}
  Therefore,
  \[
    \text{rank}(A) =
    \begin{cases}
      4 & a \neq 0 \\
      3 & a = 0
    \end{cases}
  \]
\end{proof}


% Problem 3
\begin{problem}
  Let $A = \begin{bmatrix} 2 & 4 & 6 \\ 0 & 2 & 3 \\ 0 & 0 & 2\end{bmatrix}$. Define $B = A - 2I$.
  \begin{enumerate}
    \item Compute by hand $B^k$ for any $k\in \mathbb{N}.$
    \item Find $A^n$ for $n \in \mathbb{N}.$
    \item Compute $\left(I + \frac{B}{2}\right) \left(I - \frac{B}{2} + \frac{B^2}{4}\right)$
    \item Find $A^{-n}$ for $n \in \mathbb{N}.$
  \end{enumerate}
\end{problem}

\begin{proof}

 \begin{enumerate}
    \item Note that $B = A - 2I = \begin{bmatrix}0 & 4 & 6 \\ 0 & 0 & 3 \\ 0 & 0 & 0\end{bmatrix}$
      is a nilpotent matrix since it is an upper triangular matrix where each
      entry along the diagonal is a $0$. Thus, $B^k = 0$ for $k \geq 3$ and we
      need only compute $B^2$ to find all $B^k$ for $k \in \mathbb{N}$.
      It is easy to see that
      \[
        B^2 = \begin{bmatrix}0 & 4 & 6 \\ 0 & 0 & 3 \\ 0 & 0 & 0\end{bmatrix} \begin{bmatrix}0 & 4 & 6 \\ 0 & 0 & 3 \\ 0 & 0 & 0\end{bmatrix} = \begin{bmatrix}0 & 0 & 12 \\ 0 & 0 & 0 \\ 0 & 0 & 0\end{bmatrix}
      \]
      and we are done.
    \item It is clear that $A = B + 2I$, so that $A^n = (B + 2I)^n$. Since $B$ and
      $2I$ commute, we can use the binomial theorem to find $(B + 2I)^n$. Thus,
      \begin{align*}
        A^n
        &= (B + 2I)^n \\
        &= \sum_{k=0}^n {n \choose k} B^{n-k} (2I)^k \\
        &= \sum_{k=0}^n 2 ^k {n \choose k} B^{n-k},
      \end{align*}
      where we take special note that $B^{n-k} = 0$ if $n - k \geq 3$
    \item It is easy to see that
      \begin{align*}
        \left(I + \frac{B}{2}\right) \left(I - \frac{B}{2} + \frac{B^2}{4}\right)
        &= \left(I - \frac{B}{2} + \frac{B^2}{4}\right) \left(I + \frac{B}{2}\right) \\
      \end{align*}
      and
      \begin{align*}
        \left(I + \frac{B}{2}\right) \left(I - \frac{B}{2} + \frac{B^2}{4}\right) &= \left(I - \frac{B}{2} + \frac{B^2}{4} + \frac{B}{2} - \frac{B^2}{4} + \frac{B^3}{8}\right) = I,
      \end{align*}
      since $B^3 = 0$. Thus, $\left(I + \frac{B}{2}\right)^{-1} = \left(I - \frac{B}{2} + \frac{B^2}{4}\right)$.
      Since $\frac{1}{2}A =  \left(I + \frac{B}{2}\right)$, $2A^{-1} = \left(I - \frac{B}{2} + \frac{B^2}{4}\right)$.
      Therefore,
      \begin{align*}
        A^{-1} &= \left(\frac{I}{2} - \frac{B}{4} + \frac{B^2}{8}\right) \\
        &= \begin{bmatrix}1/2 & -1 & -3/4\\ 0 & 1/2 & -3/4 \\ 0 & 0 & 1/2\end{bmatrix}.
      \end{align*}
    \item Note that $A^{-n} = (A^{-1})^n$. Using the same technique above,
      define
      \[
        C = A^{-1} - \frac{1}{2}I
      \] Then $C$ is a nilpotent
      matrix and due to the reasons above, we need only calculate $C^2$, where
      \begin{align*}
        C^2 &= \begin{bmatrix}0 & -1 & -3/4\\ 0 & 0 & -3/4 \\ 0 & 0 & 0\end{bmatrix}\begin{bmatrix}0 & -1 & -3/4\\ 0 & 0 & -3/4 \\ 0 & 0 & 0\end{bmatrix} \\
        &= \begin{bmatrix}0 & 0 & 3/4 \\ 0 & 0 & 0 \\ 0 & 0 & 0\end{bmatrix}.
      \end{align*}
      Since $A^{-1} = (C + \frac{1}{2}I)$ and we know that $C$ and $\frac{1}{2}I$
      commute,
      \begin{align*}
        (A^{-1})^n
        &= \left(C - \frac{1}{2}I\right)^n \\
        &= \sum_{k=0}^n {n \choose k} C^{n-k} \left(-\frac{1}{2}I\right)^k \\
        &= \sum_{k=0}^n \frac{(-1)^k}{2^k} {n \choose k} C^{n-k},
      \end{align*}
      where we take special note that $C^{n-k} = 0$ if $n - k \geq 3$.
  \end{enumerate}
\end{proof}


\end{document}