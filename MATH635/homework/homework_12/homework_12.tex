\documentclass[12pt]{article}
\usepackage[margin=1in]{geometry}
\usepackage{amsmath, amsthm, amssymb, amsfonts, breqn, graphicx}


\theoremstyle{definition}
\newtheorem{problem}{Problem}
\renewcommand*{\proofname}{Solution}
\newenvironment{custompbm}[1]
  {\renewcommand\theproblem{#1}\problem}
  {\endproblem}
\renewcommand{\theenumi}{\alph{enumi}}


\newcommand{\E}{\text{E}}
\newcommand{\V}{\text{Var}}
\newcommand{\Co}[2]{\text{Cov}\left({#1}, {#2}\right)}
\newcommand{\pdf}{\text{pdf}}
\newcommand{\pmf}{\text{pmf}}
\newcommand{\me}{\mathrm{e}}
\newcommand*\diff{\mathop{}\!\mathrm{d}}
\newcommand{\vect}[1]{\boldsymbol{#1}}
\newcommand{\mx}[1][t]{\mu_X({#1})}
\newcommand{\gx}[2]{\gamma_X({#1}, {#2})}
\newcommand\norm[1]{\left\lVert#1\right\rVert}


\title{Homework Assignment 12}
\author{Matthew Tiger}


\begin{document}


\maketitle


\begin{custompbm}{1}
  Use the methods of this section to approximate the solution to
  \[
    y'' + y = 3x^2, \quad y(0) = 0, y(2) = 3.5
  \]
  For basis functions, take $n=2$ and $\phi_1(x) = x(x-2), \phi_2(x) = x^2(x-2)$.
\end{custompbm}

\begin{proof}
  Note that $u(x) = \frac{7}{4}x$ satisfies the boundary conditions of the problem,
  i.e.\ $u(0) = 0$ and $u(2) = 3.5$. Therefore our approximation to the differential equation is
  given by a linear combination of the basis functions $\phi_1(x), \phi_2(x)$ and $u(x)$.
  So the approximation is given by
  \begin{align}\label{y_2}
    y_2 = u(x) + a_1 \phi_1(x) + a_2 \phi_2(x)
  \end{align}
  to the solution of the original differential equation which also satisfies
  the boundary conditions.

  We wish to find values of the coefficients $a_1, a_2$ such that
  \begin{align}\label{system}
    \int_0^2 \left(y_2'' + y_2 - 3x^2\right)\phi_i(x)dx = 0 \quad \text{for $i = 1,2.$}
  \end{align}
  Using our definition of the approximation found in \eqref{y_2}, we carry out the
  computations in eqref{system} with MATLAB to arrive at the following system
  of equations
  \begin{align*}
    \begin{bmatrix}
      8/5 & 8/5 \\
      8/5 & 64/21
    \end{bmatrix}
    \begin{bmatrix}
      a_1 \\
      a_2
    \end{bmatrix} =
    \begin{bmatrix}
      37/15 \\
      18/5
    \end{bmatrix}.
  \end{align*}
  The solution to this system yields that the coefficients
  are given by $a_1 = 173/228$ and $a_2 = 119/152$.

  Therefore, the approximation to the solution to the original differential equation
  is given by
  \[
    y_2(x) = \frac{7}{4}x + \frac{173}{228}x(x-2) + \frac{119}{152}x^2(x-2).
  \]
\end{proof}


\begin{custompbm}{3}
  Use the methods of this section to approximate the solution to
  \[
    y'' + y^2 = x,\quad y(0) = y(1) = 0.
  \]
  Let $n=1$ and for a basis function take $\phi_1(x) = x\sin(\pi x)$.
\end{custompbm}

\begin{proof}
  Note that $\phi_1(x) = x\sin(\pi x)$ satisfies the boundary conditions, i.e.\
  $\phi_1(0) = \phi_1(1) = 0$. Therefore our approximation to the differential equation is
  given by a linear combination of the basis function $\phi_1(x)$. Thus, the approximation
  is
  \[
    y_1 = a_1 x\sin(\pi x).
  \]
  We wish to find the value of $a_1$ such that
  \[
    \int_0^1 \left(y_1'' + y_1^2 - x\right)x\sin(\pi x) dx = 0.
  \]
  Using MATLAB, we see that the above equation reduces to the following quadratic equation
  \[
    \left(\frac{2\pi^2}{3\pi^3} - \frac{40}{9\pi^3}\right)a_1^2 + \left(\frac{-\pi^2}{6} - \frac{1}{4}\right)a_1 + \frac{4 - \pi^2}{\pi^3} = 0.
  \]
  Choosing the solution to the equation $a_1 = -0.099539826159056$ give us that
  the approximate solution is
  \[
    y_1 = -0.099539826159056 \sin(\pi x).
  \]
\end{proof}


\begin{custompbm}{5}
  The solution to
  \[
    ((x+1)y')' - (x+2)y = \frac{x+2}{e-1},\quad y(0)= 0, y(1) = 1.
  \]
  is $y=\frac{e^x - 1}{e - 1}$. Use the methods of this section to compute
  approximate solutions $y_5(x)$, $y_{10}(x)$, and $y_{15}(x)$ and compare these
  approximations to the actual solution.
\end{custompbm}

\begin{proof}
  Note the above differential equation can be represented in the form
  \[
    (p(x)y')' + q(x)y' +r(x)y = f(x), \quad y(a) = A, y(b) = B,
  \]
  where $p(x) = x+1$, $q(x) = 0$, $r(x)=-(x+2)$, $f(x)=\frac{x+2}{e-1}$, and the
  boundary conditions are appropriately defined. Using the methods from the reading,
  we see that an approximation to the solution $y(x)$ of the differential equation
  is given by
  \[
    y_n(x) = \sum_{j=1}^n a_j \phi_j(x) + l(x)
  \]
  where the basis functions $\phi_j = sin(j\pi x)$ satisfy $\phi_j(a) = \phi_j(b) = 0$
  and $l(x)= A \frac{b-x}{b-a} + B \frac{x-a}{b-a} = x$. Note that
  $y_n(x)$ satisfies the boundary conditions. The coefficients $a_j$ are chosen so that they
  satisfy the following system of equations
  \begin{align}\label{system_5}
    \sum_{j=1}^n a_j &\int_a^b\left[p(x)\phi_j'(x)\phi_i(x) -q(x)\phi_j'(x)\phi_i(x) -r(x)\phi_j(x)\phi_i(x)\right]dx \notag \\
    &= -\int_a^b \left[f(x) - Ll(x)\right]\phi_i(x)dx \quad \text{for $i=1,\dots n,$}
  \end{align}
  where $L$ is the differential operator defining our differential equation.

  Using our basis functions $\phi_j(x) = \sin(j\pi x)$, we see that $\phi_j'(x) = j\pi cos(j\pi x)$. Additionally,
  we also see that $f(x) - Ll(x) = \frac{x+2}{e-1} + x^2 +2x -1$. From theseidentities
  and using our definitions of $p(x)$, $q(x)$, and $r(x)$, we see that the system in \eqref{system_5}
  simplifies to the following system
  \begin{align}\label{final_system}
    \sum_{j=1}^n a_j &\int_0^1 \left[(x+1)ij\pi^2\cos(j\pi x)\cos(i \pi x) + (x+2)\sin(j \pi x)\sin(i \pi x)\right]dx \notag \\
    &= -\int_0^1 \left[\frac{x+2}{e-1} + x^2 + 2x -1\right]\sin(i\pi x)dx\quad \text{for $i=1,\dots,n$}.
  \end{align}

  We can now find an approximation to the solution $y(x)$ in the form
  \begin{align}\label{approx}
    y_n(x) = x + \sum_{j=1}^n a_j \sin(j \pi x)
  \end{align}
  where the coefficients $a_j$ are determined by the solution to the system found in \eqref{final_system}.

  After creating a MATLAB implementation of the solution to the above system of
  equations we see that $y_5(x) = x + \sum_{j=1}^5a_j\sin(j\pi x)$ where the coefficients are given by
  \[
    \begin{bmatrix}
      a_1 \\
      a_2 \\
      a_3 \\
      a_4 \\
      a_5 \\
    \end{bmatrix}
      =
    \begin{bmatrix}
      -0.126756792130151\\
      0.007873907784233\\
      -0.005123612817319\\
      0.001007418513753\\
      -0.001164503231156\\
    \end{bmatrix}
  \].

  Similarly, $y_{10}(x) = x + \sum_{j=1}^{10}a_j\sin(j\pi x)$ where the coefficients are given by
  \[
    \begin{bmatrix}
      a_1 \\
      a_2 \\
      a_3 \\
      a_4 \\
      a_5 \\
      a_6 \\
      a_7 \\
      a_8 \\
      a_9 \\
      a_{10} \\
    \end{bmatrix}
      =
    \begin{bmatrix}
      -0.126740377319384 \\
      0.007866963591612 \\
      -0.005111897265922 \\
      0.001003683279394 \\
      -0.001111840960052 \\
      0.000300043256596 \\
      -0.000405543093287 \\
      0.000129499026931 \\
      -0.000189239173159 \\
      0.000082228427550 \\
    \end{bmatrix}
  \]
  and $y_{15}(x) = x + \sum_{j=1}^{15}a_j\sin(j\pi x)$ where the coefficients are given by
  \[
    \begin{bmatrix}
      a_1 \\
      a_2 \\
      a_3 \\
      a_4 \\
      a_5 \\
      a_6 \\
      a_7 \\
      a_8 \\
      a_9 \\
      a_{10} \\
      a_{11} \\
      a_{12} \\
      a_{13} \\
      a_{14} \\
      a_{15} \\
    \end{bmatrix}
      =
    \begin{bmatrix}
      -0.126740669728242 \\
      0.007864339443911 \\
      -0.005112288932438 \\
      0.001001882459024 \\
      -0.001112248945762 \\
      0.000298067482786 \\
      -0.000406198853140 \\
      0.000126040763869 \\
      -0.000191339879818 \\
      0.000064709821496 \\
      -0.000104942850428 \\
      0.000037631839019 \\
      -0.000063854388930 \\
      0.000023917094979 \\
      -0.000043873909881 \\
    \end{bmatrix}.
  \]

We now compare these approximations to the actual solution for each of the
above derived approximations. The following tables evaluate the approximations and the
exact solution at the points 0.25, 0.50, and 0.75 and compare the absolute difference between these values
and the relative error of the approximations.

\begin{table}[h!]
  \centering
  \bgroup
  \def\arraystretch{1.5}
  \begin{tabular}{| l | c | c | c | c |}
    \hline
    $x$ & $y(x)$ & $y_5(x)$ & $|y(x) - y_5(x)|$ & $(|y(x) - y_5(x)| / |y(x)|)*100$ \\
    \hline
    0.25 & 0.165296 & 0.165443 & 1.476306e-04 & 0.089312 \\
    0.50 &  0.377540 & 0.377202 & 3.383513e-04 &   0.089619 \\
    0.75 &  0.650067 & 0.649695 & 3.719995e-04 &  0.057224 \\
    \hline
  \end{tabular}
  \egroup
  \caption{Comparison of approximation $y_5$ to solution $y$.}
\end{table}

\begin{table}[h!]
  \centering
  \bgroup
  \def\arraystretch{1.5}
  \begin{tabular}{| l | c | c | c | c |}
    \hline
    $x$ & $y(x)$ & $y_{10}(x)$ & $|y(x) - y_{10}(x)|$ & $(|y(x) - y_{10}(x)| / |y(x)|)*100$ \\
    \hline
    0.25 & 0.165296 & 0.165354 & 5.847486e-05 &  0.035375 \\
    0.50 &  0.377540 & 0.377475 & 6.468589e-05 & 0.017133 \\
    0.75 &  0.650067 & 0.650056 &  1.163722e-05 & 0.001790 \\
    \hline
  \end{tabular}
  \egroup
  \caption{Comparison of approximation $y_{10}$ to solution $y$.}
\end{table}

\begin{table}[t!]
  \centering
  \bgroup
  \def\arraystretch{1.5}
  \begin{tabular}{| l | c | c | c | c |}
    \hline

    $x$ & $y(x)$ & $y_{15}(x)$ & $|y(x) - y_{15}(x)|$ & $(|y(x) - y_{15}(x)| / |y(x)|)*100$ \\
    \hline
    0.25 & 0.165296 & 0.165313 & 1.714344e-05 & 0.010371\\
    0.50 &  0.377540 &  0.377559 & 1.852280e-05 &  0.004906 \\
    0.75 &  0.650067 & 0.650099 &  3.119950e-05 &  0.004799 \\
    \hline
  \end{tabular}
  \egroup
  \caption{Comparison of approximation $y_{15}$ to solution $y$.}
\end{table}

  Note that as $n$ increases the value $|y(x) - y_{n}(x)|$ decreases. Moreover, this rate of decrease appears to be uniform over the entire interval of definition.

\end{proof}



\end{document}
