\documentclass[12pt]{article}
\usepackage[margin=1in]{geometry}
\usepackage{amsmath, amsthm, amssymb, amsfonts, breqn, graphicx}


\theoremstyle{definition}
\newtheorem{problem}{Problem}
\renewcommand*{\proofname}{Solution}
\newenvironment{custompbm}[1]
  {\renewcommand\theproblem{#1}\problem}
  {\endproblem}
\renewcommand{\theenumi}{\alph{enumi}}


\newcommand{\E}{\text{E}}
\newcommand{\V}{\text{Var}}
\newcommand{\Co}[2]{\text{Cov}\left({#1}, {#2}\right)}
\newcommand{\pdf}{\text{pdf}}
\newcommand{\pmf}{\text{pmf}}
\newcommand{\me}{\mathrm{e}}
\newcommand*\diff{\mathop{}\!\mathrm{d}}
\newcommand{\vect}[1]{\boldsymbol{#1}}
\newcommand{\mx}[1][t]{\mu_X({#1})}
\newcommand{\gx}[2]{\gamma_X({#1}, {#2})}


\title{Homework Assignment 7}
\author{Matthew Tiger}


\begin{document}


\maketitle


% Problem 9.1
\begin{custompbm}{9.1}
  For the following problems for 9.1, suppose a function $f: [a, b] \to \mathbb{R}$ is only known at distinct sites
  $x = [x_1, x_2, \dots, x_n]$ where $x_i \in [a,b]$, for $i=1, 2, \dots n$. Let $p_n(f, t)$
  be the Lagrange interpolating polynomial at these sites.
\end{custompbm}

\begin{custompbm}{9.1.1}
  Show that the basic quadrature $J(f) := \int_a^b p_n(f, t) \diff{t}$ satisfies
  $J(f) = \sum_{j=1}^n w_j f(x_j)$ where the weights $w_j$ depend on the Lagrange basis.
\end{custompbm}

\begin{proof}
  Note the Lagrange interpolating polynomial of $f$ through the nodes
  $x_1, x_2, \dots, x_n$ is given by
  \[
    p_n(f, t) = \sum_{j=1}^{n} f(x_j) \prod_{\substack{i=1\\i\neq j}} \frac{t - x_i}{x_j - x_i}.
  \]
  If $J(f) := \int_a^b p_n(f, t)$, then, using this definition of the Lagrange interpolating
  polynomial, it is clear that
  \begin{align*}
    J(f) = \int_a^b p_n(f, t) \diff{t}
    &= \int_a^b \left[\sum_{j=1}^{n} f(x_j) \prod_{\substack{i=1\\i\neq j}} \frac{t - x_i}{x_j - x_i}\right] \diff{t}\\
    &= \sum_{j=1}^{n} \left[\int_a^b \prod_{\substack{i=1\\i\neq j}} \frac{t - x_i}{x_j - x_i} \diff{t} \right] f(x_j) = \sum_{j=1}^{n} w_j f(x_j).
  \end{align*}
  Thus, $J(f)$ is of the form $\sum_{j=1}^{n} w_j f(x_j)$ where $w_j$ depends on
  the Lagrange basis $l_j(t) = \prod_{\substack{i=1\\i\neq j}} \frac{t - x_i}{x_j - x_i}$.
\end{proof}


\begin{custompbm}{9.1.2}
  Show that $J(f)$ has degree of precision at least $n-1$.
\end{custompbm}

\begin{proof}
  Let $q(t)$ be a polynomial of degree $n-1$. Then,
  \[
    q(t) = \sum_{j=1}^n q(x_i) \prod_{\substack{i=1\\i\neq j}} \frac{t - x_i}{x_j - x_i},
  \]
  i.e.\ the Lagrange interpolating polynomial of $q$ through the nodes $x_1, x_2, \dots x_n$ is $q$ itself.
  Hence, the exact integral of $q$, $I(q) = \int_a^b q(t) \diff{t}$, satisfies
  \begin{align*}
    I(q) = \int_a^b q(t) \diff{t}
    &= \int_a^b \sum_{j=1}^{n} q(x_j) \prod_{\substack{i=1\\i\neq j}} \frac{t - x_i}{x_j - x_i} \diff{t} \\
    &= \sum_{j=1}^{n} \left[\int_a^b \prod_{\substack{i=1\\i\neq j}} \frac{t - x_i}{x_j - x_i} \diff{t} \right] q(x_j) = J(q).
  \end{align*}
  Since $q$ is a polynomial of degree $n-1$ and $I(q) = J(q)$, we know that $J(f)$
  has degree of precision at least $n-1$.
\end{proof}


\begin{custompbm}{9.1.3}
  Show that if $f \in C^n [a,b]$, then the truncation error can be bounded
  in terms of the nodal polynomial as follows:
  \[
    |R(f)| \leq \frac{1}{n!} \max_{\substack{t\in [a,b]}} |f^{(n)}(t)| \int_a^b |\Pi_n(t)| \diff{t}
  \]
\end{custompbm}

\begin{proof}
  Let $f \in C^n([a,b])$. Note the truncation error is given by $R(f) = I(f) - J(f)$. Since
  $f \in C^n([a,b])$ and the Lagrange interpolating polynomial $p_n$ satisfies
  $p_n(f, x_i) = f(x_i)$ for $i=1,2,\dots,n$, there is a point $\xi_x$ in the smallest
  interval containing $[a, b]$ and every $x_i$ such that
  \[
    R(f) = I(f) - J(f) = \int_a^b f(t) \diff{t} - \int_a^b p_n(f, t) \diff{t} = \frac{1}{n!} \int_a^b f^{(n)}(\xi_x) \Pi_n(t) \diff{t}
  \]
  where $\Pi_n(t)$ is the nodal polynomial $\Pi_n(t) = \prod_{j=1}^n (t-x_j)$.

  From this identity, it is clear that
  \begin{align*}
    |R(f)| &= \left|\frac{1}{n!} \int_a^b f^{(n)}\left(\xi_x\right) \Pi_n(t) \diff{t}\right|\\
    &\leq \frac{1}{n!} \left|f^{(n)}\left(\xi_x\right) \right| \int_a^b \Pi_n(t) \diff{t}\\
    &\leq \frac{1}{n!} \max_{\substack{t\in [a,b]}}\left|f^{(n)}(t)\right| \int_a^b \Pi_n(t) \diff{t}
  \end{align*}
  since $\left|f^{(n)}(\xi_x)\right| \leq \max_{\substack{t\in [a,b]}}\left|f^{(n)}(t)\right|$ as $\xi_x \in [a,b]$
  and we are done.
\end{proof}


% Problem 9.3.1
\begin{custompbm}{9.3.1}
  In the following, for a function $f: [a,b] \to \mathbb{R}$, $f_i$ is shorthand
  for $f(x_i)$, with $x_i = a + (i-1)(b-a)/(n-1)$. For $n=4$, consider \textbf{Simpson's 3/8 rule}
  \[
    J_{S38}(f) = \frac{b-a}{8}(f_1 + 3f_2 + 3f_3 + f_4).
  \]
  Choose the interval $[0,1]$. Find the exact degree of precision. The error is
  given by $R_{S38}(f) = c_{S38} f^{(4)}(\xi)$. Find $c_{S38}$ using MATLAB and a
  polynomial for $f$.
\end{custompbm}

\begin{proof}
  Note that on the interval [0, 1],
  \[
    J_{S38}(f) = \frac{b-a}{f_1 + 3f_2 + 3f_3 + f_4} = \frac{1}{8}(f(0) + 3f(1/3) + 3f(2/3) + f(1)).
  \]
  To see that the exact degree of precision of this
  quadrature is 3, note that
  \[
      I(x^3) = \int_0^1 x^3 \diff{x} = \frac{1}{4} = \frac{1}{8}\left((0)^3 + 3 (1/3)^3 + 3 (2/3)^3 + (1)^3\right) = J_{S38}(x^3)
  \]
  and similarly
  \begin{align*}
      I(x^2) = \int_0^1 x^2 \diff{x} = \frac{1}{3} = \frac{1}{8}\left((0)^2 + 3 (1/3)^2 + 3 (2/3)^2 + (1)^2\right) = J_{S38}(x^2) \\
      I(x^1) = \int_0^1 x^1 \diff{x} = \frac{1}{2} = \frac{1}{8}\left((0)^1 + 3 (1/3)^1 + 3 (2/3)^1 + (1)^1\right) = J_{S38}(x^1) \\
      I(x^0) = \int_0^1 x^0 \diff{x} = 1 = \frac{1}{8}\left((0)^0 + 3 (1/3)^0 + 3 (2/3)^0 + (1)^0\right) = J_{S38}(x^0)
  \end{align*}
  but
  \[
    I(x^4) = \int_0^1 x^4 \diff{x} = \frac{1}{5} \neq \frac{11}{54} = \frac{1}{8}\left((0)^4 + 3 (1/3)^4 + 3 (2/3)^4 + (1)^4\right) = J_{S38}(x^4).
  \]
  Since $I(x^i) = J_{S38}(x^i)$ for all $0 \leq i \leq 3$ the quadrature rule is
  the same as the integral for all polynomial of degree 3 or less, but $I(x^4) \neq J_{S38}(x^4)$,
  the exact degree of precision must be 3.

  To find $c_{S38}$, choose $f(x) = x^4$. Then, as shown above, $I(x^4) = 1/5$ and $J_{S38} = 11/54$,
  so $R_{S38}(f) = I_{S38}(f) - J_{S38}(f) = -0.0037037 = c_{S38} f^{(4)}(\xi)$. Since
  $f^{(4)}(\xi) = 24$ for our choice of $f$, it follows that $c_{S38} = -0.00015432$.
\end{proof}


\begin{custompbm}{9.4.1}
  Given two real numbers $\alpha_1$ and $\alpha_2$ and two values $t_1, t_2$ in
  [-1, 1]. For a function $f: [-1,1] \to \mathbb{R}$, consider the basic quadrature
  $J_1(f):= \alpha_1 f(t_1) + \alpha_2 f(t_2)$ to approximate $\int_{-1}^{1} f(t) \diff{t}$
  How do we choose the four numbers $\alpha_1, \alpha_2, t_1,$ and $t_2$ to obtain
  the maximum degree of precision?
\end{custompbm}

\begin{proof}
  For the quadrature $J_1(f)$ as defined in the problem, we have that
  \begin{align*}
    \int_{-1}^{1} t^0 \diff{t} &= 2 = \alpha_1 f(t_1) + \alpha_2 f(t_2) = \alpha_1 + \alpha_2 \\
    \int_{-1}^{1} t^1 \diff{t} &= 0 = \alpha_1 f(t_1) + \alpha_2 f(t_2) = \alpha_1 t_1 + \alpha_2 t_2 \\
    \int_{-1}^{1} t^2 \diff{t} &= \frac{2}{3} = \alpha_1 f(t_1) + \alpha_2 f(t_2) = \alpha_1 t_1^2 + \alpha_2 t_2^2 \\
    \int_{-1}^{1} t^3 \diff{t} &= 0 = \alpha_1 f(t_1) + \alpha_2 f(t_2) = \alpha_1 t_1^3 + \alpha_2 t_2^3.
  \end{align*}
  This gives us a system of four equations in four unknowns. Solving the system
  yields the solution $\alpha_1 = \alpha_2 = 1$, and $t_1 = \frac{1}{\sqrt{3}}$ and
  $t_2 = -\frac{1}{\sqrt{3}}$.

  Since the quadrature rule is such that $I(p) = J_1(p)$ for any polynomial
  of degree less than 4, by our own definition, this quadrature rule has
  maximal degree of precision 3 as the rule does not hold for polynomials of
  degree greater than or equal to 4.
\end{proof}


\begin{custompbm}{9.4.2}
  Now we consider the quadrature $J_2(f):= \beta_1 f(-1) + \beta_2 f(0) + \beta_3 f(1)$.
  Compute the values of $\beta_i$ to obtain degree of precision 2. Is 2 the maximum
  degree of precision?
\end{custompbm}

\begin{proof}
  To obtain degree of precision 2 for the quadrature $J_2(f)$, we make sure that
  $I(p) = J_2(p)$ for every polynomial $p$ with degree less than 3. Doing so gives us
  the following system of equations in three unknowns
  \begin{align*}
    \int_{-1}^{1} t^0 \diff{t} &= 2 = \beta_1 f(-1) + \beta_2 f(0) + \beta_3 f(1) = \beta_1 + \beta_2 + \beta_3 \\
    \int_{-1}^{1} t^1 \diff{t} &= 0 = \beta_1 f(-1) + \beta_2 f(0) + \beta_3 f(1) = -\beta_1 + \beta_3 \\
    \int_{-1}^{1} t^2 \diff{t} &= \frac{2}{3} = \beta_1 f(-1) + \beta_2 f(0) + \beta_3 f(1) = \beta_1 + \beta_3
  \end{align*}
  The solution to this system is $\beta_1 = 1/3, \beta_2 = 4/3$, and $\beta_3 = 1/3$.
  Note that 2 is not the maximum degree of precision as the quadrature rule
  holds for polynomials of degree 3.

\end{proof}


\end{document}
