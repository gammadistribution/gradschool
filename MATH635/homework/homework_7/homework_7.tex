\documentclass[12pt]{article}
\usepackage[margin=1in]{geometry}
\usepackage{amsmath, amsthm, amssymb, amsfonts, breqn, graphicx}


\theoremstyle{definition}
\newtheorem{problem}{Problem}
\renewcommand*{\proofname}{Solution}
\newenvironment{custompbm}[1]
  {\renewcommand\theproblem{#1}\problem}
  {\endproblem}
\renewcommand{\theenumi}{\alph{enumi}}


\newcommand{\E}{\text{E}}
\newcommand{\V}{\text{Var}}
\newcommand{\Co}[2]{\text{Cov}\left({#1}, {#2}\right)}
\newcommand{\pdf}{\text{pdf}}
\newcommand{\pmf}{\text{pmf}}
\newcommand{\me}{\mathrm{e}}
\newcommand*\diff{\mathop{}\!\mathrm{d}}
\newcommand{\vect}[1]{\boldsymbol{#1}}
\newcommand{\mx}[1][t]{\mu_X({#1})}
\newcommand{\gx}[2]{\gamma_X({#1}, {#2})}


\title{Homework Assignment 7}
\author{Matthew Tiger}


\begin{document}


\maketitle


% Problem 9.1
\begin{custompbm}{9.1}
  For the following problems for 9.1, suppose a function $f: [a, b] \to \mathbb{R}$ is only known at distinct sites
  $x = [x_1, x_2, \dots, x_n]$ where $x_i \in [a,b]$, for $i=1, 2, \dots n$. Let $p_n(f, t)$
  be the Lagrange interpolating polynomial at these sites.
\end{custompbm}

\begin{custompbm}{9.1.1}
  Show that the basic quadrature $J(f) := \int_a^b p_n(f, t) \diff{t}$ satisfies
  $J(f) = \sum_{j=1}^n w_j f(x_j)$ where the weights $w_j$ depend on the Lagrange basis.
\end{custompbm}

\begin{proof}
  Note the Lagrange interpolating polynomial of $f$ through the nodes
  $x_1, x_2, \dots, x_n$ is given by
  \[
    p_n(f, t) = \sum_{j=1}^{n} f(x_j) \prod_{\substack{i=1\\i\neq j}} \frac{t - x_i}{x_j - x_i}.
  \]
  If $J(f) := \int_a^b p_n(f, t)$, then, using this definition of the Lagrange interpolating
  polynomial, it is clear that
  \begin{align*}
    J(f) = \int_a^b p_n(f, t) \diff{t}
    &= \int_a^b \left[\sum_{j=1}^{n} f(x_j) \prod_{\substack{i=1\\i\neq j}} \frac{t - x_i}{x_j - x_i}\right] \diff{t}\\
    &= \sum_{j=1}^{n} \left[\int_a^b \prod_{\substack{i=1\\i\neq j}} \frac{t - x_i}{x_j - x_i} \diff{t} \right] f(x_j) = \sum_{j=1}^{n} w_j f(x_j).
  \end{align*}
  Thus, $J(f)$ is of the form $\sum_{j=1}^{n} w_j f(x_j)$ where $w_j$ depends on
  the Lagrange basis $l_j(t) = \prod_{\substack{i=1\\i\neq j}} \frac{t - x_i}{x_j - x_i}$.
\end{proof}


\begin{custompbm}{9.1.2}
  Show that $J(f)$ has degree of precision at least $n-1$.
\end{custompbm}

\begin{proof}
  Let $q(t)$ be a polynomial of degree $n-1$. Then,
  \[
    q(t) = \sum_{j=1}^n q(x_i) \prod_{\substack{i=1\\i\neq j}} \frac{t - x_i}{x_j - x_i},
  \]
  i.e.\ the Lagrange interpolating polynomial of $q$ through the nodes $x_1, x_2, \dots x_n$ is $q$ itself.
  Hence, the exact integral of $q$, $I(q) = \int_a^b q(t) \diff{t}$, satisfies
  \begin{align*}
    I(q) = \int_a^b q(t) \diff{t}
    &= \int_a^b \sum_{j=1}^{n} q(x_j) \prod_{\substack{i=1\\i\neq j}} \frac{t - x_i}{x_j - x_i} \diff{t} \\
    &= \sum_{j=1}^{n} \left[\int_a^b \prod_{\substack{i=1\\i\neq j}} \frac{t - x_i}{x_j - x_i} \diff{t} \right] q(x_j) = J(q).
  \end{align*}
  Since $q$ is a polynomial of degree $n-1$ and $I(q) = J(q)$, we know that $J(f)$
  has degree of precision at least $n-1$.
\end{proof}


\begin{custompbm}{9.1.3}
  Show that if $f \in C^n [a,b]$, then the truncation error can be bounded
  in terms of the nodal polynomial as follows:
  \[
    |R(f)| \leq \frac{1}{n!} \max_{\substack{t\in [a,b]}} |f^{(n)}(t)| \int_a^b |\Pi_n(t)| \diff{t}
  \]
\end{custompbm}

\begin{proof}
  Let $f \in C^n([a,b])$. Note the truncation error is given by $R(f) = I(f) - J(f)$. Since
  $f \in C^n([a,b])$ and the Lagrange interpolating polynomial $p_n$ satisfies
  $p_n(f, x_i) = f(x_i)$ for $i=1,2,\dots,n$, there is a point $\xi_x$ in the smallest
  interval containing $[a, b]$ and every $x_i$ such that
  \[
    R(f) = I(f) - J(f) = \int_a^b f(t) \diff{t} - \int_a^b p_n(f, t) \diff{t} = \frac{1}{n!} \int_a^b f^{(n)}(\xi_x) \Pi_n(t) \diff{t}
  \]
  where $\Pi_n(t)$ is the nodal polynomial $\Pi_n(t) = \prod_{j=1}^n (t-x_j)$.

  From this identity, it is clear that
  \begin{align*}
    |R(f)| &= \left|\frac{1}{n!} \int_a^b f^{(n)}\left(\xi_x\right) \Pi_n(t) \diff{t}\right|\\
    &\leq \frac{1}{n!} \left|f^{(n)}\left(\xi_x\right) \right| \int_a^b \Pi_n(t) \diff{t}\\
    &\leq \frac{1}{n!} \max_{\substack{t\in [a,b]}}\left|f^{(n)}(t)\right| \int_a^b \Pi_n(t) \diff{t}
  \end{align*}
  since $\left|f^{(n)}(\xi_x)\right| \leq \max_{\substack{t\in [a,b]}}\left|f^{(n)}(t)\right|$ as $\xi_x \in [a,b]$
  and we are done.
\end{proof}


\end{document}
