\begin{problem}
  Consider the differential equation
  \begin{align}\label{diffeq}
    y'' - y = -x,\quad  0<x<1 \quad y(0) = y(1) = 0
  \end{align}
  as in Example 15.12 on page 502.
  Use the basis $\{\phi_j(x)\} = \{x^j(1-x)^j\}$, as in section 15.5.1, to
  compute approximations to the exact solution using the finite-element method.

  Provide relative errors at the points 0.25, 0.50, and 0.75 of the approximations
  using the first $n=2,3,4$ basis functions. Plot
  the corresponding approximations $y_2$, $y_3$, $y_4$, and the exact solution
  $y$. Then find the first value of $j$ for which the relative error at all
  three points is less than 0.5\%.
\end{problem}

\begin{proof}
  The exact solution to the differential equation \eqref{diffeq}, $y(x)$, is
  a continuous function. This fact combined with the fact that $\{\phi_j(x)\}$
  form a basis of the function space shows that the continuous function $y(x)$
  can be approximated with a linear combination of the basis functions.
  Therefore, we wish to find an approximation $y_n(x)$ to the exact solution
  $y(x)$ where
  \begin{align}\label{finite_approximation}
    y_n(x) = \sum_{j=1}^n a_j \phi_j(x).
  \end{align}
  Note that the basis functions $\phi_j(x) = x^j(1-x)^j$ satisfy the boundary
  conditions $\phi_j(0) = \phi_j(1) = 0$ so that $y_n(x)$ also satisfies the
  boundary conditions.

  Corollary 15.2 suggests that if
  \[
    \int_0^1 (y_n'' - y_n + x)\phi_i(x) dx = 0 \quad \text{for $i=1,\dots,n$}
  \]
  then $y_n'' - y_n + x = 0$, i.e\ $y_n(x)$ satisfies the differential
  equation \eqref{diffeq}. If $y_n(x)$ satisfies the
  differential equation and the boundary conditions, then we know that $y_n(x)$
  approximates the exact solution $y(x)$.

  Therefore, we choose the coefficients $a_j$ such that they satisfy the system of
  equations
  \begin{align}\label{first_system}
    \sum_{j=1}^n a_j \int_0^1 \phi_j''(x)\phi_i(x) - \phi_j(x)\phi_i(x) dx = -\int_0^1 x \phi_i(x) dx \quad \text{for $i=1,\dots,n$}.
  \end{align}

  The above system unnecessarily uses the second derivative of the basis
  functions. We can rewrite the coefficients of the above system to use only
  the first derivative of the basis functions. To see this,
  note that we can rewrite the differential equation \eqref{diffeq} in the form
  \begin{align}\label{alternate_diffeq}
    (p(x)y')' + q(x)y' + r(x)y = f(x)
  \end{align}
  by choosing $p(x) = 1$, $q(x) = 0$, $r(x) = -1$, and $f(x) = -x$. With this form of the
  differential equation we would require the approximation \eqref{finite_approximation}
  to satisfy the following equations
  \[
    \int_0^1 ((p(x)y_n')' + r(x)y_n)\phi_i(x) dx = \int_0^1f(x)\phi_i(x)dx \quad \text{for $i=1,\dots,n$}.
  \]
  Making use of the fact that the basis functions are 0 on the boundary we see that
  \begin{align*}
    \int_0^1(p(x)y_n')'\phi_i(x) dx
    &= \phi_i(x)p(x)y_n'\rvert_0^1 - \int_0^1 p(x)y_n'\phi_i'(x) dx \\
    &= - \int_0^1 p(x)y_n'\phi_i'(x) dx.
  \end{align*}
  With this and the definitions of the functions $p(x)$, $r(x)$, and $f(x)$,
  the system of equations \eqref{first_system} becomes
  \begin{align}\label{system_imp}
    \sum_{j=1}^n a_j \int_0^1 -\phi_j'(x)\phi_i'(x) - \phi_j(x)\phi_i(x) dx = -\int_0^1 x \phi_i(x) dx \quad \text{for $i=1,\dots,n$}.
  \end{align}

  Finding the solution to the system of equations \eqref{system_imp} identifies
  the coefficients $a_j$ that define our approximation.

  In this instance, we have chosen the basis $\left.\{\phi_j(x)\}\right._{j=1}^n$ where $\phi_j(x) = x^j(1-x)^j$.
  Thus,
  \begin{align*}
    \phi_j'(x)
    &= \left(x^j\right)'(1-x)^j + x^j\left((1-x)^j\right)' \\
    &= jx^{j-1}(1-x)^j -jx^j(1-x)^{j-1}
  \end{align*}
  for $j=1,\dots,n$.
\end{proof}