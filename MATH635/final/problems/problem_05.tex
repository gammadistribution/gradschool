\begin{problem}
\end{problem}

\begin{proof}
  The polynomial basis does not practically approximate the exact solution for any
  point outside of small neighborhoods of $x=0.50$

  The trigonometric basis provides great approximations with relatively few iterations
  if trying to achieve relative error 0.5\%. It also appears that the convergence is
  uniform in the interval of definition suggesting that the approximation is useful
  for low iterations across the entire interval. It is also computationally inexpensive
  to calculate the coefficients of the approximation as the coefficient matrix is a
  diagonal matrix.

  The hat basis only requires 3 basis functions in order to achieve given relative error percents.
  These errors increase on the next iteration and the errors are not uniform in the
  interval, but it is computationally cheap to
  compute these approximations. The reason for this is that the coefficient matrix
  associated to these basis functions is a sparse matrix and by leveraging MATLAB's system of
  equations solver for sparse coefficient matrices we are able to quickly obtain the
  coefficients needed for the approximation. Additionally only two of the basis functions
  are needed to compute a given point of the approximation as all other basis
  functions will be zero for any given point, reducing the number of computations
  needed to evaluate an approximation.

  In order to obtain more accuracy using the trigonometric basis, you must increase
  the number of basis functions used in the approximation. This has almost no effect on
  solving the system of equations, as there is nothing to solve with a tridiagonal matrix,
  but computing the values at each of the sine basis functions increases the computational
  cost of using this approximation. The hat basis allows us to quickly obtain
  more and more accurate approximations requiring more basis functions, but not more time.

  To illustrate this, suppose we are interested in a relative accuracy of 0.0005\%
  at each of the three points 0.25, 0.50, and 0.75.
  Since the hat basis is faster it's probably going to be better for more accurate
  approximations but the trigonometric basis is probably the best if you are only
  looking for an approximation involving at most 6 terms.
\end{proof}