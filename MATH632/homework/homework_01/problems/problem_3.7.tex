\begin{problem}{3.7}
  Suppose $p(x, y, z)$, the joint probability mass function of the random variables
  $X$, $Y$, and $Z$, is given by
  \begin{align*}
    \renewcommand\arraystretch{2}
    \begin{array}{ll}
      p(1, 1, 1) = \frac{1}{8}, & p(2, 1, 1) = \frac{1}{4},\\
      p(1, 1, 2) = \frac{1}{8}, & p(2, 1, 2) = \frac{3}{16},\\
      p(1, 2, 1) = \frac{1}{16}, & p(2, 2, 1) = 0,\\
      p(1, 2, 2) = 0, & p(2, 2, 2) = \frac{1}{4}.
    \end{array}
  \end{align*}
  What is $E[X | Y=2]$? What is $E[X|Y=2, Z=1]$?
\end{problem}

\begin{proof}
  Recall that the conditional probability mass function of $X$ given that $Y = y$
  is given by
  \begin{align*}
    p_{X|Y}(x|y) = P\{X=x|Y=y\} = \frac{P\{X=x, Y=y\}}{P\{Y=y\}}.
  \end{align*}
  As a natural extension, we have that the conditional expectation of $X$ given
  that $Y=y$ is given by
  \begin{align*}
    E[X|Y=y] = \sum_x x P\{X=x|Y=y\} = \sum_x x p_{X|Y}(x|y).
  \end{align*}

  Thus, in order to find the conditional expectation of $X$ given that $Y=2$, i.e.\ $E[X | Y=2]$,
  we first need to determine $p_{X|Y}(x|2)$. We note from the above joint probability mass function
  that
  \begin{align*}
    P\{Y=2\} = \sum_{x,z}p(x,2,z) = p(1, 2, 1) + p(2, 2, 1) + p(1, 2, 2) + p(2, 2, 2) = \frac{5}{16}.
  \end{align*}
  Similarly, we have from the above joint probability mass function that
  \begin{align*}
    P\{X=x, Y=2\} = \sum_z p(x, 2, z) = p(x, 2, 1) + p(x, 2, 2).
  \end{align*}
  Thus, the conditional probability mass function of $X$ given that $Y=2$ is given by
  \begin{align*}
    p_{X|Y}(x|2) = \frac{P\{X=x, Y=2\}}{P\{Y=2\}} =
    \begin{cases}
      \frac{p(1,2,1) + p(1,2,2)}{5/16} = \frac{1}{5} \text{\quad if $x=1$}\\
      \frac{p(1,2,1) + p(1,2,2)}{5/16} = \frac{4}{5} \text{\quad if $x=2$}.
    \end{cases}
  \end{align*}
  Using $p_{X|Y}(x|2)$, we readily see that
  \begin{align*}
    E[X|Y=2] = \sum_x x p_{X|Y}(x|2) = 1 \cdot p_{X|Y}(1|2)  + 2 \cdot p_{X|Y}(2|2) = \frac{9}{5}.
  \end{align*}

  In order to find the conditional expectation of $X$ given that $Y=2$ and $Z=1$, i.e.\ $E[X | Y=2, Z=1]$,
  we proceed in a similar manner as previously by first finding $p_{X|Y,Z}(x|2,1)$. We note from the above joint probability mass function
  that
  \begin{align*}
    P\{Y=2, Z=1\} = \sum_{x}p(x,2,1) = p(1, 2, 1) + p(2, 2, 1) = \frac{1}{16}
  \end{align*}
  Similarly, we have from the above joint probability mass function that
  \begin{align*}
    P\{X=x, Y=2, Z=1\} = p(x, 2, 1).
  \end{align*}
  Thus, the conditional probability mass function of $X$ given that $Y=2$ and $Z=1$ is given by
  \begin{align*}
    p_{X|Y,Z}(x|2,1) = \frac{P\{X=x, Y=2, Z=1\}}{P\{Y=2, Z=1\}} =
    \begin{cases}
      \frac{p(1,2,1)}{1/16} = 1 \text{\quad if $x=1$}\\
      \frac{p(2,2,1)}{1/16} = 0 \text{\quad if $x=2$}.
    \end{cases}
  \end{align*}
  Using $p_{X|Y,Z}(x|2,1)$, we readily see that
  \begin{align*}
    E[X|Y=2,Z=1] = \sum_x x p_{X|Y,Z}(x|2,1) = 1 \cdot p_{X|Y,Z}(1|2,1)  + 2 \cdot p_{X|Y,Z}(2|2,1) = 1.
  \end{align*}

\end{proof}
\newpage
