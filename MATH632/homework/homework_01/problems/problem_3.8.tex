\begin{problem}{3.8}
  An unbiased die is successively rolled. Let $X$ and $Y$ denote, respectively, the number
  of rolls necessary to obtain a six and a five. Find:
  \begin{enumerate}
    \item $E[X]$,
    \item $E[X|Y=1]$,
    \item $E[X|Y=5]$.
  \end{enumerate}
\end{problem}

\begin{proof}
  The experiment of rolling a die, assuming the die is six-sided, has
  six possible outcomes: the die lands oriented such that the side with 1, 2, 3, 4, 5, or 6
  pips is face-up. Assuming the die is unbiased, each outcome occurs with probability
  $p=1/6$ and each trial of rolling the die is independent of any other trial.
  If $X$ and $Y$ denote, respectively, the number
  of rolls necessary to obtain a six and a five, then under the given assumptions,
  $X$ and $Y$ are both geometric random variables with parameter $p=1/6$.
  The probability mass function for these random variables is given by $p(n) = (1-p)^{n-1}p = (5/6)^{n-1}(1/6)$.

  For the following computations, we make use the fact that that the infinite series
  of a geometric sequence $a_n = q^n$ is uniformly convergent on its interval of convergence,
  in particular on the interval $[0, 1)$, given that $|q| < 1$. To demonstrate this, take $0 \leq q < 1$.
  Note that for $q < 1$, there exists $\varepsilon > 0$ such that $q < q + \varepsilon < 1$.
  Now let $a_n = q ^ n$ and $M_n = (q + \varepsilon) ^ n$ and note that for $n > 0$,
  we have that $|a_n| <  M_n$ and that $\sum_{n=1}^{\infty}M_n$ converges since $q + \varepsilon$
  is in the interval of convergence of the infinite series. Thus, the sequence
  $a_n$ meets the criteria of Weierstrass's M-test and $\sum_{n=1}^\infty a_n = \sum_{n=1}^\infty q^n$ converges uniformly on $[0, 1)$.
  Since the series $\sum_{n=1}^\infty q^n$ is uniformly convergent for $0 \leq q < 1$, we
  may switch the order of summation and differentiation on this series.

  \begin{enumerate}
    \item Suppose that $Z$ is a geometric random variable with parameter $p$.
      Then, by definition, we have that the probability mass function of $Z$ is
      given by $p(n) = (1-p)^{n-1}p$ and that the expectation of $Z$ is given by
      \begin{align*}
        E[Z] = \sum_{n=1}^\infty n p(n) = \sum_{n=1}^\infty np(1-p)^{n-1} = p\sum_{n=1}^\infty nq^{n-1}.
      \end{align*}
      We know that since $0 \leq q < 1$ the power series $\sum_{n=1}^\infty q^{n}$ converges uniformly. Thus,
      \begin{align}\label{uniform}
        \frac{d}{dq}\left[\sum_{n=1}^\infty q^n\right] = \sum_{n=1}^\infty \frac{d}{dq}\left[q^n\right] = \sum_{n=1}^\infty n q^{n-1}
      \end{align}
      Using \eqref{uniform}, we see that
      \begin{align*}
        E[Z] &= p\sum_{n=1}^\infty nq^{n-1} = p \frac{d}{dq}\left[\sum_{n=1}^\infty q^n\right] = p \frac{d}{dq}\left[\frac{1}{1-q}\right] = \frac{p}{(1-q)^2}.
      \end{align*}
      Therefore, since $q = 1 - p$ we have that $E[Z] = 1/p$.

      This result shows that for the random geometric variable $X$ with parameter $p = 1/6$,
      we have that $E[X] = 1 / (1/6) = 6$. Therefore, we expect to have to cast the die 6
      times in order to roll a six.
    \item
    \item
  \end{enumerate}
\end{proof}
\newpage
