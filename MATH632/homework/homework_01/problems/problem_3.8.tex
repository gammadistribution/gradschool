\begin{problem}{3.8}
  An unbiased die is successively rolled. Let $X$ and $Y$ denote, respectively, the number
  of rolls necessary to obtain a six and a five. Find:
  \begin{enumerate}
    \item $E[X]$,
    \item $E[X|Y=1]$,
    \item $E[X|Y=5]$.
  \end{enumerate}
\end{problem}

\begin{proof}
  The experiment of rolling a die, assuming the die is six-sided, has
  six possible outcomes: the die lands oriented such that the side with 1, 2, 3, 4, 5, or 6
  pips is face-up. Assuming the die is unbiased, each outcome occurs with probability
  $p=1/6$ and each trial of rolling the die is independent of any other trial.
  If $X$ and $Y$ denote, respectively, the number
  of rolls necessary to obtain a six and a five, then under the given assumptions,
  $X$ and $Y$ are both geometric random variables with parameter $p=1/6$.
  The probability mass function for these random variables is given by $p(n) = (1-p)^{n-1}p = (5/6)^{n-1}(1/6)$.

  \begin{enumerate}
    \item Let $Z$ be the random variable defined
      as $Z=1$ if the result of the first roll is a six and $Z=0$ if the result of the first roll is not a six.
      We may compute $E[X]$ by conditioning on the variable $Z$.
      Note that, by conditioning, we obtain
      \begin{align*}
        E[X] &= \sum_z E[X|Z=z]P\{Z=z\} \\
        &= \ \left[\frac{1}{6}\right] E[X|Z=1] + \left[\frac{5}{6}\right] E[X|Z=0] .
      \end{align*}
      If $Z=1$, then the result of the first roll is a six, so the number of rolls to obtain a six is clearly 1 and $E[X|Z=1] = 1$.
      Likewise, if the result of the first roll is not a six, then the expected
      number of rolls to obtain a six given that the first roll is not a six is 1 more than
      the expected number of rolls to obtain a six so that $E[X|Z = 0] = 1 + E[X]$.
      Therefore,
      \begin{align*}
        E[X]
        &= \ \left[\frac{1}{6}\right] E[X|Z=1] + \left[\frac{5}{6}\right] E[X|Z=0]\\
        &= \frac{1}{6} + \left[\frac{1}{6}\right](1 + E[X])
      \end{align*}
      which implies that $E[X] = 6$.
    \item We wish to find $E[X|Y=1]$, i.e.\ the expected number of rolls to
      obtain a six given that the first roll is a five. Using the same reasoning
      as in part a, we know that the expected
      number of rolls to obtain a six given that the first roll is not a six (it's a five) is 1 more than
      the expected number of rolls to obtain a six. Therefore,
      \begin{align*}
        E[X|Y=1] = 1 + E[X] = 7
      \end{align*}
      where we used the result previously obtained that $E[X] = 6$.
    \item In order to calculate $E[X|Y=y]$ for some $y>1$, we first compute $p_{X|Y}(x|y)$.
      Suppose that $Y=y$ for some $y > 1$. From this we gather that the first
      $y-1$ trials result in not rolling a five while the $y$-th trial results in rolling a five.

      As a consequence, if $X=x$ where $x < y$ then the first $x$ trials have only five possible
      outcomes with the $x$-th trial resulting in a success out of those five outcomes. Thus,
      \begin{align*}
        P\{X=x | Y=y\} = \frac{1}{5}\left[\frac{4}{5}\right]^{x-1},
      \end{align*}
      i.e.\ for $x <y$ the conditional probability that $X=x$ given that $Y=y$ is
      the probability mass function of a geometric random variable with parameter $p=1/5$.

      Note that if $X=x$ where $x = y$, then
      $$P\{X=x | Y=y\} = 0$$
      since it cannot happen that on the $y$-th trial the outcome
      of the trial is that both a five and a six were rolled.

      Finally, if $X=x$ where $x > y$, then
      as mentioned, the first $y-1$ trials do not result in a five, but after the $y$-th trial
      the result obtained can in fact be a five. Thus, the first $y-1$ failures each occur with probability
      $4/5$ while the $y$-th failure occurs with probability $1$. However, after that, the failures
      of the trials $y + 1$ through $x - 1$ all occur with probability $5/6$ since it is possible for the die
      to roll a five during these trials. On the $x$-th trial the
      trial succeeds with probability 1/6. Thus, if $x > y$, then
      \begin{align*}
        P\{X=x | Y=y\} = \frac{1}{6}\left[\frac{4}{5}\right]^{y-1}\left[\frac{5}{6}\right]^{x - y - 1}.
      \end{align*}

      Combining the above statements, we see that the conditional probability mass function
      that $X=x$ given that $Y=y$ with $y > 1$ is given by
      \begin{align*}
        p_{X|Y}(x|y) =
        \begin{cases}
          \frac{1}{5}\left[\frac{4}{5}\right]^{x-1}& \text{\quad if $x < y$}\\
          0 & \text{\quad if $x = y$}\\
          \frac{1}{6}\left[\frac{4}{5}\right]^{y-1}\left[\frac{5}{6}\right]^{x - y - 1} & \text{\quad if $x > y$}\\
        \end{cases}.
      \end{align*}

      Therefore, we have that the expected value of $X$ given that $Y=5$ is
      \begin{align*}
        E[X|Y=5] &= \sum_{x=1}^\infty x p_{X|Y}(x|5)\\
        &= \frac{1}{5}\sum_{x=1}^{4} x\left[\frac{4}{5}\right]^{x-1} + \frac{1}{6}\left[\frac{4}{5}\right]^{4}\sum_{x=6}^{\infty}x\left[\frac{5}{6}\right]^{x - 6} \\
        &= \frac{3637}{625} \approx 5.82.
      \end{align*}



  \end{enumerate}
\end{proof}
\newpage
