\begin{problem}{5.2}
  Suppose that you arrive at a single-teller bank to find five other customers in
  the bank, one being served and the other four waiting in line. You join the end of the line.
  If the service times are exponential with rate $\mu$, what is the expected amount of time
  you will spend in the bank?
\end{problem}

\begin{proof}
  Let $T_i$ denote the time that the $i$-th person spends at the teller in order to complete his
  or her transaction.
  Then the total amount of time I will spend in the bank is the amount of time the other
  five customers spend at the teller plus the time that I, the sixth customer, will spend at the teller.
  If $S$ denotes the total amount of time that I spend at the bank, then
  \begin{align*}
    S = \sum_{i=1}^6 T_i.
  \end{align*}

  Note that even though the first customer is currently being served, the service time is
  exponentially distributed with rate $\mu$, i.e.\ the waiting time is memory-less so that
  the expected service time of the first customer is still $1/\mu$.
  Since the other $T_i$ are exponential random variables with mean $1/\mu$,
  we have that
  \begin{align*}
    E[S] = \sum_{i=1}^6 E[T_i] = \frac{6}{\mu}.
  \end{align*}
\end{proof}
\newpage
