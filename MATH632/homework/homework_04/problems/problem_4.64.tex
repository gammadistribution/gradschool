\begin{problem}{4.64}
  Consider a branching process having $\mu < 1$. Show that if $X_0 = 1$, then the expected
  number of individuals that ever exist in this population is given by $1/(1-\mu)$.
  What if $X_0 = n$?
\end{problem}

\begin{proof}
  If $X_n$ represents the size of the $n$-th generation, then the sum of the sizes of all generations represents
  the total number of individuals that ever exist in the population. Thus, the expected number
  of individuals is given by $E\left[\sum_{i=0}^\infty X_i\ |\ X_0 = 1 \right]$ if the size of the first generation is 1.
  By definition,
  \begin{align*}
    E\left[\sum_{i=0}^\infty X_i\ |\ X_0 = 1 \right] &= E\left[\lim_{n\to \infty} \sum_{i=0}^n X_i\ |\ X_0 = 1 \right] \\
    &= \lim_{n\to \infty} E\left[ \sum_{i=0}^n X_i\ |\ X_0 = 1 \right] \\
    &= \lim_{n\to \infty} \sum_{i=0}^n E\left[ X_i\ |\ X_0 = 1 \right].
  \end{align*}
  It was shown previously that $E\left[ X_i\ |\ X_0 = 1 \right] = \mu^i$. Therefore, if $0 \leq \mu < 1$, then
  \begin{align*}
    E\left[\sum_{i=0}^\infty X_i\ |\ X_0 = 1 \right] &= \lim_{n\to\infty}\sum_{i=0}^n \mu^i = \frac{1}{1-\mu}.
  \end{align*}

  Now suppose that $X_0 = n$. Using the previous result that $E[X_i] = \mu E[X_{i-1}]$, we have
  \begin{align*}
    E\left[ X_i\ |\ X_0 = n \right] = n \mu^i.
  \end{align*}
  Therefore, if $X_0 = n$ and $0 \leq \mu < 1$, then
  \begin{align*}
    E\left[\sum_{i=0}^\infty X_i\ |\ X_0 = 1 \right]     &= \lim_{k\to \infty} \sum_{i=0}^k E\left[ X_i\ |\ X_0 = n \right]\\
    &= n\left[\lim_{k\to\infty}\sum_{i=0}^k \mu^i\right] = \frac{n}{1-\mu}.
  \end{align*}
\end{proof}
\newpage
