\begin{problem}{4.59}
  For the gambler's ruin problem of Section 4.5.1, let $M_i$ denote the mean number
  of games that must be played until the gambler either goes broke or reaches a
  fortune of $N$, given that he starts with $i$ for $i=0,1,\dots, N$. Show that
  $M_i$ satisfies
  \begin{align*}
    M_0 = M_N = 0; \quad M_i = 1 + p M_{i+1} + q M_{i-1}, \quad i=1,\dots, N - 1.
  \end{align*}
  Solve these equations to obtain
  \begin{align*}
    M_i &=
    \begin{cases}
      i(N-i) & \text{if $p = 1/2$} \\
      \displaystyle \frac{i}{q-p} - \frac{N}{q-p}\frac{1-(q/p)^i}{1- (q/p)^N} & \text{if $p \neq 1/2$}
    \end{cases}.
  \end{align*}
\end{problem}

\begin{proof}
  It is clear that if $M_i$ is the mean number of games that must be played until
  the gambler either goes broke or reaches a fortune of $N$ given that he
  starts with $i$ for $i =0,1,\dots,N$,
  then $M_0 = M_N = 0$ since if the gambler starts with either $0$ or $N$ the process ends, i.e.\
  no games will be played.

  So suppose that $i = 1, \dots, N -1$ and let $X_n$ denote the number of games that will be played
  and let $Y = \{0, 1\}$ indicate whether the initial game is won or lost.
  Assuming that the initial start for the gambler is $i$, we have that $M_i = E[X_n \ |\ X_0 = i]$.
  Conditioning on the initial outcome of the game, i.e.\ that the gambler either wins or loses, we have that
  \begin{align*}
    E[X_n \ |\ X_0 = i] &= p E[X_n \ |\ X_0 = i, Y=1] + q E[X_n \ |\ X_0 = i, Y= 0].
  \end{align*}
  If the gambler starts with fortune $i$ and the outcome of the game is a win, then the
  number of games that will be played is 1 plus the expected number of games
  to be played given that the gambler starts with fortune $i+1$. Similarly, if the
  outcome of the game is a loss, the number of games to be played is 1 plus the expected number
  of games to be played given that the gambler starts with fortune $i -1$.
  Thus,
  \begin{align*}
    M_i = E[X_n \ |\ X_0 = i] &= p(1 + E[X_n \ |\ X_0 = i+1]) + q(1 + E[X_n \ |\ X_0 = i-1]) \\
    &= p + q + pE[X_n \ |\ X_0 = i+1] + qE[X_n \ |\ X_0 = i-1] \\
    &= 1 + p M_{i+1} + q M_{i-1}.
  \end{align*}

  Note that $p + q = 1$ so that $M_i = 1 + p M_{i+1} + q M_{i-1}$ is equivalent to
  \begin{align*}
    p M_i + q M_i = pM_{i+ 1} + q M_{i-1} + 1
  \end{align*}
  for $i = 1, \dots N-1$. Hence, we have that
  \begin{align*}
    M_{i+1}- M_i  = \frac{q}{p}(M_i - M_{i-1}) - \frac{1}{p}.
  \end{align*}
  Since $M_0 = 0$, we easily see that
  \begin{align*}
    M_2 - M_1 &= \frac{q}{p}(M_1 - M_0) - \frac{1}{p} = \frac{q}{p}M_1 - \frac{1}{p}\\
    M_3 - M_2 &= \frac{q}{p}(M_2 - M_1) - \frac{1}{p} = \left(\frac{q}{p}\right)^2 M_1 - \frac{q}{p^2} - \frac{1}{p}\\
    &\vdots \\
    M_{i+1} - M_i &= \frac{q}{p}(M_i - M_{i-1}) - \frac{1}{p} = \left(\frac{q}{p}\right)^i M_1 - \frac{1}{q}\sum_{k=1}^i \left(\frac{q}{p}\right)^k\\
    &\vdots \\
    M_{N} - M_{N-1} &= \frac{q}{p}(M_{N-1} - M_{N-2}) - \frac{1}{p} = \left(\frac{q}{p}\right)^{N-1} M_1 - \frac{1}{q}\sum_{k=1}^{N-1} \left(\frac{q}{p}\right)^k
  \end{align*}

  Adding the first $i$ equations shows that
  \begin{align}\label{mi_equation}
    M_{i} - M_1 &= \sum_{j=1}^{i-1} \left[\left(\frac{q}{p}\right)^j M_1 - \frac{1}{q}\sum_{k=1}^j \left(\frac{q}{p}\right)^k\right]\notag\\
    &= M_1 \sum_{j=1}^{i-1} \left(\frac{q}{p}\right)^j - \frac{1}{q}\sum_{j=1}^{i-1} \sum_{k=1}^j \left(\frac{q}{p}\right)^k.
  \end{align}
  These sums are finite geometric progressions, so it is easy to find their closed forms. Thus, if $p\neq 1/2$, then
  \begin{align*}
    M_{i} &=M_1 -M_1 \left[\frac{-q+p(q/p)^i}{p-q}\right] + \frac{1}{q}\sum_{j=1}^i\frac{q(-1 + (q/p)^j)}{p-q} \\
    &=M_1 -M_1 \left[\frac{-q+p(q/p)^i}{p-q}\right] - \frac{-iq + p(-1 + i + (q/p)^i)}{(p-q)^2}
  \end{align*}
  Since $M_N = 0$, we know that
  \begin{align*}
    0 = M_1 - M_1 \left[\frac{-q+p(q/p)^{N}}{p-q}\right] - \frac{-Nq + p(-1 + N + (q/p)^N)}{(p-q)^2}.
  \end{align*}
  Solving this equation shows that
  \begin{align*}
    M_1 = \frac{p - Np + Nq - p(q/p)^N}{p(p-q)(-1+(q/p)^N)}.
  \end{align*}
  Therefore, if $p\neq 1/2$, we have that
  \begin{align*}
    M_i &= \frac{i - i(q/p)^N + N(-1+(q/p)^N)}{(p-q)(-1+(q/p)^N)}\\
    &= \frac{i(1 - (q/p)^N)}{(q-p)(1-(q/p)^N)} - \frac{N(1-(q/p)^N)}{(q-p)(1-(q/p)^N)}\\
    &=\frac{i}{q-p} - \frac{N}{q-p}\frac{1-(q/p)^i}{1- (q/p)^N}.
  \end{align*}

  If on the other hand we have that $p = q = 1/2$, then from \eqref{mi_equation}, we have that
  \begin{align*}
    M_{i} - M_1 &= \sum_{j=1}^{i-1} \left[\left(\frac{q}{p}\right)^j M_1 - \frac{1}{q}\sum_{k=1}^j \left(\frac{q}{p}\right)^k\right]\notag\\
    &= M_1 (i-1) - i(i-1).
  \end{align*}
  Using the fact that $M_N = 0$, we see that $M_1 = N - 1$. Therefore, if $p=q$, then
  $M_i = i(N-1) - i(i-1)= i(N-i)$ and we have that
  \begin{align*}
    M_i &=
    \begin{cases}
      i(N-i) & \text{if $p = 1/2$} \\
      \displaystyle \frac{i}{q-p} - \frac{N}{q-p}\frac{1-(q/p)^i}{1- (q/p)^N} & \text{if $p \neq 1/2$}
    \end{cases}.
  \end{align*}
\end{proof}
\newpage
