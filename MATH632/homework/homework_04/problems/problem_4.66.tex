\begin{problem}{4.66}
  For a branching process, calculate $\pi_0$ when
  \begin{enumerate}
    \item $P_0 = \frac{1}{4}$, $P_2 = \frac{3}{4}$.
    \item $P_0 = \frac{1}{4}$, $P_1 = \frac{1}{2}$, $P_2 = \frac{1}{4}$.
    \item $P_0 = \frac{1}{6}$, $P_1 = \frac{1}{2}$, $P_3 = \frac{1}{3}$.
  \end{enumerate}
\end{problem}

\begin{proof}
  Recall for a branching process that $\mu$ is the mean number of offspring of an individual such that
  $\mu = \sum_{n=0}^\infty n P_n$ where $P_n$ is the probability that an individual will
  produce $n$ offspring.

  Note that $\pi_0$ is the probability that the population will eventually die out.
  Also note that if $\mu \leq 1$, then $\pi_0 = 1$. Otherwise, if $\mu > 1$, then
  $\pi_0$ is the smallest positive number satisfying the equation $\pi_0 = \sum_{n=0}^\infty \pi_0^n P_n.$

  \begin{enumerate}
    \item If $P_0 = \frac{1}{4}$ and  $P_2 = \frac{3}{4}$, then $\mu = \frac{3}{2}$ and $\pi_0$
      is the smallest positive number satisfying
      $$\pi_0 = P_0 + P_2 \pi_0^2.$$
      Thus, $\pi_0$ is the smallest positive root of the equation
      \begin{align*}
        P_2\pi_0^2 - \pi_0 + P_0 = \frac{3}{4}\pi_0^2 - \pi_0 + \frac{1}{4}=0.
      \end{align*}
      Solving the above equation leads to the roots $\pi_{01} = \frac{1}{3}$ and $\pi_{02} = 1$.
      Therefore, since $\pi_{01}$ is the smallest positive root satisfying the above equation, we have that
      $\pi_0 = \frac{1}{3}$.

    \item If $P_0 = \frac{1}{4}$, $P_1 = \frac{1}{2}$, $P_2 = \frac{1}{4}$,
      then $\mu = 1$ and therefore we must have that $\pi_0 = 1$.

    \item If $P_0 = \frac{1}{6}$, $P_1 = \frac{1}{2}$, and $P_3 = \frac{1}{3}$,
      then $\mu = \frac{3}{2}$ and $\pi_0$ is the smallest positive number
      satisfying
      $$\pi_0 = P_0 + P_1 \pi_0 + P_3 \pi_0^3.$$
      Thus, $\pi_0$ is the smallest positive root of the equation
      \begin{align*}
        P_3\pi_0^3 + (P_1- 1)\pi_0 + P_0 = \frac{1}{3}\pi_0^3 -\frac{1}{2} \pi_0 + \frac{1}{6}=0.
      \end{align*}
      Solving the above equation leads to the roots $\pi_{01} = 1$, $\pi_{02} = (-1 - \sqrt{3})/2$,
      and $\pi_{03} = (-1 + \sqrt{3})/2$.
      Therefore, since $\pi_{03}$ is the smallest positive root satisfying the above equation, we have that
      $\pi_0 =  (-1 + \sqrt{3})/2$.
  \end{enumerate}
\end{proof}
\newpage
