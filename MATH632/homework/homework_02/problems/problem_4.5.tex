\begin{problem}{4.5}
  A Markov chain $\{X_n: n \geq 0\}$ with states 0,1,2, has the transition
  probability matrix
  \begin{align*}
    \renewcommand\arraystretch{1.5}
    \boldsymbol{P} =
    \begin{bmatrix}
      \frac{1}{2} & \frac{1}{3} & \frac{1}{6} \\
      0 & \frac{1}{3} & \frac{2}{3} \\
      \frac{1}{2} & 0 & \frac{1}{2} \\
    \end{bmatrix}.
  \end{align*}

  If $P\{X_0 = 0\} = P\{X_0 = 1\} = 1/4$, find $E[X_3]$.
\end{problem}

\begin{proof}
  If $\{X_n: n \geq 0\}$ is a Markov chain with state space $\mathcal{M}=\{0, 1, 2\}$,
  then we know that $\{X_n: n \geq 0\}$ is a stochastic process and that
  this stochastic process has the property that
  \begin{align*}
    P\{X_{n} = j\ |\ X_{n-1} = i, X_{n-2} =i_{n-2}, \dots, X_0 = i_0\} = P\{X_{n}=j\ |\ X_{n-1} = i\}
  \end{align*}
  for any time $n\in\mathbb{Z}^+$, i.e.\ the probability that $X_{n}$ is in state $j$
  depends only on the probability that $X_{n-1}$ is in state $i$.
  We denote $P\{X_{n}=j\ |\ X_{n-1} = i\}$ by $P_{ij}$ and $P\{X_{n}=j\ |\ X_{0} = i\}$ by $P_{ij}^n$.

  We wish to find
  \begin{align*}
    E[X_3] = \sum_{j=0}^2 x P\{X_3 = j\}.
  \end{align*}
  If we know the probability that $X_0 = i$ for all states $i\in\mathcal{M}$,
  then we can condition $P\{X_3 = j\}$ on the probability that $X_0$ is in state $i\in\mathcal{M}$ for all states, i.e.\
  \begin{align}\label{prob_state}
    P\{X_3 = j\} &= \sum_{i\in\mathcal{M}} P\{X_3 = j \ |\ X_0 = i\} P\{X_0 = i\} \notag \\
    &= \sum_{i\in\mathcal{M}} P_{ij}^3 P\{X_0 = i\}
  \end{align}
  By assumption, we know that $P\{X_0 = 0\} = P\{X_0 = 1\} = 1/4$.
  Since $\{X_n: n \geq 0\}$ is a stochastic process, $X_0$ is
  a random variable so that $P\{X_0 = 2\} > 0$ and in particular
  \begin{align*}
    P\{X_0 = 2\} = 1 - \sum_{i\in\mathcal{M}, i\neq 2}P\{X_0 = i\} = \frac{1}{2}.
  \end{align*}
  With this, we are able to compute $P\{X_0 = i\}$ for all $i\in\mathcal{M}$ and use these
  probabilities to find \eqref{prob_state}.

  Lastly, in order to compute \eqref{prob_state}, we need to compute $P_{ij}^3$.
  Note that the transition matrix gives the probability of transitioning from state $i$ to state $j$
  i.e.\ $\boldsymbol{P} = (P_{ij})$. Let $\boldsymbol{P}^{(n)}$ be the matrix
  of $n$-step transition probabilities $P_{ij}^n$. By the Chapman-Kolmogorov
  equations, we have that $\boldsymbol{P}^{(n)} = \boldsymbol{P}^n$
  so that the $n$-step transition probability matrix can be found through multiplication
  of the transition matrix $\boldsymbol{P}$. Thus,
  \begin{align}\label{trans}
    \renewcommand\arraystretch{1.5}
    \boldsymbol{P}^{(3)} =
    \begin{bmatrix}
      \frac{1}{2} & \frac{1}{3} & \frac{1}{6} \\
      0 & \frac{1}{3} & \frac{2}{3} \\
      \frac{1}{2} & 0 & \frac{1}{2} \\
    \end{bmatrix}^3
    =
    \begin{bmatrix}
      \frac{13}{36} & \frac{11}{54} & \frac{47}{108} \\
      \frac{4}{9} & \frac{4}{27} & \frac{11}{27} \\
      \frac{5}{12} & \frac{2}{9} & \frac{13}{36} \\
    \end{bmatrix}
  \end{align}
  and $P_{ij}^3$ is the $ij$-th entry of $\boldsymbol{P}^{(3)}$.

  Using the transition matrix \eqref{trans} and equation \eqref{prob_state}, we thus
  have that
  \begin{align*}
        P\{X_3 = 0 \} &= \sum_{i\in\mathcal{M}} P_{i0}^3 P\{X_0 = i\} \\
        &= \frac{13}{36} \cdot \frac{1}{4} + \frac{4}{9} \cdot \frac{1}{4} + \frac{5}{12}\cdot\frac{1}{2} \\
        &= \frac{59}{144} \\ \\
        P\{X_3 = 1 \} &= \sum_{i\in\mathcal{M}} P_{i1}^3 P\{X_0 = i\} \\
        &= \frac{11}{54} \cdot \frac{1}{4} + \frac{4}{27} \cdot \frac{1}{4} + \frac{2}{9}\cdot\frac{1}{2} \\
        &= \frac{43}{216} \\ \\
        P\{X_3 = 2 \} &= \sum_{i\in\mathcal{M}} P_{i2}^3 P\{X_0 = i\} \\
        &= \frac{47}{108} \cdot \frac{1}{4} + \frac{11}{27} \cdot \frac{1}{4} + \frac{13}{36}\cdot\frac{1}{2} \\
        &= \frac{169}{432}.
  \end{align*}
  Therefore,
  \begin{align*}
    E[X_3] &= \sum_{j=0}^2 x P\{X_3 = j\} \\
    &= P\{X_3 = 1\} + 2 P\{X_3 = 2\} \\
    &= \frac{43}{216} + 2 \cdot \frac{169}{432} = \frac{53}{54}.
  \end{align*}
\end{proof}
\newpage
