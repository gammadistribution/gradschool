\begin{problem}{4.14}
  Specify the classes of the following Markov chains and determine whether they
  are transient or recurrent:
  \begin{align*}
    \renewcommand\arraystretch{1.5}
    \begin{array}{ll}
      \boldsymbol{P_1} =
      \begin{Vmatrix}
        0           & \frac{1}{2} & \frac{1}{2} \\
        \frac{1}{2} & 0           & \frac{1}{2} \\
        \frac{1}{2} & \frac{1}{2} & 0 \\
      \end{Vmatrix}
      &
      \boldsymbol{P_2} =
      \begin{Vmatrix}
        0           & 0           & 0           & 1 \\
        0           & 0           & 0           & 1 \\
        \frac{1}{2} & \frac{1}{2} & 0           & 0 \\
        0           & 0           & 1           & 0\\
      \end{Vmatrix} \\\\
      \boldsymbol{P_3} =
      \begin{Vmatrix}
        \frac{1}{2} & 0           & \frac{1}{2} & 0           & 0           \\
        \frac{1}{4} & \frac{1}{2} & \frac{1}{4} & 0           & 0           \\
        \frac{1}{2} & 0           & \frac{1}{2} & 0           & 0           \\
        0           & 0           & 0           & \frac{1}{2} & \frac{1}{2} \\
        0           & 0           & 0           & \frac{1}{2} & \frac{1}{2} \\
      \end{Vmatrix}
      &
      \boldsymbol{P_4} =
      \begin{Vmatrix}
        \frac{1}{4} & \frac{3}{4} & 0           & 0           & 0           \\
        \frac{1}{2} & \frac{1}{2} & 0           & 0           & 0           \\
        0           & 0           & 1           & 0           & 0           \\
        0           & 0           & \frac{1}{3} & \frac{2}{3} & 0           \\
        1           & 0           & 0           & 0           & 0           \\
      \end{Vmatrix}
    \end{array}
  \end{align*}
\end{problem}

\begin{proof}
  In order to determine the equivalence classes of a Markov chain with state space $\mathcal{M}$,
  we must partition the state space by the relation communicates, i.e.
  states $i$ and $j$ are in the same equivalence class if there are $m, n \in \mathbb{N}$ with $P_{ij}^m > 0$ and $P_{ji}^n > 0$.
  From there, we classify the classes as transient or recurrent depending on whether
  $\sum_{n=1}^\infty P_{ii}^n$ is finite or infinite, respectively.

  We begin with the Markov chain defined by the transition matrix $\boldsymbol{P_1}$ with state space $\mathcal{M}_1 = \{0, 1, 2\}$.
  Note that from the definition of $\boldsymbol{P_1}$, we readily see that $P_{01} = 1/2 > 0$ and that $P_{10} = 1/2 > 0$ so that state 0 communicates with state 1.
  Similarly, we see that $P_{02} = 1/2 > 0$ and that $P_{20} = 1/2 > 0$ so that state 0 communicates with state 2. Since
  every state communicates with itself by definition, we have that state 0 communicates with all $i\in\mathcal{M}_1$.
  Therefore the only equivalence class of the Markov chain is $\mathcal{M}_1$ itself and this Markov chain is
  irreducible. From a previous result, since $\mathcal{M}_1$ is finite and the Markov chain is irreducible,
  every state is recurrent.

  Consider now the Markov chain defined by the transition matrix $\boldsymbol{P_2}$ with state space $\mathcal{M}_2 = \{0, 1, 2, 3\}$.
  It can be shown that $\boldsymbol{P_2}^4 = \boldsymbol{P_2}$. Thus, we must have for $m > 0 $ that $\boldsymbol{P_2}^m = \boldsymbol{P_2}^n$
  where $n = 1,2,3$. So in order to classify the states of $\mathcal{M}_2$ we need only look at $\boldsymbol{P_2}, \boldsymbol{P_2}^2$, and $\boldsymbol{P_2}^3$.
  Computing these powers of the initial transition matrix show that
  \begin{align*}
    \renewcommand\arraystretch{1.5}
    \begin{array}{lll}
      \boldsymbol{P_2} =
      \begin{Vmatrix}
        0           & 0           & 0           & 1 \\
        0           & 0           & 0           & 1 \\
        \frac{1}{2} & \frac{1}{2} & 0           & 0 \\
        0           & 0           & 1           & 0\\
      \end{Vmatrix},
      &
      \boldsymbol{P_2}^2 =
      \begin{Vmatrix}
        0           & 0           & 1           & 0 \\
        0           & 0           & 1           & 0 \\
        0           & 0           & 0           & 1 \\
        \frac{1}{2} & \frac{1}{2} & 0           & 0\\
      \end{Vmatrix},
      &
      \boldsymbol{P_2}^3 =
      \begin{Vmatrix}
        \frac{1}{2} & \frac{1}{2} & 0           & 0 \\
        \frac{1}{2} & \frac{1}{2} & 0           & 0 \\
        0           & 0           & 1           & 0 \\
        0           & 0           & 0           & 1 \\
      \end{Vmatrix}.
    \end{array}
  \end{align*}
  From these transition matrices we see that:
  \begin{align*}
    P_{01}^3 = 1/2 > 0 &\text{ and } P_{10}^3 = 1/2 >0, \\
    P_{02}^2 = 1 > 0 &\text{ and } P_{20} = 1/2 >0, \\
    P_{03} = 1 > 0 &\text{ and } P_{30}^2 = 1/2 >0.
  \end{align*}
  Thus, state 0 communicates with all states $i\in\mathcal{M}_2$ and $\mathcal{M}_2$
  is the only equivalence class of this Markov chain. Therefore, as mentioned above,
  the Markov chain is irreducible and every state is recurrent.

  Now consider now the Markov chain defined by the transition matrix $\boldsymbol{P_3}$ with state space $\mathcal{M}_3 = \{0, 1, 2, 3, 4\}$.
  Note that
  \begin{align*}
    \renewcommand\arraystretch{1.5}
    \boldsymbol{P_3} =
    \begin{Vmatrix}
      \frac{1}{2} & 0           & \frac{1}{2} & 0           & 0           \\
      \frac{1}{4} & \frac{1}{2} & \frac{1}{4} & 0           & 0           \\
      \frac{1}{2} & 0           & \frac{1}{2} & 0           & 0           \\
      0           & 0           & 0           & \frac{1}{2} & \frac{1}{2} \\
      0           & 0           & 0           & \frac{1}{2} & \frac{1}{2} \\
    \end{Vmatrix}
    =
    \begin{Vmatrix}
      \boldsymbol{A} & \boldsymbol{0} \\
      \boldsymbol{0} & \boldsymbol{B}
    \end{Vmatrix},
  \end{align*}
  a diagonal matrix. Thus, for any $n > 0$,
  \begin{align}\label{power}
    \boldsymbol{P_3}^n
    =
    \begin{Vmatrix}
      \boldsymbol{A} & \boldsymbol{0} \\
      \boldsymbol{0} & \boldsymbol{B}
    \end{Vmatrix}^n
    =
    \begin{Vmatrix}
      \boldsymbol{A}^n & \boldsymbol{0} \\
      \boldsymbol{0} & \boldsymbol{B}^n
    \end{Vmatrix}.
  \end{align}
  Note that $P_{34} = 1/2 > 0$ and $P_{43} = 1/2 > 0$ so that state 3 communicates with state 4.
  We can readily see from \eqref{power} that for all $m\geq 0$ we have that
  $P_{i3}^m =0$ and $ P_{i4}^m = 0$ for $i=0,1,2$ so that states 0, 1, and 2 do not communicate with states 3 and 4.
  Thus $\{3, 4\}$ forms an equivalence class of this Markov chain.

  We can also see that $P_{02} = 1/2 > 0$ and $P_{20} = 1/2 > 0$ so that state 0 communicates with state 2.
  However, it can be shown through induction that
  \begin{align}\label{a_n}
    \renewcommand\arraystretch{1.5}
    \boldsymbol{A}^n =
    \begin{bmatrix}
      \frac{1}{2}           & 0             & \frac{1}{2} \\
      \frac{2^{n}-1}{2^{n+1}} & \frac{1}{2^n} & \frac{2^{n}-1}{2^{n+1}} \\
      \frac{1}{2}           & 0             & \frac{1}{2}
    \end{bmatrix}
  \end{align}
  so that it is clear for all $m\geq 0$ that $P_{01}^m = 0$ and $P_{21}^m = 0$. Thus,
  states 0 and 2 do not communicate with state 1. Thus, the other equivalence classes are formed
  by $\{0, 2\}$ and $\{1\}$. In order to classify these equivalences classes
  of this Markov chain we examine the probabilities $P_{ii}^n$. Recall by a previous proposition that a state
  $i$ is recurrent if
  \begin{align*}
    \sum_{n=1}^{\infty} P_{ii}^n = \infty
  \end{align*}
  and is transient if the sum is finite. Note that $\boldsymbol{B}$ is an idempotent
  matrix, i.e.\ $\boldsymbol{B}^2 = \boldsymbol{B}$. Thus, $\boldsymbol{B}^n = \boldsymbol{B}$ and for $n > 0$, we have
  that $P_{33}^n = P_{44}^n = 1/2$. From \eqref{a_n} we see that for $n > 0$ we have that
  $P_{00}^n = P_{22}^n = 1/2$ while $P_{11} = 2^{-n}$. Now,
  we see that for $i=0,2,3,4$ we have that
  \begin{align*}
    \sum_{n=1}^{\infty} P_{ii}^n = \sum_{n=1}^{\infty} \frac{1}{2} = \infty
  \end{align*}
  so that states $i=0,2,3,4$ are recurrent. However,
  \begin{align*}
    \sum_{n=1}^{\infty} P_{11}^n = \sum_{n=1}^{\infty} \frac{1}{2^n} = 1 < \infty
  \end{align*}
  so that state 1 is transient. Therefore, the classes $\{0, 2\}$ and $\{3, 4\}$
  are recurrent and $\{1\}$ is transient.

  Lastly, consider the Markov chain defined by the transition matrix $\boldsymbol{P_4}$ with state space $\mathcal{M}_4 = \{0, 1, 2, 3, 4\}$.
  It can be shown that $\boldsymbol{P_4}$ is a lower-triangular block matrix and that by induction we have that
  \begin{align}\label{power_4}
    \renewcommand\arraystretch{1.5}
    \boldsymbol{P_4}^n =
    \begin{Vmatrix}
      \frac{3}{5} \left(4 + (-1)^n\left(\frac{1}{4}\right)^n\right)     & -\frac{3}{5} \left(4 + (-1)^n\left(\frac{1}{4}\right)^n\right) & 0           & 0           & 0           \\
      \frac{2}{5} \left(6 + (-1)^{n+1} \left(\frac{1}{4}\right)^n\right) & \frac{2}{5} \left(-6 + \left(-\frac{1}{4}\right)^n\right) & 0           & 0           & 0           \\
      0           & 0           & 1           & 0           & 0           \\
      0           & 0           & 1 - \left(\frac{2}{3}\right)^n & \left(\frac{2}{3}\right)^n & 0           \\
      \frac{12}{5} \left(1 + (-1)^{n+1} \left(\frac{1}{4}\right)^n\right)           & \frac{12}{5} \left(-1 + \left(-\frac{1}{4}\right)^n\right)             & 0           & 0           & 0           \\
    \end{Vmatrix}.
  \end{align}
  Thus, from the initial transition matrix we see that $P_{01} = 3/4 > 0$ and $P_{10} = 1/2 > 0$
  so that state 0 communicates with state 1. From \eqref{power_4} we see that for all $m \geq 0$ we have that
  $P_{0i}^m = 0$ and $P_{1i}^m = 0$ for $i=2,3,4$ so that states 0 and 1 do not communicate with state 2,3, nor 4.
  Thus, $\{0, 1\}$ forms an equivalence class.
  We also see from \eqref{power_4} that for all $m\geq 0$ we have that $P_{2i}^m = 0$ for $i = 0, 1, 3, 4$
  so that state 2 does not communicate with states 0, 1, 3, nor 4 and $\{2\}$ forms an equivalence class.
  Similarly, we see from \eqref{power_4} that for all $m\geq 0$ we have that $P_{i3}^m = 0$ and $P_{j4}^m = 0$
  for $i=0,1,2,4$ and $j=0,1,2,3$. Thus state 3 does not communicate with any other states and state 4 does not communicate
  with any other states. Thus, the last equivalence classes are formed by $\{3\}$ and $\{4\}$.
  Finally, from \eqref{power_4} we can see that
  \begin{align*}
    \sum_{n=1}^\infty P_{00}^n &= \sum_{n=1}^\infty \frac{3}{5} \left(4 + (-1)^n\left(\frac{1}{4}\right)^n\right) =\infty \\
    \sum_{n=1}^\infty P_{22}^n &= \sum_{n=1}^\infty 1 =\infty \\
  \end{align*}
  and the equivalence classes $\{0,1\}$ and $\{2\}$ are recurrent while
  \begin{align*}
    \sum_{n=1}^\infty P_{33}^n &= \sum_{n=1}^\infty \left(\frac{2}{3}\right)^n = 2 < \infty \\
    \sum_{n=1}^\infty P_{44}^n &= 0 < \infty
  \end{align*}
  so the equivalence classes $\{3\}$ and $\{4\}$ are transient.

\end{proof}
\newpage
