\begin{problem}{4.29}
  An organization has $N$ employees where $N$ is a large number. Each employee
  has one of three possible job classifications and changes classifications (independently)
  according to a Markov chain with transition probabilities
  \begin{align*}
    \bold{P} =
    \begin{bmatrix}
      0.7 & 0.2 & 0.1 \\
      0.2 & 0.6 & 0.2 \\
      0.1 & 0.4 & 0.5
    \end{bmatrix}
  \end{align*}
  What percentage of employees are in each classification?
\end{problem}

\begin{proof}
  Let $\mathcal{M} = \{0,1,2\}$ be the states of this Markov process.
  If $N$ is large, the percentage of employees in classification $0, 1, 2$ are given by the long-run
  proportions of the stated Markov chain $\pi_0, \pi_1, \pi_2$, respectively.
  It is clear that this Markov chain is irreducible and positive recurrent. Thus, the long-run
  proportions of the Markov chain satisfy the following equations:
  \begin{align*}
     \bold{P}^\intercal \begin{bmatrix} \pi_0 \\ \pi_1 \\ \pi_2 \\ \end{bmatrix} &= \begin{bmatrix} \pi_0 \\ \pi_1 \\ \pi_2 \\\end{bmatrix},\\
     \sum_{j\in \mathcal{M}} \pi_j &= 1.
  \end{align*}

  Thus, we must solve the system of equations
  \begin{align*}
    0.2 \pi_0 + 0.6 \pi_1 + 0.4 \pi_2 &= \pi_1 \\
    0.1 \pi_0 + 0.2 \pi_1 + 0.5 \pi_2 &= \pi_2 \\
    \phantom{0.1} \pi_0 + \phantom{0.1}\pi_1 + \phantom{0.1} \pi_2 &= 1.
  \end{align*}
  Using a computer algebra system we see that the solution to the system is given
  by
  \begin{align*}
    \pi_0 &= \frac{6}{17} \approx 0.352941 \\
    \pi_1 &= \frac{7}{17} \approx 0.411765 \\
    \pi_2 &= \frac{4}{17} \approx 0.235294.
  \end{align*}

  Therefore, if $N$ is large, approximately 35.29\% of employees are in classification 0,
  41.18\% of employees are in classification 1, and 23.53\% of employees are in classification 2.
\end{proof}
\newpage
