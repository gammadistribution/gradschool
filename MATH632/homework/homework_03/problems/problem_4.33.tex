\begin{problem}{4.33}
  A professor continually gives exams to her students. She can give three possible types
  of exams, and her class is graded as either having done well or badly. Let $p_i$
  denote the probability that the class does well on a type $i$ exam, and suppose that
  $p_1 = 0.3$, $p_2 = 0.6$, and $p_3 = 0.9$. If the class does well on an exam,
  then the next exam is equally likely to be any of the three types. If the class
  does badly, then the next exam is always type 1. What proportion of exams
  are type $i$?
\end{problem}

\begin{proof}
  Let $\{X_n : n\geq 0\}$ be the stochastic process of test types assigned on the $n$-th
  test where if $X_n = i$, then the $n$-test is of type $i$. The assumptions of the problem
  imply that the probability that the next test is of a certain type depends only on the
  type of the previous test so that this stochastic process may be modeled as
  a Markov chain with state space $\mathcal{M} = \{1, 2, 3\}$.

  Since the probability that the next test is of type $j$ is dependent upon the
  knowledge that the current test is of type $i$ and also whether or not the class does
  well on the current test, we can find $P_{ij}$ by conditioning on the stochastic process
  $\{S_n : n \geq 0\}$ where we define $S_n = 1$ to indicate that the class does
  well on test $n$ and $S_n = 0$ to indicate that the class does not do well on test $n$.
  Thus,
  \begin{align*}
    P_{ij} &= P\{X_{n+1} = j \ |\ X_n = i\} \\
    &= \sum_{k}P\{X_{n+1} = j \ |\ X_n = i, S_n = k\}P\{S_n = k\}.
  \end{align*}

  Note that if the current test is of type $i$ then the class does well on the exam
  with probability $p_i$, i.e.\ $P\{S_n = 1\} = p_i$ and hence $P\{S_n = 0\} = 1 - p_i$.
  We now examine $P\{X_{n+1} = j \ |\ X_n = i, S_n = k\}$ for $k=0,1$. It is clear
  that
  \begin{align*}
    P\{X_{n+1} = j \ |\ X_n = i, S_n = k\} = \frac{P\{X_{n+1} = j, X_n = i, S_n = k\}}{P\{X_n = i, S_n = k\}}.
  \end{align*}
  If $k = 0$, then the class does not do well on the current exam so that $P\{X_n = i, S_n = 0\} = 1 - p_i$.
  If the class does not do well on the exam, then the next test they receive is of type 1.
  Thus,
  \begin{align*}
    P\{X_{n+1} = j, X_n = i, S_n = 0\} =
    \begin{cases}
     1 - p_i & \text{if $j = 1$} \\
     0 & \text{if $j \neq 1$}
    \end{cases}.
  \end{align*}
  Similarly, if $k=1$, then the class does well on the exam so that $P\{X_n = i, S_n = 1\} = p_i$.
  If the class does not do well on the exam, then the next test is of type 1, 2, 3 and occurs with equal probability.
  Thus,
  \begin{align*}
    P\{X_{n+1} = j, X_n = i, S_n = 1\} = \frac{p_i}{3}.
  \end{align*}
  Using these relations, we see that
  \begin{align*}
    P\{X_{n+1} = j \ |\ X_n = i, S_n = k\} =
    \begin{cases}
      \frac{1-p_i}{1-p_i}(1 - p_i) & \text{if $k=0$ and $j = 1$}\\
      0 & \text{if $k=0$ and $j \neq 1$}\\
      \frac{p_i}{3p_i}(p_i) & \text{if $k=1$}
    \end{cases}.
  \end{align*}
  Therefore, the transition matrix $\bold{P}$ of the Markov chain is given by
  \begin{align*}
    \bold{P}
    =
    \renewcommand\arraystretch{1.5}
    \begin{bmatrix}
      \frac{p_1}{3} + 1 - p_1 & \frac{p_1}{3} & \frac{p_1}{3} \\
      \frac{p_2}{3} + 1 - p_2 & \frac{p_2}{3} & \frac{p_2}{3} \\
      \frac{p_3}{3} + 1 - p_2 & \frac{p_3}{3} & \frac{p_3}{3} \\
    \end{bmatrix}
    = \
    \begin{bmatrix}
      \frac{4}{5} & \frac{1}{10} & \frac{1}{10} \\
      \frac{3}{5} & \frac{1}{5} & \frac{1}{5} \\
      \frac{2}{5} & \frac{3}{10} & \frac{3}{10} \\
    \end{bmatrix}.
  \end{align*}

  It is clear that this Markov chain is irreducible and positive recurrent. Thus,
  the long-run proportions of the Markov chain satisfy the following equations:
  \begin{align*}
     \bold{P}^\intercal \begin{bmatrix} \pi_1 \\ \pi_2 \\ \pi_3 \\ \end{bmatrix} &= \begin{bmatrix} \pi_1 \\ \pi_2 \\ \pi_3 \\ \end{bmatrix},\\
     \sum_{j\in \mathcal{M}} \pi_j &= 1.
  \end{align*}
  Thus, in order to determine the long-run proportions we must solve the system of
  equations
  \begin{align*}
    \begin{array}{rrrrrrr}
      (1/10)\pi_1 &+& (1/5)\pi_2 &+& (3/10)\pi_3 &=& \pi_2 \\
      (1/10)\pi_1 &+& (1/5)\pi_2 &+& (3/10)\pi_3 &=& \pi_3 \\
      \pi_1       &+& \pi_2      &+& \pi_3       &=& 1 \\
    \end{array}.
  \end{align*}
  Using a computer algebra system, we see that the solution to the above system is given
  by
  \begin{align*}
    \pi_1 = \frac{5}{7}, \quad \pi_2 = \frac{1}{7}, \quad \pi_3 = \frac{1}{7}.
  \end{align*}
Therefore, 5/7 of the exams are of type 1, and the rest of the exams are split equally between types 2 and 3.
\end{proof}
\newpage
