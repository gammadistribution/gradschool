\documentclass{article}


\usepackage{amsmath}
\usepackage{cite}
\usepackage{enumitem}
\usepackage{outlines}


\title{Thesis Outline \\ Applied and Industrial Mathematics Program \\ Towson University}
\author{Matthew Tiger \\ Thesis advisors:  Mr. Jason Muhlenkamp and Dr. M. D. Voisei \\ Spring 2018}

\setenumerate[1]{label=\Roman*.}
\setenumerate[2]{label=\Alph*.}
\setenumerate[3]{label=\roman*.}
\setenumerate[4]{label=\alph*.}

\begin{document}

\maketitle

\begin{outline}[enumerate]
    \1 \textbf{Introduction}
        \2 \textbf{Thesis Statement}
            We want to improve accuracy of impression concentration forecasts at the media plan level
            using sample data from audience measurement source. In addition to improving accuracy, we
            want to also expose variance around forecasted media plan outcome.

   \1 \textbf{Motivation and Background}
       \2 \textbf{Problem Description}
           Suppliers give us forecasted TV average audience data for their selling titles. We use
           audience measurement data to come up with forecasted average audience percentages for the selling titles.
           These forecasted average audience percentages our used by our optimization engine to recommend media plans
           given certain objectives and constraints. The suppliers give these media plan as recommendations to
           advertisers and are mostly interested in the outcome of the media plan as a whole.

           We need to explain the following terms in the problem description; their explanation should be used
           cohesively and will help guide the problem description:
           \begin{enumerate}
               \item Selling title
               \item Broadcast week
               \item TV rating
               \item Average Audience
               \item Media plan
               \item Telecast
               \item Schedule
               \item Post-Campaign Reporting
           \end{enumerate}

       \2 \textbf{Data}
           \3 \textbf{Measurement Data}
               We are given audience measurement data that provides TV viewing habits
               of the panel sample. For each historical telecast, we are given the respondents
               that were watching and their total seconds of commercial viewing and total viewing.
           \3 \textbf{Supplier Data}
               A given supplier provides us with a schedule that list all content airings
               historically and their tentative future airings. Additionally, for each selling title / week
               airing in the future

   \1 \textbf{Model Description}
       \2 Hierarchical Logistic Regression Model

   \1 \textbf{Model Implementation}
       \2 \textbf{pymc3}
           We use pymc3 to sample from the posterior distribution. pymc3 uses Hamiltonian Monte Carlo sampling techniques.
           Sampling with this method took upwards of 48 hours so we
           \3 We approximate sampling using ADVI mini-batch in order to improve sampling speed.
               ADVI works best when posterior is Gaussian without too many uncorrelations.

   \1 \textbf{Results}
       \2 Allocation level
       \2 Media plan level

   \1 \textbf{Conclusion}


\end{outline}

\end{document}