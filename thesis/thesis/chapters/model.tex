\begin{chapter}{Model}


\begin{section}{Preliminaries}
  In the following section we will provide preliminary information necessary to
  understand the model.
  \begin{subsection}{Units of analysis and observation}
    We are most interested in the content that airs.

    the units of observation are selling title airings in the schedule. The outcome
    variable we are measuring for each unit $i$ is $r_i^A$ the average number of panelists
    associated to target $A$ given the average number of panelists associated to target $B$
    where $A \subseteq B$. Of primary interest is the proportion of panelists in target $A$ relative to target $B$.

  \end{subsection}

  \begin{subsection}{Covariates}

  Using the conclusions of paper TV ratings, we include the following covariates into the model:

  \begin{enumerate}
     \item Broadcast Month - The broadcast month associated to the start of a selling title airing.
       ``The key link between the broadcast and Gregorian calendars is that the first week of every broadcast month always contains the Gregorian calendar first of the month.''
       %(``https://web.archive.org/web/20111007010739/http://www.tvb.org/planning_buying/179695/5658'')
     \item Day of Week - The day of the week associated to the start of a selling title
       airing. These are encoded from 0 - 6 with 0 being Monday and 6 being Saturday
     \item Stratified Hour - We define the following groupings of hours as this covariate.
       These groupings are adapted from the measurement source.
           morning -  airing start hour is in (6, 7, 8, 9),
           daytime - airing start hour is in  (10, 11, 12, 13, 14),
           early\_fringe - airing start hour is in (15, 16, 17, 18),
           prime\_1900\_2000 - airing start hour is 19,
           prime\_2000\_2100 - airing start hour is 20,
           prime\_2100\_2200  - airing start hour is 21,
           prime\_2200\_2300  - airing start hour is 22,
           late\_fringe - airing start hour is in (23, 0, 1),
           graveyard - airing start hour is in (2, 3, 4, 5)
     \item Content Id - The id denoting the content associated with the scheduled airing.
       e.g. Famous Funny Animated Cartoon show
     \item Lead-in Content Id - The id denoting the content associated with the preceding airing.
     \item first-run - Denotes if the airing is not a repeat airing. This is a 1 if it is not a repeat and 0 if it is a repeat.
       This information is obtained through the measurement source.
     \item genre-id - Denotes the genre associated to the airing. This conveys the general flavor of the aired content.
       We will enumerate the different genre-ids and their meaning.
       This information is obtained through the measurement source.
  \end{enumerate}

  \end{subsection}
\end{section}


\begin{section}{Assumptions}
  We assume that the observations are exchangeable given the parameters of the model and the covariates of the selling title airing.
  We describe the definition of exchangeable here. Exchangeability is important since
  it allows us to use Bayes' Theorem to compute the posterior.
  The number of draws from the binomial aren't independent, but we use this as an approximation. this assumption is imperfect.
\end{section}

\begin{section}{Description}
  See below for a formal description of the model. Since the data we observe is
  the average number of panelists in target $A$ given the average number of panelists in target $B$ with $A \subseteq B$,
  it is natural to model the likelihood as a binomial distribution with unknown probability parameter $\pi$.

  \begin{align*}
    \theta_i &= \beta_0 + \sum_{j=1}^m X_i^\intercal \beta_j, \quad \beta_j \sim \text{N}(0, 4^2) \\
    \omega_i &= \logit^{-1}(\theta_i) \\
    \kappa_i &\sim \text{Exp}(X_i ^\intercal \lambda_l ), \quad \lambda_l \sim \text{U}(1e-3, 1)\\
    \pi_i &\sim \text{Beta}\left(\omega_i \kappa_i + 1, (1 - \omega_i) \kappa_i + 1\right) \\
    y_i &| n_i, \theta_i, X_i \sim \text{Bin}(n_i, \pi_i)
  \end{align*}

  where $\logit^{-1}(\alpha) = \frac{\exp{\alpha}}{1 + \exp{\alpha}}$.
\end{section}

\begin{section}{Prior Distribution Choice}
  The coefficients are chosen so that the main mass of the distribution on the logit scale falls between 1e-5 and 1 - 1e-5.

  We chose the lambda to be uniform on 1e-3 and 1 since the kappa variable is the concentration
  parameter which is roughly the number of trials in the binomial likelihood.
  For any given telecast airing we will have at most the number of trials be
  the number of panelists which cannot exceed 100K. This prior on lambda and
  the exponential reflects that.

  We chose to separate variance based off of is-first-run since we noticed
  that this variable is indicative of how many trials there will be for an airing.
\end{section}

\begin{section}{Inference}
  We perform the inference using \texttt{pymc3}. We will display convergence checks
  such as gelman-rubin statistics, effective number of parameters among other things
  to ensure that posterior distribution obtained are representative of the actual
  posterior distribution.

  Show the calculation of rhat and neff and display summary statistics of values to
  show that inferences have approximate convergence and that trace objects are representative
  of
\end{section}

\begin{section}{Validation}
  posterior predictive checking.

  Given the posterior distribution obtained through inference, we make draws
  from the distribution to arrive at replicated data, $y^{\text{rep}}$. We check
  the distribution of the replicated data using test statistics comparing replicated data to the observed
  test statistic in the data.

  - We check the resulting posteriors using test statistics
  of the mean, min, max, and std of the number of in-target panelists. This makes
  sure that the data generated by the model are consistent with the observed data.

  - ordered discrete data check to see if data is under/overdispersed compared to replicated data.

  - binned residual plots against observed in-target panelists to check that the
  regression fits make sense.

  - The data was collected sequentially in time so we must check if residuals
  have autocorrelation.

  - outlier check to see if any impressions conc predictions are greater than 1-2 std of impressions conc.

  - realized residual plots for impressions conc
\end{section}

\begin{section}{Model selection and Sensitivity Analysis}
  We use the WAIC to determine the model that has the most predictive accuracy.
  The WAIC is chosen because it is Bayesian and able to be compared across models
  with same likelihood.

  Go through the definition of WAIC and explain how it is computed etc.

  Here we will also check to see if the choice in prior is affecting inference.

  - Remove the hierarchical nature on the variance parameter and evaluate how inference is affected
  - change normal distribution on coefficients to cauchy or student t to see if inference is affected
\end{section}

\end{chapter}