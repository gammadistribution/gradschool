\begin{chapter}{Introduction}\label{introduction}
  The world of advertising today consists of a multitude of options
  for delivering a message to a targeted audience, but the relevance of TV advertising
  remains as important as ever.

  Thus, the need to correctly account for and forecast audience viewership is not surprising.
  Throughout this paper we discuss why that need exists, the challenges of providing such forecasts,
  and a solution to the issue at hand.

\begin{section}{Background}
  The TV advertising landscape consists of TV sellers, who have airtime available
  for sale to be used to air advertisements, and TV buyers, who purchase airtime
  from TV sellers in order to air their desired advertisements.

  In the normal course of events, TV buyers specify the target metrics that
  they wish to achieve through advertising such as meeting a certain number of impressions within their demographic target audience,
  i.e.\ a certain number of advertisement views.
  The TV sellers then create a media plan, which is a collection of advertising spots/units, to present to the TV buyer.
  The basis of the media plan is the forecasted number of demographic (or buy-demo) impressions for each unit.
  These demographic impressions are called the buy-demo impressions because the TV buyer is charged
  based off of the total number of forecasted demographic impressions in a media plan .
  The TV buyer then accepts or proposes changes to the media plan and eventually
  a deal is struck and the units are set to air. These buy-demo impressions are guaranteed to
  the TV buyer and additional units will be added in order to achieve that guarantee \cite{tvforecasting}. Thus, the need
  for accurate forecasts becomes clear under this paradigm.

  In addition to having a demographic target audience, the TV buyer may also have some sub-population
  target in mind, called a strategic target audience. It is then the job of the TV seller to
  provide forecasted in-target impressions for each unit in order to generate the media plan.
  The forecasted strategic target impressions are generated based off the
  forecasted buy-demo impressions since it is the latter
  that form the currency of the media plan.

  Note that the forecasts that are used by TV sellers are based off of
  television viewership as measured through audience measurement companies. Some companies
  provide such measurements
  using a statistical sample of the TV viewing population
  and track the viewership of that sample, extrapolating to the population at large.
  However, such approaches have disadvantages as will be discussed in the coming sections.
\end{section}

\begin{section}{Challenges}
  For the audience measurement source used for this paper, the sample sizes
  present on an individual program airing are orders of magnitude smaller than the statistical panel sample.
  The issue is exacerbated on networks that have low viewership and is made worse when
  the strategic target is small relative to the national population.
  This is the major challenge of providing forecasts for media plans;
  the measurements provided through the measurement company are based off of a sample
  and results in noisy data for the typical past program airing.

  A related problem exists in evaluating the performance of baseball players:
  players are judged by their batting average (percentage of hits) but this metric is not informative when
  the player has few at-bats. With more information about the league and past historical performances we are able to
  come up with a better estimate that takes such factors into account
  as well as the observed at-bats through Bayesian inference.

  Thus, we adopt the use of a similar approach in order to create better forecasts when
  small sample sizes occur in past airings data.
\end{section}

\begin{section}{Problem Description}
  Given the challenges outlined above, the problem this paper aims to solve is whether or not
  we can devise a model using
  the noisy measurement data as well as the TV supplier data that will accurately
  forecast the strategic target impressions for future program airings
  and provide improvement from the current forecasting methodology.
  Further, we wish to understand if is possible to accurately forecast the range of likely
  outcomes for each future airing. If so, then it will be possible to understand the
  probability that an advertiser will reach their targeting goals given a media plan.

  To develop a solution to these problems, we employ the theory of Bayesian statistics to devise such a model
  and provide the desired forecasts.
\end{section}
\end{chapter}