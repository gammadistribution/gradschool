\begin{chapter}{Introduction}\label{introduction}
  The world of advertising today consists of

  We are trying to provide better linear tv media plans.

  We will use the theory of Bayesian statistics to develop a model that
  will forecast future telecast airings' audience impression concentration
  distributions corrected for small sample sizes present in the measurement
  training data.

\begin{section}{Background}
  TV sellers have airtime available for sale to be used to air advertisements.
  TV buyers purchase airtime from sellers in order to air their desired advertisemensts.

  TV buyers want to buy ad airtime that will help them share their message
  with a target audience.

  Audience measurement companies provide measurements television viewership.
  Some companies take these measurements using a statistical sample of the
  TV viewing population and tracking the viewership of that sample and extrapolating
  to the population at large.

  TV sellers bundle together airtime into a unit called a selling title.
  This unit contains logically grouped airings of content. Sometimes the content
  within the selling title is specific, other times it's a blend of content packaged together.

  An example of a piece of content available is \textit{Funny Animated Cartoon}
  and an example of a selling title woudl be \textit{Funny Animated Cartoon M-F 8PM}.

  TV sellers accept media plans allocated at the selling title/week level and then
  as the flight date approaches determine the actual ad schedule that will air.
  This is known as trafficking the ads.

  The planning is done based off of demo estimates, which are impression estimates of
  a particular age/gender demographic. The TV seller provided these estimates
  using their substantive knowledge. The planning for strategic targets is
  done by forecasting a percentage. The resulting forecast is then the percentage multiplied by the forecasted
  demo estimate
\end{section}

\begin{section}{Motivation}
  The sample sizes used for the majority of telecasts in the measurement source
  are too small to be trusted on their own. This is the major challenge of providing
  forecasts for media plans; using the data on its own will produce inaccurate
  forecasts that do not take into account the small sample size.

  % Show picture of sample sizes in networks

  A motivating example is related to e-commerce reviews: players are judged by their
  batting average but this metric is not informative with few at-bats. With more
  information about the league and past historical performances we can use bayesian
  inference to come up with a better estimate that takes such factors into account
  as well as the observed at-bats.

  Simiarly, we hope to use the data from past airings to perform better inference
  on telecasts with small sample sizes.
\end{section}

\begin{section}{Challenges}
  Low sample size must be addressed.
  Variance within selling-title/week.
  Impression estimates are tied to demo forecasts.

  For the purposes of this thesis we will be addressing the first issue alone.
  We will assume for testing that we know the outcome of the demo with prescience.
\end{section}

\end{chapter}