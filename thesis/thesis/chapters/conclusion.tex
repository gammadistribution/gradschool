\begin{chapter}{Conclusion}

The aim of this paper was to discover if a model could be created that will accurately
produce strategic target forecasts, as well as the range of likely outcomes, for future program airings
and improve upon the existing forecasting methodology.
As we have seen in the previous section, the creation of such a model is possible.
Using the measurement data, we were able to perform statistical inference, understand
the major parameters that guide the generation the data, and exploit that structure
in order to provide more accurate forecasts, in general.

However, this model has its flaws; there are nuances of the ETMT network that are not
captured under this model and as such the base model outperforms the model when considering
aggregations of units. Nonetheless, the model still has its merits and can be used by
TV sellers in order to provide probabilities of reaching certain target goals of media plans as well
as understanding the range of likely impression values for each future unit.

\begin{section}{Future Work}
  Due to time constraints, we were unable to provide an improvement to the model for the ETMT network.
  Future work would be spent expanding the existing model to account for the misfit seen in the ETMT network.
  Also due to lack of time, this model was evaluated against one particular set of audiences. Practical
  media planning requires the use of a multitude of audiences and it would behoove us to evaluate
  the model against a range of audiences to understand if there are any other shortcomings. Finally,
  we wish to provide better inference summaries through the use of a utility function that captures
  the risk of not meeting the forecasted impressions quoted. Such a utility function could take into account the
  price of the spot, for instance.
\end{section}

\end{chapter}