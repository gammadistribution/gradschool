\begin{chapter}{Data}\label{data}
  In order to provide the forecasts necessary for media planning, we make use
  of several data sets. Namely, we make use of TV supplier data, which contains the inventory to plan against, and
  audience measurement data, which contains historical program airings and the
  viewing associated to those airings for various audiences. We discuss
  these data sets in more detail in the sections below.

\begin{section}{TV Supplier Data}
  TV suppliers provide us with a holistic view of their inventory that is for sale.
  The most granular unit in their inventory is a selling title during a future broadcast week, i.e.\ a
  week that starts on a Monday and ends the following Sunday. This
  unit is a grouping of content that will air during a given broadcast week, and the grouping can be
  based on either similar time of airing or similar content. The two most important
  pieces of data provided by TV suppliers are the media schedule, which contains the exact air times of
  selling titles, and the forecasted demographic impressions, which contains the
  expected number of impressions for each selling title per buy-demographic, per broadcast week. We
  provide an explanation for these data below.

  \begin{subsection}{Media Schedule}
    The media schedule lists all scheduled program airings on a given network.
    Associated to each airing is an identifier for the selling title and an identifier for the content.
    Note that the selling title indicates the grouping of content that is sold while the content
    indicates the nature of the programming that will be shown. Example records
    from a media schedule are shown in Table \ref{tab:tvsched}.

    \begin{table}[h!]
      \centering
      \begin{adjustbox}{width=1.1\textwidth,center=\textwidth}
        \large
        \begin{tabular}{lllllllll}
          network & selling title & selling title name & content & content name & start datetime & end datetime \\
          \hline
          BCST & 100 & Adult Cartoon 8PM & 10 & Adult Cartoon & 2017-04-02 20:00:00 & 2017-04-02 20:30:00 \\
          BCST & 101 & Adult Cartoon 8:30PM & 10 & Adult Cartoon & 2017-04-02 20:30:00 & 2017-04-02 21:00:00 \\
        \end{tabular}
      \end{adjustbox}
      \caption{Example media schedule data for two selling title airings.}\label{tab:tvsched}
    \end{table}

    As can be seen from Table \ref{tab:tvsched}, the same content identifier can be associated
    to multiple selling titles. This indicates that the TV supplier wishes to differentiate
    the content airing at 8:00 PM from the content airing at 8:30 PM, usually to price the airings differently.

  \end{subsection}

  \begin{subsection}{Forecasted Demographic Impressions}
    In the advertising industry, an impression is synonymous with an advertisement viewing.
    Thus, this data contains for each selling title, broadcast week, and demographic audience the expected impressions per unit, i.e.\
    the expected number of audience members that will watch a commercial airing during the given selling title and broadcast week.

    Example records from the forecasted demographic impressions data are shown in Table \ref{tab:tvdem}

    \begin{table}[h!]
      \centering
        \begin{tabular}{llll}
          selling title & broadcast week & demographic & impressions per unit \\
          \hline
          100 & 2017-03-27 06:00:00 & F45-49 & 150000 \\
          100 & 2017-03-27 06:00:00 & P18-49 & 1500000 \\
          101 & 2017-03-27 06:00:00 & F45-49 & 120000 \\
          101 & 2017-03-27 06:00:00 & P18-49 & 1000000 \\
        \end{tabular}
      \caption{Example forecasted demographic impressions data for two selling title weeks.}\label{tab:tvdem}
    \end{table}
  \end{subsection}
\end{section}


\begin{section}{Audience Measurement Data}
  As mentioned previously, the audience measurement data contains the programs that
  have aired historically and who watched them according to the audience measurement source.
  This data is stored in two separate tables, the program table and the viewing table.

  The program table details the historical program airings and metadata about those airings.
  A program in the table corresponds to an identifier of a logical grouping of
  aired content per the measurement source, while the telecast corresponds
  to an identifier of a single airing of a program.
  Some example information included is the network the program aired on, the start and end datetime of the
  program airing, and the genre associated to the program.
  Note that the genre information serves to categorize the programs aired by grouping together
  programs with similar content. Additional information about the airings in the measurement data
  is discussed in Section \ref{model:covariates}.
  Some example program data is provided in Table \ref{tab:amprogram}.

  \begin{table}[h!]
    \centering
    \begin{adjustbox}{width=1.1\textwidth,center=\textwidth}
      \large
      \begin{tabular}{lllllllllll}
        airing & network & program & telecast & program name & start datetime & end datetime & genre & is first run & is live\\
        \hline
        35 & BCST & 1000 & 301 & Adult Cartoon & 2017-04-02 20:02:00 & 2017-04-02 20:30:00 & Animation & 1 & 0 \\
        36 & BCST & 1000 & 302 & Adult Cartoon & 2017-04-02 20:30:00 & 2017-04-02 21:00:00 & Animation & 1 & 0
      \end{tabular}
    \end{adjustbox}
    \caption{Example program table data for two historical program airings.}\label{tab:amprogram}
  \end{table}

  Associated with the program table is the viewing table which details the respondents' viewing of
  historical programs. This data is stored at the minute level and contains
  which respondents were watching each minute of a given telecast, how many commercial seconds aired
  during the minute, and the total number of commercial seconds that aired.
  Additionally, limited identifying information about the
  respondent is stored in this table such as the respondent's age and gender.
  Some example viewing data is provided in Table \ref{tab:amviewing}.

  \begin{table}[h!]
    \centering
    \begin{adjustbox}{width=1.1\textwidth,center=\textwidth}
      \large
    \begin{tabular}{lllllllll}
      program & telecast & respondent & minute & comm secs & weight & age & gender & total comm secs\\
      \hline
      1000 & 301 & 2 & 3 & 60 & 2050 & 48 & F & 120\\
      1000 & 301 & 2 & 4 & 45 & 2050 & 48 & F & 120\\
      1000 & 301 & 2 & 15 & 60 & 2050 & 48 & F & 120\\
      1000 & 301 & 2 & 16 & 30 & 2050 & 48 & F & 120\\
      1000 & 302 & 2 & 22 & 15 & 2050 & 48 & F & 100\\
      1000 & 302 & 2 & 23 & 60 & 2050 & 48 & F & 100
    \end{tabular}
    \end{adjustbox}
    \caption{Example viewing table data for one respondent's viewing on historical program airings from Table \ref{tab:amprogram}.}\label{tab:amviewing}
  \end{table}

  Of primary interest from this data set is the impressions per unit,
  or average commercial minute (ACM), for a given audience.
  The definition of a program airing's ACM measurement is the weighted average
  of the in-target commercial viewing seconds over the number of commercial seconds
  that aired during the program's airing.

  Formally, let $A$ be the set of panelists that are considered in-target out of a total of $n$ possible panelists.
  Suppose that program airing $i$ has $t_i$ minutes of programming, and let
  $s_{ij}$ be the number of commercial seconds during minute $j$ of the airing.
  Further, let $p_{ijk} = 1$ if panelist $k$ was watching during the $j$-th minute of program airing $i$ and 0 otherwise,
  and let $w_{k}$ be the weight assigned to the panelist by the measurement source. Then, $m_{i}^A$, the ACM of program airing $i$,
  is
  \begin{align}\label{acm_def}
    m_{i}^A = \left\lceil\frac{\sum_{k=1}^n w_k \sum_{j=1}^{t_i} s_{ij} p_{ijk} \textbf{1}_A(p_{k})}{\sum_{j=1}^{t_i} s_{ij}}\right\rceil.
  \end{align}
  Throughout, we will refer to impressions and ACM interchangeably as the ACM is in fact a measurement
  of the number of impressions a commercial airing receives.

  A related quantity for program airing $i$, the impression concentration of target $A$ relative to target $B$ where $A \subseteq B$, is
  \begin{align*}
    c_i = \frac{m_{i}^A}{m_{i}^B}.
  \end{align*}
  This quantity is used to measure the proportion of impressions attributed to target $A$ relative to target $B$.

  As an example, consider target $A$ to be the set containing only respondent 2. Then
  from Table \ref{tab:amviewing}, we see for program 1000 and telecast 302, i.e.\ program airing 36, that $m_{36}^A = 1538$.
\end{section}

\begin{section}{Model Training Data}\label{data:used}
  For the purposes of this project, we limit the data used from the above data sets
  to three networks labeled BCST, ETMT, and SPTS. These networks are a broadcast network
  that has large reach, a cable entertainment network, and a cable sports network, respectively.
  Further, we limit the audience measurement and media schedule data to only airings that occur
  between 2015-12-28 06:00:00 and 2018-01-01 06:00:00 which coincides with broadcast years 2016 through 2017.
  We shall consider the in-target audience, target $A$, to be Females aged 45 through 49 and the buy-demographic audience, target $B$,
  to be Persons aged 18 through 49. The buy-demographic audience was chosen as it is representative
  of the demographic audience typically chosen when constructing a media plan while the in-target
  audience was chosen as it is small relative to the total number of TV viewers in the United States, roughly 3.3\%.
  As a consequence, the size of the in-target panel in the audience measurement data is small and the target is
  exemplary of the issues this project aims to solve.

  The data used for training and testing the model is combined from the above tables to create a new data set.
  This new data set is created by associating to each selling title airing in the media schedule, the
  program airing in the measurement data whose start and end datetimes most overlap with
  the start and end datetimes listed in the media schedule.
  We then assign
  the selling title airing the ACM of the matched program airing for target $A$ and target $B$.
  Example data is shown in Table \ref{tab:data}. Throughout this paper, we will consider the training set
  to be the airings occurring during broadcast year 2016 and the test, or hold-out, set to
  be the airings occurring during broadcast year 2017.

  \begin{table}[h!]
    \begin{adjustbox}{width=1.1\textwidth,center=\textwidth}
      \large
      \begin{tabular}{llllllllll}
        network & selling title & content & start datetime & end datetime & program & telecast & ACM A & ACM B\\
        \hline
        BCST & 100 & 10 & 2017-04-02 20:00:00 & 2017-04-02 20:30:00 & 1000 & 301 & 110560 & 1203560\\
        BCST & 101 & 10 & 2017-04-02 20:30:00 & 2017-04-02 21:00:00 & 1000 & 302 & 210560 & 1501000\\
      \end{tabular}
    \end{adjustbox}
    \caption{Example model training data set used for two selling title airings.}\label{tab:data}
  \end{table}


\end{section}

\end{chapter}