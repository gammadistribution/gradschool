\begin{chapter}{Data}\label{data}

We make use of several datasets in order to provide the necessary forecasts.

\begin{section}{Audience measurement data}
  The audience measurement data contains the programs that aired and who
  watched them according the audience measurement source.

  Each program in the data contains certain covariates such as the hour in which
  a program began and on which day of the week etc.

  Each program contains the list of panelists that watched the program and how much
  viewing is associated to the program. We filter down to only commercial viewing
  since this is what advertisers are most interested in.

  We are going to use the measurement source data for three networks labeled BCST
  for a broadcast network that has large reach, ETMT for a cable entertainment network,
  and SPTS for a cable sports network.

  This data is stored at the minute-level
  and stores which respondents were watching each minute of a given telecast.

  The definition of the number of impressions per equivalized unit airing
  during a telecast is the weighted average of the in-target commercial viewing seconds over the number of commercial seconds that aired during the telecast.
  Formally, let $A$ be the set of panelists that are considered in-target out of a total of $n$ possible panelists.
  Suppose that telecast $i$ has $t_i$ minutes of programming and let
  $s_{ij}$ be the number of commercial seconds during minute $j$ of the telecast.
  Further, let $p_{ijk} = 1$ if panelist $k$ was watching during the $j$-th minute of telecast $i$ and 0 otherwise,
  and let $w_{k}$ be the weight assigned to the panelist by the measurement source. Then, $m_{i}^A$, the average commercial minute (ACM) of telecast $i$
  is
  \begin{align}\label{acm_def}
    m_{i}^A = \frac{\sum_{k=1}^n w_k \sum_{j=1}^{t_i} s_{ij} p_{ijk} \textbf{1}_A(p_{k})}{\sum_{j=1}^{t_i} s_{ij}}
  \end{align}

  A related quantity is the unweighted average commercial minute, denoted $r_{i}^A$, which is
  derived from \eqref{acm_def} by setting the weights $w_k = 1$ for all $k$ and taking the ceiling of the resulting quantity, i.\ e.\
  \begin{align*}
    r_{i}^A = \left\lceil\frac{\sum_{k=1}^n \sum_{j=1}^{t_i} s_{ij} p_{ijk} \textbf{1}_A(p_{k})}{\sum_{j=1}^{t_i} s_{ij}}\right\rceil
  \end{align*}

  This quantity represents the average number of in-target panelists viewing commercials on telecast $i$.

  The impression concentration of target $A$ relative to target $B$ where $A \subseteq B$ for telecast $i$ is
  \begin{align*}
    c_i = \frac{m_{i}^A}{m_{i}^B}.
  \end{align*}
  Similarly, the unweighted impression concentration for target $A$ relative
  to target $B$ for telecast $i$ is
  \begin{align*}
    u_i = \frac{r_{i}^A}{r_{i}^B}.
  \end{align*}

  As to be expected $r_i^A$ and $m_i^A$ are nearly perfectly correlated. Further,
  the Q-Q plots show that the distributions of $c_i$ and $u_i$ for a given network agree
  after a linear transformation.

  (include QQplots)

  This suggests that the  distribution of unweighted impression concentrations is
  similar to distribution of impression concentrations. Thus, the ratio of in-target
  panelists to demo panelists is similar to the ratio of in-target average audience
  to demo average audience and we may use the raw sample sizes to estimate a population proportion
  that will be proportional to the actual impression concentration.

  The purpose of the preceding paragraph is to justify the assumption that we can use
  the raw proportion of in-target to in-demo panelists to estimate the weighted impression concentration.

\end{section}

\begin{section}{TV supplier data}
  TV suppliers provide us with a future schedule of selling title/content airings.
  They also provide the demo estimates for a variety of demos for future selling title/weeks

  \begin{subsection}{Linear Media Schedule}
    Here we will talk in more detail about the schedule and what it contains.
    We will also provide an example of such data.
    (include example record)
  \end{subsection}

  \begin{subsection}{Forecasted Demographic Estimates}
    These are estimates of forecasted demo impressions per unit. This means
    the supplier expects to get a certain amount of demo impresions for
    each allocated unit.
    (include example record)
  \end{subsection}
\end{section}

\end{chapter}