\documentclass[12pt]{article}
\usepackage[margin=1in]{geometry}
\usepackage{amsmath, amsthm, amssymb, amsfonts, enumitem}

\theoremstyle{definition}
\newtheorem{problem}{Problem}
\renewcommand*{\proofname}{Solution}
\renewcommand{\theenumi}{\alph{enumi}}


\title{Homework Assignment 1}
\author{Matthew Tiger}


\begin{document}


\maketitle


% Problem 1
\begin{problem}
  To be comprehensive, the second derivative test for two-variable functions $f = f(x, y)$
  studied in Calculus III should contain (among others) the cases:
  \begin{enumerate}
    \item $D(a, b) > 0$ and $f_{xx}(a, b) = 0,$
    \item $D(a, b) = 0$ and $f_{xx}(a, b) = 0.$
  \end{enumerate}
  Why aren't these cases considered? Explain.
\end{problem}

\begin{proof}
  Throughout, we assume that $f: S \subset \mathbb{R}^2 \to \mathbb{R}$ and that $f \in C^2(S)$ so
  that $f_{xy}(a,b) = f_{yx}(a,b)$. Therefore,
  \begin{align*}
    D(a, b)
    &= f_{xx}(a,b) f_{yy}(a,b) - f_{xy}(a,b)f_{yx}(a,b) \\
    &= f_{xx}(a,b) f_{yy}(a,b) - f_{xy}(a,b)^2.
  \end{align*}

  \begin{enumerate}
    \item To illustrate that this case can never happen, suppose to the contrary
      that $D(a, b) > 0$ and $f_{xx}(a, b) = 0.$ Since
      $D(a, b) = f_{xx}(a,b) f_{yy}(a,b) - f_{xy}(a,b)^2,$
      we see that $0 < D(a, b) = - f_{xy}(a,b)^2$ which is a contradiction since
      $f_{xy}(a,b)^2 > 0$. Therefore, this case cannot happen.
    \item Now suppose that $D(a, b) = 0$ and $f_{xx}(a, b) = 0$. As
      $D(a, b) = f_{xx}(a,b) f_{yy}(a,b) - f_{xy}(a,b)^2,$
      it is true under our supposition that $f_{xy}(a,b)^2 = 0$, i.e.\ $f_{xy}(a,b) = 0$.
      We cannot conclusively state whether the point is a local extrema or saddle point as the function could be increasing or decreasing in the direction of $x$ or $y$.

      To illustrate, take as an example $f_1(x, y) = -x^4 - y^4$ and $f_2(x, y) = x^4 + y^4$.
      Note that $f_1$ and $f_2$ both satisfy $D(a,b) = 0$ and $f_{xx}(a,b) = 0$ for the point $(a, b) = (0, 0)$.
      However, upon further inspection $f_1$ obtains a local maximum at $(0, 0)$, yet
      $f_2$ obtains a local minimum at $(0, 0)$. Thus, two different results occur for two different functions
      in the case where $D(a, b) = 0$ and $f_{xx}(a, b) = 0$ and we conclude that the
      test is inconclusive in such cases.

  \end{enumerate}
\end{proof}
\newpage


% Problem 2
\begin{problem}
  Recall that
  \begin{itemize}
    \item $(a, b)$ is called an \textit{absolute maximum} of $f = f(x, y)$ on a domain
      $D \subset \mathbb{R}^2$ if $f(x, y) \leq f(a, b)$
      for every $(x, y) \in D$.
    \item (The Extreme Value Theorem) If $f$ is continuous and $D$ is closed and bounded,
      then $f$ attains both an absolute maximum value and an absolute minimum value.
  \end{itemize}
  \begin{enumerate}
    \item Describe in steps (and in words) how one finds absolute extrema for a two-variable
      function $f = f(x, y)$ on a closed bounded $D \subset \mathbb{R}^2$.
    \item Apply your procedure derived in (a) to find absolute extrema for
      $f(x, y) = 2x^3 + xy^2 + 5x^2 + y^2$
      over the rectangle $D := \{(x, y)\ | -2 \leq x \leq 3, 0 \leq y \leq 2\}$.
  \end{enumerate}
\end{problem}

\begin{proof}
  \begin{enumerate}
    \item The steps below outline the process to obtain the absolute extreme for
      a two-variable, continuous function $f = f(x, y)$ on a closed bounded $D \subset \mathbb{R}^2$.
      \begin{enumerate}[label=\Roman*.]
        \item First, identify the critical points of the function, i.e. find the points
          $(x_i, y_i)$ such that $$\bigtriangledown f(x_i, y_i) = \langle f_x(x_i, y_i), f_y(x_i, y_i) \rangle = \langle 0, 0 \rangle$$
          or such that $f_x(x_i, y_i)$ or $f_y(x_i, y_i)$ do not exist.
        \item Suppose that $S_f$ is the set of critical points obtained in step I. Then
          $P = S_f \cap D$ is the set of possible points at which
          the function $f$ obtains its absolute minimum and maximum on the closed bounded domain $D$.
        \item Note that our function satisfies
          the assumptions of The Extreme Value Theorem and as a result, using the set $P$ obtained in step II,
          $\max{f(P)}$ is the absolute maximum of the function $f$
          and $\min{f(P)}$ is the absolute minimum of the function $f$.
      \end{enumerate}
    \item Let $f(x, y) = 2x^3 + xy^2 + 5x^2 + y^2$ where $f: D = \{(x, y)\ | -2 \leq x \leq 3, 0 \leq y \leq 2\} \to \mathbb{R}^2$. Then
      \begin{align*}
        \bigtriangledown f(x, y) = \langle f_x(x, y), f_y(x, y) \rangle = \langle 2x(3x + 5) + y^2, 2y(x + 1)\rangle.
      \end{align*}
      Note that $f_y(x,y) = 0$ if $x = -1$ or $y = 0$ as the real numbers form a field and thus form an integral domain.
      Also note that $f_x(x, y) = 0$ if $x = -1$ and $y = \pm 2$ or $x = -5/3$ and $y = 0$ or $x = 0$ and $y = 0$. Thus,
      $\bigtriangledown f(x, y) = \langle 0, 0 \rangle$ if $(x, y) \in \{(-5/3, 0), (-1, -2), (-1, 2), (0, 0)\} = S_f$.  Since the partial derivatives
      of $f$ exist everywhere, the set $S_f$ contains every critical point of the function $f$.

      Now, $P = S_f \cap D = \{(-5/3, 0), (-1, 2), (0, 0)\}$ and $f(P) = \{125/27, 3, 0\}$. Therefore,
      the absolute maximum of $f$ is $\max{f(P)} = 125/27$ which occurs at the point $(-5/3, 0)$ and the absolute
      minimum of $f$ is $\min{f(P)} = 0$ which occurs at the point $(0, 0)$.
  \end{enumerate}
\end{proof}
\newpage


% Problem 3
\begin{problem}
  Consider the optimization problem:
  \begin{align*}
    \begin{array} {lcl}
      \text{Min (Max) } & f(x_1, x_2, \dots, x_n) & \\
      \text{subject to } & g_1(x_1, x_2, \dots, x_n) &= k_1 \\
      & g_2(x_1, x_2, \dots, x_n) &= k_2 \\
      & \vdots & \\
      & g_m(x_1, x_2, \dots, x_n) &= k_m \\
    \end{array}
  \end{align*}
  \begin{enumerate}
    \item Formulate the Lagrangean and describe how we should proceed in order
      to solve such a problem.
    \item Find the relative extrema of $f(x, y, z) = x + 2y + 3z$ subject to
      $x - y + z = 1, x^2 + y^2 = 1$.
  \end{enumerate}
\end{problem}

\begin{proof}
\end{proof}
\newpage


% Problem 4
\begin{problem}
  Solve the shipping problem studied in MATH 111 if we replace the constraint
  $x + 2y \leq 100$ by the constraint $x + 2y \leq 625/6$. Use Mathematica to
  (at least) graph the feasible set.
\end{problem}

\begin{proof}
\end{proof}
\newpage


% Problem 5
\begin{problem}
  Suppose that $f, f_1, f_2$ are convex functions and $a \geq 0$. Prove that
  $af$ and $f_1 + f_2$ are convex functions.
\end{problem}

\begin{proof}
\end{proof}
\newpage


% Problem 6
\begin{problem}
  For $f: \mathbb{R}^n \to \mathbb{R} \cup \{+\infty\}$ we define its \textit{epigraph}
  as the set
  \[
   \text{epi}\ f = \{(x, \beta) \in \mathbb{R}^n \times \mathbb{R} | f(x) \leq \beta \} \subset \mathbb{R}^{n+1}.
  \]
  Prove that $f$ is convex if and only if $\text{epi}\ f$ is convex.
\end{problem}

\begin{proof}
\end{proof}


\end{document}
