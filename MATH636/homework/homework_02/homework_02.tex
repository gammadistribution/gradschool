\documentclass[12pt]{article}
\usepackage[margin=1in]{geometry}
\usepackage{amsmath, amsthm, amssymb, amsfonts, enumitem, graphicx}
\usepackage{fancyhdr}

\theoremstyle{definition}
\newtheorem{problem}{Problem}
\renewcommand*{\proofname}{Solution}
\renewcommand{\theenumi}{\alph{enumi}}

\newcommand{\vect}[1]{\boldsymbol{#1}}


\pagestyle{fancy}
\fancyhf{}
\rhead{Homework Assignment 2}
\lhead{Matthew Tiger}
\cfoot{\thepage}


\title{Homework Assignment 2}
\author{Matthew Tiger}


\begin{document}


\maketitle


% Problem 1
\begin{problem}
  Convert the following linear programming problem to \textit{standard form}:
  \begin{align*}
    \begin{array}{rll}
      \text{maximize} & 2x_1 + x_2 &\\
      \text{subject to} & 0 \leq x_1 &\leq 2 \\
      & x_1 + x_2 &\leq 3 \\
      & x_1 + 2x_2 &\leq 5 \\
      & x_2 \geq 0 &
    \end{array}
  \end{align*}
\end{problem}

\begin{proof}
  In order to convert this linear programming problem into standard form, we
  must transform the objective from \textit{maximize} to \textit{minimize} and
  the constraints must be transformed from linear inequalities into linear equations.

  Our first step will be to rewrite the objective function as a minimization problem
  and write each constraint as a linear inequality as so:
  \begin{align*}
    \begin{array}{rll}
      \text{minimize} & -2x_1 - x_2 &\\
      \text{subject to} & x_1 &\leq 2 \\
      & x_1 + x_2 &\leq 3 \\
      & x_1 + 2x_2 &\leq 5 \\
      & x_1 \geq 0, x_2 \geq 0 &
    \end{array}
  \end{align*}

  We can then introduce three slack variables $x_3, x_4, x_5$ to turn the linear
  inequalities into linear equations:
  \begin{align*}
    \begin{array}{rllllll}
      \text{minimize} & -2x_1 &- x_2  & & & &\\
      % \text{subject to} & x_1 & &+ x_3 & & &= 2 \\
      \text{subject to} & x_1 &+ x_2 &+ x_3 & & &= 2 \\
      & x_1 &+ x_2 & &+ x_4 & &= 3 \\
      & x_1 &+ 2x_2 & & &+ x_5&= 5 \\
      & x_1 \geq 0, &x_2 \geq 0, &x_3 \geq 0, &x_4 \geq 0, &x_5 \geq 0
    \end{array}
  \end{align*}

  As the above linear programming problem is written as
  \begin{align*}
    \begin{array}{rl}
      \text{minimize} & \vect{c}^\intercal \vect{x}\\
      \text{subject to} & A\vect{x} = \vect{b} \\
      & \vect{x} \geq 0
    \end{array}
  \end{align*}
  where
  \begin{align*}
    \vect{c}^\intercal =
    \begin{bmatrix}
      -2 \\
      -1 \\
    \end{bmatrix}^\intercal,
    \quad
    A =
    \begin{bmatrix}
      1 & 1 & 1 & 0 & 0 \\
      1 & 1 & 0 & 1 & 0 \\
      1 & 2 & 0 & 0 & 1 \\
    \end{bmatrix},\quad
    \vect{x} =
    \begin{bmatrix}
      x_1 \\
      x_2 \\
      x_3 \\
      x_4 \\
      x_5 \\
    \end{bmatrix}, \quad
    \vect{b} =
    \begin{bmatrix}
      2 \\
      3 \\
      5 \\
    \end{bmatrix}
  \end{align*}
  the linear programming problem is in standard form and we are done.
\end{proof}
\newpage


% Problem 2
\begin{problem}
  Solve the system $Ax = b$ where
  \begin{align*}
    A =
    \begin{bmatrix}
      2 & -1 & 2 & -1 & 3 \\
      1 & 2 & 3 & 1 & 0 \\
      1 & 0 & -2 & 0 & -5 \\
    \end{bmatrix}
    , \quad
    b = \begin{bmatrix}
      14 \\
      5 \\
      -10 \\
    \end{bmatrix}.
  \end{align*}
  If possible, generate a non-basic feasible solution of the system from which
  you derive next a basic feasible one.
\end{problem}

\begin{proof}
\end{proof}
\newpage


% Problem 3
\begin{problem}
  Does every linear programming problem in standard form have a nonempty feasible set?
  If ``yes'', provide a proof. If ``no'', provide a counter-example.

  Does every linear programming problem in standard form (assuming a nonempty feasible
  set) have an optimal solution? If ``yes'', provide a proof. If ``no'', provide a counter-example.
\end{problem}

\begin{proof}
\end{proof}
\newpage


% Problem 4
\begin{problem}
  \begin{enumerate}
    \item Solve the following linear program graphically:
      \begin{align*}
        \begin{array}{ll}
          \text{maximize} & 2x_1 + 5x_2 \\
          \text{subject to} & 0 \leq x_1 \leq 4 \\
          & 0 \leq x_2 \leq 6 \\
          & x_1 + x_2 \leq 8
        \end{array}
      \end{align*}
    \item Solve the linear program in (b) the same way Example 15.15 was solved in class.
      Compute only the vertices that lead to the optimal vertex found at (a).
  \end{enumerate}
\end{problem}

\begin{proof}
\end{proof}

\end{document}
