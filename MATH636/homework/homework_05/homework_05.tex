\documentclass[12pt]{article}
\usepackage[margin=1in]{geometry}
\usepackage{amsmath, amsthm, amssymb, amsfonts, enumitem, graphicx}
\usepackage{fancyhdr}
\usepackage{tikz}
\usepackage{mathrsfs}
\newcommand*\circled[1]{\tikz[baseline=(char.base)]{
            \node[shape=circle,draw,inner sep=2pt] (char) {#1};}}

\theoremstyle{definition}
\newtheorem{problem}{Problem}
\renewcommand*{\proofname}{Solution}
\renewcommand{\theenumi}{\alph{enumi}}

\newcommand{\vc}[1]{\boldsymbol{#1}}

\newenvironment{amatrix}[1]{%
  \left[\begin{array}{@{}*{#1}{c}|c@{}}
}{%
  \end{array}\right]
}

\newcommand{\grstep}[2][\relax]{%
   \ensuremath{\mathrel{
       {\mathop{\longrightarrow}\limits^{#2\mathstrut}_{
                                     \begin{subarray}{l} #1 \end{subarray}}}}}}
\newcommand{\swap}{\leftrightarrow}
\newcommand{\tran}{\mathsf{T}}
\newtheorem{theorem}{Theorem}



\pagestyle{fancy}
\fancyhf{}
\rhead{Homework Assignment 5}
\lhead{Matthew Tiger}
\cfoot{\thepage}


\title{Homework Assignment 5}
\author{Matthew Tiger}


\begin{document}


\maketitle


% Problem 1
\begin{problem}
  \begin{enumerate}
    \item Explain in a specific example why, when $A$ and $\vc{b}$ have integer
      components, a general integer programming problem
      \begin{align*}
        \begin{array}{rlrl}
          (GILP) & \text{Minimize (Maximize)} & f(\vc{x}) = \vc{c}^\tran \vc{x} & \\
          & \text{subject to} & A\vc{x} &\leq (\geq, =) \vc{b} \\
          & & \vc{x} \geq (\leq) \vc{0}, \vc{x}\in\mathbb{Z}^n
        \end{array}
      \end{align*}
      can be reduced (or is equivalent) to a standard integer programming problem
      \begin{align*}
        \begin{array}{rlrl}
          (ILP) & \text{Minimize (Maximize)} & f(\vc{X}) = \vc{C}^\tran\vc{X}& \\
          & \text{subject to} & \mathscr{A}\vc{X} &= \vc{B} \\
          && \vc{X} \geq \vc{0}, \vc{X} \in \mathbb{Z}^n
        \end{array}
      \end{align*}
      by adding variables or any of the transformations discussed in class that change $\vc{x}$ into $\vc{X}$.

      More precisely, explain why (GILP) has a solution $\vc{x} \in \mathbb{Z}^n$ if and only if
      (ILP) has a solution $\vc{X} \in \mathbb{Z}^n$.
    \item How do we solve (GILP) when $A$ or $\vc{b}$ do not have integer components?
  \end{enumerate}
\end{problem}

\begin{proof}
\end{proof}
\newpage


% Problem 2
\begin{problem}
  Solve the shipping problem studied in MATH 111 with the replaced constraints over integers using the
  Gomory Cutting-Plane Method.

  More precisely, solve:
  \begin{align*}
    \begin{array}{lll}
      \text{Maximize} & 9x_1 + 13x_2 & \\
      \text{subject to} & 4x_1 + 3x_2 &\leq 300 \\
       & x_1 + 2x_2 &\leq 625/6 \\
       & -2x_1 + x_2 &\leq 0 \\
       & x_1, x_2 \geq 0 & \\
       & x_1, x_2 \in \mathbb{Z}&
    \end{array}.
  \end{align*}
\end{problem}

\begin{proof}
\end{proof}
\newpage


% Problem 3
\begin{problem}
  Let $f:\mathbb{R}^n \to \mathbb{R}^m$ and $\Omega \subset \mathbb{R}^n$ be an open subset.
  Explain the meaning of $f\in C^1(\Omega)$. More precisely, give all the definitions needed and present some examples
  and results concerning $C^1(\Omega)$ functions.
\end{problem}

\begin{proof}
\end{proof}
\newpage


\end{document}
