\documentclass[12pt]{article}
\usepackage[margin=1in]{geometry}
\usepackage{amsmath, amsthm, amssymb, amsfonts, enumitem, graphicx}
\usepackage{fancyhdr}
\usepackage{tikz}
\usepackage{mathrsfs}
\newcommand*\circled[1]{\tikz[baseline=(char.base)]{
            \node[shape=circle,draw,inner sep=2pt] (char) {#1};}}

\theoremstyle{definition}
\newtheorem{problem}{Problem}
\renewcommand*{\proofname}{Solution}
\renewcommand{\theenumi}{\alph{enumi}}

\newcommand{\vc}[1]{\boldsymbol{#1}}

\newenvironment{amatrix}[1]{%
  \left[\begin{array}{@{}*{#1}{c}|c@{}}
}{%
  \end{array}\right]
}

\newcommand{\grstep}[2][\relax]{%
   \ensuremath{\mathrel{
       {\mathop{\longrightarrow}\limits^{#2\mathstrut}_{
                                     \begin{subarray}{l} #1 \end{subarray}}}}}}
\newcommand{\swap}{\leftrightarrow}
\newcommand{\tran}{\mathsf{T}}
\newtheorem{theorem}{Theorem}



\pagestyle{fancy}
\fancyhf{}
\rhead{Homework Assignment 5}
\lhead{Matthew Tiger}
\cfoot{\thepage}


\title{Homework Assignment 5}
\author{Matthew Tiger}


\begin{document}


\maketitle


% Problem 1
\begin{problem}
  \begin{enumerate}
    \item Explain in a specific example why, when $A$ and $\vc{b}$ have integer
      components, a general integer programming problem
      \begin{align*}
        \begin{array}{rlrl}
          (\text{GILP}) & \text{Minimize (Maximize)} & f(\vc{x}) = \vc{c}^\tran \vc{x} & \\
          & \text{subject to} & A\vc{x} &\leq (\geq, =) \vc{b} \\
          & & \vc{x} \geq (\leq) \vc{0}, \vc{x}\in\mathbb{Z}^n
        \end{array}
      \end{align*}
      can be reduced (or is equivalent) to a standard integer programming problem
      \begin{align*}
        \begin{array}{rlrl}
          (\text{ILP}) & \text{Minimize (Maximize)} & f(\vc{X}) = \vc{C}^\tran\vc{X}& \\
          & \text{subject to} & \mathscr{A}\vc{X} &= \vc{B} \\
          && \vc{X} \geq \vc{0}, \vc{X} \in \mathbb{Z}^n
        \end{array}
      \end{align*}
      by adding variables or any of the transformations discussed in class that change $\vc{x}$ into $\vc{X}$.

      More precisely, explain why (GILP) has a solution $\vc{x} \in \mathbb{Z}^n$ if and only if
      (ILP) has a solution $\vc{X} \in \mathbb{Z}^n$.
    \item How do we solve (GILP) when $A$ or $\vc{b}$ do not have integer components?
  \end{enumerate}
\end{problem}

\begin{proof}
  \begin{enumerate}
    \item In general, the transformations needed to change the constraints in the (GILP) to the standard form
      (ILP) all involve operations on integers so that the requirement that the constraints all involve integer components is satisfied.
      As such, if either the (GILP) or (ILP) has a solution, then the other is guaranteed
      to have a feasible solution since the additional slack or excess variables all involve integer operations. Thus,
      one of the feasible solutions will be the optimum and the other problem is solved.

      To illustrate this idea, take the (GILP)
      \begin{align*}
        \begin{array}{rlrl}
          (\text{GILP}) & \text{Minimize} & f(\vc{x}) = \begin{bmatrix}-4 & -3\end{bmatrix}\begin{bmatrix}x_1 & x_2\end{bmatrix}^\tran& \\
          & \text{subject to} & \begin{bmatrix}2 & -1\\ -1 & 2\end{bmatrix}\begin{bmatrix}x_1\\x_2\\\end{bmatrix} &\leq \begin{bmatrix}5\\3\end{bmatrix} \\
          & & \vc{x}\geq \vc{0}, \vc{x}\in\mathbb{Z}^n
        \end{array}
      \end{align*}
      The solution to (GILP) is $\vc{x}^\tran = \begin{bmatrix}4&3\end{bmatrix}$.

      In order to change the problem (GILP) to standard form (ILP) we need to add slack variables to the matrix $A$. Doing so
      results in a matrix $\mathscr{A}$ that still has all integer components and $\vc{b}$ remains unchanged so that it too
      still has all integer components. To see this, in standard form the problem becomes
      \begin{align*}
        \begin{array}{rlrl}
          (\text{ILP}) & \text{Minimize} & f(\vc{x}) = \begin{bmatrix}-4 & -3 & 0 & 0\end{bmatrix}\begin{bmatrix}x_1 & x_2 & x_3 & x_4\end{bmatrix}^\tran& \\
          & \text{subject to} & \begin{bmatrix}2 & -1 & 1 & 0\\ -1 & 2 & 0 & 1\end{bmatrix}\begin{bmatrix}x_1\\x_2\\x_3\\x_4\end{bmatrix} &= \begin{bmatrix}5\\3\\0\\0\end{bmatrix} \\
          & & \vc{X}\geq \vc{0}, \vc{X}\in\mathbb{Z}^n
        \end{array}.
      \end{align*}
      the solution of which is $\vc{x}^\tran = \begin{bmatrix}4 & 3 & 0 & 0\end{bmatrix}$.
    \item When the matrix $A$ or the constraint column $\vc{b}$ do not have integer components, we can solve (GILP) by converting the problem to
      standard form and performing the simplex method in conjunction with the Gomory Cutting-Plan Method to obtain an integer solution.
  \end{enumerate}
\end{proof}
\newpage


% Problem 2
\begin{problem}
  Solve the shipping problem studied in MATH 111 with the replaced constraints over integers using the
  Gomory Cutting-Plane Method.

  More precisely, solve:
  \begin{align*}
    \begin{array}{lll}
      \text{Maximize} & 9x_1 + 13x_2 & \\
      \text{subject to} & 4x_1 + 3x_2 &\leq 300 \\
       & x_1 + 2x_2 &\leq 625/6 \\
       & -2x_1 + x_2 &\leq 0 \\
       & x_1, x_2 \geq 0 & \\
       & x_1, x_2 \in \mathbb{Z}&
    \end{array}.
  \end{align*}
\end{problem}

\begin{proof}
\end{proof}
\newpage


% Problem 3
\begin{problem}
  Let $f:\mathbb{R}^n \to \mathbb{R}^m$ and $\Omega \subset \mathbb{R}^n$ be an open subset.
  Explain the meaning of $f\in C^1(\Omega)$. More precisely, give all the definitions needed and present some examples
  and results concerning $C^1(\Omega)$ functions.
\end{problem}

\begin{proof}
  The set $\Omega$ is open in $\mathbb{R}^n$ if $\Omega$ is a union of open balls where an open ball of radius $r$ centered at $\vc{x_0}$ is defined
  to be the set $B_r(\vc{x_0}) = \{\vc{x}\in\mathbb{R}^n\ |\ ||\vc{x} - \vc{x_0}|| < r\}$ where $||\vc{x}-\vc{x_0}||$ can be taken to be the Euclidean distance.

  The space $C^1(\Omega)$ is the set of all continuous functions that have continuous partial
  derivatives over the open set $\Omega$. Thus, if $f\in C^1(\Omega)$, then the function $f:\Omega \subset \mathbb{R}^n \to \mathbb{R}^m$
  is necessarily continuous and all partial derivatives of $f$ exist and are continuous.

  Let us make this more precise.
  If $f$ is a function from the open set $\Omega$ to $\mathbb{R}^m$, then $f$ has $m$ component functions $f_1, \dots, f_m$,i.e.\
  $f(\vc{x}) = [f_1(\vc{x}),\dots,f_m(\vc{x})]$. We require first that $f$ be continuous. Thus, for every $\vc{x}, \vc{a} \in \Omega$,
  we require that
  \begin{align*}
    \lim_{\vc{x}\to\vc{a}}f(\vc{x}) = \lim_{\vc{x}\to\vc{a}}[f_1(\vc{x}), \dots, f_m(\vc{x})]= [f_1(\vc{a}),\dots,f_m(\vc{a})] = f(\vc{a}),
  \end{align*}
  i.e.\ for all $\epsilon > 0$, there exists a $\delta >0$ such that if $||\vc{x}-\vc{a}|| < \delta$, then $||f(\vc{x}) - f(\vc{a})|| < \epsilon$.
  We also require that all of the partial derivatives of $f$ to exist and be continuous. That is,
  for all $1 \leq i \leq n$and $1\leq k \leq m$ the partial derivatives $\partial f_k(\vc{x})/\partial x_i$ exist and
  \begin{align*}
    \lim_{\vc{x}\to\vc{a}}\frac{\partial f(\vc{x})}{\partial x_i}
    = \lim_{\vc{x}\to\vc{a}} \left[\frac{\partial f_1(\vc{x})}{\partial x_i},\dots,\frac{\partial f_m(\vc{x})}{\partial x_i}\right]
    = \left[\left.\frac{\partial f_1(\vc{x})}{\partial x_i}\right|_{\vc{x}=\vc{a}}  ,\dots, \left.\frac{\partial f_m(\vc{x})}{\partial x_i}\right|_{\vc{x}=\vc{a}}\right]
    = \left.\frac{\partial f(\vc{x})}{\partial x_i}\right|_{\vc{x}=\vc{a}}.
  \end{align*}

  An example of a function in this space, take for instance $f:\mathbb{R}^2 \to \mathbb{R}$
  where $f(x_1,x_2) = x_1 + x_2$. This function is clearly $C^1$ smooth over the open
  set $\Omega = \{\vc{x}\in\mathbb{R}^2\ |\ ||\vc{x}||<1\}$ since it is continuous over this set
  and its partial derivatives are both the constant 1, guaranteeing that its partial derivatives are continuous as well.

  One result concerning $C^1(\Omega)$ functions is that the set of all such functions forms a vector space.


\end{proof}
\newpage


\end{document}
