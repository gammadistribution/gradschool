\documentclass[12pt]{article}
\usepackage[margin=1in]{geometry}
\usepackage{amsmath, amsthm, amssymb, amsfonts, enumitem, graphicx}
\usepackage{fancyhdr}
\usepackage{tikz}
\newcommand*\circled[1]{\tikz[baseline=(char.base)]{
            \node[shape=circle,draw,inner sep=2pt] (char) {#1};}}

\theoremstyle{definition}
\newtheorem{problem}{Problem}
\renewcommand*{\proofname}{Solution}
\renewcommand{\theenumi}{\alph{enumi}}

\newcommand{\vc}[1]{\boldsymbol{#1}}

\newenvironment{amatrix}[1]{%
  \left[\begin{array}{@{}*{#1}{c}|c@{}}
}{%
  \end{array}\right]
}

\newcommand{\grstep}[2][\relax]{%
   \ensuremath{\mathrel{
       {\mathop{\longrightarrow}\limits^{#2\mathstrut}_{
                                     \begin{subarray}{l} #1 \end{subarray}}}}}}
\newcommand{\swap}{\leftrightarrow}
\newcommand{\tran}{\mathsf{T}}


\pagestyle{fancy}
\fancyhf{}
\rhead{Homework Assignment 4}
\lhead{Matthew Tiger}
\cfoot{\thepage}


\title{Homework Assignment 4}
\author{Matthew Tiger}


\begin{document}


\maketitle


% Problem 1
\begin{problem}
  Find the dual of the following linear programs via the symmetric form of duality:
  \begin{enumerate}
    \item Maximize $f(x) = c^\tran x$ subject to $Ax=b$.
    \item Maximize $2x_1 + 5x_2 + x_3$ subject to $\begin{cases}
        \ 2x_1 - x_2 + 7x_3 \leq 6 \\
        \ x_1 + 3x_2 + 4x_3 \leq 9 \\
        \ 3x_1 + 6x_2 + x_3 \leq 3 \\
        \ x_1, x_2, x_3 \geq 0.
      \end{cases}$
  \end{enumerate}
\end{problem}

\begin{proof}
\end{proof}
\newpage


% Problem 2
\begin{problem}
  \begin{enumerate}
    \item Prove (via the symmetric form of duality) that the dual of the dual problem in an asymmetric
      form of duality is the primal (standard) problem.
    \item Prove the weak duality proposition for the symmetric form of duality.
    \item Prove that the primal problem is infeasible if and only if the dual problem is unbounded.
  \end{enumerate}
\end{problem}

\begin{proof}
\end{proof}
\newpage


% Problem 3
\begin{problem}
  Prove the Duality Theorem for the symmetric case.
\end{problem}

\begin{proof}
\end{proof}
\newpage


% Problem 4
\begin{problem}
  Consider the following linear program:
  \begin{align*}
    \begin{array}{rrl}
      \text{maximize} & 2x_1 + 3x_2 & \\
      \text{subject to} & x_1 + 2x_2 &\leq 4 \\
      & 2x_1 + x_2 &\leq 5 \\
      & x_1, x_2 &\geq 0.
    \end{array}
  \end{align*}
  \begin{enumerate}
    \item Use the simplex method to solve the problem.
    \item Write down the dual of the linear program and solve the dual.
  \end{enumerate}
\end{problem}

\begin{proof}
\end{proof}
\newpage


% Problem 5
\begin{problem}
  Consider the following primal problem:
  \begin{align*}
    \begin{array}{rllllll}
      \text{maximize} & x_1 &+ 2x_2 & & & & \\
      \text{subject to} & -2x_1 &+ x_2 &+ x_3 & & &= 2 \\
      & -x_1 &+ 2x_2& &+ x_4 & &= 7 \\
      & x_1 &&& &+ x_5 &= 3 \\
      &x_i \geq 0 & i= 1,2,3,4,5.& & & \\
    \end{array}
    \begin{array}{ll}
    \end{array}
  \end{align*}
  \begin{enumerate}
    \item Construct the dual problem corresponding to the primal problem above.
    \item It is known that the solution to the primal above is $\vc{x}^{*} = [3,5,3,0,0]^\tran$.
      Find the solution to the dual.
  \end{enumerate}
\end{problem}

\begin{proof}
\end{proof}
\newpage


% Problem 6
\begin{problem}
  Let $A$ be a given matrix and $\vc{b}$ a given vector. We wish to prove the
  following result: There exists a vector $\vc{x}$ such that $A\vc{x} = \vc{b}$ and
  $\vc{x} \geq \vc{0}$ if and only if for any given vector $\vc{y}$ satisfying $A^\tran \vc{y} \leq \vc{0}$
  we have $\vc{b}^\tran \vc{y} \leq \vc{0}$. This result is known as \textit{Farkas's transposition theorem}.
  Our program is based on duality theory, consisting of the parts listed below.
  \begin{enumerate}
    \item Consider the primal linear program
      \begin{align*}
        \begin{array}{rl}
          \text{minimize} & \vc{0}^\tran \vc{x} \\
          \text{subject to} & A\vc{x} = \vc{b} \\
          & \vc{x} \geq \vc{0}.
        \end{array}
      \end{align*}
      Write down the dual of this problem using the notation $\vc{y}$ for the dual variable.
    \item Show that the feasible set of the dual problem is guaranteed to be nonempty.

      \textit{Hint:} Think about an obvious feasible point.
    \item Suppose that for any $\vc{y}$ satisfying $A^\tran \vc{y} \leq \vc{0}$,
      we have $\vc{b}^\tran \vc{y} \leq 0$. In this case what can you say about whether
      or not the dual has an optimal feasible solution.

      \textit{Hint:} Think about the obvious feasible point in part b.
    \item Suppose that for any $\vc{y}$ satisfying $A^\tran \vc{y} \leq \vc{0}$,
      we have $\vc{b}^\tran \vc{y} \leq 0$. Use parts b and c to show that there exists
      $\vc{x}$ such that $A\vc{x} = \vc{b}$ and $\vc{x} \geq \vc{0}$. (This proves one
      direction of Farkas's transposition theorem.)
    \item Suppose that $\vc{x}$ satisfies $A\vc{x} = \vc{b}$ and $\vc{x} \geq \vc{0}$.
      Let $\vc{y}$ be an arbitrary vector satisfying $A^\tran \vc{y} \leq \vc{0}$.
      (This proves the other direction of Farkas's transposition theorem.)
  \end{enumerate}
\end{problem}

\begin{proof}
\end{proof}
\newpage


\end{document}
