\documentclass[12pt]{article}
\usepackage[margin=1in]{geometry}
\usepackage{amsmath, amsthm, amssymb, amsfonts, enumitem, graphicx}
\usepackage{fancyhdr}
\usepackage{tikz}
\usepackage{mathrsfs}
\newcommand*\circled[1]{\tikz[baseline=(char.base)]{
            \node[shape=circle,draw,inner sep=2pt] (char) {#1};}}

\theoremstyle{definition}
\newtheorem{problem}{Problem}
\renewcommand*{\proofname}{Solution}
\renewcommand{\theenumi}{\alph{enumi}}

\newcommand{\vc}[1]{\boldsymbol{#1}}

\newenvironment{amatrix}[1]{%
  \left[\begin{array}{@{}*{#1}{c}|c@{}}
}{%
  \end{array}\right]
}

\newcommand{\grstep}[2][\relax]{%
   \ensuremath{\mathrel{
       {\mathop{\longrightarrow}\limits^{#2\mathstrut}_{
                                     \begin{subarray}{l} #1 \end{subarray}}}}}}
\newcommand{\swap}{\leftrightarrow}
\newcommand{\tran}{\mathsf{T}}
\newtheorem{theorem}{Theorem}



\pagestyle{fancy}
\fancyhf{}
\rhead{Homework Assignment 4}
\lhead{Matthew Tiger}
\cfoot{\thepage}


\title{Homework Assignment 4}
\author{Matthew Tiger}


\begin{document}


\maketitle


% Problem 1
\begin{problem}
  Find the dual of the following linear programs:
  \begin{enumerate}
    \item Maximize $f(\vc{x}) = \vc{c}^\tran \vc{x}$ subject to $A\vc{x}=\vc{b}$.
    \item Maximize $2x_1 + 5x_2 + x_3$ subject to $\begin{cases}
        \ 2x_1 - x_2 + 7x_3 \leq 6 \\
        \ x_1 + 3x_2 + 4x_3 \leq 9 \\
        \ 3x_1 + 6x_2 + x_3 \leq 3 \\
        \ x_1, x_2, x_3 \geq 0.
      \end{cases}$
      via the symmetric form of duality.
  \end{enumerate}
\end{problem}

\begin{proof}
  \begin{enumerate}
    \item Note that for this problem, the variable $\vc{x}$ is unconstrained in sign.
      After making the substitution $\vc{x} = \vc{x_1} - \vc{x_2}$ with $\vc{x_1}, \vc{x_2} \geq \vc{0}$,
      this problem in standard form is then stated as
      \begin{align*}
        \begin{array}{rl}
          \text{minimize} & -\vc{c}^\tran(\vc{x_1} - \vc{x_2}) \\
          \text{subject to} & A(\vc{x_1} - \vc{x_2}) = \vc{b} \\
          & \vc{x_1}, \vc{x_2} \geq \vc{0}.
        \end{array}
      \end{align*}
      The realization that the equality $A(\vc{x_1} - \vc{x_2}) = \vc{b}$ can
      be represented as the system of inequalities
      \begin{align*}
        A(\vc{x_1} - \vc{x_2}) &\geq \vc{b} \\
        -A(\vc{x_1} - \vc{x_2}) &\geq -\vc{b}
      \end{align*}
      yields that the standard form of the LP is equivalent to:
      \begin{align*}
        \begin{array}{rrl}
          \text{minimize} & -\vc{c}^\tran\vc{x_1} + \vc{c}^\tran \vc{x_2} &\\
          \text{subject to} & A\vc{x_1} -A\vc{x_2} &\geq \vc{b}\\
          & -A\vc{x_1} + A\vc{x_2} &\geq -\vc{b} \\
          & \vc{x_1}, \vc{x_2} \geq \vc{0}. &
        \end{array}
      \end{align*}
      But this can be stated as
      \begin{align*}
        \begin{array}{rl}
          \text{minimize} & \begin{bmatrix}-\vc{c} \\ \vc{c}\end{bmatrix}^\tran \begin{bmatrix}\vc{x_1} \\ \vc{x_2}\end{bmatrix} \\ \\
          \text{subject to} & \begin{bmatrix}A & -A \\ -A & A\end{bmatrix} \begin{bmatrix}\vc{x_1} \\ \vc{x_2}\end{bmatrix} \geq \begin{bmatrix}\vc{b} \\ -\vc{b}\end{bmatrix}\\ \\
          & \vc{x_1}, \vc{x_2} \geq \vc{0}
        \end{array}
      \end{align*}
      or, more succinctly,
      \begin{align}\label{primal_1}
        \begin{array}{rl}
          \text{minimize} & \vc{C}^\tran \vc{X} \\
          \text{subject to} & \mathscr{A}\vc{X} \geq \vc{B} \\
          & \vc{X} \geq \vc{0}
        \end{array}
      \end{align}
      where
      \begin{align}\label{primal_def}
        \mathscr{A} = \begin{bmatrix}A & -A \\ -A & A\end{bmatrix}, \quad \vc{C} = \begin{bmatrix}-\vc{c} \\ \vc{c}\end{bmatrix}, \quad
        \vc{X} = \begin{bmatrix}\vc{x_1} \\ \vc{x_2}\end{bmatrix}, \quad \vc{B} = \begin{bmatrix}\vc{b} \\ -\vc{b}\end{bmatrix}.
      \end{align}
      By definition, the dual of the primal problem \eqref{primal_1} is
      \begin{align}\label{dual}
        \begin{array}{rl}
          \text{maximize} & \vc{B}^\tran \vc{\Lambda}\\
          \text{subject to} & \mathscr{A}^\tran \vc{\Lambda} \leq \vc{C} \\
          & \vc{\Lambda}^\tran = \begin{bmatrix}\vc{\lambda_1}^\tran \vc{\lambda_2}^\tran\end{bmatrix} \geq \vc{0}^\tran.
        \end{array}
      \end{align}

      Using the corresponding definitions found in \eqref{primal_def}, we see that
      after some algebraic manipulation the dual problem \eqref{dual} can be written as
      \begin{align*}
        \begin{array}{rl}
          \text{maximize} & \vc{b}^\tran(\vc{\lambda_1} - \vc{\lambda_2}) \\
          \text{subject to} & A^\tran(\vc{\lambda_1} - \vc{\lambda_2}) \leq -\vc{c} \\
          & A^\tran(\vc{\lambda_1} - \vc{\lambda_1}) \geq -\vc{c} \\
          & \vc{\lambda_1}, \vc{\lambda_2} \geq \vc{0}.
        \end{array}
      \end{align*}
      Noting that the system of inequalities can be written as an equality and
      making the substitution $\vc{\lambda} = (\vc{\lambda_1} - \vc{\lambda_2})$
      where $\vc{\lambda}$ is free, we see that the dual of the problem
      \begin{align*}
        \begin{array}{rl}
          \text{maximize} & \vc{c}^\tran \vc{x} \\
          \text{subject to} & A\vc{x}=\vc{b}
        \end{array}
      \end{align*}
      is
      \begin{align*}
        \begin{array}{rl}
          \text{minimize} & -\vc{b}^\tran\vc{\lambda} \\
          \text{subject to} & A^\tran\vc{\lambda}= -\vc{c}.
        \end{array}
      \end{align*}
    \item Note that the linear program
      \begin{align}\label{orig_primal}
        \begin{array}{rl}
          \text{maximize} & 2x_1 + 5x_2 + x_3 \\
              \text{subject to} & 2x_1 - x_2 + 7x_3 \leq 6 \\
              & x_1 + 3x_2 + 4x_3 \leq 9 \\
              & 3x_1 + 6x_2 + x_3 \leq 3 \\
              & x_1, x_2, x_3 \geq 0.
        \end{array}
      \end{align}
      can be written as
      \begin{align*}
        \begin{array}{rl}
          \text{maximize} & \vc{c}^\tran \vc{x} \\
          \text{subject to} & A\vc{x} \leq \vc{b} \\
          & \vc{x} \geq 0
        \end{array}
      \end{align*}
      where
      \begin{align*}
        A = \begin{bmatrix}2 & -1 & 7 \\ 1 & 3 & 4 \\ 3 & 6 & 1 \\\end{bmatrix}, \quad \vc{c} = \begin{bmatrix}2 \\ 5 \\ 1\end{bmatrix}, \quad
        \vc{x} = \begin{bmatrix}x_1 \\ x_2 \\ x_3\end{bmatrix}, \quad \vc{b} = \begin{bmatrix}6 \\ 9 \\ 3\end{bmatrix}.
      \end{align*}

      Some algebraic manipulations allows us to write the above problem as
      \begin{align}\label{primal_2}
        \begin{array}{rl}
          \text{minimize} & -\vc{c}^\tran \vc{x} \\
          \text{subject to} & -A\vc{x} \geq -\vc{b} \\
          & \vc{x} \geq \vc{0}
        \end{array}
      \end{align}

      By definition, the symmetric dual to the primal problem \eqref{primal_2} is
      \begin{align*}
        \begin{array}{rl}
          \text{maximize} & -\vc{b}^\tran \vc{\lambda} \\
          \text{subject to} & -A^\tran\vc{\lambda} \leq -\vc{c} \\
          & \vc{\lambda} = [\lambda_1, \lambda_2, \lambda_3]^\tran \geq \vc{0}.
        \end{array}
      \end{align*}

      Therefore, the dual to the primal problem \eqref{primal_2} can be written as
      \begin{align*}
        \begin{array}{rrl}
          \text{maximize} & -6\lambda_1 - 9\lambda_2 - 3\lambda_3 &\\
              \text{subject to} & -2\lambda_1 - \lambda_2 - 3\lambda_3 &\leq -2 \\
              & \lambda_1 - 3 \lambda_2 -  6\lambda_3 &\leq -5 \\
              & -7\lambda_1 -4 \lambda_2 - \lambda_3 &\leq -1 \\
              & \lambda_1, \lambda_2, \lambda_3 \geq 0 &
        \end{array}
      \end{align*}

      and the dual to the original primal problem \eqref{orig_primal} is
      \begin{align*}
        \begin{array}{rrl}
          \text{minimize} & 6\lambda_1 + 9\lambda_2 + 3\lambda_3 &\\
              \text{subject to} & 2\lambda_1 + \lambda_2 + 3\lambda_3 &\geq 2 \\
              & - \lambda_1 + 3 \lambda_2 +  6\lambda_3 &\geq 5 \\
              & 7\lambda_1 +4 \lambda_2 + \lambda_3 &\geq 1 \\
              & \lambda_1, \lambda_2, \lambda_3 \geq 0. &
        \end{array}
      \end{align*}
  \end{enumerate}
\end{proof}
\newpage


% Problem 2
\begin{problem}
  \begin{enumerate}
    \item Prove (via the symmetric form of duality) that the dual of the dual problem in an asymmetric
      form of duality is the primal (standard) problem.
    \item Prove the weak duality proposition for the symmetric form of duality.
    \item Prove that the primal problem is infeasible if and only if the dual problem is unbounded.
  \end{enumerate}
\end{problem}

\begin{proof}
  \begin{enumerate}
    \item Suppose that we have the following primal problem in standard form
      \begin{align*}
        \begin{array}{lrl}
          (P_a) & \text{minimize} & \vc{c}^\tran\vc{x} \\
          & \text{subject to} & A\vc{x} = \vc{b} \\
          & & \vc{x} \geq \vc{0}.
        \end{array}
      \end{align*}
      The dual of this problem via an asymmetric form of duality is given by
      \begin{align*}
        \begin{array}{lrl}
          (D_a) & \text{maximize} & \vc{b}^\tran\vc{\lambda} \\
          & \text{subject to} & A^\tran\vc{\lambda} \leq \vc{c}.
        \end{array}
      \end{align*}
      where $\vc{\lambda}$ is free. Making the substitution $\vc{\lambda} = \vc{\lambda_1} - \vc{\lambda_2}$
      and performing some algebraic manipulation, we can transform this dual problem into the following
      equivalent form
      \begin{align*}
        \begin{array}{lrl}
          (D_a) & \text{minimize} & -\vc{b}^\tran(\vc{\lambda_1} - \vc{\lambda_2}) \\
          & \text{subject to} & -A^\tran(\vc{\lambda_1} - \vc{\lambda_2}) \geq -\vc{c} \\
          & & \vc{\lambda_1}, \vc{\lambda_2} \geq \vc{0}.
        \end{array}
      \end{align*}

      From some more algebraic manipulation, we rewrite the dual problem into the following form
      \begin{align}\label{dual}
        \begin{array}{lrl}
          (D_a) & \text{minimize} & \vc{B}^\tran\vc{\Lambda} \\
          & \text{subject to} & \mathscr{A}^\tran\vc{\Lambda} \geq -\vc{c} \\
          & & \vc{\Lambda} \geq \vc{0}
        \end{array}
      \end{align}
      where
      \begin{align*}
        \mathscr{A} = \begin{bmatrix}-A\\ A\end{bmatrix}, \quad \vc{\Lambda} = \begin{bmatrix}\vc{\lambda_1} \\ \vc{\lambda_2}\end{bmatrix},
        \quad \vc{B} = \begin{bmatrix}-\vc{b} \\ \vc{b}\end{bmatrix}.
      \end{align*}
      Therefore, by definition, the dual via the symmetric form of duality of the dual problem \eqref{dual} is
      \begin{align*}
        \begin{array}{lrl}
          (P_s) & \text{maximize} & -\vc{c}^\tran \vc{x} \\
          & \text{subject to} & \mathscr{A}\vc{x} \leq \vc{B}\\
          & & \vc{x}\geq \vc{0}
        \end{array}
      \end{align*}
      or equivalently
      \begin{align}\label{primal_symmetric}
        \begin{array}{lrl}
          (P_s) & \text{maximize} & -\vc{c}^\tran \vc{x} \\
          & \text{subject to} & -A\vc{x} \leq -\vc{b}\\
          & & A\vc{x} \leq \vc{b}\\
          & & \vc{x}\geq \vc{0}.
        \end{array}
      \end{align}
      By rewriting the objective from \textit{maximize} to \textit{minimize}
      and combining the inequalities into one equality in \eqref{primal_symmetric}, we see that the dual via the symmetric
      form of duality of the dual in \eqref{dual} is
      \begin{align*}
        \begin{array}{lrl}
          (P_s) & \text{minimize} & \vc{c}^\tran\vc{x} \\
          & \text{subject to} & A\vc{x} = \vc{b} \\
          & & \vc{x} \geq \vc{0}.
        \end{array}
      \end{align*}
      which is equivalent to the standard form of the primal problem $(P_a)$
      and we are done.
    \item Suppose that we have the following primal problem in symmetric form and dual problem obtained in a symmetric form:
      \begin{align*}
        \begin{array}{lrl}
          (P_s) & \text{minimize} & \vc{c}^\tran \vc{x} \\
          & \text{subject to} & A\vc{x} \geq \vc{b} \\
          & & \vc{x} \geq \vc{0}
        \end{array}
        \quad
        \implies
        \quad
        \begin{array}{lrl}
          (D_s) & \text{maximize} & \vc{\lambda}^\tran \vc{b} \\
          & \text{subject to} & \vc{\lambda}^\tran A \leq \vc{c}^\tran \\
          & & \vc{\lambda} \geq \vc{0}.
        \end{array}
      \end{align*}
      We wish to show that if $\vc{x}$ and $\vc{\lambda}$ are feasible solutions
      to the primal and dual problem, respectively, then $\vc{\lambda}^\tran \vc{b} \leq \vc{c}^\tran\vc{x}$.

      Since $\vc{x}$ is a feasible solution to the primal problem, we know that
      \begin{align*}
        \vc{\lambda}^\tran \vc{b}\leq \vc{\lambda}^\tran (A\vc{x}).
      \end{align*}
      We also know, for $\vc{x} \geq \vc{0}$, that since $\vc{\lambda}$ is a feasible
      solution to the dual problem,
      \begin{align*}
        \vc{\lambda}^\tran A \leq \vc{c}^\tran \implies \vc{\lambda}^\tran A\vc{x} \leq \vc{c}^\tran\vc{x}.
      \end{align*}
      Therefore, by combining the two obtained inequalities, we have that
      \begin{align*}
        \vc{\lambda}^\tran \vc{b}\leq \vc{\lambda}^\tran (A\vc{x}) \leq \vc{c}^\tran\vc{x}
      \end{align*}
      proving the weak duality proposition.
    \item We wish to show that the primal problem is infeasible if and only if the dual problem is unbounded.
      This is equivalent to showing that the dual problem is bounded if and only if the primal problem is feasible.
      The primal problem in symmetric form and the dual problem obtained in a symmetric form are presented below:
      \begin{align*}
        \begin{array}{lrl}
          (P_s) & \text{minimize} & \vc{c}^\tran \vc{x} \\
          & \text{subject to} & A\vc{x} \geq \vc{b} \\
          & & \vc{x} \geq \vc{0}
        \end{array}
        \quad
        \implies
        \quad
        \begin{array}{lrl}
          (D_s) & \text{maximize} & \vc{\lambda}^\tran \vc{b} \\
          & \text{subject to} & \vc{\lambda}^\tran A \leq \vc{c}^\tran \\
          & & \vc{\lambda} \geq \vc{0}.
        \end{array}
      \end{align*}
      Suppose first that the primal problem is feasible. Then there exists $\vc{x_0} \geq \vc{0}$ with $A\vc{x_0} \geq \vc{b}$.
      By the Weak Duality proposition, we know that for every feasible
      $\vc{\lambda}$ that $\vc{\lambda}^\tran \vc{b} \leq \vc{c}^\tran\vc{x_0}$. Thus,
      $\max\{\vc{\lambda}^\tran \vc{b} \ |\ \vc{\lambda}^\tran A \leq \vc{c}^\tran, \vc{\lambda}\geq \vc{0} \} \leq \vc{c}^\tran\vc{x_0}$
      and the dual problem is bounded above.

      We now wish to show that if the dual problem is bounded, then the primal problem is feasible.
      This is equivalent to showing that if the primal problem is infeasible, then the dual problem is unbounded.
  \end{enumerate}

\end{proof}
\newpage


% Problem 3
\begin{problem}
  Prove the Duality Theorem for the symmetric case.
\end{problem}

\begin{proof}
  The Duality Theorem is presented below:
  \begin{theorem}[\textbf{\emph{Duality Theorem}}] If the primal problem, in symmetric form,
    has an optimal solution, then so does the dual, and the optimal values of their
    respective objective functions are equal.
  \end{theorem}

  \emph{Proof.} We define the symmetric form for the primal problem below:
  \begin{align*}
    \begin{array}{lrl}
      (P_s) & \text{minimize} & \vc{c}^\tran \vc{x} \\
      & \text{subject to} & A\vc{x} \geq \vc{b} \\
      & & \vc{x} \geq \vc{0}
    \end{array}
  \end{align*}

  Note that we can transform the symmetric problem to an equivalent asymmetric form
  by adding excess variables in the constraints:
  \begin{align*}
    \begin{array}{lrl}
      (P_a) & \text{minimize} & \vc{C}^\tran \vc{X} \\
      & \text{subject to} & \mathscr{A}\vc{X} = \vc{b} \\
      & & \vc{X} \geq \vc{0}
    \end{array}
  \end{align*}
  where
  \begin{align*}
    \mathscr{A} = \begin{bmatrix}A^\tran \\ -I\end{bmatrix}^\tran, \quad \vc{X} = \begin{bmatrix}\vc{x} \\ \vc{y}\end{bmatrix},
    \quad \vc{C} = \begin{bmatrix}\vc{c} \\ \vc{0}\end{bmatrix}.
  \end{align*}

  We wish to show that $\vc{x}$ is optimal in $(P_s)$ if and only if $\vc{X} = \begin{bmatrix}\vc{x} \\ A\vc{x} - \vc{b}\end{bmatrix}$
  is optimal in $(P_a)$.

  Suppose first that $\vc{x}$ is optimal in $(P_s)$. We begin by showing that $\vc{X}$ is feasible, i.e.\ $\vc{X} \geq \vc{0}$. Since $\vc{x}$ is a feasible solution of $(P_s)$, we have that $\vc{x} \geq \vc{0}$ and since $\vc{x}$ satisfies $A\vc{x} \geq \vc{b}$, we have that $A\vc{x} - \vc{b} \geq \vc{0}$
  showing indeed that $\vc{X} \geq \vc{0}$. Further, $\vc{X}$ is a feasible solution since
  \begin{align*}
    \mathscr{A}\vc{X} = \begin{bmatrix}A & -I\end{bmatrix}\begin{bmatrix}\vc{x} \\ A\vc{x} - \vc{b}\end{bmatrix} = A\vc{x} -(A\vc{x} - \vc{b}) = \vc{b}.
  \end{align*}
  Lastly, since $\vc{x}$ is optimal in $(P_s)$, we know that $\vc{X}$ is optimal in $(P_a)$ since
  \begin{align}\label{same_obj}
    \vc{C}^\tran\vc{X} = \begin{bmatrix}\vc{c} \\ \vc{0}\end{bmatrix}^\tran\begin{bmatrix}\vc{x} \\ A\vc{x} - \vc{b}\end{bmatrix} = \vc{c}^\tran\vc{x},
  \end{align}
  i.e.\ the minimum for both problems is the same.

  The other direction is trivial since if $\vc{X}$ is an optimal solution to $(P_a)$, then $\vc{x}$ is clearly a feasible solution to $(P_s)$ since $\vc{X} \geq \vc{0}$ and from equation
  \eqref{same_obj}, the objectives are equal so the problems obtain their optimum at the same point $ \vc{x}$.

  Also note that the dual of $(P_s)$ is equivalent to the dual of $(P_a)$. Note that the dual of the primal in the symmetric form and asymmetric form are presented below:
  \begin{align*}
    \begin{array}{lrl}
      (D_s) & \text{maximize} & \vc{\lambda}^\tran \vc{b} \\
      & \text{subject to} & \vc{\lambda}^\tran A \leq \vc{c}^\tran \\
      & & \vc{\lambda} \geq \vc{0}.
    \end{array}
    \qquad
    \begin{array}{lrl}
      (D_a) & \text{maximize} & \vc{\lambda}^\tran \vc{b} \\
      & \text{subject to} & \vc{\lambda}^\tran \mathscr{A} \leq \vc{C}^\tran \\
      & &
    \end{array}
  \end{align*}

  To see that these two problems are equivalent, one merely needs to examine the constraint $\vc{\lambda}^\tran\mathscr{A} \leq \vc{C}^\tran$.
  Note that
  \begin{align*}
    \vc{\lambda}^\tran\mathscr{A} \leq \vc{C}^\tran \implies \begin{bmatrix}\vc{\lambda}^\tran A & -\vc{\lambda}^\tran\end{bmatrix} \leq \begin{bmatrix}\vc{c}^\tran & \vc{0}^\tran\end{bmatrix}
  \end{align*}
  from which we deduce the two inequalities $\vc{\lambda}^\tran A \leq \vc{c}^\tran$ and $\vc{\lambda} \geq \vc{0}$, precisely the inequalities found in $(D_s)$.


  As was shown in class the theorem's results hold in the asymmetric case. As was just shown, the symmetric case
  can be transformed to its equivalent asymmetric case and therefore the theorem's results hold as well in the symmetric case.
\end{proof}
\newpage


% Problem 4
\begin{problem}
  Consider the following linear program:
  \begin{align*}
    \begin{array}{rrl}
      \text{maximize} & 2x_1 + 3x_2 & \\
      \text{subject to} & x_1 + 2x_2 &\leq 4 \\
      & 2x_1 + x_2 &\leq 5 \\
      & x_1, x_2 &\geq 0.
    \end{array}
  \end{align*}
  \begin{enumerate}
    \item Use the simplex method to solve the problem.
    \item Write down the dual of the linear program and solve the dual.
  \end{enumerate}
\end{problem}

\begin{proof}
\end{proof}
\newpage


% Problem 5
\begin{problem}
  Consider the following primal problem:
  \begin{align*}
    \begin{array}{rllllll}
      \text{maximize} & x_1 &+ 2x_2 & & & & \\
      \text{subject to} & -2x_1 &+ x_2 &+ x_3 & & &= 2 \\
      & -x_1 &+ 2x_2& &+ x_4 & &= 7 \\
      & x_1 &&& &+ x_5 &= 3 \\
      &x_i \geq 0 & i= 1,2,3,4,5.& & & \\
    \end{array}
    \begin{array}{ll}
    \end{array}
  \end{align*}
  \begin{enumerate}
    \item Construct the dual problem corresponding to the primal problem above.
    \item It is known that the solution to the primal above is $\vc{x}^{*} = [3,5,3,0,0]^\tran$.
      Find the solution to the dual.
  \end{enumerate}
\end{problem}

\begin{proof}
\end{proof}
\newpage


% Problem 6
\begin{problem}
  Let $A$ be a given matrix and $\vc{b}$ a given vector. We wish to prove the
  following result: There exists a vector $\vc{x}$ such that $A\vc{x} = \vc{b}$ and
  $\vc{x} \geq \vc{0}$ if and only if for any given vector $\vc{y}$ satisfying $A^\tran \vc{y} \leq \vc{0}$
  we have $\vc{b}^\tran \vc{y} \leq \vc{0}$. This result is known as \textit{Farkas's transposition theorem}.
  Our program is based on duality theory, consisting of the parts listed below.
  \begin{enumerate}
    \item Consider the primal linear program
      \begin{align*}
        \begin{array}{rl}
          \text{minimize} & \vc{0}^\tran \vc{x} \\
          \text{subject to} & A\vc{x} = \vc{b} \\
          & \vc{x} \geq \vc{0}.
        \end{array}
      \end{align*}
      Write down the dual of this problem using the notation $\vc{y}$ for the dual variable.
    \item Show that the feasible set of the dual problem is guaranteed to be nonempty.

      \textit{Hint:} Think about an obvious feasible point.
    \item Suppose that for any $\vc{y}$ satisfying $A^\tran \vc{y} \leq \vc{0}$,
      we have $\vc{b}^\tran \vc{y} \leq 0$. In this case what can you say about whether
      or not the dual has an optimal feasible solution.

      \textit{Hint:} Think about the obvious feasible point in part b.
    \item Suppose that for any $\vc{y}$ satisfying $A^\tran \vc{y} \leq \vc{0}$,
      we have $\vc{b}^\tran \vc{y} \leq 0$. Use parts b and c to show that there exists
      $\vc{x}$ such that $A\vc{x} = \vc{b}$ and $\vc{x} \geq \vc{0}$. (This proves one
      direction of Farkas's transposition theorem.)
    \item Suppose that $\vc{x}$ satisfies $A\vc{x} = \vc{b}$ and $\vc{x} \geq \vc{0}$.
      Let $\vc{y}$ be an arbitrary vector satisfying $A^\tran \vc{y} \leq \vc{0}$.
      (This proves the other direction of Farkas's transposition theorem.)
  \end{enumerate}
\end{problem}

\begin{proof}
\end{proof}
\newpage


\end{document}
