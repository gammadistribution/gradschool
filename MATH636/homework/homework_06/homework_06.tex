\documentclass[12pt]{article}
\usepackage[margin=1in]{geometry}
\usepackage{amsmath, amsthm, amssymb, amsfonts, enumitem, graphicx}
\usepackage{fancyhdr}
\usepackage{tikz}
\usepackage{mathrsfs}
\usepackage{mathtools}
\newcommand*\circled[1]{\tikz[baseline=(char.base)]{
            \node[shape=circle,draw,inner sep=2pt] (char) {#1};}}

\theoremstyle{definition}
\newtheorem{problem}{Problem}
\renewcommand*{\proofname}{Solution}
\renewcommand{\theenumi}{\alph{enumi}}

\newcommand{\vc}[1]{\boldsymbol{#1}}

\newenvironment{amatrix}[1]{%
  \left[\begin{array}{@{}*{#1}{c}|c@{}}
}{%
  \end{array}\right]
}

\newcommand{\grstep}[2][\relax]{%
   \ensuremath{\mathrel{
       {\mathop{\longrightarrow}\limits^{#2\mathstrut}_{
                                     \begin{subarray}{l} #1 \end{subarray}}}}}}
\newcommand{\swap}{\leftrightarrow}
\newcommand{\tran}{\mathsf{T}}
\newtheorem{theorem}{Theorem}



\pagestyle{fancy}
\fancyhf{}
\rhead{Homework Assignment 6}
\lhead{Matthew Tiger}
\cfoot{\thepage}


\title{Homework Assignment 6}
\author{Matthew Tiger}


\begin{document}


\maketitle


% Problem 1
\begin{problem}
  \begin{enumerate}
    \item Where is the assumption ``$x^*$ is regular'' essential in the proof of the results of section: Lagrange Multipliers?
    \item In the example on page 49 (Example 20.8 in \textit{An Introduction to Optimization}) explain in what way is $(P_0)$ equivalent to $(P_1)$.
    \item State the SOSC Theorem on p.\ 51 (Theorem 20.5 p.\ 474 in the book) for $x^*$ a local maximizer.
  \end{enumerate}
\end{problem}

\begin{proof}
\end{proof}
\newpage


% Problem 2
\begin{problem}
  Find local extremizers for the following optimization problem:
  \begin{align*}
    \begin{array}{rl}
      \text{maximize} & x_1x_2 \\
      \text{subject to} & x_1^2 + 4x_2^2 = 1.
    \end{array}
  \end{align*}
\end{problem}

\begin{proof}
\end{proof}
\newpage


% Problem 3
\begin{problem}
  Consider the problem
  \begin{align*}
    \begin{array}{rll}
      \text{minimize} & 2x_1 + 3x_2 - 4, & x_1, x_2 \in \mathbb{R} \\
      \text{subject to} & x_1 x_2 = 6.&
    \end{array}
  \end{align*}

  \begin{enumerate}
    \item Use Lagrange's theorem to find all possible local minimizers and maximizers.
    \item Use the second-order sufficient conditions to specify which points are strict
      local minimizers and which are strict local maximizers.
    \item Are the points in part b global minimizers or maximizers? Explain.
  \end{enumerate}
\end{problem}

\begin{proof}
\end{proof}
\newpage


% Problem 4
\begin{problem}
  Consider the problem of minimizing a general quadratic function subject to a linear constraint:
  \begin{align*}
    \begin{array}{rl}
      \text{minimize} & \frac{1}{2} \vc{x}^\tran Q \vc{x} - \vc{c}^\tran \vc{x} + d\\
      \text{subject to} & A\vc{x} = \vc{b},
    \end{array}
  \end{align*}
  where $Q=Q^\tran > 0$, $A \in\mathbb{R}^{m \times n}$ with $m<n$, $\text{rank} A = m$ and $d$ a constant.
  Derive a closed form solution to the problem.
\end{problem}

\begin{proof}
\end{proof}
\newpage


% Problem 5
\begin{problem}
  Consider the discrete-time linear system $x_k = 2 x_{k-1} + u_k$, $k \geq 1$, with
  $x_0 = 1$. Find the values of the control inputs $u_1$ and $u_2$ to minimize
  $$x_2^2 + \frac{1}{2}u_1^2+ \frac{1}{3}u_2^2.$$
\end{problem}

\begin{proof}
\end{proof}


\end{document}
