\documentclass[12pt]{article}
\usepackage[margin=1in]{geometry}
\usepackage{amsmath, amsthm, amssymb, amsfonts, enumitem, graphicx}
\usepackage{fancyhdr}
\usepackage{tikz}
\newcommand*\circled[1]{\tikz[baseline=(char.base)]{
            \node[shape=circle,draw,inner sep=2pt] (char) {#1};}}

\theoremstyle{definition}
\newtheorem{problem}{Problem}
\renewcommand*{\proofname}{Solution}
\renewcommand{\theenumi}{\alph{enumi}}

\newcommand{\vect}[1]{\boldsymbol{#1}}

\newenvironment{amatrix}[1]{%
  \left[\begin{array}{@{}*{#1}{c}|c@{}}
}{%
  \end{array}\right]
}

\newcommand{\grstep}[2][\relax]{%
   \ensuremath{\mathrel{
       {\mathop{\longrightarrow}\limits^{#2\mathstrut}_{
                                     \begin{subarray}{l} #1 \end{subarray}}}}}}
\newcommand{\swap}{\leftrightarrow}
\newcommand{\tran}{\mathsf{T}}


\pagestyle{fancy}
\fancyhf{}
\rhead{Homework Assignment 3}
\lhead{Matthew Tiger}
\cfoot{\thepage}


\title{Homework Assignment 3}
\author{Matthew Tiger}


\begin{document}


\maketitle


% Problem 1
\begin{problem}
  Solve the following linear program using the Simplex Algorithm in conjunction
  with Bland's rule:
  \begin{align*}
    \begin{array}{rrl}
      \text{maximize} & 2x_1 + 5x_2 &\\
      \text{subject to} & x_1 &\leq 4 \\
      & x_2 &\leq 6 \\
      & x_1 + x_2 &\leq 8 \\
      & x_1, x_2 &\geq 0.
    \end{array}
  \end{align*}
\end{problem}

\begin{proof}
  To start, we must transform this LP into standard form. This is achieved by
  changing the objective from \textit{maximize} to \textit{minimize} and
  adding three slack variables. In standard form,
  the problem becomes
  \begin{align*}
    \begin{array}{rrl}
      \text{minimize} & -2x_1 -5x_2&\\
      \text{subject to} & x_1 + x_3 &= 4 \\
      & x_2 + x_4 &= 6 \\
      & x_1 + x_2 + x_5 &= 8 \\
      & x_1, x_2, x_3, x_4, x_5 &\geq 0.
    \end{array}
  \end{align*}
  The initial tableau associated to this problem is then:
  \begin{align*}
    \begin{matrix}
      & \vect{a_1} & \vect{a_2} & \vect{a_3} & \vect{a_4} & \vect{a_5} & \vect{b} \\
      & 1 & 0 & 1 & 0 & 0 & 4 \\
      & 0 & 1 & 0 & 1 & 0 & 6 \\
      & 1 & 1 & 0 & 0 & 1 & 8 \\
      \vect{c}^\tran & -2 & -5  & 0 & 0 & 0 & 0 \\
    \end{matrix}
  \end{align*}
  Note that this tableau is in canonical form with respect to the basis
  $[\vect{a_3}, \vect{a_4}, \vect{a_5}]$. Thus, the last row of the tableau
  contains the reduced cost coefficients. Bland's rule prescribes how to choose
  the column-index $q$ and the row-index $p$ to pivot around. According to
  Bland's rule, choose
  \begin{align*}
    q &= \min\{i\ |\ r_i < 0  \}\\
    p &= \min\{j\ |\ y_{j0}/y_{jq} = \min_i\{y_{i0}/y_{iq}\ |\ y_{iq} > 0\} \}.
  \end{align*}
  Thus, we proceed by choosing the column-index to pivot around to be the smallest
  index pertaining to negative reduced cost coefficients in the bottom vector of the
  tableau and by then choosing the row-index to pivot around to be the
  index pertaining to the row with the lowest ratio between the right hand side and
  the positive coefficients of the $q$-th column in matrix $A$ of the tableau.
  If there are two such row-indexes, choose the smaller one.

  From the initial tableau, Bland's rule prescribes that we pivot around column
  $q=1$ since this is the smallest index with a negative reduced cost coefficient.
  The smallest ratio between the right hand side and the positive coefficients
  of the $q$-th column in matrix $A$ is $4/1$ so we pivot around row $p =1$. Thus, $\vect{a_1}$
  enters the basis, $\vect{a_3}$ leaves the basis, and we move from the initial tableau to the updated tableau:
  \begin{align*}
    \begin{matrix}
      & \vect{a_1} & \vect{a_2} & \vect{a_3} & \vect{a_4} & \vect{a_5} & \vect{b} \\
      & \circled{1} & 0 & 1 & 0 & 0 & 4 \\
      & 0 & 1 & 0 & 1 & 0 & 6 \\
      & 1 & 1 & 0 & 0 & 1 & 8 \\
      \vect{c}^\tran & -2 & -5  & 0 & 0 & 0 & 0 \\
      & \uparrow & & & & & \\
    \end{matrix}
    \quad
    \begin{array}{c}
    \grstep[{[4] + 2[1]}]{[3] - [1]} \\
    \end{array}
    \quad
    \begin{matrix}
      & \vect{a_1} & \vect{a_2} & \vect{a_3} & \vect{a_4} & \vect{a_5} & \vect{b} \\
                       & 1 & 0 & 1 & 0 & 0 & 4 \\
                       & 0 & 1 & 0 & 1 & 0 & 6 \\
                       & 0 & 1 & -1 & 0 & 1 & 4 \\
      \vect{c}^\tran & 0 & -5 & 2 & 0 & 0 & 8 \\
      & & & & & & \\
    \end{matrix}
  \end{align*}

  From this newly derived tableau, we notice that the only negative reduced cost coefficient
  occurs in column $q=2$. Further, the smallest ratio between the right hand side
  and the positive coefficients of the $q$-th column in matrix $A$ is $4/1$ so we pivot
  around row $p=3$. Thus, $\vect{a_2}$ enters the basis, $\vect{a_5}$ leaves the basis, and we move from this tableau to the updated tableau:
  \begin{align*}
    \begin{matrix}
      & \vect{a_1} & \vect{a_2} & \vect{a_3} & \vect{a_4} & \vect{a_5} & \vect{b} \\
                       & 1 & 0 & 1 & 0 & 0 & 4 \\
                       & 0 & 1 & 0 & 1 & 0 & 6 \\
                       & 0 & \circled{1} & -1 & 0 & 1 & 4 \\
      \vect{c}^\tran & 0 & -5 & 2 & 0 & 0 & 8 \\
      & & \uparrow & & & & \\
    \end{matrix}
    \quad
    \begin{array}{c}
    \grstep[{[4] + 5[3]}]{[2] - [3]} \\
    \end{array}
    \quad
    \begin{matrix}
      & \vect{a_1} & \vect{a_2} & \vect{a_3} & \vect{a_4} & \vect{a_5} & \vect{b} \\
                       & 1 & 0 & 1 & 0 & 0 & 4 \\
                       & 0 & 0 & 1 & 1 & -1 & 2 \\
                       & 0 & 1 & -1 & 0 & 1 & 4 \\
      \vect{c}^\tran & 0 & 0 & -3 & 0 & 5 & 28 \\
      & & & & & & \\
    \end{matrix}
  \end{align*}

  From this newly derived tableau, we notice that the only negative reduced cost coefficient
  occurs in column $q=3$. Further, the smallest ratio between the right hand side
  and the positive coefficients of the $q$-th column in matrix $A$ is $2/1$ so we pivot
  around row $p=2$. Thus, $\vect{a_3}$ enters the basis, $\vect{a_4}$ leaves the basis, and we move from this tableau to the updated tableau:
  \begin{align*}
    \begin{matrix}
      & \vect{a_1} & \vect{a_2} & \vect{a_3} & \vect{a_4} & \vect{a_5} & \vect{b} \\
                       & 1 & 0 & 1 & 0 & 0 & 4 \\
                       & 0 & 0 & \circled{1} & 1 & -1 & 2 \\
                       & 0 & 1 & -1 & 0 & 1 & 4 \\
      \vect{c}^\tran & 0 & 0 & -3 & 0 & 5 & 28 \\
      & & & \uparrow & & & \\
    \end{matrix}
    \quad
    \begin{array}{c}
    \grstep[{[3] + [2]}]{[1] - [2]} \\
    \grstep[{[4] + 3[2]}]{} \\
    \end{array}
    \quad
    \begin{matrix}
      & \vect{a_1} & \vect{a_2} & \vect{a_3} & \vect{a_4} & \vect{a_5} & \vect{b} \\
                       & 1 & 0 & 0 & -1 & 1 & 2 \\
                       & 0 & 0 & 1 & 1 & -1 & 2 \\
                       & 0 & 1 & 0 & 1 & 0 & 6 \\
      \vect{c}^\tran & 0 & 0 & 0 & 3 & 2 & 34 \\
      & & & & & & \\
    \end{matrix}
  \end{align*}

  In the final tableau we have no negative reduced cost coefficients. Therefore,
  the current basic feasible solution $\vect{x} = [2, 6, 2, 0, 0]^\tran$ of the LP in
  standard form is optimal with corresponding
  objective function value $-34$. The solution to the original problem is then
  $x_1 = 2$, $x_2 = 6$ with corresponding objective value 34.
\end{proof}
\newpage


% Problem 2
\begin{problem}
  \begin{enumerate}
    \item Prove that if (ALP) has a feasible solution $[x_1, \dots, x_n, y_1, \dots, y_m]^\tran$
      with objective function value zero then $y_1 = 0, \dots, y_m = 0$.
    \item What do you do if after Phase I (ALP) does not have any optimal feasible
      solution with objective function value zero?
  \end{enumerate}
\end{problem}

\begin{proof}
  Suppose we have the following LP in standard form:
  \begin{align*}
    \begin{array}{rl}
      \text{minimize} & \vect{c}^\tran \vect{x}\\
      \text{subject to} & A\vect{x} = \vect{b}\\
      & \vect{x} \geq 0.
    \end{array}
  \end{align*}
  The \textit{artificial problem} (ALP) associated to this problem is stated as
  \begin{align}\label{alp}
    \begin{array}{rl}
      \text{minimize} & y_1 + y_2 + \dots + y_m \\
      \text{subject to} & \begin{bmatrix}A, I_m\end{bmatrix} \begin{bmatrix}\vect{x} \\ \vect{y}\end{bmatrix} = \vect{b} \\
      & \begin{bmatrix} \vect{x} \\ \vect{y} \end{bmatrix} \geq \vect{0},\\
    \end{array}
  \end{align}
  where $\vect{y} = [y_1, \dots, y_m]^\tran$.
  \begin{enumerate}
    \item Suppose that the (ALP) has the following feasible solution
      $\begin{bmatrix}\vect{x}_0 \\ \vect{y}_0\end{bmatrix} = [x_1, \dots, x_n, y_1, \dots, y_m]^\tran$
      with corresponding objective function value zero. Then
      $\begin{bmatrix}\vect{x}_0 \\ \vect{y}_0\end{bmatrix}$ is a solution of
      \eqref{alp}, i.e.\ $A\vect{x}_0 + I_m\vect{y}_0 = \vect{b}$ where
      $\vect{x}_0 \geq \vect{0}$ and $\vect{y}_0 \geq \vect{0}$.
      As $\vect{y}_0 = [y_1, y_2, \dots, y_m]^\tran \geq \vect{0}$, every
      component of $\vect{y}_0$ is non-negative, i.e.\ $y_i \geq 0$ for
      $i=1,\dots,m$. Since the corresponding objective function value of this
      solution is zero, we know that
      \begin{align*}
        y_1 + y_2 + \dots + y_m = 0.
      \end{align*}
      Note that if $y_i \geq 0$ for $i = 1, \dots, m$ the sum $y_1 + y_2 + \dots + y_m = 0$
      if and only each $y_i = 0$. Therefore, if
      $\begin{bmatrix}\vect{x}_0 \\ \vect{y}_0\end{bmatrix} = [x_1, \dots, x_n, y_1, \dots, y_m]^\tran$
      is a feasible solution of the (ALP) with objective function value zero
      then $y_1 = 0, \dots, y_m = 0$.
    \item Proposition 16.1 found on p.\ 362 of \textit{An Introduction to Optimization}
      states that the (ALP) has an optimal feasible solution with objective
      function value zero if and only if the original LP problem has a basic feasible solution.
      Thus, if the Phase I (ALP) does not have any optimal feasible solution with objective value zero,
      then the original LP does not have a basic feasible solution.

      The Fundamental Theorem of LP states that if there exists a feasible solution
      of the LP, then there exists a basic feasible solution. Thus, if no basic
      feasible solution of the LP exists, then no feasible solution of the LP exists.

      Therefore, if the Phase I (ALP) does not have any optimal feasible solution
      with objective value zero, then the original LP does not have a basic feasible
      solution and consequently the original LP has no feasible solution, i.e.\ the original LP
      is infeasible.
  \end{enumerate}
\end{proof}
\newpage


% Problem 3
\begin{problem}
  Consider the linear program
  \begin{align*}
    \begin{array}{rrl}
      \text{maximize} & 2x_1 + x_2 &\\
      \text{subject to} & 0 \leq x_1 &\leq 5 \\
      & 0 \leq x_2 &\leq 7 \\
      & x_1 + x_2 &\leq 9.
    \end{array}
  \end{align*}
  Convert the problem to standard form and solve it using the simplex method.
\end{problem}

\begin{proof}
  The linear program (LP) above can be converted to standard form by changing the
  objective from \textit{maximize} to \textit{minimize} and introducing
  slack variables. The LP in standard form is presented below
  \begin{align*}
    \begin{array}{rrl}
      \text{minimize} & -2x_1 - x_2 & \\
      \text{subject to} & x_1 + x_3 &= 5 \\
      & x_2 + x_4 &= 7 \\
      & x_1 + x_2 + x_5 &= 9 \\
      & x_1, x_2, x_3, x_4, x_5 &\geq 0.  \\
    \end{array}
  \end{align*}
  with associated matrix form:
  \begin{align*}
    \begin{array}{rrl}
      \text{minimize} & \vect{c}^\tran\vect{x} = [-2, -1, 0, 0, 0][x_1, x_2, x_3, x_4, x_5]^\tran & \\
      \text{subject to} & A\vect{x} =
      \begin{bmatrix}
        1 & 0 & 1 & 0 & 0 \\
        0 & 1 & 0 & 1 & 0 \\
        1 & 1 & 0 & 0 & 1 \\
      \end{bmatrix}
      \begin{bmatrix}
        x_1 \\
        x_2 \\
        x_3 \\
        x_4 \\
        x_5\\
      \end{bmatrix}
      =
      \begin{bmatrix}
        5 \\
        7 \\
        9 \\
      \end{bmatrix}
      &= \vect{b}
      \\
      &\vect{x} = [x_1, x_2, x_3, x_4, x_5]^\tran &\geq \vect{0}.\\
    \end{array}
  \end{align*}
  The initial tableau associated to this problem is then
  \begin{align*}
    \begin{matrix}
      & \vect{a_1} & \vect{a_2} & \vect{a_3} & \vect{a_4} & \vect{a_5} & \vect{b} \\
      & 1 & 0 & 1 & 0 & 0 & 5 \\
      & 0 & 1 & 0 & 1 & 0 & 7 \\
      & 1 & 1 & 0 & 0 & 1 & 9 \\
      \vect{c}^\tran & -2 & -1  & 0 & 0 & 0 & 0 \\
    \end{matrix}
  \end{align*}
  which is in canonical form with respect to the basis $[\vect{a_3}, \vect{a_4}, \vect{a_5}]$.

  Following the Simplex Algorithm, we must find the $q$-th column and $p$-th row to pivot around
  to move to the next tableau. To find the $q$-th column to pivot around, we find the column associated with
  the most negative reduced cost coefficient. We then find the $p$-th row to pivot
  around by finding the row with the lowest ratio between the right hand side and
  the positive coefficients of the $q$-th column in matrix $A$ of the tableau.
  If there are no positive entries in the $q-th$ column, then the problem is
  unbounded and we can stop. Otherwise, we then pivot
  around the entry $(p, q)$ and continue the above outlined process until all of the reduced
  cost coefficients found in the bottom row of the tableau are positive.

  From the initial tableau, we pivot around column
  $q=1$ since this is the index with the most negative reduced cost coefficient.
  The smallest ratio between the right hand side and the positive coefficients
  of the $q$-th column in matrix $A$ is $5/1$ so we pivot around row $p=1$. Thus, $\vect{a_1}$
  enters the basis, $\vect{a_3}$ leaves the basis, and we move from the initial tableau to the updated tableau:
  \begin{align*}
    \begin{matrix}
      & \vect{a_1} & \vect{a_2} & \vect{a_3} & \vect{a_4} & \vect{a_5} & \vect{b} \\
      & \circled{1} & 0 & 1 & 0 & 0 & 5 \\
      & 0 & 1 & 0 & 1 & 0 & 7 \\
      & 1 & 1 & 0 & 0 & 1 & 9 \\
      \vect{c}^\tran & -2 & -1 & 0 & 0 & 0 & 0. \\
      & \uparrow & & & & & \\
    \end{matrix}
    \quad
    \begin{array}{c}
    \grstep[{[4] + 2[1]}]{[3] - [1]} \\
    \end{array}
    \quad
    \begin{matrix}
      & \vect{a_1} & \vect{a_2} & \vect{a_3} & \vect{a_4} & \vect{a_5} & \vect{b} \\
                       & 1 & 0 & 1 & 0 & 0 & 5 \\
                       & 0 & 1 & 0 & 1 & 0 & 7 \\
                       & 0 & 1 & -1 & 0 & 1 & 4 \\
      \vect{c}^\tran & 0 & -1 & 2 & 0 & 0 & 10 \\
      & & & & & & \\
    \end{matrix}
  \end{align*}

  From this newly derived tableau, we notice that the only negative reduced cost coefficient
  occurs in column $q=2$. Further, the smallest ratio between the right hand side
  and the positive coefficients of the $q$-th column in matrix $A$ is $4/1$ so we pivot
  around row $p=3$. Thus, $\vect{a_2}$ enters the basis, $\vect{a_5}$ leaves the basis,
  and we move from this tableau to the updated tableau:
  \begin{align*}
    \begin{matrix}
      & \vect{a_1} & \vect{a_2} & \vect{a_3} & \vect{a_4} & \vect{a_5} & \vect{b} \\
                       & 1 & 0 & 1 & 0 & 0 & 5 \\
                       & 0 & 1 & 0 & 1 & 0 & 7 \\
                       & 0 & \circled{1} & -1 & 0 & 1 & 4 \\
      \vect{c}^\tran & 0 & -1 & 2 & 0 & 0 & 10 \\
      & & \uparrow & & & & \\
    \end{matrix}
    \quad
    \begin{array}{c}
    \grstep[{[4] + [3]}]{[2] - [3]} \\
    \end{array}
    \quad
    \begin{matrix}
      & \vect{a_1} & \vect{a_2} & \vect{a_3} & \vect{a_4} & \vect{a_5} & \vect{b} \\
                       & 1 & 0 & 1 & 0 & 0 & 5 \\
                       & 0 & 0 & 1 & 1 & -1 & 3 \\
                       & 0 & 1 & -1 & 0 & 1 & 4 \\
      \vect{c}^\tran & 0 & 0 & 1 & 0 & 1 & 14 \\
      & & & & & & \\
    \end{matrix}
  \end{align*}

  Note that there are no negative reduced cost coefficients in this tableau so the current solution
  is optimal, i.e.\ $\vect{x} = [5, 4, 0, 3, 0]^\tran$ is optimal with objective function
  value $-14$. Thus, the solution to the original problem is $x_1 = 5$ and $x_2 = 4$
  with objective function value 14.
\end{proof}
\newpage


% Problem 4
\begin{problem}
  Solve the following linear programs using the revised simplex method:
  \begin{enumerate}
    \item
      \begin{align*}
        \begin{array}{rrl}
          \text{maximize} & -4x_1 -3x_2 &\\
          \text{subject to} &5x_1 + x_2 &\geq 11 \\
          & -2x_1 - x_2 &\leq -8 \\
          & x_1 + 2x_2 &\geq 7 \\
          & x_1, x_2 &\geq 0.
        \end{array}
      \end{align*}
    \item
      \begin{align*}
        \begin{array}{rrl}
          \text{maximize} & 6x_1 + 4x_2 + 7x_3 + 5x_4  &\\
          \text{subject to}
          & x_1 + 2x_2 + x_3 + 2x_4 &\leq 20 \\
          & 6x_1 + 5x_2 + 3x_3 + 2x_4 &\leq 100 \\
          & 3x_1 + 4x_2 + 9x_3 + 12x_4 &\leq 75 \\
          & x_1, x_2, x_3, x_4 &\geq 0.
        \end{array}
      \end{align*}
  \end{enumerate}
\end{problem}

\begin{proof}
  \begin{enumerate}
    \item In order to solve the linear program using the revised simplex method,
      the problem must be converted to standard form by changing the objective
      from \textit{maximize} to \textit{minimize}, multiplying inequalities to ensure
      the entries on the right hand side are positive, and adding excess
      variables. The problem in standard form is thus
      \begin{align*}
        \begin{array}{rll}
          \text{minimize} & 4x_1 +3x_2 &\\
          \text{subject to} &5x_1 + x_2 - x_3 &= 11 \\
          & 2x_1 + x_2 - x_4 &= 8 \\
          & x_1 + 2x_2 - x_5 &= 7 \\
          & x_1, x_2, x_3, x_4, x_5 &\geq 0.
        \end{array}
      \end{align*}
      There is no obvious basic feasible solution to the problem in standard form so
      we use the two-phase approach.

      \textbf{\textit{Phase I.}} In order to find a basic feasible solution, we introduce
      three artificial variables and the artificial objective function
      $\text{minimize}\ x_6 + x_7 + x_8$. The initial tableau is then:
      \begin{align*}
        \begin{matrix}
          & \vect{a_1} & \vect{a_2} & \vect{a_3} & \vect{a_4} & \vect{a_5} & \vect{a_6} & \vect{a_7} & \vect{a_8} & \vect{b} \\
                         & 5 & 1 & -1 & 0 & 0 & 1 & 0 & 0 & 11 \\
                         & 2 & 1 & 0 & -1 & 0 & 0 & 1 & 0 & 8 \\
                         & 1 & 2 & 0 & 0 & -1 & 0 & 0 & 1 & 7 \\
          \vect{c}^\tran  & 0 & 0 & 0 & 0 & 0 & 1 & 1 & 1 & 0 \\
        \end{matrix}
      \end{align*}

      Note that this tableau does not need to be in canonical form to use the revised
      simplex method. We start with an initial basic feasible solution to this new problem
      along with its associated $B^{-1}$ and write this information in the following tableau:
      \begin{align*}
        \begin{matrix}
          \text{variable} & B^{-1} & & & \vect{y_0} \\
          \hline \\
          x_6 & 1 & 0 & 0 & 11 \\
          x_7 & 0 & 1 & 0 & 8  \\
          x_8 & 0 & 0 & 1 & 7  \\
        \end{matrix}
      \end{align*}
      The next step is to compute the reduced cost coefficients:
      \begin{align*}
        \vect{r}_D^\tran &= \vect{c}_D^\tran - \vect{c}_B^\tran B^{-1} D \\
        &= [0, 0, 0, 0, 0] - [1, 1, 1]
        \begin{bmatrix}
          5 & 1 & -1 & 0 & 0 \\
          2 & 1 & 0 & -1 & 0 \\
          1 & 2 & 0 & 0 & -1 \\
        \end{bmatrix} \\
        &= [-8, -4, -1, -1, -1] = [r_1, r_2, r_3, r_4, r_5]
      \end{align*}
      Since $r_1$ is the most negative reduced cost coefficient, we bring $\vect{a_1}$ into the basis.
      To do so, we compute $\vect{y}_1 = B^{-1}\vect{a_1} = \vect{a_1}$ and arrive
      at the augmented revised tableau:
      \begin{align*}
        \begin{matrix}
          \text{variable} & B^{-1} & & & \vect{y_0} & \vect{y_1} \\
          \hline \\
          x_6 & 1 & 0 & 0 & 11 & 5 \\
          x_7 & 0 & 1 & 0 & 8 & 2 \\
          x_8 & 0 & 0 & 1 & 7 & 1 \\
        \end{matrix}
      \end{align*}
      To determine which column leaves the basis, we compute
      $p = \text{arg}\min_i = \{y_{i0} / y_{i1} \ | \ y_{i1} > 0\} = 1$ so that $\vect{a_6}$
      leaves the basis and pivot
      around the first row of the last column and arrive at the updated revised tableau:
      \begin{align*}
        \begin{matrix}
          \text{variable} & B^{-1} & & & \vect{y_0}\\
          \hline \\
          x_1 & 1/5 & 0 & 0 & 11/5 \\
          x_7 & -2/5 & 1 & 0 & 18/5\\
          x_8 & -1/5 & 0 & 1 & 24/5\\
        \end{matrix}
      \end{align*}

      We continue by computing the reduced cost coefficients as follows:
      \begin{align*}
        \vect{r}_D^\tran &= \vect{c}_D^\tran - \vect{c}_B^\tran B^{-1} D \\
        &= [0, 0, 0, 0, 1] - [0, 1, 1]
        \begin{bmatrix}
          1/5 & 0 & 0 \\
          -2/5 & 1 & 0\\
          -1/5 & 0 & 1\\
        \end{bmatrix}
        \begin{bmatrix}
          1 & -1 & 0 & 0 & 1\\
          1 & 0 & -1 & 0 & 0\\
          2 & 0 & 0 & -1 & 0\\
        \end{bmatrix} \\
        &= [-12/5, -3/5, 1, 1, 8/5] = [r_2, r_3, r_4, r_5, r_6]
      \end{align*}
      Since $r_2$ is the most negative reduced cost coefficient, $\vect{a_2}$
      will enter the basis. After computing $\vect{y_2} = B^{-1}\vect{a_2}$ we
      arrive at the updated augmented revised tableau:
      \begin{align*}
        \begin{matrix}
          \text{variable} & B^{-1} & & & \vect{y_0} & \vect{y_2}\\
          \hline \\
          x_1 & 1/5 & 0 & 0 & 11/5 & 1/5\\
          x_7 & -2/5 & 1 & 0 & 18/5 & 3/5\\
          x_8 & -1/5 & 0 & 1 & 24/5 & 9/5\\
        \end{matrix}
      \end{align*}
      To determine which column leaves the basis, we compute
      $p = \text{arg}\min_i = \{y_{i0} / y_{i2} \ | \ y_{i2} > 0\} = 3$ so that $\vect{a_8}$
      leaves the basis and pivot
      around the last row of the last column and arrive at the updated revised tableau:
      \begin{align*}
        \begin{matrix}
          \text{variable} & B^{-1} & & & \vect{y_0}\\
          \hline \\
          x_1 & 2/9 & 0 & -1/9 & 5/3\\
          x_7 & -1/3 & 1 & -1/3 & 2\\
          x_2 & -1/9 & 0 & 5/9 & 24/9\\
        \end{matrix}
      \end{align*}

      We continue by computing the reduced cost coefficients as follows:
      \begin{align*}
        \vect{r}_D^\tran &= \vect{c}_D^\tran - \vect{c}_B^\tran B^{-1} D \\
        &= [0, 0, 0, 1, 1] - [0, 1, 0]
        \begin{bmatrix}
          2/9 & 0 & -1/9 \\
          -1/3 & 1 & -1/3\\
          -1/9 & 0 & 5/9
        \end{bmatrix}
        \begin{bmatrix}
          -1 & 0 & 0 & 1 & 0\\
          0 & -1 & 0 & 0 & 0\\
          0 & 0 & -1 & 0 & 1\\
        \end{bmatrix} \\
        &= [-1/3, 1, -1/3, 4/3, 4/3] = [r_3, r_4, r_5, r_6, r_8]
      \end{align*}
      Since $r_3$ and $r_5$ are the most negative reduced cost coefficients, we choose $\vect{a_5}$
      to enter the basis. After computing $\vect{y_5} = B^{-1}\vect{a_5}$ we
      arrive at the updated augmented revised tableau:
      \begin{align*}
        \begin{matrix}
          \text{variable} & B^{-1} & & & \vect{y_0} & \vect{y_5}\\
          \hline \\
          x_1 & 2/9 & 0 & -1/9 & 5/3 & 1/9\\
          x_7 & -1/3 & 1 & -1/3 & 2 & 1/3\\
          x_2 & -1/9 & 0 & 5/9 & 24/9 & -5/9\\
        \end{matrix}
      \end{align*}
      To determine which column leaves the basis, we compute
      $p = \text{arg}\min_i = \{y_{i0} / y_{i5} \ | \ y_{i5} > 0\} = 2$ so that $\vect{a_7}$
      leaves the basis and pivot
      around the second row of the last column and arrive at the updated revised tableau:
      \begin{align*}
        \begin{matrix}
          \text{variable} & B^{-1} & & & \vect{y_0}\\
          \hline \\
          x_1 & 1/3 & -1/3 & 0 & 1\\
          x_5 & -1 & 3 & -1 & 6\\
          x_2 & -2/3 & 5/3 & 0 & 6\\
        \end{matrix}
      \end{align*}

      Note that the reduced cost coefficients are
      \begin{align*}
        \vect{r}_D^\tran &= \vect{c}_D^\tran - \vect{c}_B^\tran B^{-1} D \\
        &= [0, 0, 1, 1, 1] - [0, 0, 0]
        \begin{bmatrix}
          1/3 & -1/3 & 0\\
          -1 & 3 & -1\\
          -2/3 & 5/3 & 0\\
        \end{bmatrix}
        \begin{bmatrix}
          -1 & 0 & 1 & 0 & 0\\
          0 & -1 & 0 & 1 & 0\\
          0 & 0 & 0 & 0 & 1\\
        \end{bmatrix} \\
        &= [0,0,1,1,1] = [r_3, r_4, r_6, r_7, r_8]
      \end{align*}
      which are all positive implying that our current solution is optimal for the artificial problem.
      Thus, the solution to the artificial problem is $\vect{x} = [1,6,0,0,6,0,0,0]^\tran$ and the initial
      basic feasible solution for phase II is $\vect{x} = [1,6,0,0,6]^\tran$.

      \textbf{\textit{Phase II.}} The initial tableau for the problem in standard form is:
      \begin{align*}
        \begin{matrix}
          & \vect{a_1} & \vect{a_2} & \vect{a_3} & \vect{a_4} & \vect{a_5} & \vect{b} \\
                         & 5 & 1 & -1 & 0 & 0 & 11 \\
                         & 2 & 1 & 0 & -1 & 0 & 8 \\
                         & 1 & 2 & 0 & 0 & -1 & 7 \\
          \vect{c}^\tran  & 4 & 3 & 0 & 0 & 0 & 0 \\
        \end{matrix}
      \end{align*}

      We will use the final revised tableau from phase I as the initial revised tableau for phase II
      and compute the reduced cost coefficients:
      \begin{align*}
        \vect{r}_D^\tran &= \vect{c}_D^\tran - \vect{c}_B^\tran B^{-1} D \\
        &= [0, 0] - [4, 0, 3]
        \begin{bmatrix}
          1/3 & -1/3 & 0\\
          -1 & 3 & -1\\
          -2/3 & 5/3 & 0\\
        \end{bmatrix}
        \begin{bmatrix}
          -1 & 0 \\
          0 & -1 \\
          0 & 0 \\
        \end{bmatrix} \\
        &= [-2/3, 11/3] = [r_3,r_4]
      \end{align*}
      Since $r_3$ is the most negative reduced cost coefficient, $\vect{a_3}$
      will enter the basis. After computing $\vect{y_3} = B^{-1}\vect{a_3}$ we
      arrive at the updated augmented revised tableau:
      \begin{align*}
        \begin{matrix}
          \text{variable} & B^{-1} & & & \vect{y_0} & \vect{y_3}\\
          \hline \\
          x_1 & 1/3 & -1/3 & 0 & 1 & -1/3\\
          x_5 & -1 & 3 & -1 & 6 & 1\\
          x_2 & -2/3 & 5/3 & 0 & 6 & 2/3\\
        \end{matrix}
      \end{align*}
      To determine which column leaves the basis, we compute
      $p = \text{arg}\min_i = \{y_{i0} / y_{i3} \ | \ y_{i3} > 0\} = 2$ so that $\vect{a_5}$
      leaves the basis and pivot
      around the second row of the last column and arrive at the updated revised tableau:
      \begin{align*}
        \begin{matrix}
          \text{variable} & B^{-1} & & & \vect{y_0}\\
          \hline \\
          x_1 & 0 & 2/3 & -1/3 & 3\\
          x_3 & -1 & 3 & -1 & 6\\
          x_2 & 0 & -1/3 & 2/3 & 2\\
        \end{matrix}
      \end{align*}

      Note that the reduced cost coefficients are
      \begin{align*}
        \vect{r}_D^\tran &= \vect{c}_D^\tran - \vect{c}_B^\tran B^{-1} D \\
        &= [0, 0] - [4, 0, 3]
        \begin{bmatrix}
          0 & 2/3 & -1/3\\
          -1 & 3 & -1\\
          0 & -1/3 & 2/3\\
        \end{bmatrix}
        \begin{bmatrix}
          0 & 0 \\
          -1 & 0 \\
          0 & -1 \\
        \end{bmatrix} \\
        &= [5/3, 2/3] = [r_4,r_5]
      \end{align*}
      which are all positive implying that our current solution is optimal.
      Thus, the solution to the problem in standard form is $\vect{x} = [3,2,6,0,0]^\tran$
      with objective function value 18. Therefore, the solution to the original problem
      is $x_1 = 3$, $x_2 = 2$ with objective function value $-18$.

    \item In order to solve the linear program using the revised simplex method,
      the problem must be converted to standard form by changing the objective
      from \textit{maximize} to \textit{minimize} and adding slack variables.
      The problem in standard form is thus
      \begin{align*}
        \begin{array}{rrl}
          \text{minimize} & -6x_1 - 4x_2 - 7x_3 - 5x_4  &\\
          \text{subject to}
          & x_1 + 2x_2 + x_3 + 2x_4 + x_5 &= 20 \\
          & 6x_1 + 5x_2 + 3x_3 + 2x_4 + x_6 &= 100 \\
          & 3x_1 + 4x_2 + 9x_3 + 12x_4 + x_7 &= 75 \\
          & x_1, x_2, x_3, x_4, x_5, x_6, x_7 &\geq 0.
        \end{array}
      \end{align*}

      The initial tableau associated to the problem is:
      \begin{align*}
        \begin{matrix}
          & \vect{a_1} & \vect{a_2} & \vect{a_3} & \vect{a_4} & \vect{a_5} & \vect{a_6} & \vect{a_7} & \vect{b} \\
                         & 1 & 2 & 1 & 2 & 1 & 0 & 0 & 20 \\
                         & 6 & 5 & 3 & 2 & 0 & 1 & 0 & 100 \\
                         & 3 & 4 & 9 & 12 & 0 & 0 & 1 & 75 \\
          \vect{c}^\tran  & -6 & -4 & -7 & -5 & 0 & 0 & 0 & 0 \\
        \end{matrix}
      \end{align*}

      We start with an initial basic feasible solution to the above tableau
      along with its associated $B^{-1}$ and write this information in the following tableau:
      \begin{align*}
        \begin{matrix}
          \text{variable} & B^{-1} & & & \vect{y_0} \\
          \hline \\
          x_5 & 1 & 0 & 0 & 20 \\
          x_6 & 0 & 1 & 0 & 100  \\
          x_7 & 0 & 0 & 1 & 75 \\
        \end{matrix}
      \end{align*}
      The next step is to compute the reduced cost coefficients:
      \begin{align*}
        \vect{r}_D^\tran &= \vect{c}_D^\tran - \vect{c}_B^\tran B^{-1} D \\
        &= [-6, -4, -7, -5] - [0, 0, 0]
        \begin{bmatrix}
          1 & 2 & 1 & 2 \\
          6 & 5 & 3 & 2 \\
          3 & 4 & 9 & 12 \\
        \end{bmatrix} \\
        &= [-6, -4, -7, -5] = [r_1, r_2, r_3, r_4]
      \end{align*}
      Since $r_3$ is the most negative reduced cost coefficient, we bring $\vect{a_3}$ into the basis.
      To do so, we compute $\vect{y}_3 = B^{-1}\vect{a_3} = \vect{a_3}$ and arrive
      at the augmented revised tableau:
      \begin{align*}
        \begin{matrix}
          \text{variable} & B^{-1} & & & \vect{y_0} & \vect{y_3} \\
          \hline \\
          x_5 & 1 & 0 & 0 & 20 & 1 \\
          x_6 & 0 & 1 & 0 & 100& 3 \\
          x_7 & 0 & 0 & 1 & 75 & 9 \\
        \end{matrix}
      \end{align*}
      To determine which column leaves the basis, we compute
      $p = \text{arg}\min_i = \{y_{i0} / y_{i3} \ | \ y_{i3} > 0\} = 3$ so that $\vect{a_7}$
      leaves the basis and pivot around the last row of the last column and arrive at the updated revised tableau:
      \begin{align*}
        \begin{matrix}
          \text{variable} & B^{-1} & & & \vect{y_0} \\
          \hline \\
          x_5 & 1 & 0 & -1/9 & 35/3 \\
          x_6 & 0 & 1 & -3/9 & 75 \\
          x_3 & 0 & 0 & 1/9 & 75/9 \\
        \end{matrix}
      \end{align*}

      We continue by computing the reduced cost coefficients as follows:
      \begin{align*}
        \vect{r}_D^\tran &= \vect{c}_D^\tran - \vect{c}_B^\tran B^{-1} D \\
        &= [-6, -4, -5, 0] - [0, 0, -7]
        \begin{bmatrix}
          1 & 2 & 1 & 2 \\
          6 & 5 & 3 & 2 \\
          3 & 4 & 9 & 12 \\
        \end{bmatrix} \\
        &= [-11/3, -8/9, 13/3, 7/9] = [r_1, r_2, r_4, r_7]
      \end{align*}
      Since $r_1$ is the most negative reduced cost coefficient, we bring $\vect{a_1}$ into the basis.
      To do so, we compute $\vect{y}_1 = B^{-1}\vect{a_1}$ and arrive
      at the augmented revised tableau:
      \begin{align*}
        \begin{matrix}
          \text{variable} & B^{-1} & & & \vect{y_0} & \vect{y_1} \\
          \hline \\
          x_5 & 1 & 0 & -1/9 & 35/3 & 2/3\\
          x_6 & 0 & 1 & -3/9 & 75 & 5\\
          x_3 & 0 & 0 & 1/9 & 75/9 & 1/3\\
        \end{matrix}
      \end{align*}
      To determine which column leaves the basis, we compute
      $p = \text{arg}\min_i = \{y_{i0} / y_{i1} \ | \ y_{i1} > 0\} = 2$ so that $\vect{a_6}$
      leaves the basis and pivot around the second row of the last column and arrive at the updated revised tableau:
      \begin{align*}
        \begin{matrix}
          \text{variable} & B^{-1} & & & \vect{y_0} \\
          \hline \\
          x_5 & 1 & -2/15 & -1/15 & 5/3\\
          x_1 & 0 & 1/5 & -3/45 & 15\\
          x_3 & 0 & -1/15 & 2/15 & 10/3\\
        \end{matrix}
      \end{align*}

      Note that the reduced cost coefficients are
      \begin{align*}
        \vect{r}_D^\tran &= \vect{c}_D^\tran - \vect{c}_B^\tran B^{-1} D \\
        &= [-4, -5, 0, 0] - [0, -6, -7]
        \begin{bmatrix}
          16/15 & 14/15 & -2/15 & -1/15 \\
          11/15 & -2/5 & 1/5 & -1/15 \\
          1/5 & 22/15 & -1/15 & 2/15 \\
        \end{bmatrix} \\
        &= [9/5, 43/15, 11/15, 8/15] = [r_2, r_4, r_6, r_7]
      \end{align*}
      which are all positive implying that our current solution is optimal.
      Thus, the solution to the problem in standard form is $\vect{x} = [15,0,10/3,0,5/3,0,0]^\tran$
      with objective function value $-340/3$. Therefore, the solution to the original problem
      is $x_1 = 15$, $x_2 = 0$, $x_3=10/3$, $x_4=0$ with objective function value $340/3$.
  \end{enumerate}
\end{proof}
\newpage


% Problem 5
\begin{problem}
  Suppose that we apply the simplex method to a given linear programming
  problem and obtain the following canonical tableau:
  \begin{align*}
    \begin{matrix}
      0 & \beta & 0 & 1 & 4 \\
      1 & \gamma & 0 & 0 & 5 \\
      0 & -3 & 1 & 0 & 6 \\
      0 & 2 - \alpha & 0 & 0 & \delta \\
    \end{matrix}
  \end{align*}
  For each of the following conditions, find the set of all parameter values
  $\alpha, \beta, \gamma, \delta$ that satisfy the condition.
  \begin{enumerate}
    \item The problem has no solution because the objective function values are
      unbounded.
    \item The current basic feasible solution is optimal, and the corresponding
      objective function value is 7.
    \item The current basic feasible solution is not optimal, and the objective
      function value strictly decreases if we remove the first column of $A$ from
      the basis.
  \end{enumerate}
\end{problem}

\begin{proof}
  \begin{enumerate}
    \item The problem has no solution because the objective function values are
      unbounded if the current pivot column has all non-positive entries. The only
      possible pivot column is the second one. We would only continue the algorithm
      if the associated reduced cost coefficient is negative. Thus, we must have
      $2-\alpha < 0$, i.e.\ $2 < \alpha$ and $\beta, \gamma \leq 0$. Further,
      since the objective function value is unbounded, we must have
      $\delta \geq 0$.
    \item The current basic feasible solution is optimal if all reduced cost
      coefficients are non-negative. This requires that $2-\alpha \geq 0$, i.e.\
      $\alpha \leq 2$. Since, the simplex algorithm terminates at this point no restrictions are placed
      on $\beta$ or $\gamma$. Further, the objective function value is $7$ if $\delta = 7$.
    \item The current basic feasible solution is not optimal if there there is at least
      one negative reduced cost coefficient. This requires that $2-\alpha < 0$, i.e.\ $2 < \alpha$.
      In order for the algorithm to continue and remove $\vect{a_1}$ from the basis,
      we would need to pivot around the entry $(2,2)$. This can happen only if $\gamma \geq 0$.
      Further, if $\beta > 0$ we would require $5/\gamma < 4/\beta$. If $\beta \leq 0$,
      then no additional restrictions on $\gamma$ and $\beta$ are necessary. The objective function value would strictly decrease if after pivoting
      $\delta \geq 0$ and $\delta - 5(2-\alpha)/\gamma < \delta$, i.e.\ $(2-\alpha)/\gamma > 0$. However, this condition is
      contradictory to the requirement $(2-\alpha) < 0$ since $\gamma > 0$. Therefore, no value
      of $\alpha$ satisfies this scenario.
  \end{enumerate}
\end{proof}


\end{document}
