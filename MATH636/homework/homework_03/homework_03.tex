\documentclass[12pt]{article}
\usepackage[margin=1in]{geometry}
\usepackage{amsmath, amsthm, amssymb, amsfonts, enumitem, graphicx}
\usepackage{fancyhdr}
\usepackage{tikz}
\newcommand*\circled[1]{\tikz[baseline=(char.base)]{
            \node[shape=circle,draw,inner sep=2pt] (char) {#1};}}

\theoremstyle{definition}
\newtheorem{problem}{Problem}
\renewcommand*{\proofname}{Solution}
\renewcommand{\theenumi}{\alph{enumi}}

\newcommand{\vect}[1]{\boldsymbol{#1}}

\newenvironment{amatrix}[1]{%
  \left[\begin{array}{@{}*{#1}{c}|c@{}}
}{%
  \end{array}\right]
}

\newcommand{\grstep}[2][\relax]{%
   \ensuremath{\mathrel{
       {\mathop{\longrightarrow}\limits^{#2\mathstrut}_{
                                     \begin{subarray}{l} #1 \end{subarray}}}}}}
\newcommand{\swap}{\leftrightarrow}

\pagestyle{fancy}
\fancyhf{}
\rhead{Homework Assignment 3}
\lhead{Matthew Tiger}
\cfoot{\thepage}


\title{Homework Assignment 3}
\author{Matthew Tiger}


\begin{document}


\maketitle


% Problem 1
\begin{problem}
  Solve the following linear program using the Simplex Algorithm in conjunction
  with Bland's rule:
  \begin{align*}
    \begin{array}{rrl}
      \text{maximize} & 2x_1 + 5x_2 &\\
      \text{subject to} & x_1 &\leq 4 \\
      & x_2 &\leq 6 \\
      & x_1 + x_2 &\leq 8 \\
      & x_1, x_2 &\geq 0.
    \end{array}
  \end{align*}
\end{problem}

\begin{proof}
  To start, we must transform this LP into standard form. This is achieved by
  changing the objective from \textit{maximize} to \textit{minimize} and
  adding three slack variables. In standard form,
  the problem becomes
  \begin{align*}
    \begin{array}{rrl}
      \text{minimize} & -2x_1 -5x_2&\\
      \text{subject to} & x_1 + x_3 &= 4 \\
      & x_2 + x_4 &= 6 \\
      & x_1 + x_2 + x_5 &= 8 \\
      & x_1, x_2, x_3, x_4, x_5 &\geq 0.
    \end{array}
  \end{align*}
  The initial tableau associated to this problem is then:
  \begin{align*}
    \begin{matrix}
      & \vect{a_1} & \vect{a_2} & \vect{a_3} & \vect{a_4} & \vect{a_5} & \vect{b} \\
      & 1 & 0 & 1 & 0 & 0 & 4 \\
      & 0 & 1 & 0 & 1 & 0 & 6 \\
      & 1 & 1 & 0 & 0 & 1 & 8 \\
      \vect{c}^\intercal & -2 & -5  & 0 & 0 & 0 & 0 \\
    \end{matrix}
  \end{align*}
  Note that this tableau is in canonical form with respect to the basis
  $[\vect{a_3}, \vect{a_4}, \vect{a_5}]$. Thus, the last row of the tableau
  contains the reduced cost coefficients. Bland's rule prescribes how to choose
  the column-index $q$ and the row-index $p$ to pivot around. According to
  Bland's rule, choose
  \begin{align*}
    q &= \min\{i\ |\ r_i < 0  \}\\
    p &= \min\{j\ |\ y_{j0}/y_{jq} = \min_i\{y_{i0}/y_{iq}\ |\ y_{iq} > 0\} \}.
  \end{align*}
  Thus, we proceed by choosing the column-index to pivot around to be the smallest
  index pertaining to negative reduced cost coefficients in the bottom vector of the
  tableau and by then choosing the row-index to pivot around to be the
  index pertaining to the row with the lowest ratio between the right hand side and
  the positive coefficients of the $q$-th column in matrix $A$ of the tableau.
  If there are two such row-indexes, choose the smaller one.

  From the initial tableau, Bland's rule prescribes that we pivot around column
  $q=1$ since this is the smallest index with a negative reduced cost coefficient.
  The smallest ratio between the right hand side and the positive coefficients
  of the $q$-th column in matrix $A$ is $4/1$ so we pivot around row $p =1$. Thus, $\vect{a_1}$
  enters the basis, $\vect{a_3}$ leaves the basis, and we move from the initial tableau to the updated tableau:
  \begin{align*}
    \begin{matrix}
      & \vect{a_1} & \vect{a_2} & \vect{a_3} & \vect{a_4} & \vect{a_5} & \vect{b} \\
      & \circled{1} & 0 & 1 & 0 & 0 & 4 \\
      & 0 & 1 & 0 & 1 & 0 & 6 \\
      & 1 & 1 & 0 & 0 & 1 & 8 \\
      \vect{c}^\intercal & -2 & -5  & 0 & 0 & 0 & 0 \\
      & \uparrow & & & & & \\
    \end{matrix}
    \quad
    \begin{array}{c}
    \grstep[{[4] + 2[1]}]{[3] - [1]} \\
    \end{array}
    \quad
    \begin{matrix}
      & \vect{a_1} & \vect{a_2} & \vect{a_3} & \vect{a_4} & \vect{a_5} & \vect{b} \\
                       & 1 & 0 & 1 & 0 & 0 & 4 \\
                       & 0 & 1 & 0 & 1 & 0 & 6 \\
                       & 0 & 1 & -1 & 0 & 1 & 4 \\
      \vect{c}^\intercal & 0 & -5 & 2 & 0 & 0 & 8 \\
      & & & & & & \\
    \end{matrix}
  \end{align*}

  From this newly derived tableau, we notice that the only negative reduced cost coefficient
  occurs in column $q=2$. Further, the smallest ratio between the right hand side
  and the positive coefficients of the $q$-th column in matrix $A$ is $4/1$ so we pivot
  around row $p=3$. Thus, $\vect{a_2}$ enters the basis, $\vect{a_5}$ leaves the basis, and we move from this tableau to the updated tableau:
  \begin{align*}
    \begin{matrix}
      & \vect{a_1} & \vect{a_2} & \vect{a_3} & \vect{a_4} & \vect{a_5} & \vect{b} \\
                       & 1 & 0 & 1 & 0 & 0 & 4 \\
                       & 0 & 1 & 0 & 1 & 0 & 6 \\
                       & 0 & \circled{1} & -1 & 0 & 1 & 4 \\
      \vect{c}^\intercal & 0 & -5 & 2 & 0 & 0 & 8 \\
      & & \uparrow & & & & \\
    \end{matrix}
    \quad
    \begin{array}{c}
    \grstep[{[4] + 5[3]}]{[2] - [3]} \\
    \end{array}
    \quad
    \begin{matrix}
      & \vect{a_1} & \vect{a_2} & \vect{a_3} & \vect{a_4} & \vect{a_5} & \vect{b} \\
                       & 1 & 0 & 1 & 0 & 0 & 4 \\
                       & 0 & 0 & 1 & 1 & -1 & 2 \\
                       & 0 & 1 & -1 & 0 & 1 & 4 \\
      \vect{c}^\intercal & 0 & 0 & -3 & 0 & 5 & 28 \\
      & & & & & & \\
    \end{matrix}
  \end{align*}

  From this newly derived tableau, we notice that the only negative reduced cost coefficient
  occurs in column $q=3$. Further, the smallest ratio between the right hand side
  and the positive coefficients of the $q$-th column in matrix $A$ is $2/1$ so we pivot
  around row $p=2$. Thus, $\vect{a_3}$ enters the basis, $\vect{a_4}$ leaves the basis, and we move from this tableau to the updated tableau:
  \begin{align*}
    \begin{matrix}
      & \vect{a_1} & \vect{a_2} & \vect{a_3} & \vect{a_4} & \vect{a_5} & \vect{b} \\
                       & 1 & 0 & 1 & 0 & 0 & 4 \\
                       & 0 & 0 & \circled{1} & 1 & -1 & 2 \\
                       & 0 & 1 & -1 & 0 & 1 & 4 \\
      \vect{c}^\intercal & 0 & 0 & -3 & 0 & 5 & 28 \\
      & & & \uparrow & & & \\
    \end{matrix}
    \quad
    \begin{array}{c}
    \grstep[{[3] + [2]}]{[1] - [2]} \\
    \grstep[{[4] + 3[2]}]{} \\
    \end{array}
    \quad
    \begin{matrix}
      & \vect{a_1} & \vect{a_2} & \vect{a_3} & \vect{a_4} & \vect{a_5} & \vect{b} \\
                       & 1 & 0 & 0 & -1 & 1 & 2 \\
                       & 0 & 0 & 1 & 1 & -1 & 2 \\
                       & 0 & 1 & 0 & 1 & 0 & 6 \\
      \vect{c}^\intercal & 0 & 0 & 0 & 3 & 2 & 34 \\
      & & & & & & \\
    \end{matrix}
  \end{align*}

  In the final tableau we have no negative reduced cost coefficients. Therefore,
  the current basic feasible solution $\vect{x} = [2, 6, 2, 0, 0]^\intercal$ of the LP in
  standard form is optimal with corresponding
  objective function value $-34$. The solution to the original problem is then
  $x_1 = 2$, $x_2 = 6$ with corresponding objective value 34.
\end{proof}
\newpage


% Problem 2
\begin{problem}
  \begin{enumerate}
    \item Prove that if (ALP) has a feasible solution $(x_1, \dots, x_n; y_1, \dots, y_m)$
      with objective function value zero then $y_1 = 0, \dots, y_m = 0$.
    \item What do you do if after Phase I (ALP) does not have any optimal feasible
      solution with objective function value zero?
  \end{enumerate}
\end{problem}

\begin{proof}
\end{proof}
\newpage


% Problem 3
\begin{problem}
  Consider the linear program
  \begin{align*}
    \begin{array}{rrl}
      \text{maximize} & 2x_1 + x_2 &\\
      \text{subject to} & 0 \leq x_1 &\leq 5 \\
      & 0 \leq x_2 &\leq 7 \\
      & x_1 + x_2 &\leq 9.
    \end{array}
  \end{align*}
  Convert the problem to standard form and solve it using the simplex method.
\end{problem}

\begin{proof}
\end{proof}
\newpage


% Problem 4
\begin{problem}
  Solve the following linear programs using the revised simplex method:
  \begin{enumerate}
    \item
      \begin{align*}
        \begin{array}{rrl}
          \text{maximize} & -4x_1 -3x_2 &\\
          \text{subject to} &5x_1 + x_2 &\geq 11 \\
          & -2x_1 - x_2 &\leq -8 \\
          & x_1 + 2x_2 &\geq 7 \\
          & x_1, x_2 &\geq 0.
        \end{array}
      \end{align*}
    \item
      \begin{align*}
        \begin{array}{rrl}
          \text{maximize} & 6x_1 + 4x_2 + 7x_3 + 5x_4  &\\
          \text{subject to}
          & x_1 + 2x_2 + x_3 + 2x_4 &\leq 20 \\
          & 6x_1 + 5x_2 + 3x_3 + 2x_4 &\leq 100 \\
          & 3x_1 + 4x_2 + 9x_3 + 12x_4 &\leq 75 \\
          & x_1, x_2, x_3, x_4 &\geq 0.
        \end{array}
      \end{align*}
  \end{enumerate}
\end{problem}

\begin{proof}
\end{proof}
\newpage


% Problem 5
\begin{problem}
  Suppose that we apply the simplex method to a given linear programming
  problem and obtain the following canonical tableau:
  \begin{align*}
    \begin{matrix}
      0 & \beta & 0 & 1 & 4 \\
      1 & \gamma & 0 & 0 & 5 \\
      0 & -3 & 1 & 0 & 6 \\
      0 & 2 - \alpha & 0 & 0 & \delta \\
    \end{matrix}
  \end{align*}
  For each of the following conditions, find the set of all parameter values
  $\alpha, \beta, \gamma, \delta$ that satisfy the condition.
  \begin{enumerate}
    \item The problem has no solution because the objective function values are
      unbounded.
    \item The current basic feasible solution is optimal, and the corresponding
      objective function value is 7.
    \item The current basic feasible solution is not optimal, and the objective
      function value strictly decreases if we remove the first column of $A$ from
      the basis.
  \end{enumerate}
\end{problem}

\begin{proof}
\end{proof}


\end{document}
