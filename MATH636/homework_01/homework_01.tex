\documentclass[12pt]{article}
\usepackage[margin=1in]{geometry}
\usepackage{amsmath, amsthm, amssymb, amsfonts}

\theoremstyle{definition}
\newtheorem{problem}{Problem}
\renewcommand*{\proofname}{Solution}
\renewcommand{\theenumi}{\alph{enumi}}


\title{Homework Assignment 1}
\author{Matthew Tiger}


\begin{document}


\maketitle


% Problem 1
\begin{problem}
  To be comprehensive, the second derivative test for two-variable functions $f = f(x, y)$
  studied in Calculus III should contain (among others) the cases:
  \begin{enumerate}
    \item $D(a, b) > 0$ and $f_{xx}(a, b) = 0,$
    \item $D(a, b) = 0$ and $f_{xx}(a, b) = 0.$
  \end{enumerate}
  Why aren't these cases considered? Explain.
\end{problem}

\begin{proof}
\end{proof}
\newpage


% Problem 2
\begin{problem}
  Recall that
  \begin{itemize}
    \item $(a, b)$ is called an \textit{absolute maximum} of $f = f(x, y)$ on a domain
      $D \subset \mathbb{R}^2$ if $f(x, y) \leq f(a, b)$
      for every $(x, y) \in D$.
    \item (The Extreme Value) If $f$ is continuous and $D$ is closed and bounded,
      then $f$ attains both an absolute maximum value and an absolute minimum value.
  \end{itemize}
  \begin{enumerate}
    \item Describe in steps (and in words) how one finds absolute extrema for a two-variable
      function $f = f(x, y)$ on a closed bounded $D \subset \mathbb{R}^2$.
    \item Apply your procedure derived in (a) to find absolute extrema for $f(x, y) = 2x^3 + xy^2 + xy^2 + 5x^2 + y^2$
      over the rectangle $D := \{(x, y)\ | -2 \leq x \leq 3, 0 \leq y \leq 2\}$.
  \end{enumerate}
\end{problem}

\begin{proof}
\end{proof}
\newpage


% Problem 3
\begin{problem}
  Consider the optimization problem:
  \begin{align*}
    \begin{array} {lcl}
      \text{Min (Max) } & f(x_1, x_2, \dots, x_n) & \\
      \text{subject to } & g_1(x_1, x_2, \dots, x_n) &= k_1 \\
      & g_2(x_1, x_2, \dots, x_n) &= k_2 \\
      & \vdots & \\
      & g_m(x_1, x_2, \dots, x_n) &= k_m \\
    \end{array}
  \end{align*}
  \begin{enumerate}
    \item Formulate the Lagrangean and describe how we should proceed in order
      to solve such a problem.
    \item Find the relative extrema of $f(x, y, z) = x + 2y + 3z$ subject to
      $x - y + z = 1, x^2 + y^2 = 1$.
  \end{enumerate}
\end{problem}

\begin{proof}
\end{proof}
\newpage


% Problem 4
\begin{problem}
  Solve the shipping problem studied in MATH 111 if we replace the constraint
  $x + 2y \leq 100$ by the constraint $x + 2y \leq 625/6$. Use Mathematica to
  (at least) graph the feasible set.
\end{problem}

\begin{proof}
\end{proof}
\newpage


% Problem 5
\begin{problem}
  Suppose that $f, f_1, f_2$ are convex functions and $a \geq 0$. Prove that
  $af$ and $f_1 + f_2$ are convex functions.
\end{problem}

\begin{proof}
\end{proof}
\newpage


% Problem 6
\begin{problem}
  For $f: \mathbb{R}^n \to \mathbb{R} \cup \{+\infty\}$ we define its \textit{epigraph}
  as the set
  \[
   \text{epi}\ f = \{(x, \beta) \in \mathbb{R}^n \times \mathbb{R} | f(x) \leq \beta \} \subset \mathbb{R}^{n+1}.
  \]
  Prove that $f$ is convex if and only if $\text{epi}\ f$ is convex.
\end{problem}

\begin{proof}
\end{proof}


\end{document}
