\documentclass[12pt]{article}
\usepackage[margin=1in]{geometry}
\usepackage{amsmath, amsthm, amssymb, amsfonts, breqn, graphicx}


\theoremstyle{definition}
\newtheorem{problem}{Problem}
\renewcommand*{\proofname}{Solution}
\newenvironment{custompbm}[1]
  {\renewcommand\theproblem{#1}\problem}
  {\endproblem}
\renewcommand{\theenumi}{\alph{enumi}}


\newcommand{\E}{\text{E}}
\newcommand{\V}{\text{Var}}
\newcommand{\Co}[2]{\text{Cov}\left({#1}, {#2}\right)}
\newcommand{\pdf}{\text{pdf}}
\newcommand{\pmf}{\text{pmf}}
\newcommand{\me}{\mathrm{e}}
\newcommand*\diff{\mathop{}\!\mathrm{d}}
\newcommand{\vect}[1]{\boldsymbol{#1}}
\newcommand{\mx}[1][t]{\mu_X({#1})}
\newcommand{\gx}[2]{\gamma_X({#1}, {#2})}


\title{Midterm 1}
\author{Matthew Tiger}


\begin{document}


\maketitle


% Problem
\begin{custompbm}{1.a}
  Consider the process
  \begin{align}\label{arma_eqns}
    X_t + 0.4X_{t-1} - 0.32X_{t-2} = Z_t -0.8Z_{t-1} + 0.16 Z_{t-2}.
  \end{align}
  Determine whether the model is a stationary process.
\end{custompbm}

\begin{proof}
  The model $\{X_t\}$ is a stationary process if $\{X_t\}$ is a stationary solution
  of the equations \eqref{arma_eqns}. By the existence and uniqueness theorem
  of ARMA$(p, q)$ processes, a stationary solution $\{X_t\}$ of the equations
  \[
    X_t - \phi_1X_{t-1} - \dots - \phi_pX_{t-p} = Z_t + \theta_1Z_{t-1} + \dots + \theta_q Z_{t-q}
  \]
  that define the model exists if and only if
  \[
    \phi(z) = 1 - \phi_1 z - \dots - \phi_p z^p \neq 0 \text{\quad for all $|z| = 1,$}
  \]
  i.e. if and only if the roots of $\phi(z)$ do not lie on the unit circle.

  For our model, we have $\phi_1 = -0.4$ and $\phi_2 = 0.32$ so that $\phi(z) = 1 + 0.4z - 0.32z^2$.
  Note that the roots of $\phi(z)$ are $z_1 = -1.25$ and $z_2 = 2.5$. As $|z_i| \neq 1$ for $i=1,2$,
  we conclude that the roots of $\phi(z)$ do not lie on the unit circle and that
  the model $\{X_t\}$ is a stationary process assuming that $\{Z_t\} \sim \text{WN}(0, \sigma^2)$.
\end{proof}


\begin{custompbm}{1.b}
  Considering the model in problem 1.a, what is $R_3$, i.e.\ the correlation
  matrix of size 3?
\end{custompbm}

\begin{proof}
  The covariance matrix of size 3 for our model $\{X_t\}$ is given by
  \begin{align*}
    \Gamma_3 =
    \begin{bmatrix}
      \gamma(0) & \gamma(1) & \gamma(2) \\
      \gamma(1) & \gamma(0) & \gamma(1) \\
      \gamma(2) & \gamma(1) & \gamma(0) \\
    \end{bmatrix}
  \end{align*}
  where $\gamma(h)$ is the autocovariance function of the process $\{X_t\}$. For an ARMA$(p, q)$
  process $X_t -\phi_1 X_{t-1} - \dots - \phi_p X_{t-p} = Z_t + \theta_1 Z_{t-1} + \dots + \theta_q Z_{t-q}$,
  the autocovariance function $\gamma(h)$ satisfies the equations
  \[
    \gamma(k) - \phi_1 \gamma(k-1) - \dots - \phi_p \gamma(k-p) = \sigma^2 \sum_{j=0}^\infty \theta_{k+j}\psi_j \text{\quad for $0 \leq k < \max(p, q+1)$}
  \]
  where $\psi_j - \sum_{k=1}^p \phi_k \psi_{j-k} = \theta_j$ for $j \geq 0$ and $\psi_j = 0$ for $j < 0$.
  For our process,  this corresponds to the system of equations
  \begin{align}\label{cov_sys}
    \gamma(0) - \phi_1 \gamma(1) - \phi_2 \gamma(2) &= \sigma^2 (\psi_0 + \theta_1\psi_1 + \theta_2\psi_2) \notag \\
    \gamma(1) - \phi_1 \gamma(0) - \phi_2 \gamma(1) &= \sigma^2 (\theta_1\psi_0 + \theta_2\psi_1) \\
    \gamma(2) - \phi_1 \gamma(1) - \phi_2 \gamma(0) &= \sigma^2 \theta_2\psi_0 \notag
  \end{align}
  where $\psi_0 = 1$, $\psi_1 = \theta_1 + \phi_1$, and $\psi_2 = \theta_2 + \phi_1^2 + \phi_1\theta_1 + \phi_2$.
  Using the parameters $\phi_j$ and $\theta_k$ defining our model, the system of equations \eqref{cov_sys} becomes
  \begin{align*}
    \gamma(0) + 0.4 \gamma(1) - 0.32 \gamma(2) &= 2.1136\sigma^2\\ \notag
    \gamma(1) + 0.4 \gamma(0) - 0.32 \gamma(1) &= -0.992 \sigma^2 \\
    \gamma(2) + 0.4 \gamma(1) - 0.32 \gamma(0) &= 0.16 \sigma^2 \notag
  \end{align*}
  the solution of which is $\gamma(0) = 5 \sigma^2$, $\gamma(1) = -4.4 \sigma^2$,
  and $\gamma(2) = 3.52 \sigma^2$. Thus, the covariance matrix $\Gamma_3$ is given by
  \[
    \Gamma_3 =
    \sigma^2
    \begin{bmatrix}
      \phantom{-}5.00  & -4.40 & \phantom{-}3.52 \\
      -4.40 & \phantom{-}5.00  & -4.40  \\
      \phantom{-}3.52 & -4.40 & \phantom{-}5.00  \\
    \end{bmatrix}.
  \]
  Note that the correlation matrix $R_3$ is given by $(1/\gamma(0))\Gamma_3$. Therefore,
  \[
    R_3 =
    \begin{bmatrix}
      \phantom{-}1.000  & -0.880 & \phantom{-}0.704 \\
      -0.880 & \phantom{-}1.000  & -0.880  \\
      \phantom{-}0.704 & -0.880 & \phantom{-}1.000  \\
    \end{bmatrix}.
  \]

\end{proof}


\begin{custompbm}{1.c}
  Express the process in problem 1.a as a pure MA process in the form of $X_t = \sum_{j=0}^\infty\psi_jZ_t$.
\end{custompbm}

\begin{proof}
  For our process, the roots of the equation $\phi(z) = 1 + 0.4 z - 0.32 z^2 = 0$ are
  $z_1 = -1.25$ and $z_2 = 2.5$. As $|z_i| > 1$ for $i=1,2$, this process is causal and
  can be represented as an MA($\infty$) process, i.e.\ $X_t = \sum_{j=0}^\infty \psi_j Z_{t-j}$,
  where the coefficients $\psi_j$ are determined by the equations $\psi_j - \sum_{k=1}^p \phi_k\psi_{j-k} = \theta_j$
  for $j \geq 0$ and $\psi_j = 0$ for $j < 0$.

  Note that for an ARMA($p, q$) process, as $\theta_j = 0$ for $j > q$, the equations
  determining the coefficients are difference equations determined by the boundary conditions
  \[
    \psi_j - \sum_{k=1}^p \phi_k \psi_{j-k} = \theta_j \text{\quad \text{for $0 \leq j < \max(p, q+1)$}}
  \]
  and the homogeneous equation
  \[
    \psi_j - \sum_{k=1}^p \phi_k \psi_{j-k} = 0 \text{\quad \text{for $j \geq \max(p, q+1)$}}.
  \]

  For our process, the characteristic equation of these difference equations is $\phi(z)$. The roots
  of this characteristic equation are, as shown above, $z_1 = -1.25$ and $z_2 = 2.5$. As these roots
  are distinct, the solution to the homogeneous difference equation is
  \[
    \psi_j = \alpha_1 z_1 ^ {-j} + \alpha_2 z_2 ^{-j} = \alpha_1 (-1.25)^{-j} + \alpha_2 (2.5)^{-j} \text{\quad for $j \geq 1$}
  \]
  where the coefficients are determined by the boundary conditions $\psi_0 = 1$,
  $\psi_1 = \theta_1 + \phi_1 = -1.2$, and $\psi_2 = \theta_2 + \phi_1^2 + \phi_1\theta_1 + \phi_2 = 0.96$.
  Using the method of undetermined coefficients, we can see that $\alpha_1 = 1.5$ and $\alpha_2 = 0$.
  Therefore $\psi_j = 1.5(-1.25)^{-j}$ for $j \geq 1$, $\psi_0 = 1$, and
  \[
    X_t = \sum_{j=0}^\infty\psi_j Z_{t-j} = Z_t + 1.5 \sum_{j=1}^\infty (-1.25)^{-j}Z_{t-j}.
  \]
\end{proof}


\begin{custompbm}{2.a}
  Let $X_t$ be the AR(2) process such that $X_t = 0.8X_{t-2} + Z_t$ where $\{Z_t\}\sim \text{WN}(0, \sigma^2)$.
  Find the autocorrelation function of $X_t$.
\end{custompbm}

\begin{proof}
  This AR(2) process is defined by the parameters $\phi_1 = 0$ and $\phi_2 = 0.8$.
  This process has characteristic equation $\phi(z) = 1 - 0.8z^2 = 0$ of which
  the roots are $z_1 = 1.11803$ and $z_2 = -1.11803$. As these roots lie outside the unit
  circle this process is causal.

  Note that $\{X_t\}$ can be represented as $(1 - \xi_1^{-1}B)(1 - \xi_2^{-1}B)X_t = Z_t$ where
  $0 = \phi_1 = \xi_1^{-1} + \xi_2^{-1}$ and $0.8 = \phi_2 =-\xi_1^{-1} \xi_2^{-1}$. Thus,
  $\xi_1^{-1} = - \frac{2}{\sqrt{5}}$ and $\xi_2^{-1} = \frac{2}{\sqrt{5}}$ so
  \[
    X_t - 0.8 X_{t-2} = \left(1 + \frac{2}{\sqrt{5}}B\right)\left(1 - \frac{2}{\sqrt{5}}B\right)X_t = Z_t.
  \]
  The covariance function of this AR(2) process is given by
  \[
    \gamma(h) = \frac{\sigma^2 \xi_1^2 \xi_2^2}{(\xi_1 \xi_2 - 1)(\xi_2 - \xi_1)}\left[\frac{\xi_1^{1-|h|}}{\xi_1^2 - 1} - \frac{\xi_2^{1-|h|}}{\xi_2^2 - 1}\right].
  \]
  Using $\xi_1 = -\frac{\sqrt{5}}{2} $ and $\xi_2 = \frac{\sqrt{5}}{2}$, we see that for our process,
  \[
    \gamma(h) = \frac{5\sqrt{5}\sigma^2}{9}\left[\left(\frac{\sqrt{5}}{2}\right)^{1-|h|} - \left(\frac{-\sqrt{5}}{2}\right)^{1-|h|}\right].
  \]
  As $\gamma(0) = \frac{25 \sigma^2}{9}$, the autocorrelation function of this process is given by
  \[
    \rho(h) = \frac{\gamma(h)}{\gamma(0)} = \frac{\sqrt{5}}{5}\left[\left(\frac{\sqrt{5}}{2}\right)^{1-|h|} - \left(\frac{-\sqrt{5}}{2}\right)^{1-|h|}\right].
  \]
\end{proof}


\begin{custompbm}{2.b}
  Let $X_t$ be the AR(2) process such that $X_t = 0.8X_{t-2} + Z_t$ where $\{Z_t\}\sim \text{WN}(0, \sigma^2)$.
  Find the partial autocorrelation function of $X_t$.
\end{custompbm}

\begin{proof}
  The partial autocorrelation function $\alpha(h)$ is defined as $\alpha(0) = 1$,
  and for $h > 0$, $\alpha(h) = \phi_{hh}$ where $\phi_{hh}$ is the last component of
  \[
    \phi_h =
    \begin{bmatrix}
      \gamma(0) & \gamma(1) & \dots & \gamma(h-1) \\
      \gamma(1) & \gamma(0) & \dots & \gamma(h-2) \\
      \vdots & \vdots & \ddots & \vdots \\
      \gamma(h - 1) & \gamma(h-2) & \dots & \gamma(0)
    \end{bmatrix}^{-1}
    \begin{bmatrix}
      \gamma(1) \\
      \gamma(2) \\
      \vdots\\
      \gamma(h) \\
    \end{bmatrix}.
  \]
  Note for an $AR(p)$ process that $\alpha(h) = 0$ if $h > p$ and $\alpha(p) = \phi_p$. So for our process, we need only determine
  $\alpha(1)$. From the above,
  \[
    \alpha(1) = \frac{\gamma(1)}{\gamma(0)} = 0.
  \]
  Therefore, for our AR(2) process, the partial autocorrelation function is
  \[
    \alpha(h) =
    \begin{cases}
      1 & \text{if $h = 0$}\\
      0 & \text{if $|h| = 1$}\\
      0.8 & \text{if $|h| = 2$} \\
      0 & \text{if $|h| > 2$}
    \end{cases}.
  \]
\end{proof}


\end{document}
