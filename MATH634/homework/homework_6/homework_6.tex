\documentclass[12pt]{article}
\usepackage[margin=1in]{geometry}
\usepackage{amsmath, amsthm, amssymb, amsfonts, breqn}


\theoremstyle{definition}
\newtheorem{problem}{Problem}
\renewcommand*{\proofname}{Solution}
\newenvironment{custompbm}[1]
  {\renewcommand\theproblem{#1}\problem}
  {\endproblem}
\renewcommand{\theenumi}{\alph{enumi}}


\newcommand{\E}{\text{E}}
\newcommand{\V}{\text{Var}}
\newcommand{\Co}[2]{\text{Cov}({#1}, {#2})}
\newcommand{\pdf}{\text{pdf}}
\newcommand{\pmf}{\text{pmf}}
\newcommand{\me}{\mathrm{e}}
\newcommand*\diff{\mathop{}\!\mathrm{d}}
\newcommand{\vect}[1]{\boldsymbol{#1}}
\newcommand{\mx}[1][t]{\mu_X({#1})}
\newcommand{\gx}[2]{\gamma_X({#1}, {#2})}


\title{Homework Assignment 6}
\author{Matthew Tiger}


\begin{document}


\maketitle


% Problem 3.1
\begin{custompbm}{3.1}
  Determine which of the following ARMA processes are causal and which of them
  are invertible. (In each case $\{ Z_t \}$ denotes white noise.)
  \begin{enumerate}
    \item $X_t + 0.2X_{t-1} - 0.48X_{t-2} = Z_t$
    \item $X_t + 1.9X_{t-1} + 0.88X_{t-2} = Z_t + 0.2Z_{t-1} + 0.7Z_{t-2}$
    \item $X_t + 0.6X_{t-1} = Z_t + 1.2Z_{t-1}$
    \item $X_t + 1.8X_{t-1} + 0.81X_{t-2} = Z_t$
    \item $X_t + 1.6X_{t-1} = Z_t - 0.4Z_{t-1}  + 0.04Z_{t-2} $
  \end{enumerate}
\end{custompbm}

\begin{proof}
  A general $\text{ARMA}(p, q)$ process
  $X_t - \phi_1 X_{t-1} - \dots - \phi_p X_{t-p} = Z_t + \theta_1 Z_{t-1} + \dots + \theta_q Z_{t-q}$
  can be rewritten as $\phi(B)X_t = \theta(B)Z_t$, where $B$ is the backshift
  operator and $\phi(x) = 1 - \phi_1 x - \dots - \phi_p x^p$ and
  $\theta(x) = 1 + \theta_1 x + \dots + \theta_q x^q$. A stationary solution of the
  above process exists if and only if $\phi(x) = 0$ for $|x| \ne 1$. Similarly,
  if a stationary solution to the process exists, then the process is causal
  if and only the solutions to $\phi(x) = 0$ have norm greater than 1 and the
  process is invertible if the solutions to $\theta(x) = 0$ have norm greater
  than 1.
  \begin{enumerate}
    \item For this process, $\phi(x) = 1 - (-0.2)x - (0.48)x^2$ and
      $\theta(x) = 1$. The solutions to $\phi(x) = 0$ are $x_1 = 1.\overline{66}$ and $x_2=-1.25$
      both of which have norm greater than 1. Therefore a stationary solution
      exists and the process is causal. Similarly, $\theta(x) \ne 0$ for any
      $x$ so it is invertible as well.
    \item For this process, $\phi(x) = 1 - (-1.9)x - (-0.88)x^2$ and
      $\theta(x) = 1 + 0.2x + 0.7x^2$. The solutions to $\phi(x) = 0$ are
      $x_1 = -0.\overline{90}$ and $x_2 =-1.25$. Therefore a stationary solution
      exists however since $|x_1| \leq 1$ the process is not causal. Similarly, the
      solutions to $\theta(x) = 0$ are $x_1 = -0.1429 + 1.1867i$
      and $x_2 =  -0.1429 - 1.1867i$ both of
      which have norm greater than 1. Therefore the process is invertible.
    \item For this process, $\phi(x) = 1 - (-0.6)x$ and
      $\theta(x) = 1 + 1.2x$. The solution to $\phi(x) = 0$ is
      $x_1 = -1.\overline{66}$ which has norm greater than 1. Therefore a stationary solution
      exists and the process is causal. Similarly, the
      solution to $\theta(x) = 0$ is $x_1 = -0.8\overline{33}$ which has norm
      less than 1. Therefore the process is not invertible.
    \item For this process, $\phi(x) = 1 - (-1.8)x - (-0.81)x^2$ and
      $\theta(x) = 1$. The solution to $\phi(x) = 0$ is $x_1 = -1.\overline{11}$ with multiplicity 2
      which has norm greater than 1. Therefore a stationary solution
      exists and the process is causal. Similarly, $\theta(x) \ne 0$ for any
      $x$ so it is invertible as well.
    \item For this process, $\phi(x) = 1 - (-1.6)x$ and
      $\theta(x) = 1 + (-0.4)x + (0.04)x^2$. The solution to $\phi(x) = 0$ is
      $x_1 = -0.625$ which has norm less than 1. Therefore a stationary solution
      exists however the process is not causal. Similarly, the
      solution to $\theta(x) = 0$ is $x = 5$ with multiplicity 2 which has norm
      greater than 1. Therefore the process is invertible.
  \end{enumerate}
\end{proof}


% Problem 3.3
\begin{custompbm}{3.3}
  For those processes in Problem 3.1 that are causal, compute the first six
  coefficients $\psi_0, \psi_1, \dots, \psi_5$ in the causal representation
  $X_t = \sum_{j=0}^\infty \psi_j Z_{t-j}$ of $\{X_t\}$.
\end{custompbm}

\begin{proof}
\end{proof}


\end{document}
