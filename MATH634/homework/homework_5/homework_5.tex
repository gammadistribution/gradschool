\documentclass[12pt]{article}
\usepackage[margin=1in]{geometry}
\usepackage{amsmath, amsthm, amssymb, amsfonts, breqn}


\theoremstyle{definition}
\newtheorem{problem}{Problem}
\renewcommand*{\proofname}{Solution}
\newenvironment{custompbm}[1]
  {\renewcommand\theproblem{#1}\problem}
  {\endproblem}
\renewcommand{\theenumi}{\alph{enumi}}


\newcommand{\E}{\text{E}}
\newcommand{\V}{\text{Var}}
\newcommand{\Co}[2]{\text{Cov}({#1}, {#2})}
\newcommand{\pdf}{\text{pdf}}
\newcommand{\pmf}{\text{pmf}}
\newcommand{\me}{\mathrm{e}}
\newcommand*\diff{\mathop{}\!\mathrm{d}}
\newcommand{\vect}[1]{\boldsymbol{#1}}
\newcommand{\mx}[1][t]{\mu_X({#1})}
\newcommand{\gx}[2]{\gamma_X({#1}, {#2})}


\title{Homework Assignment 5}
\author{Matthew Tiger}


\begin{document}


\maketitle


% Problem 2.7
\begin{custompbm}{2.7}
   Show, using the geometric series $1/(1-x) = \sum_{j=0}^\infty x^j$ for $|x| < 1$,
   that $1/(1-\phi z) = -\sum_{j=1}^\infty \phi^{-j} z^{-j}$ for $|\phi| > 1$ and $|z| \geq 1$.
\end{custompbm}

\begin{proof}
  If $|\phi| > 1$ and $|z| \geq 1$, then $|\phi z| = |\phi| |z| > 1$ and
  $|1 / \phi z| = 1 / |\phi z|  < 1$. Now,
  \[
    \sum_{j=0}^\infty \phi^{-j} z ^ {-j} = \sum_{j=0}^\infty (\phi z )^{-j} = \sum_{j=0}^\infty \left(\frac{1}{\phi z}\right)^j.
  \]
  Note that this is a geometric series where $|\frac{1}{\phi z}| < 1$. So the
  series converges and
  \[
    \sum_{j=0}^\infty \left(\frac{1}{\phi z}\right)^j = \frac{1}{1 - \frac{1}{\phi z}} = \frac{\phi z}{\phi z - 1}.
  \]
  This implies that
  \[
    \frac{\phi z}{\phi z - 1}  = \sum_{j=0}^\infty \left(\frac{1}{\phi z}\right)^j = 1 + \sum_{j=1}^\infty \left(\frac{1}{\phi z}\right)^j
  \]
  so that
  \[
    \sum_{j=1}^\infty \left(\frac{1}{\phi z}\right)^j = \frac{\phi z}{\phi z - 1} - 1 = \frac{\phi z - (\phi z - 1)}{\phi z - 1} = \frac{1}{\phi z - 1}.
  \]
  From this identity it is clear that
  \[
    - \sum_{j=1}^\infty \phi^{-j} z ^ {-j} = - \sum_{j=1}^\infty \left(\frac{1}{\phi z}\right)^j = \frac{1}{-(\phi z -1)} = \frac{1}{1 - \phi z}
  \]
  and we are done.
\end{proof}


% Problem 2
\begin{problem}
  Prove that if $|\phi|>1$, then the unique stationary solution is given by (2.3.4).
\end{problem}

\begin{proof}
  Let $\phi(z) = 1 - \phi z$ and $\theta(z) = 1 + \theta z$
  and suppose that $|\phi| > 1$. Note that the ARMA(1, 1) process is defined
  by the equation $\phi(B)X_t = \theta(B)Z_t$ where $B$ is the back-shift
  operator. Using our definition of $\phi(z)$, we can see that the power series
  expansion of $1/\phi(z)$ is given by
  \[
    \frac{1}{\phi(z)} = \frac{1}{1-\phi z} = - \sum_{j=1}^{\infty} \phi^{-j}z^{-j}
  \]
  which converges since $|1/\phi|<1$. From this it is clear that
  $X_t = (1/\phi(B))\theta(B)Z_t$ and, evaluating $(1/\phi(B))\theta(B)$, we see that
  \begin{align*}
    \frac{1}{\phi(B)}\theta(B)
    &= -(\phi^{-1} B^{-1} + \phi^{-2} B^{-2} + \phi^{-3} B^{-3} + \dots )(1 + \theta B)\\
    &= (\phi^{-1} B^{-1} + \phi^{-2} B^{-2} + \phi^{-3} B^{-3} + \dots )(-1 - \theta B)\\
    &= -\phi^{-1}B^{-1} - \phi^{-2}B^{-1} - \phi^{-3}B^{-2} + \dots
    + -\theta\phi^{-1} - \theta\phi^{-2}B^{-1} - \theta\phi^{-3} B^{-2} + \dots\\
    &= -\theta\phi^{-1} + (-\theta\phi^{-2} - \phi^{-1})B^{-1} + (-\theta\phi^{-3} - \phi^{-2})B^{-2} + \dots\\
    &= -\theta\phi^{-1} - (\theta + \phi)\sum_{j=1}^{\infty} \phi^{-j-1}B^{-j}.
  \end{align*}
  Using this derivation we notice that
  \begin{align}
    X_t
    &= \frac{1}{\phi(B)}\theta(B) Z_t \notag \\
    &= -\theta\phi^{-1}Z_t - (\theta + \phi)\sum_{j=1}^{\infty} \phi^{-j-1}B^{-j} Z_t \notag \\
    &= -\theta\phi^{-1}Z_t - (\theta + \phi)\sum_{j=1}^{\infty} \phi^{-j-1}Z_{t+j} \notag \\
    &= -\theta\phi^{-1}Z_t - (\theta + \phi)\sum_{j=-\infty}^{-1} \phi^{j-1}Z_{t-j} \label{sum}.
  \end{align}
  It is then clear that $X_t$ is of the form $\sum_{j=-\infty}^\infty \psi_j Z_{t-j}$
  where $\psi_0 = -\theta\phi^{-1}$ and $\psi_j = -(\theta + \phi)\phi^{j-1} \text{for $j\leq -1$}$.
  If $|\phi|>1$, then $\sum_{j=-\infty}^\infty |\psi_j| =\sum_{j=-\infty}^{-1} |\psi_j| < \infty$. This fact
  combined with the fact that $\{Z_t\}$ is stationary shows that $\{X_t\}$ must be stationary by proposition 2.2.1.
  Therefore \eqref{sum} is the unique stationary solution to the ARMA(1, 1) equation when $|\phi| > 1$.
\end{proof}


% Problem 3
\begin{problem}
  Prove that if $|\theta| < 1$, then the ARMA(1, 1) process is invertible.
\end{problem}

\begin{proof}
  Note that the ARMA(1, 1) process is invertible if $Z_t$ is expressed in
  terms of $X_s$ for $s\leq t$. Let $\phi(z) = 1 - \phi z$ and $\theta(z) = 1 + \theta z$
  and suppose that $|\theta| < 1$.
  Then the ARMA(1, 1) equation can be written as $\phi(B)X_t = \theta(B)Z_t$ where
  $B$ is the back-shift operator. Note that the power series expansion of $1/\theta(z)$
  is given by
  \[
    \frac{1}{\theta(z)} = \frac{1}{1+\theta z} = \sum_{j=0}^{\infty} (-\theta)^j z^j
  \]
  which converges since $|\theta| < 1$. From this it is clear that
  $Z_t = (1/\theta(B))\phi(B)X_t$ and, evaluating $(1/\theta(B))\phi(B)$, we see that
  \begin{align*}
    \frac{1}{\theta(B)}\phi(B)
    &= (1 -\theta B + (-\theta)^2 B^2 + \dots )(1 - \phi B)\\
    &= 1 + (-\theta) B + (-\theta)^2 B^2 + \dots + -\phi B + (-1)^2 \phi \theta B^2 + (-1)^3 \phi \theta^2 B^3 + \dots \\
    &= 1 + -(\theta + \phi)B + (-1)^2(\theta^2 + \phi\theta)B^2 + (-1)^3(\theta^3 + \phi\theta^2)B^3 + \dots \\
    &= 1 -(\theta + \phi)\sum_{j=1}^\infty (-\theta)^{j-1}B^{j}.
  \end{align*}
  Using this derivation we notice that
  \begin{align*}
    Z_t
    &= \frac{1}{\theta(B)}\phi(B) X_t \\
    &= X_t -(\theta + \phi)\sum_{j=1}^\infty (-\theta)^{j-1}B^{j}X_t \\
    &= X_t -(\theta + \phi)\sum_{j=1}^\infty (-\theta)^{j-1}X_{t-j}.
  \end{align*}
  But this shows that $Z_t$ is expressed in terms of $X_s$ where $s\leq t$,
  showing that when $|\theta| < 1$, the ARMA(1, 1) process is invertible.
\end{proof}


% Problem 4
\begin{problem}
  Prove that if $|\theta| > 1$, then the ARMA(1, 1) process is non-invertible.
\end{problem}

\begin{proof}
  Note that the ARMA(1, 1) process is non-invertible if $Z_t$ is expressed in
  terms of $X_s$ for $s\geq t$. Let $\phi(z) = 1 - \phi z$ and $\theta(z) = 1 + \theta z$
  and suppose that $|\theta| > 1$.
  Then the ARMA(1, 1) equation can be written as $\phi(B)X_t = \theta(B)Z_t$ where
  $B$ is the back-shift operator. Note that the power series expansion of $1/\theta(z)$
  is given by
  \[
    \frac{1}{\theta(z)} = \frac{1}{1+\theta z} = -\sum_{j=1}^{\infty} (-\theta)^{-j} z^{-j}.
  \]
  Since $|\theta| > 1$, we know that $|1/\theta| < 1$ and this series converges.
  From this it is clear that $Z_t = (1/\theta(B))\phi(B)X_t$ and,
  evaluating $(1/\theta(B))\phi(B)$, we see that
  \begin{align*}
    \frac{1}{\theta(B)}\phi(B)
    &= -((-\theta)^{-1} B^{-1} + (-\theta)^{-2} B^{-2} + (-\theta)^{-3} B^{-3} + \dots )(1 - \phi B)\\
    &= ((-\theta)^{-1} B^{-1} + (-\theta)^{-2} B^{-2} + (-\theta)^{-3} B^{-3} + \dots )(\phi B - 1)\\
    &= -\phi\theta^{-1} + (-1)^2\phi\theta^{-2}B^{-1} + (-1)^3\phi\theta^{-3}B^{-2} + \dots \\
    & \hspace{5mm} + (-1)^2\theta^{-1}B^{-1} + (-1)^{3}\theta^{-2}B^{-2} + (-1)^4\theta^{-3} B^{-3} + \dots\\
    &= -\phi\theta^{-1} + (-1)^2(\phi\theta^{-2} + \theta^{-1})B^{-1} + (-1)^3(\phi\theta^{-3} + \theta^{-2})B^{-2} + \dots\\
    &= -\phi\theta^{-1} + (\phi + \theta)\sum_{j=1}^{\infty} (-\theta)^{-j-1}B^{-j}
  \end{align*}
  Using this derivation we notice that
  \begin{align*}
    Z_t
    &= \frac{1}{\theta(B)}\phi(B) X_t \\
    &= -\phi\theta^{-1}X_t +(\phi + \theta)\sum_{j=1}^\infty (-\theta)^{-j-1}B^{-j}X_t \\
    &= -\phi\theta^{-1}X_t +(\phi + \theta)\sum_{j=1}^\infty (-\theta)^{-j-1}X_{t+j}.
  \end{align*}
  But this shows that $Z_t$ is expressed in terms of $X_s$ where $s\geq t$,
  showing that when $|\theta| > 1$, the ARMA(1, 1) process is non-invertible.
\end{proof}


\end{document}
