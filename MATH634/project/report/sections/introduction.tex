With the advent of social media, the way in which society communicates has evolved. Popular
social media networks document these new forms of communications and as a result, a rich
set of data surrounding these interactions emerges. In particular, Twitter
remains one of the most popular social media platforms
to date. Twitter was first launched in July 2006 as a social networking service
designed to allow its users to communicate via short 140-character messages
called ``tweets''. These tweets are sometimes affixed by the sender with a meta-label
called a ``hashtag'', denoted by a string leading with the \# symbol, that is
meant to categorize the information contained in the
message.

In this report, we will analyze data pertaining to a popular television show
collected from Twitter's streaming API over the course of three weeks. This
analysis will consist of measuring the number of tweets that contain a certain
hashtag sent every hour over this timeframe. We will then fit a time
series model to these measurements and provide a forecasting model of the next
week's projected data in order to to predict the number of tweets that will occur during the show's next airing.