Twitter's streaming API along with our custom program allowed us to gather tweets
containing the hashtag ``\#thewalkingdead'' over a three week time period. Isolating
that data to 48 hour cycles centered at the show's air time gave us a time series
to analyze, namely the number of tweets containing the hashtag over that three week period.

Utilizing the time series plots of our data $\{X_t\}$ combined with differencing
techniques allowed us to determine a potential family of seasonal ARIMA models
to fit to the transformed mean-corrected data $\{Y_t\} = \{\log(X_t) - E(X_t)\}$.
Choosing the model that minimizes the AIC statistic, we arrived at the SARIMA
model in \eqref{sarima_model} with period 48.

With a model describing the transformed data, we applied R's forecasting software
to create a forecast of the next 48 hour cycle for the data. The shape of the forecast
seems plausible, however. The forecast is not very accurate in predicting the number of
tweets that will occur during the show's airing due to the large variance between the
bounds of the 90\% confidence interval as can be seen from Figure \ref{forecast_plot}
and entry 169 of the table in Appendix \ref{forecast_table}.

Thus, the model currently does not achieve its main goal of accurately predicting the number of
tweets at the show's future airing. However, the techniques outlined in this
report can be repeated with more Twitter data to derive a more accurate model.
